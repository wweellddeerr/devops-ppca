%%%%%%%%%%%%%%%%%%%%%%%%%%%%%%%%%%%%%%%%
% Classe do documento
%%%%%%%%%%%%%%%%%%%%%%%%%%%%%%%%%%%%%%%%

% Opções:
%  - Graduação: bacharelado|engenharia|licenciatura
%  - Pós-graduação: [qualificacao], mestrado|doutorado, ppca|ppginf

% \documentclass[engenharia]{UnB-CIC}%
\documentclass[mestrado,ppca]{UnB-CIC}%

\usepackage{pdfpages}% incluir PDFs, usado no apêndice

\usepackage{placeins}%

\usepackage{booktabs}

\newcommand{\cat}[1]{{\textbf{\emph{#1}}}}

\usepackage{array}

%%%%%%%%%%%%%%%%%%%%%%%%%%%%%%%%%%%%%%%%
% Informações do Trabalho
%%%%%%%%%%%%%%%%%%%%%%%%%%%%%%%%%%%%%%%%
\orientador{\prof \dr Rodrigo Bonif\'acio de Almeida}{CIC/UnB}%
\coordenador[a]{\prof[a] \dr[a] Alet\'eia Patr\'icia Favacho de Ara\'ujo}{CIC/UnB}%
\diamesano{15}{junho}{2018}%

\membrobanca{\prof[a] \dr[a] Edna Dias Canedo}{FGA/UnB}%
\membrobanca{\prof \dr Paulo Roberto Miranda Meirelles}{FGA/UnB}%

\autor{Welder Pinheiro}{Luz}%

\titulo{Uma Caracterização da Adoção de DevOps Utilizando Grounded Theory}%

\palavraschave{TCU, DevOps, desenvolvimento de software, operações de software,
Grounded Theory, adoção de DevOps}%
\keywords{TCU, DevOps, software development, software operations, Grounded Theory,
DevOps adoption}%

\newcommand{\unbcic}{\texttt{UnB-CIC}}%

%%%%%%%%%%%%%%%%%%%%%%%%%%%%%%%%%%%%%%%%
% Texto
%%%%%%%%%%%%%%%%%%%%%%%%%%%%%%%%%%%%%%%%
\begin{document}%
    \capitulo{1_Introducao}{Introdução}%
    \capitulo{2_Fundamentacao_Teorica}{Fundamentação Teórica}%
    \capitulo{3_GT_Adocao_DevOps}{Uma \textit{Grounded Theory} Sobre a Adoção de \textit{DevOps} no Mercado}%
    \capitulo{4_Modelo_Adocao_DevOps}{Um Modelo para Adoção de \textit{DevOps} e sua Aplicação no \acrshort{TCU}}%
    \capitulo{5_Avaliacao_Empirica}{Avaliação Empírica}%
    \capitulo{6_Conclusoes}{Conclusões}%

    %\apendice{Apendice_Fichamento}{Fichamento de Artigo Científico}%
    %\anexo{Anexo1}{Documentação Original \unbcic\ (parcial)}%
\end{document}%
