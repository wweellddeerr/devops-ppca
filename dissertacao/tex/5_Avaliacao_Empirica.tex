No capítulo anterior, foi descrito como tem se dado a implantação do modelo
lá proposto na adoção de \textit{DevOps} do TCU. As descrições lá constantes
são fruto de uma percepção individual. Este capítulo apresenta uma avaliação
empírica da situação atual do processo de adoção de \textit{DevOps} no \acrshort{TCU},
bem como da aplicação do modelo proposto no capítulo anterior, com base na
percepção dos profissionais envolvidos no processo. Esta avaliação empírica
foi conduzida com o objetivo de produzir uma descrição com menor risco de viés
do que as apresentadas no capítulo anterior.

\section{Grupo Focal}

Para obtenção da avaliação empírica proposta, foi utilizado o método qualitativo
de pesquisa \emph{grupo focal}.

A população envolvida no processo de adoção de \textit{DevOps} no TCU é
relativamente pequena e há um maior número de profissionais nos times de
desenvolvimento do que no time de operações. Para que as percepções obtidas
na avaliação não possuíssem o viés do time de desenvolvimento, foi adotada a
estratégia de ouvir um número similar de profissionais de cada um dos times.
Com essa limitação do número de profissionais do time de desenvolvimento que
poderiam ser ouvidos na pesquisa, o número total de participantes na amostra
(estimado entre 8 e 10 pessoas) mostrou-se pequeno para realização de um
questionário.

Ao mesmo tempo, a realização de entrevistas além de alongar a obtenção do
\textit{feedback} desejado, poderia gerar um grande número de respostas repetitivas, uma
vez que são todos profissionais envolvidos no mesmo processo de adoção de
\textit{DevOps}.

Segundo F. Shull et al. \cite{shull2007guide}, grupos focais são projetados
para obter percepções pessoais de membros de um ou mais grupos envolvidos em
uma área definida de interesse de pesquisa e possuem como benefícios a produção
de informações cândidas, muitas vezes perspicazes, com um baixo custo e rápida
execução. Estas características tornam o grupo focal uma alternativa adequada
aos propósitos desta pesquisa.

\section{\textit{Background}}
Apresentar aqui o background a respeito de grupo focal.

\section{Questões de Pesquisa}
Foram definidas cinco questões de pesquisa (QP1-5) para realização da avaliação
empírica usando grupo focal. As questões de pesquisa estão listadas na Tabela
\ref{tabela_qp} junto com as respectivas hipóteses para cada uma delas.

\begin{table}[hb!]
\centering
\caption{Questões de Pesquisa - Grupo Focal}
\label{tabela_qp}
\begin{tabular}{|p{1.cm}|p{7cm}|p{7cm}|} \hline
    & \textbf{Questão}                                                                                    & \textbf{Hipótese}                                                                                                                                                                                                             \\
\hline
QP1 & Qual a importância de o \acrshort{TCU} ter passado a focar na construção da cultura de colaboração? & Redução dos conflitos que persistiam, entre os times, mesmo com o uso de ferramentas \textit{DevOps}. \\
\hline
QP2 & Qual a importância do modelo apresentado para a mudança de foco do \acrshort{TCU}?                  & O modelo tornou clara a direção que deveria ser buscada, evitando ameaças relacionada ao foco em pontos errados (como ferramental e automação, por exemplo). \\
\hline
QP3 & Como o \acrshort{TCU} tem utilizado os \emph{facilitadores DevOps}?                                 & Na construção e no fortalecimento da cultura de colaboração. \\
\hline
QP4 & Como o \acrshort{TCU} tem utilizado as \emph{saídas DevOps}?                                        & Explorando-os como consequências obtidas da adoção de \textit{DevOps} e como resposta aos riscos inerentes à entrega contínua de \textit{software}, também propiciada pela adoção de \textit{DevOps}. \\
\hline
QP5 & Quais os próximos passos da adoção de \textit{DevOps} no \acrshort{TCU}?                            & Ajustes na estrutura organizacional para comportar o conceito de responsabilidades compartilhadas que faz parte da cultura de colaboração. \\
\hline
\end{tabular}
\end{table}

Para responder às questões de pesquisa QP2-3, foram utilizadas perguntas
envolvendo as categorias relativas a \emph{facilitadores DevOps} e
\emph{saídas DevOps}. Por exemplo, \cat{automação} é uma das categorias que são
consideradas \emph{facilitadores DevOps}, isso ocasionou a inclusão da seguinte
pergunta: ``\textit{Como automação tem sido utilizada na adoção de
\textit{DevOps} no \acrshort{TCU}?}''. O mesmo foi feito para cada uma das
demais categorias. Já as questões QP1-2 e QP5 foram feitas diretamente aos
participantes do grupo focal.

As discussões foram semiformais. Um pesquisador atuou como facilitador
fornecendo as perguntas aos participantes e conduzindo a discussão.
O grupo focal foi divido em sessões, uma para cada pergunta. Cada sessão iniciou
com a distribuição da respectiva pergunta a cada participante. Os participantes
foram orientados a escrever suas ideias e palavras-chave sobre a pergunta em
notas \textit{post-it}. Depois disso, as notas foram postas em um quadro branco.
As notas serviram como ponto de partida para a realização de discussões com o
propósito de se obter conclusões coletivas a respeito da respectiva pergunta.

Foram realizados dois grupos focais com duração de 1h30m cada. Foram convidados
a participar, dois profissionais do time de operações e dois de times
de desenvolvimento, todos com participação ativa na adoção de \textit{DevOps}
no \acrshort{TCU}.

\section{Resultados}

Nesta seção são apresentados os resultados das sessões do grupo focal para cada
questão de pesquisa.

\subsection{QP1 - Qual a importância de o \acrshort{TCU} ter passado a focar na
construção da cultura de colaboração?}

\subsection{QP2 - Qual a importância do modelo apresentado para a mudança de
foco do \acrshort{TCU}?}

\subsection{QP3 - Como o \acrshort{TCU} tem utilizado os \emph{facilitadores
DevOps}?}

\subsection{QP4 - Como o \acrshort{TCU} tem utilizado as \emph{saídas
DevOps}?}

\subsection{QP5 - Quais os próximos passos da adoção de \textit{DevOps} no
\acrshort{TCU}?}
