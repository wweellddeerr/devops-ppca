\section{A Categoria Principal: Cultura de Colaboração}\label{secao_core_category}

A categoria principal da adoção de {\it DevOps} é \cc. A \cc visa essencialmente
remover os silos entre os times de desenvolvimento e operações, o que até certo
ponto se confunde com os próprios objetivos de {\it DevOps}. Inicialmente, uma
\cc envolve que as tarefas tipicamente de operações --- como {\it deployment},
provisionamento e gerenciamento de infraestrutura e monitoramento --- devem ser
consideradas atividades regulares, parte do dia a dia de desenvolvimento de
{\it software}. Os entrevistados relatam que nas suas organizações não são
mais aguardados momentos específicos para performar essas atividades, elas são
executadas continuamente. Isso leva ao primeiro conceito relacionado à
categoria principal: {\bf operações no dia a dia de desenvolvimento}.

\begin{mq}
``\emph{Uma etapa muito importante foi trazer o {\it deployment} para dentro do
dia a dia de desenvolvimento, sem ter que ficar esperando um dia específico
da semana ou do mês. Nós queríamos fazer {\it deploy} o tempo inteiro, mesmo
que em um primeiro momento não fosse em produção, um ambiente de {\it staging}
era suficiente. [...] O que a gente queria era incorporar o {\it deployment} ao
desenvolvimento. É claro que para que a gente pudesse fazer o {\it deployment}
continuamente, a gente tinha que prover toda a infraestrutura necessária no
mesmo ritmo.}'' (P14, Gerente de \acrshort{TI}, Brasil)
\end{mq}

Sem {\it DevOps}, um cenário comumente descrito é a existência de um processo acelerado de
desenvolvimento de {\it software} sem preocupações relacionadas a operações.
No final, quando o time de desenvolvimento tem um mínimo produto viável, ele
o envia para o time de operações para publicação. Conhecendo poucas coisas sobre
a natureza do {\it software} e como ele foi produzido, o time de operações tem
que criar e configurar o ambiente necessário e publicar o {\it software}. Neste
cenário, a entrega de {\it software} tipicamente atrasa e os conflitos entre
os times se manifestam. Quando uma \cc é fomentada, os times colaboram para
executar as tarefas desde o primeiro dia do desenvolvimento do {\it software}.
Com o constante exercício das práticas de provisionamento, gerenciamento,
configuração e {\it deployment}, a entrega de {\it software} se torna mais
natural, reduzindo os atrasos e, consequentemente, os conflitos entre os times.

\begin{mq}
``\emph{Nós trabalhamos usando uma abordagem ágil, com {\it sprints} de quinze
dias, onde a gente foca em produzir {\it software} e gera novas versões em
altíssima frequência. Mas, na hora da entrega do {\it software} é que as
complicações começavam a aparecer. O trabalho de construir todo o ambiente
e fazer o {\it deploy} não fazia parte das {\it sprints}, a gente focava apenas
em codificar a aplicação. As entregas frequentemente atrasavam,
e a gente tinha que entregar com atrasos de semanas, o que não era bom nem para
nós e nem para os clientes.}'' (P6, {\it Developer}, Portugal)
\end{mq}

Um dos resultados da construção de uma \cc é que o time de desenvolvimento não
precisa mais parar o seu trabalho aguardando pela disponibilização de um
servidor de aplicação, pela execução de um {\it script} na base de dados, ou
ainda pela publicação de uma nova versão do {\it software} em um ambiente de
testes. Todos os envolvidos precisam conhecer a maneira como todas essas coisas
são feitas e, com a colaboração do time de operações, isso pode ser executado
de maneira regular. Se uma tarefa pode ser executada pelo time de desenvolvimento
e há confiança entre os times, essa tarefa será incorporada ao processo de
desenvolvimento de uma maneira natural, manifestando o segundo conceito
relacionado à categoria \cc: {\bf empoderamento do desenvolvimento de \emph{software}}.

\begin{mq}
``\emph{Não era viável a gente ter tantos desenvolvedores produzindo artefatos e
parando o trabalho deles para aguardar outro time completamente separado
publicar. Ou também precisar de um ambiente de teste e ter que esperar o time
de operações prover isto quando possível. Essas atividades tinham que estar
disponíveis para servir rapidamente ao time de desenvolvimento. Com DevOps
nós conseguimos suprir essa necessidade de liberdade e mais poder para executar
algumas tarefas que são intrisicamente relacionadas ao trabalho deles.}''
(P5, {\it Systems Engineer}, Espanha)
\end{mq}

Uma \cc requer {\bf pensamento de produto}, em substituição a pensamento de
desenvolvimento ou de operações. O time de desenvolvimento precisa compreender
que o {\it software} é um produto que não se encerra após o ``{\it push}'' do código
para o repositório de código-fonte, e o time de operações precisa entender que
o processo também não se inicia quando um artefato é recebido para publicação.
{\bf Pensamento de produto} é o terceiro conceito relacionado à categoria
principal.

\begin{mq}
``\emph{Nós alteramos o perfil profissional buscado em nossas contratações. Nós
queríamos contratar pessoas que tivessem uma visão de produto. Pessoas que eram
capazes de olhar para um problema e pensar na melhor solução para ele. Mas não
apenas pensar em uma solução de software, pensar também no momento em que essa
aplicação vai ser publicada. Nós também reunimos os desenvolvedores para
reforçar que todos deviam atuar dessa maneira. Todos deviam pensar no produto e
não apenas em seu código ou em sua infraestrutura.}'' (P12, {\it Cloud Engineer},
Estados Unidos)
\end{mq}

Deve haver uma {\bf comunicação direta} entre os times. Sistemas de {\it
ticket} são citados como um meio típico e inadequado de comunicação entre os
times de desenvolvimento e operações. A comunicação face a face é a melhor
opção, mas considerando que nem sempre é viável, o uso contínuo de ferramentas
como \emph{Slack}\footnote{https://slack.com/} e \emph{Hipchat}\footnote{https://www.hipchat.com/}
é citado como opção apropriada. {\bf Comunicação direta} é o quarto conceito
relacionado à categoria principal.

\begin{mq}
``\emph{Nós também usamos essa ferramenta ({\it Hipchat})
como uma maneira de facilitar a comunicação entre os times de desenvolvimento e
operações. O ritmo de trabalho é bastante acelerado, e por isso não é viável ter
uma comunicação burocrática. Pra agilizar as coisas, a gente usa bastante o
{\it Hipchat}, todos estão sempre atentos às mensagens, as respostas são
rápidas e a gente tem um controle bem adequado por lá. Isso deu muita liberdade
nas tarefas de desenvolvimento, em caso de qualquer dúvida, a equipe de
operações está ao alcance de uma mensagem.}'' (P5, {\it Systems Engineer}, Espanha)
\end{mq}

Existe uma {\bf responsabilidade compartilhada} de identificar e corrigir os
problemas de um {\bf \emph{software}} ao fazer a transição para produção. A estratégia
de fugir da responsabilidade deve ser evitada. A equipe de desenvolvimento não
deve afirmar que uma determinada questão é um problema na infraestrutura, então
é responsabilidade da equipe de operações. Ou o contrário, a equipe de operações
não deve afirmar que uma determinada falha foi motivada por um problema na
aplicação, então é responsabilidade da equipe de desenvolvimento. Um contexto de
{\bf \emph{blameless}} deve ser fomentado. Os times precisam focar na resolução dos
problemas e não em encontrar um culpado e fugir da responsabilidade. O contexto
de {\bf responsabilidade compartilhada} envolve não apenas a resolução de
problemas, mas também qualquer outra responsabilidade inerente ao produto de
{\it software} deve ser compartilhada. {\bf \emph{Blameless}} e {\bf responsabilidade
compartilhada} são os conceitos restantes da categoria principal.

\begin{mq}
``\emph{Como consequência dessa busca contínua por melhoria da qualidade, em um
momento já avançado do processo, quando nós já tínhamos um bom nível de
colaboração, automação e tudo mais, nós identificamos um ponto de melhoria na
nossa cultura. Nós percebemos que algumas pessoas tinham medo de cometer erros.
Nossa cultura não era forte o suficiente para fazer com que todos se sentissem
à vontade para inovar e experimentar sem medo de errar. Nós fizemos um grande
esforço para espalhar essa ideia de que não há culpados por qualquer problema
que possa ocorrer. Nós fazemos todo o possível para evitar falhas, mas elas
vão acontecer, e apenas sem essa busca por culpados nós vamos ser capazes de
resolver os problemas rapidamente.}'' (P8, {\it DevOps Engineer}, Brasil)
\end{mq}

À primeira vista, considerar a criação e o fortalecimento da \cc como o passo
mais importante para a adoção de {\it DevOps} pode parecer um tanto óbvio, mas
os próprios entrevistados citaram alguns equívocos que consideram recorrentes
em não priorizar esse aspecto na adoção de {\it DevOps}:

\begin{mq}``\emph{Na adoção de {\it DevOps}, há uma questão cultural muito forte
que os times muitas vezes não estão adaptados. Relacionado a isso, uma coisa
que me incomoda muito e que eu vejo acontecer muito é que as pessoas tomam
{\it DevOps} exclusivamente por ferramentas ou automação}'' (P9, Gerente de
\acrshort{TI}, Brasil)
\end{mq}

\begin{mq}
``\emph{DevOps envolve ferramental, mas DevOps não é ferramental. Ou seja, as
pessoas muitas vezes focam no uso de ferramentas que são chamadas de
`ferramentas Dev-Ops', acreditando que DevOps é isto. Eu sempre insisto que
DevOps não é uma ferramenta, DevOps envolve o uso de ferramentas para melhorar
os procedimentos de desenvolvimento de software.}'' (P2, Consultor {\it DevOps}, Brasil)
\end{mq}

Além da categoria principal (\cc), foram identificados três outros conjuntos de
categorias: os \emph{facilitadores} da adoção de {\it DevOps}, as \emph{saídas}
da adoção de {\it DevOps}, e as categorias que são tanto facilitadores como
saídas. Uma explicação mais abrangente sobre esses dois papéis desempenhados
pelas categorias de conceitos em uma adoção de {\it DevOps} será apresentada
na seção \ref{secao_teoria}. Nas próximas seções as categorias são descritas
por meio dos seus conceitos relacionados.

\section{Facilitadores \emph{DevOps}}\label{secao_facilitadores}

Aqui serão detalhadas as duas categorias que sustentam a adoção das práticas
{\it DevOps}: \cat{automação} e \cat{compartilhamento e transparência}.

\subsection{Automação} \label{ssec:automation}

Essa é a categoria que apresenta o maior número de conceitos relacionados. Isso
ocorre porque procedimentos manuais são considerados fortes candidatos para
propiciar a formação de um silo, dificultando a criação de uma \cc. Se uma tarefa
é manual, uma pessoa ou um time será responsável por executá-la. Apesar de
\cat{compartilhamento e transparência} poderem ser usados para garantir a
colaboração mesmo em tarefas manuais, com a \cat{automação}, os pontos onde os
silos podem surgir são minimizados.

\begin{mq}
``\emph{Quando um desenvolvedor precisava criar uma nova aplicação, o workflow
antigo exigia que ele criasse um ticket para a equipes de operações, que então
avaliava e resolvia manualmente o problema solicitado. Essa tarefa podia levar
muito tempo e não havia visibilidade entre os times sobre o que estava
acontecendo [\ldots]. Hoje, esses silos não existem mais dentro da empresa,
em particular porque não é mais necessário executar todas essas tarefas
manualmente, tudo foi automatizado.}'' (P12, {\it Cloud Engineer}, Estados Unidos)
\end{mq}

Além de contribuir para transparência, a \cat{automação} de procedimentos
também é considerada importante para garantir \emph{reprodutibilidade} de
tarefas, reduzindo retrabalho e risco de falha humana. Consequentemente,
\cat{automação} aumenta a confiança entre os times o que é um aspecto importante
da \cc.

\begin{mq}
``\emph{Antes de nós adotarmos DevOps, havia muito trabalho manual. Por exemplo,
se você precisasse criar um esquema no banco de dados, era um processo manual;
se você precisava criar um servidor de banco de dados, era um processo manual;
se você precisasse criar instâncias EC2\footnote{{\it Amazon Elastic Compute Cloud}}
adicionais, mais uma vez um processo manual. Este trabalho manual era demorado
e muitas vezes causava erros e retrabalho.}'' (P1, {\it DevOps Developer}, Irlanda)
\end{mq}

\begin{mq}
``\emph{Nossa principal motivação para adotar DevOps foi basicamente reduzir o
retrabalho. Quase toda semana a gente tinha que basicamente construir novos
servidores e iniciá-los manualmente, o que era muito demorado.}'' (P4, Técnico
em Computação, Brasil)
\end{mq}

Os oito conceitos da categoria \cat{automação} são detalhados a seguir.
Todas as entrevistas continham explicações sobre (1) {\bf automação do \emph{deployment}},
como parte da adoção de {\it DevOps}. A entrega de {\it software} é a
manifestação mais clara da entrega de valor no desenvolvimento de {\it software}.
Em caso de problemas no {\it deployment}, a expectativa de entregar valor ao
negócio pode rapidamente gerar conflitos e manifestar a existência de silos.
Desta forma, a \cat{automação} normalmente aumenta a agilidade e a confiabilidade.
Alguns outros conceitos de automação giram exatamente em torno da automação do
{\it deployment}.

É importante observar que a ocorrência frequente de {\it deployments} bem
sucedidos não é suficiente para garantir a geração de valor para o negócio.
Certamente, a qualidade do {\it software} é mais relevante. Portanto, para que possam
fazer parte do {\it pipeline} do {\it deployment}, as verificações de qualidade
também precisam ser automatizadas, como é o caso da (2) {\bf automação de testes}.
Além disso, para automatizar o {\it deployment} de aplicações, o ambiente em
que elas serão executadas precisa estar disponível. Portanto, a (3) {\bf automação
do provimento da infraestrutura} também deve ser considerada no processo. Além
de estar disponível, o ambiente precisa ser configurado adequadamente,
incluindo a quantidade de memória e CPU disponibilizada, as versões corretas de
bibliotecas  e a estrutura do banco de dados. Se a configuração de algum
desses aspectos não tiver sido automatizada, o {\it deployment} automatizado
pode não funcionar. Portanto, a (4) {\bf automação do gerenciamento da
infraestrutura} é outro conceito da categoria \cat{automação}.

{\it Software} moderno é tipicamente construído em torno de serviços.
Microsserviços foram comumente citados como um aspecto da adoção de {\it DevOps}.
Para Fowler e Lewis \cite{martinfowler2014microservices}, no estilo arquitetural
de microsserviços, os serviços precisam ser independentemente implantáveis por
mecanismos de {\it deployment} totalmente automatizados. Essa parte das
características de microsserviços relacionada a \cat{automação} foi aqui
denominada de (5) {\bf serviços autônomos}. Os relatos sobre a adoção de {\it DevOps}
tipicamente citam a (6) {\bf conteinerização} como uma maneira de automatizar o
provisionamento do ambiente onde esses serviços autônomos são executados: os
contêineres. (7) {\bf Automação do monitoramento} e (8) {\bf automação da
recuperação} são os conceitos restantes. O primeiro refere-se à capacidade de
monitorar as aplicações e a infraestrutura subjacente sem intervenção humana.
Um exemplo clássico é o uso generalizado de ferramentas para enviar mensagens
relatando alarmes - por meio de SMS, {\it Slack} / {\it Hipchat}, ou até mesmo
chamadas de celular - em caso de incidentes relacionados às aplicações
detectados automaticamente. E o segundo está relacionado à capacidade de
substituir um componente que não está funcionando adequadamente ou reverter uma
falha no {\it deployment} sem intervenção humana.

\subsection{Compartilhamento e Transparência}

Esta categoria representa o agrupamento de conceitos emergidos
nas entrevistas a respeito de atividades que ajudam a
disseminar conhecimento técnico e procedimental entre os times, de modo a incrementar a
colaboração entre eles. Ações de treinamento interno e externo, palestras,
discussões em grupos e {\it round tables} são exemplos desses eventos. Criar
canais usando ferramentas de comunicação é outra alteranativa recorrentemente
citada no \cat{compartilhamento e transparência} ao longo do processo de adoção de
{\it DevOps}. De acordo com o conteúdo e o meio onde há o compartilhamento,
foram identificados inicialmente três conceitos para esta categoria:

\begin{enumerate}
\item {\bf Compartilhamento de conhecimento}: os profissionais entrevistados
mencionaram que existe uma ampla gama de habilidades técnicas e culturais que
precisaram adquirir durante a adoção de {\it DevOps}. Como meio para suavizar a
curva de aprendizagem existente foi mencionada a realização de eventos de
compartilhamento de conhecimento, tais quais, ações estruturadas de treinamento
(cursos) com participação de profissionais dos dois times e participação em
eventos da comunidade de desenvolvimento de {\it software}.

\item {\bf Compartilhamento de atividades}: esse conceito trata de ações onde o
foco é em compartilhar a maneira como tarefas simples foram realizadas, por
exemplo, como um erro específico foi corrigido, ou detalhes de configuração de
alguma ferramenta específica. As próprias ferramentas de comunicação, junto com
existência de comitês e a realização de {\it round tables} foram citados como
fóruns adequados para compartilhamento deste tipo de conteúdo.

\item {\bf Compartilhamento de processos}: aqui, o foco é em compartilhar um
processo de trabalho como um todo. Por exemplo: como configurar um {\it pipeline}
no {\it Jenkins} utilizando {\it Jenkinsfile}. O conteúdo é mais abrangente do
que no compartilhamento de atividades. A realização de apresentações e palestras
foi citada como meio mais comum para o compartilhamento de processos.
\end{enumerate}

\begin{mq}
``\emph{Hoje em dia eu vejo as pessoas na empresa muito preocupadas em que todo
mundo saiba o que ela está fazendo e como ela está fazendo. Por isso a gente
tem essas ações estruturadas que eu te falei, se alguma pessoa quer repassar
algum conteúdo relevante para o restante da equipe, ela tem total liberdade para
reservar a sala e realizar a própria tech talk. Antigamente tinha um negócio de
meio que fazer as coisas às escondidas para o cara até mesmo tentar valorizar
o passe dele, `ah, só fulano sabe como fazer isso'. Agora com esse investimento
em horizontalizar a cultura, o cara sabe que tudo tem que ser compartilhado e
transparente para todos, e isso gera um ciclo positivo que aumenta o senso
de colaboração e faz ele querer compartilhar ainda mais o que ele faz.}''
(P7, Analista de Suporte, Brasil)
\end{mq}

Estes conceitos de compartilhamento contribuem com a \cc. Por exemplo, todos os
membros dos times ganham uma melhor compreensão sobre todo o processo de produção
de {\it software}, com um sólido entendimento de que as responsabilidades devem
ser compartilhadas. Um vocabulário compartilhado também tende a se formar por
meio dessas ações de compartilhamento e isso facilita a comunicação.

O uso de {\bf infraestrutura como código} foi recorrentemente citado como um
meio de se garantir que todos saibam como o ambiente de execução de aplicação
é provido e gerenciado. Abaixo, é apresentada um trecho de transcrição de
entrevista que resume bem este conceito:

\begin{mq}
``\emph{Então, aqui nós adotamos este tipo de estratégia
que é a infraestrutura como código, consequentemente, nós temos um versionamento
de toda a nossa infraestrutura em uma linguagem comum, de tal maneira que
qualquer pessoa, um desenvolvedor, um arquiteto, o cara de operações, ou mesmo
o gerente, ele olha e consegue descrever que a configuração da aplicação x é
y. Então, isto agrega muito valor para nós exatamente com mais transparência.}'' (P12, {\it Cloud Engineer}, Estados Unidos)
\end{mq}

Em relação a \cat{compartilhamento e transparência}, foi também identificado o
conceito de \textbf{compartilhamento em bases regulares}, que sugere que as
ações de compartilhamento devem ser incorporadas no processo de
desenvolvimento de {\it software}, de modo a contribuir de maneira eficaz
para a transparência. Conforme será detalhado no conceito de \emph{integração
contínua} da categoria \cat{agilidade} (subseção \ref{subsecao_agilidade}), uma
maneira comum de se integrar todas as tarefas é um {\it pipeline}. Aqui existe
o conceito de \textbf{\emph{pipelines} compartilhados}, que indica que o
código dos {\it pipelines} deve ser acessível a todos, buscando fomentar a
transparência.

\begin{mq}
``\emph{O código de como a infraestrutura é feita é aberto aos desenvolvedores
e os sysadmins precisam conhecer alguns aspectos de como o código da aplicação
é construído. O código dos nossos pipelines é acessível a todos na empresa
para saberem como as atividades estão automatizadas.}''
(P13, Gerente de Tecnologia, Brasil)
\end{mq}

\section{\emph{Saídas} da Adoção de \emph{DevOps}}\label{secao_saidas}

Nesta seção são detalhadas as categorias que correspondem às consequências
esperadas do processo de adoção de {\it DevOps}: \cat{agilidade} e \cat{resiliência}.

\subsection{Agilidade}\label{subsecao_agilidade}

A maior {\it agilidade} dos times foi frequentemente descrita como um dos
principais resultados da adoção do {\it DevOps}. Com mais colaboração entre os
times, a {\bf integração contínua} com execução de {\it pipelines}
multidisciplinares é possível e este é um conceito relacionado a agilidade que
é frequentemente explorado. Esses {\it pipelines} podem conter provimento
de infraestrutura, testes automatizados, análise de código, {\it deployment}
automatizado e qualquer outra tarefa considerada importante no processo de
desenvolvimento para que possa ser executada continuamente.

\begin{mq}
``\emph{O nosso pipeline tem etapas com toda uma estrutura para criar e destruir
contêineres de maneira rápida e gerar ambientes rapidamente utilizando Docker.
Ele também passa pelo Sonar\footnote{https://www.sonarqube.org/} para
analisar o código utilizando os plugins e bibliotecas dele [...]. Tem também os
testes automatizados e em qualquer dessas etapas que estou falando o pipeline
pode ser interrompido com envio de mensagem para que o desenvolvedor ajuste
algum aspecto no código. Tem também pontos específicos de uso do Nessus\footnote{{\it Scanner} de Vulnerabilidades}
para checagem de vulnerabilidade, então, se a aplicação consegue passar por
todas essas etapas do pipeline, a gente não tem motivos para se preocupar em
fazer o deployment dessa versão.}'' (P7, Analista de Suporte, Brasil)
\end{mq}

Estes {\it pipelines} ocasionam dois outros conceitos da categoria \cat{agilidade}:
{\bf privisionamento contínuo de infraestrutura} e {\bf \emph{deployment} contínuo}.
Este último é um dos conceitos mais recorrentes identificados na análise das
entrevistas. Sem {\it DevOps}, o {\it deployment} é visto como um grande evento
com alto risco de {\it down-time} e falhas. Com {\it DevOps}, a sensação de
risco no {\it deployment} diminui e essa atividade torna-se mais natural e
frequente. Alguns praticantes afirmam realizar dezenas de {\it deployments}
diariamente.

\begin{mq}
``\emph{Quando a gente adotou DevOps, a nossa curva de ganho foi muito grande.
Então a gente foi de um deploy a cada 15 dias para 40 deploys por dia. Então,
é uma curva muito gigante, a entrega de valor para a empresa foi gigante.}''
(P12, {\it Cloud Engineer}, Estados Unidos)
\end{mq}

\subsection{Resiliência}

Também relacionada à uma saída esperada do processo de adoção de {\it DevOps}, a
categoria \cat{resiliência} refere-se à capacidade que as aplicações
desenvolvidas possuem de se adaptar rapidamente a situações adversas. O primeiro
conceito relacionado é {\bf auto \emph{scaling}}, que indica a presença de
mecanismos que possibilitam a alocação automática de mais ou menos recursos para
aplicações que aumentam ou diminuem sua demanda de acesso em momentos
específicos. Outro conceito relacionado à categoria de \cat{resiliência} é
\textbf{automação da recuperação}, que é a capacidade de as aplicações e
infraestrutura subjacente se recuperarem em caso de falhas. Foram relatados dois
casos típicos de automação de recuperação: (1) em casos de alguma instabilidade
no ambiente de execução de uma aplicação (um contêiner, por exemplo) ocorre
uma reinicialização automática desse ambiente; e (2) em casos de implantação
de nova versão, se ela não funcionar adequadamente, a anterior é automaticamente
restaurada. Essa restauração automática de uma versão anterior diminui a chance
de inatividade devido a erros em versões específicas, que é o conceito de {\bf
zero \emph{down-time}}, que é o último da categoria de \cat{resiliência} e indica
que uma aplicação que esteja executanto corretamente não sofrerá
indisponibilidade por conta de falhas que possam ser evitadas.

\begin{mq}
``\emph{Quando era necessário dar um deploy em alguma das aplicações que a
gente tinha sempre acontecia um downtime de alguns minutos e, obviamente, se
tinha o downtime e o deploy não dava certo, o downtime era ainda maior. Mas com
a adoção de DevOps a gente conseguiu justamente diminuir para, na verdade a
gente diminuiu para quase nada, acho que era em torno de um minuto ou menos e
posteriormente a gente conseguiu eliminar qualquer downtime, utilizando o
Kubernetes\footnote{https://kubernetes.io/}}''
(P1, {\it DevOps Developer}, Irlanda)
\end{mq}

\section{Categorias que são tanto \emph{Facilitadores} como \emph{Saídas} da Adoção de \emph{DevOps}}\label{secao_facilitadores_e_saidas}

Finalmente, aqui são detalhadas as categorias que aparecem tanto como
\emph{facilitadores} quanto como \emph{saídas} no processo de adoção de
{\it DevOps}: \cat{medição contínua} e \cat{garantia da qualidade}.


\subsection{Medição Contínua}

Como a responsabilidade pela execução das atividades de medição e monitoramento
é tida como atribuição típica do time de operações, à medida que ela passa a
ser executada continuamente e de maneira transparente, manifesta-se como um
\emph{facilitador DevOps} pois fomenta a \cc. Ademais, a coleta
contínua de métricas reforça a confiança entre os times pois há um incremento
na proatividade, o que também é uma característica importante da \cc.

\begin{mq}
``\emph{Antes, a gente tinha só aquelas olhadas esporádicas no zabbix\footnote{\url{https://www.zabbix.com/}}
para verificar se estava tudo OK. No máximo, alguém parava para verificar o
consumo de memória e CPU. Para manter a qualidade dos serviços, expandimos essa
questão da coleta de métricas para que ela se tornasse parte do produto de
software. Depois, a gente começou a coletar métricas continuamente e com
responsabilidades compartilhadas. Por exemplo, se acontecer um overflow no
número de conexões do banco de dados, todos recebem um alerta e são responsáveis
por buscar soluções para esse problema. Este (número de conexões de banco) é um
exemplo interessante de métrica que todos começaram a ficar mais atentos, não
só o time de operações.}''
(P3, {\it DevOps Developer}, Irlanda)
\end{mq}

Já considerando as \emph{saídas DevOps}, a coleta contínua de métricas das
aplicações e da infraestrutura é tida pelos entrevistados como uma consequência
requerida da adoção de {\it DevOps}. Isto ocorre porque a agilidade resultante
do processo aumenta o risco de algo dar errado. Os times devem ser capazes de
reagir rapidamente em caso de problemas, e a medição contínua possibilita essa
proatividade e resiliência.

\begin{mq}
``\emph{Hoje é viável que a gente faça o deploy o tempo todo e, naturalmente,
houve a necessidade de maior controle do que estava acontecendo. Então, nós
usamos grafana\footnote{\url{https://grafana.com/}} e prometheus\footnote{\url{https://prometheus.io/}}
para acompanhar tudo o que está acontecendo na infraestrutura e nas aplicações.
Nós temos um painel completo em tempo real, extraímos relatórios e, quando algo
dá errado, somos os primeiros a saber.}''
(P10, Administrador de Redes, Brasil)
\end{mq}

A \cat{medição contínua} envolve (1) {\bf monitoramento de logs de aplicações},
um conceito que corresponde ao uso do {\it log} produzido pelas aplicações e pela
infraestrutura como fonte de dados. O conceito de (2) {\bf monitoramento
contínuo de infraestrutura} indica que o monitoramento não é realizado por uma
pessoa ou time específicos em um momento específico. A responsabilidade de
monitorar a infraestrutura é compartilhada e é executada continuamente.
Já o {\bf monitoramento contínuo de aplicações} refere-se à instrumentação para
fornecer métricas que são usadas para avaliar o comportamento das aplicações em
execução e, muitas vezes, direcionar decisões de evolução ou de negócios. Todas
essas medições/monitoramentos podem ocorrer de forma automatizada, o conceito de
{\bf automação do monitoramento} já foi descrito na subseção \ref{ssec:automation}.


\subsection{Garantia da Qualidade}

Da mesma forma que a \cat{medição contínua}, a \cat{garantia de qualidade} é
uma categoria que pode funcionar tanto como \emph{facilitador} quanto como \emph{saída}
do processo de adoção de {\it DevOps}. Como \emph{facilitador}, porque um aumento na
qualidade é descrito como responsável por gerar mais confiança entre os times,
o que, no final, gera um ciclo virtuoso de colaboração. Como \emph{saída}, o
princípio é de que não é viável criar um cenário de entrega contínua de {\it software}
sem um controle rigoroso da qualidade dos produtos e seus respectivos processos
de produção.

Os entrevistados apontaram para a necessidade de um controle sofisticado a
respeito de quais partes de código devem fazer parte das entregas que são
realizadas continuamente. O \emph{Git Flow}\footnote{https://nvie.com/posts/a-successful-git-branching-model/}
foi recorrentemente citado como modelo utilizado para atender à essas necessidades
de (1) {\bf ramificação de código}, o primeiro conceito de \cat{garantia de
qualidade}. Em uma seção anterior, foi explorada a face de automação relacionada
a microsserviços e testes. Esses elementos também têm uma face de garantia de
qualidade. Uma característica do estilo arquitetural de microsserviços é a
necessidade de os serviços serem de pequeno porte com foco em fazer apenas uma coisa.
Esses pequenos serviços são mais fáceis de dimensionar e estruturar, o que
manifesta um conceito de garantia de qualidade: (2) {\bf serviços coesos}. Em
relação aos testes, a outra face mencionada é o (3) \textbf{teste contínuo}.
Para garantir a qualidade dos produtos de {\it software}, foi identificado que
os testes (bem como outras verificações de qualidade) devem ocorrer
continuamente. Executar os testes de maneira contínua é considerado uma tarefa
desafiadora sem o uso de automação, e isso reforça a necessidade de testes
automatizados.

Os outros dois conceitos citados como parte da \cat{garantia de qualidade} na
adoção de {\it DevOps} são o uso de (4) \textbf{análise estática de código
fonte} para calcular e avaliar continuamente métricas de qualidade no
código-fonte e \textbf{paridade entre os ambientes} para reforçar a
transparência e a colaboração durante o desenvolvimento de {\it software}.

\begin{mq}
``\emph{Com essa questão do deploy automatizado a gente tinha que se preocupar
bastante com testar direito a aplicação, foi inevitável a gente automatizar os
testes. [...] aí também a gente começou a usar o Git Flow para acabar com as
inconsistências que aconteciam, era commit de muita gente no mesmo repositório,
sem pull request, sem nada e às vezes isso comprometia o processo de entrega,
tinha que ficar fazendo uma branch separada só para release.}''
(P6, Developer, Portugal)
\end{mq}
