\section{A Categoria Principal: Cultura de Colaboração}\label{secao_core_category}

A categoria principal da adoção de {\it DevOps} é \cc. A \cc visa essencialmente
remover os silos entre os times de desenvolvimento e operações, o que até certo
ponto se confunde com os próprios objetivos de {\it DevOps}. Inicialmente, uma
\cc envolve que as tarefas tipicamente de operaçõe---como {\it deployment},
provisionamento e gerenciamento de infraestrutura e monitoramento---devem ser
consideradas atividades regulares, parte do dia a dia de desenvolvimento de
{\it software}. As organizações relatam que não aguardam mais momentos específicos
para performar essas atividades, elas são executadas continuamente. Isso leva
ao primeiro conceito relacionado à categoria principal: {\bf operações no dia
a dia de desenvolvimento}.

\begin{mq}
``\emph{Uma etapa muito importante foi trazer o {\it deployment} para dentro do
dia a dia de desenvolvimento, sem ter que ficare esperando um dia específico
da semana ou do mês. Nós queríamos fazer {\it deploy} o tempo inteiro, mesmo
que em um primeiro momento não fosse em produção, um ambiente de {\it staging}
era suficiente. [...] É claro que para que a gente pudesse fazer o {\it deployment}
continuamente, a gente tinha que prover toda a infraestrutura necessária no
mesmo ritmo.}" (P14, Gerente de \acrshort{TI}, Brasil)
\end{mq}

Sem {\it DevOps}, um cenário comum é a existência de um processo acelerado de
desenvolvimento de {\it software} sem preocupações relacionadas a operações.
No final, quando o time de desenvolvimento tem um mínimo produto viável, ele
o envia para o time de operações para publicação. Conhecendo poucas coisas sobre
a natureza do {\it software} e como ele foi produzido, o time de operações tem
que criar e configurar o ambiente necessário e publicar o {\it software}. Neste
cenário, a entrega de {\it software} tipicamente atrasa e os conflitos entre
os times se manifestam. Quando uma \cc é fomentada, os times colaboram para
executar as tarefas desde o primeiro dia do desenvolvimento do {\it software}.
Com o constante exercício das práticas de provisionamento, gerenciamento,
configuração e {\it deployment}, a entrega de {\it software} se torna mais
natural, reduzindo os atrasos e, consequentemente, os conflitos entre os times.

\begin{mq}
``\emph{Nós trabalhamos usando uma abordagem ágil, com {\it sprints} de quinze
dias, onde a gente foca em produzir {\it software} e gera novas versões em
altíssima frequência. Mas, na hora da entrega do {\it software} é que as
complicações começavam a aparecer. [...] As entregas frequentemente atrasavam,
e a gente tinha que entregar com atrasos de semanas, o que não era bom nem para
nós e nem para os clientes.}'' (P6, {\it Developer}, Portugal)
\end{mq}

Um dos resultados da construção de uma \cc é que o time de desenvolvimento não
precisa mais parar o seu trabalho aguardando pela disponibilização de um
servidor de aplicação, ou pela execução de um {\it script} na base de dados, ou
ainda pela publicação de uma nova versão do {\it software} em um ambiente de
testes. Todos os envolvidos precisam conhecer a maneira como todas essas coisas
são feitas e, com a colaboração do time de operações, isso pode ser executado
de maneira regular. Se uma tarefa pode ser executada pelo time de desenvolvimento
e há confiança entre os times, essa tarefa será incorporada ao processo de
desenvolvimento de uma maneira natural, manifestando o segundo conceito
relacionado à categoria \cc: {\bf empoderamento do desenvolvimento de {\it software}}.

\begin{mq}
``\emph{
Não era viável a gente ter tantos desenvolvedores produzindo artefatos e
parando o trabalho deles para aguardar outro time completamente separado
publicar. Ou também precisar de um ambiente de teste e ter que esperar o time
de operações prover isto quando possível. Essas atividades tinham que estar
disponíveis para servir rapidamente ao time de desenvolvimento. Com DevOps
nós conseguimos suprir essa necessidade de liberdade e mais poder para executar
algumas tarefas que são intrisicamente relacionadas ao trabalho deles.}''
(P5, {\it Systems Engineer}, Espanha)
\end{mq}

Uma \cc requer {\bf pensamento de produto}, em substituição a pensamento de
desenvolvimento ou de operações. O time de desenvolvimento precisa compreender
que o {\it software} é um produto que não se encerra após o ``{\it push}'' do código
para o repositório de código-fonte, e o time de operações precisa entender que
o processo também não se inicia quand oum artefato é recebido para publicação.
{\bf Pensamento de produto} é o terceiro conceito relacionado à categoria
principal.

\begin{mq}
``\emph{We wanted to
contract people who could have a product vision. People who could see the
problem and think in the better solution to it. But not only think in a
software solution, also think about the moment when that application will be
published. We also bring together developers to reinforce that everyone
%should think that way. Everyone
has to think in the product and not only in
their code or in their infrastructure}" (P12, Cloud Engineer, United States)
\end{mq}

There should be a \textbf{straightforward communication} between teams. Ticketing
systems are cited as a typical and inappropriate means of communication
between development and operations teams. Face-to-face communication is the best
option, but considering that it is not always feasible, the continuous use of
tools like \emph{Slack} and \emph{Hip Chat} was cited as appropriate options.

\begin{mq}
``\emph{We also use this tool (Hip Chat) as a way to facilitate communication between
development and operations teams. The pace of work there is very accelerated, and thus
it is not feasible to have a bureaucratic communication. (...) This gave us a lot of
freedom to the development activities, in case of any doubt, the operations staff
is within reach of a message.}" (P5, Systems Engineer, Spain)
\end{mq}

There is a \emph{shared} responsibility to identify and fix the issues
of a software when transitioning to production. The strategy of avoiding liability should be kept away.
The development team must not say that a given issue is a problem in the infrastructure, then
it is responsibility of the operations team. Or the opposite, the operations team
must not say that a given failure was motivated by a problem in the application, then it is
responsibility of development team. A \textbf{blameless} context must exist.
The teams need to focus on solving problems, not on finding blame and
running away from responsibility. The context of \textbf{shared
responsibilities} involves not only solving problems, but also any other
responsibility inherent in the software product must be shared.
\textbf{Blameless} and \textbf{shared responsibilities} are the remaining
concepts of the core category.

\begin{mq}
``\emph{We realized that some people were afraid of making mistakes. Our
culture was not strong enough to make everyone feel comfortable to innovate and
experiment without fear of making mistakes. We made a great effort to spread
this idea that there are no blame for any problem that may occur. We take every possible
measure to avoid failures, but they
will happen, and only without blame we will be able to solve a problem quickly.}" (P8,
DevOps Engineer, Brazil)
\end{mq}

At first glance, considering the creation and strengthening of the \cc as the most important step towards DevOps adoption seems somewhat obvious, but
the respondents cited some mistakes that they consider recurrent in not
prioritizing this aspect in a DevOps adoption:

\begin{mq}``\emph{In a DevOps adoption, there is a very strong cultural issue that the teams
sometimes are not adapted. Related to this, one thing that bothers me a lot and
that I see happen a lot is people hitching DevOps exclusively by tooling or
automation.}" (P9, IT Manager, Brazil)
\end{mq}

Besides the core category (\cc), we identified
three other sets of categories: the enablers
of DevOps adoption, the consequences of adopting
DevOps, and the categories that are both enablers and consequences.

\section{Facilitadores DevOps}\label{secao_facilitadores}

Here we detail the categories that support the adoption of
DevOps practices, including \cat{automation}, \cat{sharing and transparency}.

\subsection{Automation} \label{ssec:automation}

This category presents the higher number of related concepts. This
occurs because manual proceedings are considered strong candidates to
propitiate the formation of a silo, hindering the construction
of a \cc. If a task is manual, a single person or
team will be responsible to execute it. Although \cat{transparency} and \cat{sharing} can
be used to ensure collaboration even in manual tasks, with automation the
points where silos may arise are minimized.

\begin{mq}
``\emph{When a developer needed to build a new application, the previous workflow demanded her
to create a ticket to the operations teams, which should then manually evaluate and solve
the requested issue. This task could take a lot of time and there was no
visibility between teams about what was going on (\ldots). Today, those silos do not exist
anymore within the company, in particular because it is not necessary to execute all these tasks manually,
everything has been automated.}" (P12, Cloud Engineer, United States)
\end{mq}

In addition to contribute to \cat{transparency}, \cat{automation} is also considered
important to ensure \emph{reproducibility} of tasks, reducing rework and risk of
human failure. Consequently, \cat{automation} increases the confidence
between teams, which is an important aspect of the \cc.

\begin{mq}
``\emph{Before we adopted DevOps, there was a lot of manual work. For example, if you
needed to create a database schema, it was a manual process; if you needed to create a
database server, it was a manual process; if you needed to create additional EC2 \footnote{Amazon Elastic
Compute Cloud} instances, such a process was also manual.
This manual work was time consuming and often caused errors and
rework.}" (P1, DevOps Developer, Ireland)
\end{mq}

%\begin{mq}
%``\emph{Our main motivation to adopt DevOps was basically reduce rework. Almost every
%week, we had to basically make new servers and start them manually, which was
%very time-consuming.}" (P4, Computer Technician, Brazil)
%\end{mq}

The eight concepts of the \cat{automation} category will be detailed next.
In all interviews we extracted explanations about \textbf{deployment
automation} (1), as part of DevOps adoption. Software delivery is the clearest
manifestation of value delivery in software development. In case of problems
in deployment, the expectation of delivering value to business can quickly
generate conflicts and manifest the existence of silos.
In this way, \cat{automation} typically increases agility and reliability. Some other
concepts of automation go exactly around deployment automation.

It is important to note that frequent and successfully
deployments are not sufficient to generate value to business. Surely, the quality of
the software is more relevant. Therefore, quality checks need to be automated as well, so they can be part of the
deployment pipeline, as is the case of \textbf{test automation} (2). In addition, to
automate application deployment, the environment where the
application will run needs to be available. So, \textbf{infrastructure
provisioning automation} (3) must be also considered in the process. Beyond being available,
the environment needs to be properly configured, including the amount of memory and CPU,
availability of the correct libraries versions, and database structure. If the configuration of some of these concerns
has not been automated, the deployment activity can go wrong. Therefore,
the automation of \textbf{infrastructure management} (4) is another
concept of the \cat{automation} category.

Modern software is built around services. Microservices  was commonly cited
as one aspect of DevOps adoption. To Fowler and Lewis
\cite{martinfowler2014microservices}, in the
microservice architectural style, services need to be independently deployable
by fully automated deployment machinery. We call this part of microservices
characteristics of \textbf{autonomous services} (5). \textbf{Containerization}
(6) is also mentioned as a way to automate the provisioning of containers---the
environment where these autonomous services will execute.
\textbf{Monitoring automation} (7) and \textbf{recovery automation} (8) are the
remaining concepts. The first refers to the ability to monitor the
applications and infrastructure without human intervention. One classic example
is the widespread use of tools for sending messages reporting
alarms---through SMS, Slack/Hip Chat, or even
cellphone calls-- in case of incidents. And the second is related to the ability
to either replace a component that is not working or
roll back a failed deployment without human intervention.

\subsection{Transparency and Sharing} Represents the grouping of concepts
emerged from recurrent interviews that help to disseminate concepts and
activities. Training, tech talks, committees lectures, and round tables
are examples of these events. Creating
channels using communication tools is another recurrent topic
related to \cat{sharing} along the processes of DevOps adoption.
According with the content of what is shared, we
have identified three main concepts:

\begin{itemize}
\item Knowledge sharing: the professionals interviewed mention a wide range of
skills they need to acquire during the adoption of DevOps, citing
structured events of sharing to smooth the learning curve of both technical and
cultural knowledge.


\item Activities sharing: where the focus is on sharing how simple tasks can or
should be performed. Communication tools, committees, and round tables are the common
forum for sharing this type of content.

\item Process sharing: here, the focus is on sharing whole working processes. The
content is more comprehensive than in sharing activities. Tech talks and
lectures are the common forum for sharing processes.

\end{itemize}

Sharing concepts contribute with the \cc. For example,
all team members gain best insight about the entire software production
process, with a solid understanding of shared responsibilities. A shared vocabulary also
emerged from \cat{sharing} and this facilitates communication.

The use of \textbf{infrastructure as code} was
recurrently cited as a means for guaranteeing that everyone knows how the execution environment of
an application is provided and managed. Bellow, we present an interview
transcript which sums up this concept.

\begin{mq}
``\emph{So, here we adopted this type of strategy that is the infrastructure as code,
consequently we have the versioning of our entire infrastructure in a common
language, in such a way that any person, a developer, an architect, the
operations guy of even the manager, he looks and is able to describe that the
configuration of application x is y. So, it aggregates too much value for us
exactly with more transparency.}" (P12, Cloud Engineer, United States)
\end{mq}

Regarding cat{transparency and sharing}, we also found the concept of \textbf{sharing on a regular basis}, which suggests
that sharing should be embedded in the process of software
development, in order to contribute effectively to transparency.
As we will detail in the \emph{continuous integration} concept of
the \cat{agility} category, a common way to integrate all tasks is a pipeline. Here, there is the
concept of \textbf{shared pipelines}, which indicates that the code of pipelines
must be accessible to everyone, in order to foment transparency.

\begin{mq}
``\emph{The code of how the infrastructure is
made is open to developers and the sysadmins need to know some aspects of how
the application code is built. The code of our pipelines is accessible to
everyone in the company to know how activities are automated}" (P13, Technology
Manager, Brazil)
\end{mq}


\section{Categorias Relacionadas a \emph{Saídas} da Adoção de {\it DevOps}}
\label{secao_saidas}

%In this section we detail the categories that correspond to
%the expected consequences with the adoption of
%DevOps practices, including \cat{agility} and \cat{resilience};
%as discussed as follows.

\subsection{Agility}

Agility is frequently discussed as a major outcome of DevOps adoption. With more
collaboration between teams, \textbf{continuous integration} with execution of
multidisciplinary pipelines is possible and it is an agile related concept
frequently explored. These pipelines might contain
infrastructure provisioning, automated regression testing, code analysis,
automated deployment and any other task considered important to
execute continuously.

These pipelilnes encourage two other agile concepts: \textbf{continuous
infrastructure provisioning} and \textbf{continuous deployment}. The latter is
one of the most recurrent concepts identified in the interview analysis. Before
DevOps, deployment had been seen as a major event with high risk of downtime and
failure involved. After DevOps, the sensation of risk in deployment decreases
and this activity became more natural and frequent. Some practitioners claim
to perform dozens of deployments daily.

\subsection*{Resilience}

Also related to an expected outcome of adopting DevOps, \cat{resilience} in this
context refers to the ability of applications to adapt quickly to adverse situations.
The first related concept is \textbf{auto scaling}---i.e.,
allocating more or less resources to applications that increase or
decrease on demand. Another concept related to
the \cat{resilience} category is \textbf{recovery automation}, that is
the capability of the applications and infrastructure to recovery itself in case of
failures. There are two typical cases of recovery automation: (1) in cases
of some instability in the execution environment of an application (a
container, for example) occurs an auto restart of that environment; and (2) in
cases of new version deployment, if the new version does not work properly, the
previous one must be restored. This auto restore of a previous version
decrease the chance of downtimes due to errors in specific versions, which
is the concept of \textbf{zero down-time}, the last one of the \cat{resilience} category.

\section{Categorias que são tanto \emph{Facilitadores} como \emph{Saídas} da Adoção de {\it DevOps}}
\label{secao_facilitadores_e_saidas}

Finally, here we detail the categories that are both enablers
and outcomes, including \cat{continuous measurement}
and \cat{quality assurance}; as discussed as follows.

\subsection{Continuous Measurement}

As an enabler, performing regularly the
activities of measurement and sharing
contributes to avoid existing silos and reinforce the \cc, because it is
considered a typical responsibility of the operations team.

\begin{mq}
``\emph{Before, we had only sporadic looks to
zabbix\footnote{\url{https://www.zabbix.com/}} to check if everything was OK.
At most someone would stop to look memory and CPU consumption. To maintain a
the quality of services, we expanded this view of metrics collection so that it
became part of the software product. We then started to collect metrics continuously
and with shared responsibilities. For example, if an overflow occurred in the
number of database connections, everyone received an alert and had
the responsibility to find solutions to that problem. %This is an interesting example of
%metrics that everyone started to be more attentive, not only the operations
%team.
}" (P3, DevOps Developer, Ireland)
\end{mq}

As an outcome, the continuously collection of metrics from applications and
infrastructure is a required consequence of DevOps adoption. It occurs because
the resultant agility increases the risk of something going wrong. The team
should be able to react quickly in case of problems, and the continuous
measurement allows it to be proactive and resilient.

\begin{mq}
``\emph{With DevOps we can do deployment all the time and, consequently, there was
the need of greater control of what was happening. So, we used
grafana\footnote{\url{https://grafana.com/}} and
prometheus\footnote{\url{https://prometheus.io/}} to follow everything that is
happening in the infrastructure and in the applications. We have a complete
dashboard in real time, we extract reports and, when something goes wrong, we
are the first to know.}" (P10, Network Administrator, Brazil)
\end{mq}

Continuous monitoring involves \textbf{application log monitoring} (1), a
concept that corresponds to the use of the log produced by
applications and infrastructure as data source. The concept of
\textbf{continuous infrastructure monitoring} (2) indicates that the monitoring
is not performed by a specific person or team in a specific moment. The
responsibility to monitor the infrastructure is shared and it is executed in
daily. \textbf{Continuous application measurement} (3), in turn, refers to
the instrumentation to provide metrics that are used to evaluate aspects and
often direct evolution or business decisions. All these monitoring/measurement
can occur in an automated way, the \textbf{monitoring automation} already been
detailed in subsection \ref{ssec:automation}.

\subsection{Quality Assurance}

In the same way as continuous measurement, quality assurance is a category that
can work both as enabler and as outcome. As enabler because an increasing quality
leads to more confidence between the teams, which in the end generates a virtuous
cycle of collaboration. As outcome, the principle is that it is not
feasible to create a scenario of continuous delivery of software without control
about the quality of the products and its production processes.

Respondents pointed to the need for sophisticated control of which code should
be part of deliverables that are continuously delivered. Git Flow was
recurrently cited as suitable \textbf{code branching} (1) model, the first
concept of quality assurance.
In a previous section, we explored the automation face of
microservices and testing. These elements have also a quality assurance face.
Another characteristic of microservices is the need for small services focusing
in doing only one thing. These small services are easier to scale and
structure, which manifest a quality assurance concept: \textbf{cohesive
services} (2). Regarding testing, another face is \textbf{continuous
testing} (3). To ensure quality in software products, we found that
tests (as well as other quality checks) should occur continuously. Continuous testing
is considered challenging without automation, and this reinforces the need for automated
tests.

Another two concepts cited as part of quality assurance in DevOps adoption are
the use of \textbf{source code static analysis} (4) to compute quality metrics in
source code and the \textbf{parity between environments} to
reinforce transparency and collaboration during software development.
