O advento e a consolidação dos métodos ágeis de desenvolvimento de
\textit{software} possibilitaram uma maior eficiência na produção de
\textit{software} com entregas mais rápidas em um período de tempo menor
\cite{agile_sw_dev}. Todavia, essa mudança gerou um aumento na demanda da equipe de
operações, tipicamente responsável pelas atividades relacionadas à
publicação dos artefatos de \textit{software}, e que não participava
do processo de desenvolvimento ágil \cite{humble2010continuous}.

Visando reduzir as barreiras entre os times, trazendo as atividades típicas
do time operações para um contexto de agilidade já presente durante o
desenvolvimento, o conceito de \textit{DevOps} emergiu na indústria de
desenvolvimento de \textit{software}. O termo \textit{DevOps} é uma
junção das palavras ``\textit{development}'' e ``\textit{operations}'' que
foi usado pela primeira vez em 2009 \cite{devops_for_developers}.

Antes mesmo de o termo \textit{DevOps} existir, artigos e apresentações
já tratavam da aplicação dos princípios ágeis também às atividades de operações,
da rediscussão sobre os papéis dos times de desenvolvimento e operações, e dos
benefícios de uma mudança de paradigma com maior aproximação e colaboração
entre estes times~\cite{devops_for_developers,agile_infra_operations,flickr}.

Os benefícios relacionados a \textit{DevOps} incluem aumento na performance
e na produtividade organizacional de \acrshort{TI}, redução de custos no ciclo de vida
de \textit{software}, melhoria na eficiência e eficácia operacional e maior
alinhamento com o negócio entre os times de desenvolvimento e operações
\cite{characterizing_devops}.

O \textit{State of DevOps Report} aponta que, após mais de 27000
respostas ao \textit{DevOps survey} ao longo dos últimos 6 anos, foi possível
concluir que a utilização das práticas \textit{DevOps} possibilita uma maior
performance de \acrshort{TI} e que \textit{DevOps} contribui para melhorias nos
ciclos de entrega de \textit{software}, na qualidade e segurança de \textit{software}
e na capacidade de obter \textit{feedback} rápido sobre o desenvolvimento de produtos.
Ademais, \textit{DevOps} contribui para o atingimento das missões de qualquer
tipo de organização, independente da indústria ou setor \cite{state_of_devops}.

Os resultados do \textit{agile survey}, realizado entre entre agosto e dezembro
de 2017 e publicados no \textit{State of Agile Report}, apontam que 71\% dos 1492
respondentes ou estão implementando \textit{DevOps} ou planejando adotar nos
próximos doze meses \cite{state_of_agile}.

Alinhado a esse contexto, o \acrfull{TCU} está buscando ampliar o uso da abordagem
\textit{DevOps} no desenvolvimento de suas aplicações corporativas. O
entendimento é de que os rígidos procedimentos criados como paliativo para o
problema da baixa colaboração entre os times de desenvolvimento e operações
ocasionam atrasos na entrega de \textit{software} do órgão que são considerados
inadequados.

Os times de desenvolvimento e operações do \acrshort{TCU} historicamente
trabalham em silos. Metodologias ágeis são utilizadas durante o processo de
desenvolvimento e a equipe de operações (\acrfull{SINAP}) participa da produção
de \textit{software} apenas em momentos específicos, notadamente quando uma nova
infraestrutura precisa ser provida para publicação de algum artefato de
\textit{software}, e também na publicação propriamente dita.

Essa estrutura em silos ocasionou alguns conflitos ao longo do tempo. Como
exemplos de problemas ocorridos que levaram à manifestação
destes conflitos, destacam-se: (1) o não funcionamento em ambiente de produção
de um \textit{software} que funciona adequadamente
em outros ambientes, como ambiente de desenvolvimento, por exemplo; e (2) o
atraso em entregas de \textit{software} consideradas importantes ocasionado
pelo não provisionamento tempestivo da infraestrutura necessária.

Como resposta aos conflitos ocorridos ao longo do tempo, o \acrshort{TCU}
investiu na padronização de procedimentos. O provimento de infraestrutura
para publicação de novas aplicações passou a funcionar em conformidade com um
\acrfull{SLA} que prevê um prazo de sete dias para conclusão por parte do time
de operações. O provimento de infraestrutura deve ser pedido por qualquer
dos times de desenvolvimento por meio do registro de uma solicitação no sistema
de \textit{servicedesk} interno, e o prazo inicia após este registro.
Na solicitação devem constar os detalhes da configuração (quantidade de memória
e CPU, por exemplo) necessária, bem como estimativa de número de usuários que
irão acessar a aplicação.

No tocante à publicação do \textit{software}, foi criada uma padronização de
procedimentos denominada \emph{publicação programada}, que é vista como a maior
manifestação atual da falta de colaboração entre os times de desenvolvimento
de operações do \acrshort{TCU}. A publicação programada
é um processo que prevê um complexo encadeamento de ações com horários bem
definidos visando reduzir o risco de problemas na publicação dos artefatos. Os
detalhes desse processo são apresentados a seguir.

A versão de todas as aplicações desenvolvidas é congelada semanalmente
às 03h00m de terça-feira. Para que esse congelamento fosse viabilizado, foi
criado um segundo servidor \textit{Artifactory}\footnote{Gerenciador de
repositórios de artefatos de \textit{software} \url{https://jfrog.com/artifactory/}}
e um segundo servidor \textit{Jenkins}\footnote{Servidor de integração contínua
\url{https://jenkins.io/}}
cujos \textit{jobs} apenas podem ser executados no
contexto da publicação programada. Uma primeira construção é executada com
base no código contante na \textit{branch} master das aplicações no horário
estabelecido. Os artefatos produzidos são então publicados no servidor
\textit{artifactory} onde só podem ser alterados pelo \acrshort{SINAP}.

As equipes de desenvolvimento tem então um prazo até
as 13h00m para verificar se os artefatos foram gerados adequadamente. Após
encerrado este prazo, o \acrshort{SINAP} efetua a publicação
em um ambiente denominado pré-produção. Caso algum artefato não tenha sido
gerado adequadamente, os times devem efetuar as correções necessárias e
solicitar a reexecução dos relativos \textit{jobs} no servidor \textit{Jenkins}
que opera apenas no contexto da publicação programada. Cumpre ressaltar que
antes de serem reexecutados, os \textit{jobs} precisam ser ajustados para
congelar uma nova \textit{branch} do repositório de código fonte, não mais a
\textit{branch} master que pode já possuir código que não faz parte do escopo
da publicação da semana em tela.

Pré-produção é um ambiente mantido pelo time de operações que possui o
propósito de ser \emph{o mais similar possível} ao ambiente de produção,
buscando-se evitar o não funcionamento de alguma aplicação apenas em ambiente
de produção.

Após a publicação em pré-produção, o \acrshort{SINAP} envia um e-mail de
``abertura de temporada de testes em pré-produção'', a partir do qual os times
de desenvolvimento executam testes manuais em suas respectivas aplicações.

É comum que versões específicas de artefatos necessitem da execução de
\textit{scripts} em banco de dados para funcionar adequadamente, esta execução
de \textit{scripts} também é padronizada. A sua versão definitiva
precisa ser concluída e incluída junto com os artefatos para publicação até o
limite de 12h00 de terça-feira. Ressalte-se que para que um \textit{script}
esteja em versão definitiva, é necessário ter passado pela equipe de operações
para validação, para a qual não há prazo definido. Um \textit{script}
proposto por algum dos times de desenvolvimento durante o dia de segunda-feira
tipicamente não é validado pelo time de operações até o limite de prazo e isso
ocasiona erros na publicação dos artefatos relacionados.

Após a publicação em ambiente de pré-produção (13h00m de terça-feira), os times
de desenvolvimento podem solicitar a substituição dos artefatos a qualquer
momento até o limite de 18h00m de quinta-feira, quando os artefatos são
congelados definitivamente para publicação em ambiente de produção. Novas
publicações em pré-produção ocorrem às 18h00m de terça-feira e às 14h00m de
quarta e quinta-feira. Essas novas publicações são chamadas de \emph{corretivas}.
A geração de artefatos para publicações corretivas, assim como a reexecução
de \textit{jobs} para geração de artefatos que falharam, é um procedimento
executado manualmente e requer diversos cuidados relacionados à criação de novas
\textit{branchs} e ajuste dos \textit{jobs} para evitar a inclusão de código
inserido por outro desenvolvedor com o propósito de publicação apenas na semana
seguinte.

A publicação em ambiente de produção ocorre \textbf{após} o término do
expediente de quinta-feira, que ocorre às 20h00m em dias quando não há sessão
plenária ou logo após o seu término (não há horário máximo estipulado).
Os procedimentos de publicação são executados manualmente pelo time de
operações e ocorre um \textit{downtime} que pode variar de 10 minutos a algumas
horas.

O \acrshort{SLA} entre as equipes prevê que em caso de erros na publicação em
produção, o \acrshort{SINAP} irá reverter a versão anterior e os novos
artefatos só serão publicados na semana seguinte. Está previsto ainda que
não haverá publicação programada em semanas que contenham feriados ou pontos
facultativos na quinta ou sexta-feira.

A publicação programada ajuda a lidar com alguns dos problemas existentes
previamente. Todavia, alguns outros problemas de ordem prática passam a existir:

\begin{itemize}
\item Uma eventual evolução pontual em alguma aplicação, cujo código tenha sido
disponibilizado pelo desenvolvedor na manhã de uma terça-feira, só será
disponibilizada para o usuário final em ambiente de produção na sexta-feira da
semana seguinte, uma demora de dez dias que nem sempre é aceitável;

\item Por mais que o ambiente de pré-produção seja semelhante ao de produção,
ainda ocorrem erros de versões que funcionam naquele mas não neste;

\item A existência de servidores \textit{artifactory} e \textit{jenkins}
duplicados aumenta o retrabalho e o risco de erros durante o processo de
geração de artefatos para publicação;

\item A publicação manual e com \textit{downtime} requer a alocação de
profissionais para trabalho fora do horário regulamentar do \acrshort{TCU},
gerando a necessidade de pagamento de horas extras, cujo número se acentua
em dias quando ocorre sessão plenária;

\item O término do horário de expediente em Brasília diverge de algumas outras
capitais. Em caso de horário de verão, a diferença para Rio Branco, no Acre,
por exemplo, chega a ser de 3 horas. Ou seja, o \textit{downtime} de 10 minutos
a algumas horas impede a continuidade dos trabalhos em algumas secretarias
de controle externo que funcionam nos estados;

\item Combinações de feriados com publicações com erros podem ocasionar longos
períodos sem publicação de \textit{software} em ambiente de produção. Por exemplo, já
ocorreu de em uma semana a publicação ser revertida pela ocorrência de erros,
na semana seguinte ser feriado e na terceira ocorrer novo erro na publicação.
Neste cenário, é possível que uma nova funcionalidade pronta tenha passado um mês
inteiro até ser publicada;

\item Nos casos de não funcionamento de aplicações em ambiente de produção,
a postura de resolução não é colaborativa. A preocupação do time de operações
tende a ser tão somente assegurar que não é um problema na infraestrutura,
enquanto que os times de desenvolvimento buscam defender que não é um problema
na aplicação.

\end{itemize}

Toda essa rigidez procedimental, com tarefas e horários encadeados ilustra o
quão complexas podem se tornar soluções que visam contornar o problema da falta
de colaboração entre os times de desenvolvimento e operações.

O \acrshort{TCU} possui diversos comitês envolvidos no processo de desenvolvimento
de suas aplicações corporativas. Um desses comitês se chama \acrfull{CPA} que é
constituído por representantes de todas as equipes da área de \acrshort{TI} e cuja
função é orientar e padronizar as decisões tecnológicas e arquiteturais no
âmbito do processo de desenvolvimento de \textit{software}.

O \acrfull{PDTI} do \acrshort{TCU} para o biênio 2017/2018 contém como um de seus
indicadores: \emph{o aprimoramento do uso de práticas ágeis em todas as equipes
e \textbf{ampliação do uso da abordagem DevOps}}.

Inicialmente, para possibilitar o cumprimento do indicador constante no
\acrshort{PDTI}, o \acrshort{CPA} deliberou no sentido da experimentação e
realização de provas de conceito em ferramentas \textit{DevOps}. Após um
período de avaliação, o \acrshort{CPA} avaliou que, embora
houvesse competência técnica interna para implantação de ferramentas, existia
uma carência de melhor entendimento a respeito de como se adotar \textit{DevOps}.
Formou-se então novo entendimento de que experiências bem sucedidas na adoção de
\textit{DevOps} no mercado deveriam ser buscadas através de contato direto
e participações em eventos. Essas experiências deveriam ser apresentadas
internamente no sentido de pautar a adoção de \textit{DevOps} no \acrshort{TCU}.

\section{Problema de Pesquisa}%

A ampliação do uso da abordagem \textit{DevOps} faz parte dos indicadores
do \acrshort{PDTI} do \acrshort{TCU}. O comitê responsável por orientar e
padronizar as decisões tecnológicas e arquiteturais (\acrshort{CPA}) formou
entendimento de que para que a adoção de \textit{DevOps} no \acrshort{TCU}
seja bem sucedida, ela deve ser orientada pelo intercâmbio de experiências com
praticantes de mercado e pautada em experiências bem sucedidas da adoção de
\textit{DevOps}.

%%%%%%%%%%%%%%%%%%%%%%%%%%%%%%%%%%%%%%%%%%%%%%%%%%%%%%%%%%%%%%%%%%%%%%%%%%%%%%%%
%%%%%%%%%%%%%%%%%%%%%%%%%%%%%%%%%%%%%%%%%%%%%%%%%%%%%%%%%%%%%%%%%%%%%%%%%%%%%%%%
%%%%%%%%%%%%%%%%%%%%%%%%%%%%%%%%%%%%%%%%%%%%%%%%%%%%%%%%%%%%%%%%%%%%%%%%%%%%%%%%
\section{Justificativa}%

A adoção de \textit{DevOps} no \acrshort{TCU} está alinhada aos direcionadores
estratégicos da área de \acrshort{TI} do órgão e tem possibilitado uma
mudança de paradigma necessária para tratar adequadamente os problemas
decorrentes da baixa colaboração entre os times de desenvolvimento e operações.

A produção de um modelo para orientar a adoção de \textit{DevOps} com base em
experiências de mercado bem sucedidas, com implantação na adoção de
\textit{DevOps} no \acrshort{TCU}, representa um
incremento ao conhecimento existente sobre \textit{DevOps}, com destaque
para o melhor entendimento de \emph{como} se adotar \textit{DevOps} com base na
prática de mercado.

O intercâmbio de experiências e conhecimentos com praticantes do mercado
durante o processo de adoção de \textit{DevOps} no TCU possibilita uma redução
das chances de o \acrshort{TCU} incorrer em falhas comuns, já superadas em
outros contextos.

%%%%%%%%%%%%%%%%%%%%%%%%%%%%%%%%%%%%%%%%%%%%%%%%%%%%%%%%%%%%%%%%%%%%%%%%%%%%%%%%
%%%%%%%%%%%%%%%%%%%%%%%%%%%%%%%%%%%%%%%%%%%%%%%%%%%%%%%%%%%%%%%%%%%%%%%%%%%%%%%%
%%%%%%%%%%%%%%%%%%%%%%%%%%%%%%%%%%%%%%%%%%%%%%%%%%%%%%%%%%%%%%%%%%%%%%%%%%%%%%%%
\section{Objetivos}

\subsection{Objetivo Geral}

Este trabalho tem como objetivo geral propor e implantar um modelo aderente à prática
de mercado para guiar a adoção de \textit{DevOps} no desenvolvimento das
aplicações corporativas do \acrshort{TCU}.

\subsection{Objetivos Específicos}
Com o intuito de atingir o objetivo geral deste trabalho, os seguintes objetivos
específicos foram definidos:

\begin{itemize}
\item Investigar a adoção de \textit{DevOps} por praticantes bem sucedidos no
mercado de desenvolvimento de \textit{software};
\item Propor um modelo de adoção de \textit{DevOps} com base na prática de mercado;
\item Implantar o modelo na adoção de \textit{DevOps} no \acrshort{TCU}; e
\item Realizar um estudo empírico para avaliar o estágio atual da adoção de
\textit{DevOps} no \acrshort{TCU} bem como para validar a relevância do modelo
proposto neste contexto.
\end{itemize}

%\section{Resultados Esperados}
%O resultado esperado para essa pesquisa é a construção e implantação no
%\acrshort{TCU} de um modelo para adoção de \textit{DevOps} com base em
%experiências bem sucedidas coletadas empiricamente da prática na indústria.

\section{Metodologia}

Para investigar a adoção bem sucedida de \textit{DevOps} no mercado foi
utilizada a variação clássica do método \textit{Grounded Theory} \cite{glaser1967discovery}.
O uso de \textit{Grounded Theory} se justifica por quatro motivos:

\begin{enumerate}

\item O método possibilita a construção de um entendimento independente e original,
o que é adequado para coletar evidência empírica diretamente da prática de
mercado sem o viés de pesquisas anteriores, alinhado às necessidades do \acrshort{TCU}.
A evidência coletada só é reintegrada com a literatura existente após a
construção da teoria \cite{reconciling_perspectives,stol2016grounded}.

\item \acrshort{GT} é uma metodologia consolidada em outras áreas de
pesquisa, como sociologia médica \cite{gt_medical_sociology}, nutrição
\cite{gt_nursing}, educação \cite{gt_education} e administração
\cite{gt_management,locke2001grounded}.

\item \acrshort{GT} é considerado um método adequado para caracterizar cenários
sob uma perspectiva pessoal daqueles envolvidos em uma disciplina ou atividade \cite{stol2016grounded},
que é exatamente o cenário que se almeja caracterizar aqui: quais são os caminhos
seguidos por praticantes que adotaram \textit{DevOps} de uma maneira bem sucedida?

\item \acrshort{GT} tem sido cada vez mais aplicada para estudar tópicos de
engenharia de \textit{software} \cite{hoda2017becoming,Waterman:2015:ICSE,stol2016grounded},
com destaque para o recente trabalho de Hoda et al. \cite{hoda2017becoming} que
caracterizou a adoção de métodos ágeis em uma maneira similar ao que se propõe
fazer aqui em relação a \textit{DevOps}.

\end{enumerate}

O modelo para adoção de \textit{DevOps} foi produzido com base na teoria
construída utilizando o método \textit{Grounded Theory}.

Já a avaliação empírica foi realizada utilizando um grupo focal, pois este
método é considerado adequado para obter novos \textit{insights} sobre um
assunto de um pequeno grupo de pessoas nele envolvido, de maneira rápida, ao
invés de se buscar fornecer respostas quantificáveis a perguntas específicas
obtidas a partir de uma amostra grande da população \cite{focus_group_handbook,shull2007guide}.

\section{Estrutura do Trabalho}

O restante deste documento está organizado em cinco capítulos cujos conteúdos
foram estruturados da seguinte forma:

\begin{itemize}
\item \textbf{Capítulo 2.} Introduz o tema \textit{DevOps} por meio da
apresentação de seu histórico, definições e elementos que o caracterizam;
\item \textbf{Capítulo 3.} Apresenta o detalhamento da pesquisa utilizando
\textit{\acrfull{GT}} que possibilitou a caracterização da adoção de
\textit{DevOps} com base na percepção de praticantes do mercado;
\item \textbf{Capítulo 4.} Contém detalhes de um modelo para adoção de
\textit{DevOps} proposto com base na teoria construída no Capítulo 3, os
resultados da avaliação empírica e os respectivos detalhes da adoção de {\it DevOps}
e uso do modelo no \acrshort{TCU};
\item \textbf{Capítulo 5.} Contém as considerações finais do trabalho com
designação das suas limitações bem como de oportunidades para trabalhos futuros.
\end{itemize}


%\section{Normas CIC}

% \href{http://monografias.cic.unb.br/dspace/normasGerais.pdf}{Política de Publicação de Monografias e Dissertações no Repositório Digital do CIC}%
% \href{http://monografias.cic.unb.br/dspace/}{Repositório do Departamento de Ciência da Computação da UnB}

% \href{http://bdm.bce.unb.br/}{Biblioteca Digital de Monografias de Graduação e Especialização}
