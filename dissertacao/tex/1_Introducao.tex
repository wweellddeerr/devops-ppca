O advento e a consolida\c{c}\~ao dos m\'etodos \'ageis de desenvolvimento de
\textit{software} possibilitaram uma maior efici\^encia na produ\c{c}\~ao de
\textit{software} com entregas mais r\'apidas em um per\'iodo de tempo menor
\cite{}. Todavia, essa mudan\c{c}a gerou um aumento na demanda da equipe de
opera\c{c}\~oes, tipicamente respons\'avel pelas atividades relacionadas a
publica\c{c}\~ao dos artefatos de \textit{software} e que n\~ao participava
do processo de desenvolvimento \'agil \cite{humble2010continuous}.

Visando reduzir as barreiras entre os times de desenvolvimento e opera\c{c}\~oes,
trazendo as atividades t\'ipicas de opera\c{c}\~oes para um contexto de agilidade
j\'a presente durante o desenvolvimento, o conceito de \textit{DevOps} emergiu
na ind\'ustria de desenvolvimento de software. O termo \textit{DevOps} \'e uma
jun\c{c}\~ao das palavras ``\textit{development}'' e ``\textit{operations}'' que
foi usado pela primeira vez em 2009 \cite{}. \textit{DevOps} emergiu sem uma
clara defini\c{c}\~ao, todavia, estudos recentes prop\~oem defini\c{c}\~oes
dentre as quais destaco a de Fran\c{c}a et al. \cite{characterizing_devops}
que definem \textit{DevOps} como um neologismo que representa um movimento de
profissionais de tecnologia da informa\c{c}\~ao e comunica\c{c}\~ao buscando uma
diferente atitude no que se refere \`a entrega de \textit{software},
atrav\'es da colabora\c{c}\~ao entre o desenvolvimento de sistemas de
\textit{software} e as fun\c{c}\~oes de opera\c{c}\~oes, com base em um conjunto
de pr\'aticas, tais como cultura, automa\c{c}\~ao, medi\c{c}\~ao e
compartilhamento.

A necessidade de se rediscutir o papel do time de opera\c{c}\~oes no
desenvolvimento de software j\'a havia sido apontada mesmo antes de o termo
\textit{DevOps} existir. Uma apresenta\c{c}\~ao da \textit{Flickr} \cite{flickr}
onde uma rediscuss\~ao sobre os pap\'eis dos times de desenvolvimento e
opera\c{c}\~oes \'e proposta e os benef\'icios de uma mudan\c{c}a de paradigma
com maior aproxima\c{c}\~ao e colabora\c{c}\~ao entre estes times s\~ao
explorados \'e considerada um dos pontos de partida para o surgimento de
\textit{DevOps} \cite{devops_for_developers}.

Os benef\'icios relacionados a \textit{DevOps} incluem aumento na performance
e na produtividate organizacional de TI, redu\c{c}\~ao de custos no ciclo de vida
de \textit{software}, melhoria na efici\^encia e efic\'acia operacional e maior
alinhamento com o negócio entre os times de desenvolvimento e opera\c{c}\~oes
\cite{characterizing_devops}.

Nesse contexto, a adoção de \textit{DevOps} no desenvolvimento de sistemas
corporativos do \acrfull{TCU} foi inserida como um dos objetivos estratégicos da
Secretaria de Soluções de Tecnologia da Informação em seu planejamento estratégico
para o período entre julho de 2017 e junho de 2018, uma vez que existe o
entendimento de que os rígidos procedimentos criados como paleativo para o
problema da baixa colaboração entre os times de desenvolvimento e operações
ocasionam atrasos na entrega de software do órgão que são considerados inadequados.

%%%%%%%%%%%%%%%%%%%%%%%%%%%%%%%%%%%%%%%%%%%%%%%%%%%%%%%%%%%%%%%%%%%%%%%%%%%%%%%%
%%%%%%%%%%%%%%%%%%%%%%%%%%%%%%%%%%%%%%%%%%%%%%%%%%%%%%%%%%%%%%%%%%%%%%%%%%%%%%%%
%%%%%%%%%%%%%%%%%%%%%%%%%%%%%%%%%%%%%%%%%%%%%%%%%%%%%%%%%%%%%%%%%%%%%%%%%%%%%%%%
\section{Problema de Pesquisa}%

%%%%%%%%%%%%%%%%%%%%%%%%%%%%%%%%%%%%%%%%%%%%%%%%%%%%%%%%%%%%%%%%%%%%%%%%%%%%%%%%
%%%%%%%%%%%%%%%%%%%%%%%%%%%%%%%%%%%%%%%%%%%%%%%%%%%%%%%%%%%%%%%%%%%%%%%%%%%%%%%%
%%%%%%%%%%%%%%%%%%%%%%%%%%%%%%%%%%%%%%%%%%%%%%%%%%%%%%%%%%%%%%%%%%%%%%%%%%%%%%%%
\section{Justificativa}%

%conforme ilustrado na \refFig{latexvsword}.%
%\figuraBib{miktex}{\LaTeX\ vs MS Word}{pinteric_latex_2004}{latexvsword}{width=.45\textwidth}%

%%%%%%%%%%%%%%%%%%%%%%%%%%%%%%%%%%%%%%%%%%%%%%%%%%%%%%%%%%%%%%%%%%%%%%%%%%%%%%%%
%%%%%%%%%%%%%%%%%%%%%%%%%%%%%%%%%%%%%%%%%%%%%%%%%%%%%%%%%%%%%%%%%%%%%%%%%%%%%%%%
%%%%%%%%%%%%%%%%%%%%%%%%%%%%%%%%%%%%%%%%%%%%%%%%%%%%%%%%%%%%%%%%%%%%%%%%%%%%%%%%
\section{Objetivos}
Este trabalho tem como objetivo propor e implantar um modelo aderente à prática
de mercado para guiar a adoção de \textit{DevOps} no desenvolvimento das
aplicações corporativas do \acr{TCU}.

Para atingir esse objetivo, os seguintes objetivos específicos foram definidos:

\begin{itemize}
\item Elaborar uma \textit{grounded theory} que explique como se deu a adoção
de \textit{DevOps} por praticantes bem sucedidos na indústria e desenvolvimento
de software;
\item Propor e implantar um modelo de adoção de \textit{DevOps} com base na
\textit{grounded theory} produzida;
\item Realizar um estudo empírico para avaliar o modelo proposto no contexto
do desenvolvimento de uma aplicação corporativa do \acr{TCU}.
\end{itemize}

\section{Resultados Esperados}

\section{Estrutura do Trabalho}

%\section{Normas CIC}
% \href{http://monografias.cic.unb.br/dspace/normasGerais.pdf}{Política de Publicação de Monografias e Dissertações no Repositório Digital do CIC}%
% \href{http://monografias.cic.unb.br/dspace/}{Repositório do Departamento de Ciência da Computação da UnB}

% \href{http://bdm.bce.unb.br/}{Biblioteca Digital de Monografias de Graduação e Especialização}
