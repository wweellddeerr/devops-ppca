O advento e a consolidação dos métodos ágeis de desenvolvimento de
\textit{software} possibilitaram uma maior eficiência na produção de
\textit{software} com entregas mais rápidas em um período de tempo menor
\cite{}. Todavia, essa mudança gerou um aumento na demanda da equipe de
operações, tipicamente responsável pelas atividades relacionadas a
publicação dos artefatos de \textit{software} e que não participava
do processo de desenvolvimento ágil \cite{humble2010continuous}.

Visando reduzir as barreiras entre os times de desenvolvimento e operações,
trazendo as atividades típicas de operações para um contexto de agilidade
já presente durante o desenvolvimento, o conceito de \textit{DevOps} emergiu
na indústria de desenvolvimento de software. O termo \textit{DevOps} é uma
junção das palavras ``\textit{development}'' e ``\textit{operations}'' que
foi usado pela primeira vez em 2009 \cite{devops_for_developers}.

A necessidade de se rediscutir o papel do time de operações durante o
desenvolvimento de software já havia sido apontada mesmo antes de o termo
\textit{DevOps} existir. Uma apresentação da \textit{Flickr} \cite{flickr}
onde é proposta uma rediscussão sobre os papéis dos times de desenvolvimento e
operações, e os benefícios de uma mudança de paradigma
com maior aproximação e colaboração entre estes times são
explorados é considerada um dos pontos de partida para o surgimento de
\textit{DevOps} \cite{devops_for_developers}.

Os benefícios relacionados a \textit{DevOps} incluem aumento na performance
e na produtividate organizacional de TI, redução de custos no ciclo de vida
de \textit{software}, melhoria na eficiência e eficácia operacional e maior
alinhamento com o negócio entre os times de desenvolvimento e operações
\cite{characterizing_devops}.

Nesse contexto, a adoção de \textit{DevOps} no desenvolvimento de sistemas
corporativos do \acrfull{TCU} foi inserida como um dos objetivos estratégicos da
Secretaria de Soluções de Tecnologia da Informação em seu planejamento estratégico
para o período entre julho de 2017 e junho de 2018, uma vez que existe o
entendimento de que os rígidos procedimentos criados como paleativo para o
problema da baixa colaboração entre os times de desenvolvimento e operações
ocasionam atrasos na entrega de software do órgão que são considerados inadequados.

%%%%%%%%%%%%%%%%%%%%%%%%%%%%%%%%%%%%%%%%%%%%%%%%%%%%%%%%%%%%%%%%%%%%%%%%%%%%%%%%
%%%%%%%%%%%%%%%%%%%%%%%%%%%%%%%%%%%%%%%%%%%%%%%%%%%%%%%%%%%%%%%%%%%%%%%%%%%%%%%%
%%%%%%%%%%%%%%%%%%%%%%%%%%%%%%%%%%%%%%%%%%%%%%%%%%%%%%%%%%%%%%%%%%%%%%%%%%%%%%%%
\section{Problema de Pesquisa}%

%%%%%%%%%%%%%%%%%%%%%%%%%%%%%%%%%%%%%%%%%%%%%%%%%%%%%%%%%%%%%%%%%%%%%%%%%%%%%%%%
%%%%%%%%%%%%%%%%%%%%%%%%%%%%%%%%%%%%%%%%%%%%%%%%%%%%%%%%%%%%%%%%%%%%%%%%%%%%%%%%
%%%%%%%%%%%%%%%%%%%%%%%%%%%%%%%%%%%%%%%%%%%%%%%%%%%%%%%%%%%%%%%%%%%%%%%%%%%%%%%%
\section{Justificativa}%

%conforme ilustrado na \refFig{latexvsword}.%
%\figuraBib{miktex}{\LaTeX\ vs MS Word}{pinteric_latex_2004}{latexvsword}{width=.45\textwidth}%

%%%%%%%%%%%%%%%%%%%%%%%%%%%%%%%%%%%%%%%%%%%%%%%%%%%%%%%%%%%%%%%%%%%%%%%%%%%%%%%%
%%%%%%%%%%%%%%%%%%%%%%%%%%%%%%%%%%%%%%%%%%%%%%%%%%%%%%%%%%%%%%%%%%%%%%%%%%%%%%%%
%%%%%%%%%%%%%%%%%%%%%%%%%%%%%%%%%%%%%%%%%%%%%%%%%%%%%%%%%%%%%%%%%%%%%%%%%%%%%%%%
\section{Objetivos}
Este trabalho tem como objetivo propor e implantar um modelo aderente à prática
de mercado para guiar a adoção de \textit{DevOps} no desenvolvimento das
aplicações corporativas do \acrshort{TCU}.

Para atingir esse objetivo, os seguintes objetivos específicos foram definidos:

\begin{itemize}
\item Elaborar uma \textit{grounded theory} que explique como se deu a adoção
de \textit{DevOps} por praticantes bem sucedidos na indústria e desenvolvimento
de software;
\item Propor e implantar um modelo de adoção de \textit{DevOps} com base na
\textit{grounded theory} produzida;
\item Realizar um estudo empírico para avaliar o modelo proposto no contexto
do desenvolvimento de uma aplicação corporativa do \acrshort{TCU}.
\end{itemize}

\section{Resultados Esperados}

\section{Estrutura do Trabalho}

%\section{Normas CIC}
% \href{http://monografias.cic.unb.br/dspace/normasGerais.pdf}{Política de Publicação de Monografias e Dissertações no Repositório Digital do CIC}%
% \href{http://monografias.cic.unb.br/dspace/}{Repositório do Departamento de Ciência da Computação da UnB}

% \href{http://bdm.bce.unb.br/}{Biblioteca Digital de Monografias de Graduação e Especialização}
