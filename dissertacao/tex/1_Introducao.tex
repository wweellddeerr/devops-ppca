O advento e a consolidação dos métodos ágeis de desenvolvimento de
\textit{software} possibilitaram uma maior eficiência na produção de
\textit{software} com entregas mais rápidas em um período de tempo menor
\cite{agile_sw_dev}. Todavia, essa mudança gerou um aumento na demanda da equipe de
operações, tipicamente responsável pelas atividades relacionadas à
publicação dos artefatos de \textit{software}, e que não participava
do processo de desenvolvimento ágil \cite{humble2010continuous}.

Visando reduzir as barreiras entre os times, trazendo as atividades típicas
do time operações para um contexto de agilidade já presente durante o
desenvolvimento, o conceito de \textit{DevOps} emergiu na indústria de
desenvolvimento de \textit{software}. O termo \textit{DevOps} é uma
junção das palavras ``\textit{development}'' e ``\textit{operations}'' que
foi usado pela primeira vez em 2009 \cite{devops_for_developers}.

Antes mesmo de o termo \textit{DevOps} existir, artigos e apresentações
já tratavam da aplicação dos princípios ágeis também às atividades de operações,
da rediscussão sobre os papéis dos times de desenvolvimento e operações, e dos
benefícios de uma mudança de paradigma com maior aproximação e colaboração
entre estes times~\cite{devops_for_developers,agile_infra_operations,flickr}.

Os benefícios relacionados a \textit{DevOps} incluem aumento na performance
e na produtividade organizacional de \acrshort{TI}, redução de custos no ciclo de vida
de \textit{software}, melhoria na eficiência e eficácia operacional e maior
alinhamento com o negócio entre os times de desenvolvimento e operações
\cite{characterizing_devops}.

O interesse em {\it DevOps} é crescente. O \textit{State of DevOps Report}
aponta que, após mais de 27000 respostas ao \textit{DevOps survey} ao longo dos
últimos 6 anos, foi possível concluir que a utilização das práticas \textit{DevOps}
possibilita uma maior performance de \acrshort{TI} e que \textit{DevOps} contribui para melhorias nos
ciclos de entrega de \textit{software}, na qualidade e segurança de \textit{software}
e na capacidade de obter \textit{feedback} rápido sobre o desenvolvimento de produtos.
Ademais, \textit{DevOps} contribui para o atingimento das missões de qualquer
tipo de organização, independente da indústria ou setor \cite{state_of_devops}.

Já os resultados do \textit{agile survey}, realizado entre entre agosto e dezembro
de 2017 e publicados no \textit{State of Agile Report}, apontam que 71\% dos 1492
respondentes ou estão implementando \textit{DevOps} ou planejando adotar nos
próximos doze meses \cite{state_of_agile}.

Alinhado a esse contexto, o \acrfull{TCU} está buscando ampliar o uso da abordagem
\textit{DevOps} no desenvolvimento de suas aplicações corporativas. O
entendimento é de que os rígidos procedimentos criados como paliativo para o
problema da baixa colaboração entre os times de desenvolvimento e operações
ocasionam atrasos na entrega de \textit{software} do órgão que são considerados
inadequados.

Os times de desenvolvimento e operações do \acrshort{TCU} historicamente
trabalham em silos. Metodologias ágeis são utilizadas durante o processo de
desenvolvimento e a equipe de operações (\acrfull{SINAP}) participa da produção
de \textit{software} apenas em momentos específicos, notadamente quando uma nova
infraestrutura precisa ser provida para publicação de algum artefato de
\textit{software}, e também na publicação propriamente dita.

Essa estrutura em silos ocasionou alguns conflitos ao longo do tempo. Como
exemplos de problemas ocorridos que levaram à manifestação
destes conflitos, destacam-se: (1) o não funcionamento em ambiente de produção
de um \textit{software} que funciona adequadamente
em outros ambientes, como ambiente de desenvolvimento, por exemplo; e (2) o
atraso em entregas de \textit{software} consideradas importantes ocasionado
pelo não provisionamento tempestivo da infraestrutura necessária.

Como resposta aos conflitos ocorridos ao longo do tempo, o \acrshort{TCU}
investiu na padronização de procedimentos. O provimento de infraestrutura
para publicação de novas aplicações passou a funcionar em conformidade com um
\acrfull{SLA} que prevê prazos para conclusão por parte do time
de operações. No tocante à publicação do \textit{software}, foi criada uma padronização de
procedimentos denominada \emph{publicação programada}, que é vista como a maior
manifestação atual da falta de colaboração entre os times de desenvolvimento
e de operações do \acrshort{TCU}. A publicação programada
é um processo que prevê um complexo encadeamento de ações com horários bem
definidos visando reduzir o risco de problemas na publicação dos artefatos.
A versão de todas as aplicações desenvolvidas é congelada no início de cada
semana em horários preestabelecidos e existe uma grande quantidade de
procedimentos formais que cada um dos times deve seguir para que no final da
semana possa ocorrer a publicação em ambiente de produção.

A publicação programada e o SLA ajudaram a lidar com alguns dos problemas
existentes previamente, todavia, alguns outros problemas de ordem prática
passaram a existir e, por esse motivo, o \acrfull{PDTI} do \acrshort{TCU} para o
biênio 2017/2018 contém como um de seus indicadores: \emph{o aprimoramento do
uso de práticas ágeis em todas as equipes e \textbf{ampliação do uso da
abordagem DevOps}}.

O \acrshort{TCU} possui diversos comitês envolvidos no processo de desenvolvimento
de suas aplicações corporativas. Um desses comitês se chama \acrfull{CPA} que é
constituído por representantes de todas as equipes da área de \acrshort{TI} e cuja
função é orientar e padronizar as decisões tecnológicas e arquiteturais no
âmbito do processo de desenvolvimento de \textit{software}.

Inicialmente, para possibilitar o cumprimento do indicador constante no
\acrshort{PDTI}, o \acrshort{CPA} deliberou no sentido da experimentação e
realização de provas de conceito em ferramentas \textit{DevOps}. Após um
período de avaliação, o \acrshort{CPA} avaliou que, embora
houvesse competência técnica interna para implantação de ferramentas, existia
uma carência de melhor entendimento a respeito de como se adotar \textit{DevOps}.
Formou-se então novo entendimento de que experiências bem sucedidas na adoção de
\textit{DevOps} no mercado deveriam ser buscadas através de contato direto
e participações em eventos. Essas experiências deveriam ser apresentadas
internamente no sentido de pautar a adoção de \textit{DevOps} no \acrshort{TCU}.

\section{Problema de Pesquisa}%

A rigidez procedimental existente no \acrshort{TCU}, ilustra o
quão complexas podem se tornar soluções que visam contornar o problema da falta
de colaboração entre os times de desenvolvimento e operações. Neste sentido, a
adoção de {\it DevOps} é tida como necessária e faz parte dos indicadores do
\acrshort{PDTI} do órgão.

No entanto, a adoção de DevOps ainda é uma tarefa desafiadora, porque existe
uma infinidade de informações, práticas e ferramentas relacionadas a {\it DevOps},
mas ainda não está claro como praticantes de mercado têm utilizado de forma
organizada e estruturada essa rica, porém ainda dispersa, quantidade de
informação durante um processo de adoção de {\it DevOps}.

%%%%%%%%%%%%%%%%%%%%%%%%%%%%%%%%%%%%%%%%%%%%%%%%%%%%%%%%%%%%%%%%%%%%%%%%%%%%%%%%
%%%%%%%%%%%%%%%%%%%%%%%%%%%%%%%%%%%%%%%%%%%%%%%%%%%%%%%%%%%%%%%%%%%%%%%%%%%%%%%%
%%%%%%%%%%%%%%%%%%%%%%%%%%%%%%%%%%%%%%%%%%%%%%%%%%%%%%%%%%%%%%%%%%%%%%%%%%%%%%%%
\section{Justificativa}%

A adoção de \textit{DevOps} no \acrshort{TCU} está alinhada aos direcionadores
estratégicos da área de \acrshort{TI} do órgão e tem possibilitado uma
mudança de paradigma necessária para tratar adequadamente os problemas
decorrentes da baixa colaboração entre os times de desenvolvimento e operações.

A produção de um modelo para orientar a adoção de \textit{DevOps} com base em
experiências de mercado bem sucedidas, com implantação na adoção de
\textit{DevOps} no \acrshort{TCU}, representa um
incremento ao conhecimento existente sobre \textit{DevOps}, com destaque
para o melhor entendimento de \emph{como} se adotar \textit{DevOps} com base na
prática de mercado.

O intercâmbio de experiências e conhecimentos com praticantes do mercado
durante o processo de adoção de \textit{DevOps} no TCU possibilita uma redução
das chances de o \acrshort{TCU} incorrer em falhas comuns, já superadas em
outros contextos.

%%%%%%%%%%%%%%%%%%%%%%%%%%%%%%%%%%%%%%%%%%%%%%%%%%%%%%%%%%%%%%%%%%%%%%%%%%%%%%%%
%%%%%%%%%%%%%%%%%%%%%%%%%%%%%%%%%%%%%%%%%%%%%%%%%%%%%%%%%%%%%%%%%%%%%%%%%%%%%%%%
%%%%%%%%%%%%%%%%%%%%%%%%%%%%%%%%%%%%%%%%%%%%%%%%%%%%%%%%%%%%%%%%%%%%%%%%%%%%%%%%
\section{Objetivos}

\subsection{Objetivo Geral}

Este trabalho tem como objetivo geral investigar a adoção de {\it DevOps} no
mercado de desenvolvimento de {\it software} e utilizar os resultados e as
experiências obtidas durante a investigação para fomentar a adoção de \textit{DevOps}
no desenvolvimento das aplicações corporativas do \acrshort{TCU}.

\subsection{Objetivos Específicos}
Com o intuito de atingir o objetivo geral deste trabalho, os seguintes objetivos
específicos foram definidos:

\begin{itemize}
\item Investigar a adoção de \textit{DevOps} por praticantes bem sucedidos no
mercado de desenvolvimento de \textit{software};
\item Propor um modelo de adoção de \textit{DevOps} com base na prática de mercado;
\item Introduzir o modelo e as experiências obtidas durante a sua construção na
adoção de \textit{DevOps} no \acrshort{TCU}; e
\item Realizar um estudo empírico para avaliar o estágio atual da adoção de
\textit{DevOps} no \acrshort{TCU} bem como para validar a relevância do modelo
proposto neste contexto.
\end{itemize}

\section{Metodologia}

Para investigar a adoção bem sucedida de \textit{DevOps} no mercado foi
utilizada a variação clássica do método \textit{Grounded Theory} \cite{glaser1967discovery}.
O uso de \textit{Grounded Theory} se justifica por quatro motivos:

\begin{enumerate}

\item O método possibilita a construção de um entendimento independente e original,
o que é adequado para coletar evidência empírica diretamente da prática de
mercado sem o viés de pesquisas anteriores, alinhado às necessidades do \acrshort{TCU}.
A evidência coletada só é reintegrada com a literatura existente após a
construção da teoria \cite{reconciling_perspectives,stol2016grounded}.

\item \acrshort{GT} é uma metodologia consolidada em outras áreas de
pesquisa, como sociologia médica \cite{gt_medical_sociology}, nutrição
\cite{gt_nursing}, educação \cite{gt_education} e administração
\cite{gt_management,locke2001grounded}.

\item \acrshort{GT} é considerado um método adequado para caracterizar cenários
sob uma perspectiva pessoal daqueles envolvidos em uma disciplina ou atividade \cite{stol2016grounded},
que é exatamente o cenário que se almeja caracterizar aqui: quais são os caminhos
seguidos por praticantes que adotaram \textit{DevOps} de uma maneira bem sucedida?

\item \acrshort{GT} tem sido cada vez mais aplicada para estudar tópicos de
engenharia de \textit{software} \cite{hoda2017becoming,Waterman:2015:ICSE,stol2016grounded},
com destaque para o recente trabalho de Hoda et al. \cite{hoda2017becoming} que
caracterizou a adoção de métodos ágeis em uma maneira similar ao que se propõe
fazer aqui em relação a \textit{DevOps}.

\end{enumerate}

O modelo para adoção de \textit{DevOps} foi produzido com base na teoria
que foi construída utilizando o método \textit{Grounded Theory}.

Já a avaliação empírica foi realizada utilizando um grupo focal, pois este
método é considerado adequado para obter novos \textit{insights} sobre um
assunto de um pequeno grupo de pessoas nele envolvido, de maneira rápida, ao
invés de se buscar fornecer respostas quantificáveis a perguntas específicas
obtidas a partir de uma amostra grande da população \cite{focus_group_handbook,shull2007guide}.

\section{Estrutura do Trabalho}

O restante deste documento está organizado em cinco capítulos cujos conteúdos
foram estruturados da seguinte forma:

\begin{itemize}
\item \textbf{Capítulo 2.} Introduz o tema \textit{DevOps} por meio da
apresentação de seu histórico, definições, elementos que o caracterizam e
desafios relacionados;
\item \textbf{Capítulo 3.} Apresenta o detalhamento da pesquisa utilizando
\textit{\acrfull{GT}} que possibilitou a caracterização da adoção de
\textit{DevOps} com base na percepção de praticantes do mercado;
\item \textbf{Capítulo 4.} Contém detalhes e resultados da avaliação empírica
a respeito da adoção de {\it DevOps} e uso do modelo no \acrshort{TCU};
\item \textbf{Capítulo 5.} Contém as considerações finais do trabalho com
designação das suas limitações bem como de oportunidades para trabalhos futuros.
\end{itemize}


%\section{Normas CIC}

% \href{http://monografias.cic.unb.br/dspace/normasGerais.pdf}{Política de Publicação de Monografias e Dissertações no Repositório Digital do CIC}%
% \href{http://monografias.cic.unb.br/dspace/}{Repositório do Departamento de Ciência da Computação da UnB}

% \href{http://bdm.bce.unb.br/}{Biblioteca Digital de Monografias de Graduação e Especialização}
