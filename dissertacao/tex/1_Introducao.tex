O advento e a consolidação dos métodos ágeis de desenvolvimento de
\textit{software} possibilitaram uma maior eficiência na produção de
\textit{software} com entregas mais rápidas em um período de tempo menor
\cite{agile_sw_dev}. Todavia, essa mudança gerou um aumento na demanda da equipe de
operações, tipicamente responsável pelas atividades relacionadas a
publicação dos artefatos de \textit{software}, e que não participava
do processo de desenvolvimento ágil \cite{humble2010continuous}.

Visando reduzir as barreiras entre os times, trazendo as atividades típicas
do time operações para um contexto de agilidade já presente durante o
desenvolvimento, o conceito de \textit{DevOps} emergiu na indústria de
desenvolvimento de \textit{software}. O termo \textit{DevOps} é uma
junção das palavras ``\textit{development}'' e ``\textit{operations}'' que
foi usado pela primeira vez em 2009 \cite{devops_for_developers}.

Antes mesmo de o termo \textit{DevOps} existir, artigos e apresentações
já tratavam da aplicação dos princípios ágeis também às atividades de operações,
da rediscussão sobre os papéis dos times de desenvolvimento e operações, e dos
benefícios de uma mudança de paradigma com maior aproximação e colaboração
entre estes times~\cite{devops_for_developers,flickr}.

Os benefícios relacionados a \textit{DevOps} incluem aumento na performance
e na produtividate organizacional de \acrshort{TI}, redução de custos no ciclo de vida
de \textit{software}, melhoria na eficiência e eficácia operacional e maior
alinhamento com o negócio entre os times de desenvolvimento e operações
\cite{characterizing_devops}.

Nesse contexto, o \acrfull{TCU} inseriu como um dos direcionadores presentes
em seu \acrfull{PDTI} para o biênio 2017/2018: \emph{o
aprimoramento do uso de práticas ágeis em todas as equipes e ampliação do uso da
abordagem DevOps}. O entendimento é de que os rígidos procedimentos criados
como paleativo para o problema da baixa colaboração entre os times de
desenvolvimento e operações ocasionam atrasos na entrega de \textit{software}
do órgão que são considerados inadequados.

%%%%%%%%%%%%%%%%%%%%%%%%%%%%%%%%%%%%%%%%%%%%%%%%%%%%%%%%%%%%%%%%%%%%%%%%%%%%%%%%
%%%%%%%%%%%%%%%%%%%%%%%%%%%%%%%%%%%%%%%%%%%%%%%%%%%%%%%%%%%%%%%%%%%%%%%%%%%%%%%%
%%%%%%%%%%%%%%%%%%%%%%%%%%%%%%%%%%%%%%%%%%%%%%%%%%%%%%%%%%%%%%%%%%%%%%%%%%%%%%%%
\section{Problema de Pesquisa}%
O \acrshort{TCU} possui sete times de desenvolvimento de \textit{software} e um
único time de operações, que historicamente trabalham em silos.
A equipe de operações (\acrfull{SINAP}) participa da produção de
\textit{software} apenas em momentos específicos, notadamente quando uma nova
infraestrutura precisa ser provida para publicação de algum artefato de
\textit{software}, e também na publicação propriamente dita.

Essa estrutura em silos ocasionou alguns conflitos ao longo do tempo. Como
exemplos de problemas ocorridos que levaram à manifestação
destes conflitos, destacam-se: (1) o não funcionamento em ambiente de produção
de um \textit{software} que funciona adequadamente
em outros ambientes, como ambiente de desenvolvimento, por exemplo; e (2) o
atraso em entregas de \textit{software} consideradas importantes ocasionado
pelo não provisionamento tempestivo da infraestrutura necessária.

Como resposta aos conflitos ocorridos ao longo do tempo, o \acrshort{TCU}
investiu na padronização de procedimentos. O provimento de infraestrutura
para publicação de novas aplicações passou a funcionar em conformidade com um
\acrfull{SLA} que prevê um prazo de sete dias para conclusão por parte do time
de operações. O provimento de infraestrutura deve ser pedido por qualquer
dos times de desenvolvimento por meio do registro de uma solicitação no sistema
de \textit{servicedesk} interno, e o prazo inicia após este registro.
Na solicitação devem constar os detalhes da configuração (quantidade de memória
e CPU, por exemplo) necessária, bem como estimativa de número de usuários que
irão acessar a aplicação.

No tocante à publicação do \textit{software}, foi criada uma padronização de
procedimentos denominada \emph{publicação programada}, que é vista como a maior
manifestação atual da falta de colaboração entre os times de desenvolvimento
de operações do \acrshort{TCU}. A publicação programada
é um processo que prevê um complexo encadeamento de ações com horários bem
definidos visando reduzir o risco de problemas na publicação dos artefatos. Os
detalhes desse processo são apresentados a seguir.

A versão de todas as aplicações desenvolvidas é congelada semanalmente
às 03h00m de terça-feira. Para que esse congelamento fosse viabilizado, foi
criado um segundo servidor \textit{Artifactory}\footnote{Gerenciador de
repositórios de artefatos de \textit{software} \url{https://jfrog.com/artifactory/}}
e um segundo servidor \textit{Jenkins}\footnote{Servidor de integração contínua
\url{https://jenkins.io/}}
cujos \textit{jobs} apenas podem ser executados no
contexto da publicação programada. Uma primeira construção é executada com
base no código contante na \textit{branch} master das aplicações no horário
estabelecido. Os artefatos produzidos são então publicados no servidor
\textit{artifactory} onde só podem ser alterados pelo \acrshort{SINAP}.

As equipes de desenvolvimento tem então um prazo até
as 13h00m para verificar se os artefatos foram gerados adequadamente. Após
encerrado este prazo, o \acrshort{SINAP} efetua a publicação
em um ambiente denominado pré-produção. Caso algum artefato não tenha sido
gerado adequadamente, os times devem efetuar as correções necessárias e
solicitar a reexecução dos relativos \textit{jobs} no servidor \textit{Jenkins}
que opera apenas no contexto da publicação programada. Cumpre ressaltar que
antes de serem reexecutados, os \textit{jobs} precisam ser ajustados para
congelar uma nova \textit{branch} do repositório de código fonte, não mais a
\textit{branch} master que pode já possuir código que não faz parte do escopo
da publicação da semana em tela.

Pré-produção é um ambiente mantido pelo time de operações que possui o
propósito de ser \emph{o mais similar possível} ao ambiente de produção,
buscando-se evitar o não funcionamento de alguma aplicação apenas em ambiente
de produção.

Após a publicação em pré-produção, o \acrshort{SINAP} envia um e-mail de
``abertura de temporada de testes em pré-produção'', a partir do qual os times
de desenvolvimento executam testes manuais em suas respectivas aplicações.

É comum que versões específicas de artefatos necessitem da execução de
\textit{scripts} em banco de dados para funcionar adequadamente, esta execução
de \textit{scripts} também é padronizada. A sua versão definitiva
precisa ser concluída e incluída junto com os artefatos para publicação até o
limite de 12h00 de terça-feira. Ressalte-se que para que um \textit{script}
esteja em versão definitiva, é necessário ter passado pela equipe de operações
para validação, para a qual não há prazo definido. Um \textit{script}
proposto por algum dos times de desenvolvimento durante o dia de segunda-feira
tipicamente não é validado pelo time de operações até o limite de prazo e isso
ocasiona erros na publicação dos artefatos relacionados.

Após a publicação em ambiente de pré-produção (13h00m de terça-feira), os times
de desenvolvimento podem solicitar a substituição dos artefatos a qualquer
momento até o limite de 18h00m de quinta-feira, quando os artefatos são
congelados definitivamente para publicação em ambiente de produção. Novas
publicações em pré-produção ocorrem às 18h00m de terça-feira e às 14h00m de
quarta e quinta-feira. Essas novas publicações são chamadas de \emph{corretivas}.
A geração de artefatos para publicações corretivas, assim como a reexecução
de \textit{jobs} para geração de artefatos que falharam, é um procedimento
executado manualmente e requer diversos cuidados relacionados à criação de novas
\textit{branchs} e ajuste dos \textit{jobs} para evitar a inclusão de código
inserido por outro desenvolvedor com o propósito de publicação apenas na semana
seguinte.

A publicação em ambiente de produção ocorre \textbf{após} o término do
expediente de quinta-feira, que ocorre às 20h00m em dias quando não há sessão
plenária ou logo após o seu término (não há horário máximo estipulado).
Os procedimentos de publicação são executados manualmente pelo time de
operações e ocorre um \textit{downtime} que pode variar de 10 minutos a algumas
horas.

O \acrshort{SLA} entre as equipes prevê que em caso de erros na publicação em
produção, o \acrshort{SINAP} irá reverter a versão anterior e os novos
artefatos só serão publicados na semana seguinte. Está previsto ainda que
não haverá publicação programada em semanas que contenham feriados ou pontos
facultativos na quinta ou sexta-feira.

A publicação programada ajuda a lidar com alguns dos problemas existentes
previamente. Todavia, alguns outros problemas de ordem prática passam a existir:

\begin{itemize}
\item Uma eventual evolução pontual em alguma aplicação, cujo código tenha sido
disponibilizado pelo desenvolvedor na manhã de uma terça-feira, só será
disponibilizada para o usuário final em ambiente de produção na sexta-feira da
semana seguinte, uma demora de dez dias que nem sempre é aceitável;

\item Por mais que o ambiente de pré-produção seja semelhante ao de produção,
ainda ocorrem erros de versões que funcionam naquele mas não neste;

\item A existência de servidores \textit{artifactory} e \textit{jenkins}
duplicados aumenta o retrabalho e o risco de erros durante o processo de
geração de artefatos para publicação;

\item A publicação manual e com \textit{downtime} requer a alocação de
profissionais para trabalho fora do horário regulamentar do \acrshort{TCU},
gerando a necessidade de pagamento de horas extras, cujo número se acentua
em dias quando ocorre sessão plenária;

\item O término do horário de expediente em Brasília diverge de algumas outras
capitais. Em caso de horário de verão, a diferença para Rio Branco, no Acre,
por exemplo, chega a ser de 3 horas. Ou seja, o \textit{downtime} de 10 minutos
a algumas horas impede a continuidade dos trabalhos em algumas secretarias
de controle externo que funcionam nos estados;

\item Combinações de feriados com publicações com erros podem ocasionar longos
períodos sem publicação de \textit{software} em ambiente de produção. Por exemplo, já
ocorreu de em uma semana a publicação ser revertida pela ocorrência de erros,
na semana seguinte ser feriado e na terceira ocorrer novo erro na publicação.
Neste cenário, é possível que uma nova funcionalidade pronta tenha passado um mês
inteiro até ser publicada;

\item Nos casos de não funcionamento de aplicações em ambiente de produção,
a postura de resolução não é colaborativa. A preocupação do time de operações
tende a ser tão somente assegurar que não é um problema na infraestrutura,
enquanto que os times de desenvolvimento buscam defender que não é um problema
na aplicação.

\end{itemize}

Toda essa rigidez procedimental, com tarefas e horários encadeados ilustra o
quão complexa podem se tornar soluções que visam contornar o problema da falta
de colaboração entre os times de desenvolvimento e operações.

Ademais, diante da necessidade de adoção de \textit{DevOps} no TCU, foi
identificado que existe uma carência de orientações na literatura existente
sobre \textit{DevOps} a respeito de caminhos a serem seguidos por novos
praticantes.

%%%%%%%%%%%%%%%%%%%%%%%%%%%%%%%%%%%%%%%%%%%%%%%%%%%%%%%%%%%%%%%%%%%%%%%%%%%%%%%%
%%%%%%%%%%%%%%%%%%%%%%%%%%%%%%%%%%%%%%%%%%%%%%%%%%%%%%%%%%%%%%%%%%%%%%%%%%%%%%%%
%%%%%%%%%%%%%%%%%%%%%%%%%%%%%%%%%%%%%%%%%%%%%%%%%%%%%%%%%%%%%%%%%%%%%%%%%%%%%%%%
\section{Justificativa}%

O modelo produzido para orientar a adoção de \textit{DevOps} com base em
experiências de mercado bem sucedidas, produzido durante a pesquisa e
devidamente implementado no \acrshort{TCU}, representa um
incremento ao conhecimento existente sobre \textit{DevOps}, com destaque
para o melhor entendimento de \emph{como} se adotar \textit{DevOps} com base na
prática de mercado.

Ademais, a realização deste estudo contribui para o intercâmbio de experiências
e conhecimentos que apóiam o processo de adoção de \textit{DevOps} no TCU
para que ocorra alinhado às práticas de mercado e, portanto, com menor
chance de incorrer em falhas comuns, já superadas em outros contextos.

A adoção de \textit{DevOps} no \acrshort{TCU} está alinhada aos direcionadores
estratégicos da área de \acrshort{TI} do órgão e tem possibilitado uma
mudança de paradigma necessária para tratar adequadamente os problemas
decorrentes da baixa colaboração entre os times de desenvolvimento e operações.

%%%%%%%%%%%%%%%%%%%%%%%%%%%%%%%%%%%%%%%%%%%%%%%%%%%%%%%%%%%%%%%%%%%%%%%%%%%%%%%%
%%%%%%%%%%%%%%%%%%%%%%%%%%%%%%%%%%%%%%%%%%%%%%%%%%%%%%%%%%%%%%%%%%%%%%%%%%%%%%%%
%%%%%%%%%%%%%%%%%%%%%%%%%%%%%%%%%%%%%%%%%%%%%%%%%%%%%%%%%%%%%%%%%%%%%%%%%%%%%%%%
\section{Objetivos}
Este trabalho tem como objetivo propor e implantar um modelo aderente à prática
de mercado para guiar a adoção de \textit{DevOps} no desenvolvimento das
aplicações corporativas do \acrshort{TCU}.

Para atingir esse objetivo, os seguintes objetivos específicos foram definidos:

\begin{itemize}
\item Elaborar uma \textit{grounded theory} que explique a adoção
de \textit{DevOps} por praticantes bem sucedidos na indústria de desenvolvimento
de software;
\item Propor um modelo de adoção de \textit{DevOps} com base na
\textit{grounded theory} produzida;
\item Implantar o modelo para guiar a adoção de \textit{DevOps} no \acrshort{TCU}; e
\item Realizar um estudo empírico para avaliar o modelo proposto no contexto
do desenvolvimento de uma aplicação corporativa do \acrshort{TCU}.
\end{itemize}

\section{Resultados Esperados}
O resultado esperado para essa pesquisa é a construção e implantação no
\acrshort{TCU} de um modelo para adoção de \textit{DevOps} com base em
experiências bem sucedidas coletadas empiricamente da prática na indústria.

\section{Estrutura do Trabalho}

O restante deste documento está organizado em cinco capítulos cujos conteúdos
foram estruturados da seguinte forma:

\begin{itemize}
\item \textbf{Capítulo 2.} Introduz o tema \textit{DevOps} por meio da
apresentação de seu histórico, definições e elementos que o caracterizam;
\item \textbf{Capítulo 3.} Apresenta o detalhamento da pesquisa utilizando
\textit{\acrfull{GT}} que possibilitou a caracterização da adoção de
\textit{DevOps} com base na percepção de praticantes da indústria;
\item \textbf{Capítulo 4.} Contém detalhes de um modelo para adoção de
\textit{DevOps} proposto com base na teoria construída no Capítulo 2 e os
respectivos detalhes da sua implantação no \acrshort{TCU};
\item \textbf{Capítulo 5.} Apresenta os resultados de uma avaliação empírica
utilizada para obter as percepções relacionadas à adoção de \textit{DevOps} no
\acrshort{TCU}.
\item \textbf{Capítulo 6.} Contém as conclusões do trabalho com designação
das suas limitações bem como de oportunidades para trabalhos futuros.
\end{itemize}


%\section{Normas CIC}

% \href{http://monografias.cic.unb.br/dspace/normasGerais.pdf}{Política de Publicação de Monografias e Dissertações no Repositório Digital do CIC}%
% \href{http://monografias.cic.unb.br/dspace/}{Repositório do Departamento de Ciência da Computação da UnB}

% \href{http://bdm.bce.unb.br/}{Biblioteca Digital de Monografias de Graduação e Especialização}
