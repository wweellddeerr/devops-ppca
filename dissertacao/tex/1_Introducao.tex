O advento e a consolidação dos métodos ágeis de desenvolvimento de
\textit{software} possibilitaram uma maior eficiência na produção de
\textit{software} com entregas mais rápidas em um período de tempo menor
\cite{}. Todavia, essa mudança gerou um aumento na demanda da equipe de
operações, tipicamente responsável pelas atividades relacionadas a
publicação dos artefatos de \textit{software} e que não participava
do processo de desenvolvimento ágil \cite{humble2010continuous}.

Visando reduzir as barreiras entre os times de desenvolvimento e operações,
trazendo as atividades típicas de operações para um contexto de agilidade
já presente durante o desenvolvimento, o conceito de \textit{DevOps} emergiu
na indústria de desenvolvimento de software. O termo \textit{DevOps} é uma
junção das palavras ``\textit{development}'' e ``\textit{operations}'' que
foi usado pela primeira vez em 2009 \cite{devops_for_developers}.

A necessidade de se rediscutir o papel do time de operações durante o
desenvolvimento de software já havia sido apontada mesmo antes de o termo
\textit{DevOps} existir. Uma apresentação da \textit{Flickr} \cite{flickr}
onde é proposta uma rediscussão sobre os papéis dos times de desenvolvimento e
operações, e os benefícios de uma mudança de paradigma
com maior aproximação e colaboração entre estes times são
explorados é considerada um dos pontos de partida para o surgimento de
\textit{DevOps} \cite{devops_for_developers}.

Os benefícios relacionados a \textit{DevOps} incluem aumento na performance
e na produtividate organizacional de \acrshort{TI}, redução de custos no ciclo de vida
de \textit{software}, melhoria na eficiência e eficácia operacional e maior
alinhamento com o negócio entre os times de desenvolvimento e operações
\cite{characterizing_devops}.

Nesse contexto, o \acrfull{TCU} inseriu como um dos direcionadores da Secretaria
de Soluções de Tecnologia da Informação para o biênio 2017/2018: \emph{o
aprimoramento do uso de práticas ágeis em todas as equipes e ampliação do uso da
abordagem DevOps}. O entendimento é de que os rígidos procedimentos criados
como paleativo para o problema da baixa colaboração entre os times de
desenvolvimento e operações ocasionam atrasos na entrega de software do órgão
que são considerados inadequados.

%%%%%%%%%%%%%%%%%%%%%%%%%%%%%%%%%%%%%%%%%%%%%%%%%%%%%%%%%%%%%%%%%%%%%%%%%%%%%%%%
%%%%%%%%%%%%%%%%%%%%%%%%%%%%%%%%%%%%%%%%%%%%%%%%%%%%%%%%%%%%%%%%%%%%%%%%%%%%%%%%
%%%%%%%%%%%%%%%%%%%%%%%%%%%%%%%%%%%%%%%%%%%%%%%%%%%%%%%%%%%%%%%%%%%%%%%%%%%%%%%%
\section{Problema de Pesquisa}%
O \acrshort{TCU} possui sete times de desenvolvimento de \textit{software} e um
único time de operações chamado de \acrfull{SINAP}. Esses times de
desenvolvimento e operações trabalham historicamente em silos.
O \acrshort{SINAP} participa do processo apenas em
momentos específicos, notadamente quando uma infraestrutura precisa ser provida
para publicação de algum artefato de \textit{software} e na publicação
propriamente dita.

Essa estrutura em silos ocasionou alguns conflitos ao longo do tempo. Como
exemplos de problemas ocorridos que levaram à manifestação
de conflitos de interesses entre os times, destacam-se: (1) o não
funcionamento em ambiente de produção de um software que funciona adequadamente
em ambiente de desenvolvimento; e (2) o atraso em entregas de software
importantes pelo não provisionamento tempestivo da infraestrutura necessária.

Como resposta aos conflitos ocorridos ao longo do tempo, o \acrshort{TCU}
investiu na padronização de procedimentos. O provimento de infraestrutura
para publicação de novas aplicações passou a funcionar de acordo com um
\acrfull{SLA} que previa um prazo de sete dias desde a abertura por qualquer
dos times de desenvolvimento de uma solicitação no sistema de
\textit{servicedesk} interno. Na solicitação deveriam constar os detalhes
de configuração necessária bem como estimativa de número de usuários que
acessariam a aplicação.

No tocante à publicação do software foi criada uma padronização de
procedimentos denominada publicação programada. A publicação programada prevê
que uma versão de todas as aplicações desenvolvidas será congelada semanalmente
às 03h00m de terça-feira. As equipes de desenvolvimento tem então um prazo até
as 13h00m para verificar se os artefatos foram congelados adequadamente. Após
encerrado este prazo, o \acrshort{SINAP} efetua a publicação dos artefatos
em um ambiente denominado pré-produção.

Pré-produção é um ambiente mantido pelo time de operações que possui o
propósito de ser o mais similar possível com o ambiente de produção, buscando-se
evitar o não funcionamento de alguma aplicação apenas em ambiente de produção.

Após a publicação em pré-produção, o \acrshort{SINAP} envia um e-mail de
``abertura de temporada de testes em pré-produção'', a partir do qual os times
de desenvolvimento executam testes manuais em suas respectivas aplicações.

Caso seja necessária a execução de \textit{scripts} em banco de dados, a versão
definitiva desses \textit{scripts} precisa ser concluída e incluída junto com
os artefatos para publicação até o limite de 12h00 de terça-feira.

Após a publicação em ambiente de pré-produção (terça-feira) os artefatos podem
ser substituídos pelo time de desenvolvimento a qualquer momento até o limite
de 18h00m de quinta-feira, quando os artefatos são congelados definitivamente
para publicação em ambiente de produção. Novas publicações em pré-produção
ocorrem às 18h00m de terça-feira e às 14h00m de quarta e quinta-feira. Essas
novas publicações são chamadas de corretivas. A geração de artefatos para
publicações corretivas é um procedimento executado manualmente pelo
desenvolvedor que deseja a nova publicação e requer diversos cuidados para
evitar a inclusão de código inserido por outro desenvolvedor com o propósito
de publicação apenas na semana seguinte.

A publicação em ambiente de produção ocorre \textbf{após} o término do
expediente de quinta-feira que ocorre às 20h00m em dias quando não há sessão
plenária ou logo após o seu término que não há horário máximo estipulado.
Os procedimentos de publicação são executados manualmente pelo time de
operações e gera um \textit{downtime} que pode variar de 10 minutos a algumas
horas.

O \acrshort{SLA} entre as equipes prevê que em caso de erros na publicação em
produção, o \acrshort{SINAP} irá reverter a versão anterior e os novos
artefatos só serão publicados na semana seguinte. Está previstos ainda que
não haverá publicação programada em semanas que contenham feriados ou pontos
facultativos na quinta ou sexta-feira.

A publicação programada ajuda a lidar com alguns dos problemas existentes
previamente. Todavia, alguns outros problemas de ordem prática passam a existir:

\begin{itemize}
\item Uma eventual evolução pontual em alguma aplicação, cujo código tenha sido
disponibilizado pelo desenvolvedor na manhã de uma terça-feira, só será
disponibilizada para o usuário final em ambiente de produção na sexta-feira da
semana seguinte, uma demora de 10 dias que nem sempre é aceitável;

\item Por mais que o ambiente de pré-produção seja semelhante ao de produção,
ainda ocorrem erros de versões que funcionam naquele mas não neste;

\item downtime no Acre

\item banco de dados sincronização

\item cenários com feriados e erros (time-to-market)

\end{itemize}

Toda essa rigidez procedimental, com tarefas e horários encadeados ilustra o
quão complexa podem se tornar soluções que visam contornar o problema da falta
de colaboração entre os times de desenvolvimento e operações.

%%%%%%%%%%%%%%%%%%%%%%%%%%%%%%%%%%%%%%%%%%%%%%%%%%%%%%%%%%%%%%%%%%%%%%%%%%%%%%%%
%%%%%%%%%%%%%%%%%%%%%%%%%%%%%%%%%%%%%%%%%%%%%%%%%%%%%%%%%%%%%%%%%%%%%%%%%%%%%%%%
%%%%%%%%%%%%%%%%%%%%%%%%%%%%%%%%%%%%%%%%%%%%%%%%%%%%%%%%%%%%%%%%%%%%%%%%%%%%%%%%
\section{Justificativa}%

O modelo produzido para orientar a adoção de \textit{DevOps} com base em
experiências de mercado bem sucedidas, produzido durante a pesquisa e
devidamente implementado no \acrshort{TCU}, representa um
incremento ao conhecimento existente sobre \textit{DevOps}, com destaque
para o melhor entendimento de como se adotar \textit{DevOps} com base na
prática de mercado.

Ademais, a realização deste estudo contribui para o intercâmbio de experiências
e conhecimentos que apóiam o processo de adoção de \textit{DevOps} no TCU
para que ocorra alinhado às práticas de mercado e, portanto, com menor
chance de incorrer em falhas comuns, já superadas em outros contextos.

A adoção de \textit{DevOps} no \acrshort{TCU} está alinhada aos direcionadores
estratégicos da área de \acrshort{TI} do órgão e irá possibilitar uma
mudança de paradigma necessária para tratar adequadamente os problemas
decorrentes da baixa colaboração entre os times de desenvolvimento e operações.

%conforme ilustrado na \refFig{latexvsword}.%
%\figuraBib{miktex}{\LaTeX\ vs MS Word}{pinteric_latex_2004}{latexvsword}{width=.45\textwidth}%

%%%%%%%%%%%%%%%%%%%%%%%%%%%%%%%%%%%%%%%%%%%%%%%%%%%%%%%%%%%%%%%%%%%%%%%%%%%%%%%%
%%%%%%%%%%%%%%%%%%%%%%%%%%%%%%%%%%%%%%%%%%%%%%%%%%%%%%%%%%%%%%%%%%%%%%%%%%%%%%%%
%%%%%%%%%%%%%%%%%%%%%%%%%%%%%%%%%%%%%%%%%%%%%%%%%%%%%%%%%%%%%%%%%%%%%%%%%%%%%%%%
\section{Objetivos}
Este trabalho tem como objetivo propor e implantar um modelo aderente à prática
de mercado para guiar a adoção de \textit{DevOps} no desenvolvimento das
aplicações corporativas do \acrshort{TCU}.

Para atingir esse objetivo, os seguintes objetivos específicos foram definidos:

\begin{itemize}
\item Elaborar uma \textit{grounded theory} que explique a adoção
de \textit{DevOps} por praticantes bem sucedidos na indústria de desenvolvimento
de software;
\item Propor um modelo de adoção de \textit{DevOps} com base na
\textit{grounded theory} produzida;
\item Implantar o modelo para guiar a adoção de \textit{DevOps} no \acrshort{TCU}; e
\item Realizar um estudo empírico para avaliar o modelo proposto no contexto
do desenvolvimento de uma aplicação corporativa do \acrshort{TCU}.
\end{itemize}

\section{Resultados Esperados}

\section{Estrutura do Trabalho}

%\section{Normas CIC}
% \href{http://monografias.cic.unb.br/dspace/normasGerais.pdf}{Política de Publicação de Monografias e Dissertações no Repositório Digital do CIC}%
% \href{http://monografias.cic.unb.br/dspace/}{Repositório do Departamento de Ciência da Computação da UnB}

% \href{http://bdm.bce.unb.br/}{Biblioteca Digital de Monografias de Graduação e Especialização}
