\section{Uma Teoria Sobre a Adoção de \textit{DevOps}}\label{secao_teoria}

Os resultados de um estudo utilizando {\it grounded theory}, como o próprio
nome do método sugere, são fundamentados nos dados coletados, de modo que as
hipóteses emergem dos dados, ao invés de serem definidas no início da pesquisa
e validadas utilizando algum modelo estatítico. Hoda et. al~\cite{hoda2017becoming}
explicam que uma teoria produzida utilizando \emph{grounded theory} deve produzir
um conjunto de hipóteses inter-relacionadas e que o termo \emph{hipótese} neste
contexto se refere às descrições das relações-chave entre as categorias que
compõem uma determinada teoria.

Nas próximas seções serão apresentadas as hipóteses que descrevem as principais
relações entre as categorias que fazem parte da adoção de {\it DevOps}. Estas
hipóteses constituem a teoria produzida e indicam que a adoção
de {\it DevOps} é explicada por meio de uma rede de categorias que contribuem
para o desenvolvimento da \cc, os facilitadores. E que as demais categorias
relacionadas são consideradas resultados esperados do processo de adoção de
{\it DevOps}, as {\emph saídas DevOps}. É ainda explicado que algumas categorias
em alguns momentos podem ser consideradas facilitadores e em outros saídas.

\subsection{Uma Abordagem Geral para a Adoção de DevOps}

Aqui é apresentado um possível caminho que pode ser utilizado por
novos praticantes que desejam adotar {\it DevOps}. Esta abordagem foi construída
com base nas análises realizadas conforme detalhado na subseção~\ref{subsecao:abordagem_gt}.
Conforme já mencionado anteriormente, a categoria principal na adoção de {\it DevOps}
é \cc. Isto implica que a principal preocupação na adoção de {\it DevOps} deve
ser a formação e o desenvolvimento de uma \cc entre as equipes de desenvolvimento
e operações de {\it software}. De acordo com as análises, as outras categorias,
muitas das quais também estão presentes em outros estudos que investigaram
{\it DevOps}, só podem ser consideradas como efetivas para a adoção de {\it DevOps}
se as práticas e conceitos relacionados a elas contribuírem para o nível de \cc
ou levarem às conseqüências esperadas de uma \cc. Esse entendimento induz algumas
hipóteses, como discutido a seguir.

\begin{mh}
{\bf Hipótese 1:} {\it Existe um conjunto de categorias relacionadas à adoção de
{\it DevOps} que apenas fazem sentido se usadas para aumentar o nível da \cc.
Esse conjunto de categorias é chamado de {\bf facilitadores DevOps}}.
\end{mh}

\iffalse
Com base nesta primeira hipótese, a maturidade da adoção de DevOps não
avança em situações em que apenas uma equipe é responsável por entender, adaptar ou
evoluir a automação --- mesmo quando essa automação suportar diferentes atividades, como implantação, provisionamento de infraestrutura,
monitoramento. O mesmo vale para as outras categorias \emph{enable}. Ou seja, nas situações em que
\cat{transparência e compartilhamento} não contribuem para
o \cc, eles não contribuem para a adoção de DevOps como um todo. Alguns exemplos
que suportam nossa primeira hipótese incluem:

%\begin{mq}
%``\emph{Olhe, dentro do setor de operações houve algum grau de automação. O cara
% tinha armazenado em sua própria máquina bash scripts que o ajudaram na criação de um
% server ou ao criar uma nova instância de banco de dados. No entanto, não houve DevOps
% porque não havia relação intrínseca dessa automação com o
% processo de desenvolvimento} "(P11, DevOps Supervisor, Brazil)
% \ end {mq}


\begin{mq}
``\emph{DevOps envolve ferramentas, mas DevOps não é uma ferramenta. Isto é, as pessoas muitas vezes
focam no uso de ferramentas que são chamadas de `DevOps tools ', acreditando que DevOps é
isto. Eu sempre insisto que DevOps não é uma ferramenta, DevOps envolve o uso de
ferramentas para melhorar os procedimentos de desenvolvimento de software.} "(P2, DevOps
Consultor, Brasil)
\end{mq}


%% \begin{mq}
%% ``\emph{Manter a cultura viva continua a ser um desafio para nós e é muito
%% importante. Aqui na nossa empresa, por exemplo, temos Tech Talks que são
%% conversas mensais que temos com as equipes. O objetivo destes Tech
%% Talks é compartilhar conhecimentos sobre tecnologias e processos de trabalho aumentando a
%% transparência de como tudo funciona. Nós também temos um canal Slack chamado
%% DevOps como cultura, onde discutimos as coisas da cultura de DevOps. A ideia é não
%% deixar a cultura morrer, estamos sempre alimentando-a com alguma coisa, porque isso é
%% a essência de DevOps para nós.} "(P12, Cloud Engineer, Estados Unidos)
%% \end{mq}

\begin{mh}
\textbf{Hipótese 2:} \textit{Há um grupo de categorias relacionadas à adoção do DevOps
que não contribuem para aumentar o nível} \ cc \ emph {, mas
apontado como a adoção de DevOps relacionada, porque eles emergem como um
conseqüência necessária da adoção. Essas categorias representam o grupo de
\ textbf {resultados}}.
\end{mh}

Em um primeiro momento, o simples fato de que uma equipe é mais
\ cat {agile} na entrega de software, ou mais \ cat {resilient} na recuperação de falhas, não
contribui diretamente para aproximar as equipes de operações das equipes de desenvolvimento.
No entanto, um sinal de uma adoção madura de DevOps é um aumento da capacidade de continuar
entregando software (e sendo assim mais \ cat {ágil})
e para construir infraestruturas \ cat {resilient}.

\begin{mh}
\textbf{Hypothesis 3:} \textit{As categorias \ cat {Continuous Measurement} e \ cat {Quality Assurance}
ambas estão relacionadas ao DevOps permitindo capacidade e aos resultados de DevOps}.
\end{mh}

A medição é citada como uma responsabilidade típica da equipe de operações.
Ao mesmo tempo que compartilhar essa responsabilidade reduz os silos,
também é citado que a medição é uma consequência necessária da adoção de DevOps. Particularmente porque
a entrega contínua de software requer mais controle,
que é fornecido por conceitos relacionados à categoria \ cat {mensuração contínua}.
A mesma premissa é válida para a categoria \ cat {quality assurance}. À primeira vista,
\ cat {quality assurance} aparece como uma resposta ao contexto de agilidade nas operações
como resultado da adoção do DevOps. Mas, os esforços na garantia de qualidade de produtos de software
aumentam a confiança entre as equipes de desenvolvimento e operações, aumentando o nível de
de \ cc.

\subsection{DevOps Enablers}

Ao todo, os facilitadores DevOps são os meios comumente usados para aumentar o nível de
\ cc em um processo de adoção do DevOps.
Nós identificamos cinco categorias de ativadores do DevOps:
\ cat {Automation}, \ cat {Medição Contínua}, \ cat {Quality Assurance},
\ cat {Sharing} e {\ cat {Transparency}. Outra descoberta do nosso
estudo leva à nossa quarta hipótese.

\begin{mh}
\textbf{Hypothesis 4:} \textit{Não há precedência entre os facilitadores em um processo de adoção de DevOps}.
\end{mh}

Percebemos que o processo de adoção pode não ter que priorizar nenhum ativador, e uma empresa que visa implementar
DevOps deve começar com os facilitadores que parecem mais apropriados (em termos
das suas especificidades). Assim, não encontramos nenhuma evidência de que um enebler
é mais eficiente que outro para criar um \ cc. \cat{Automation} é a categoria
que aparece com mais frequência em nosso estudo, embora vários participantes
É claro que associar o DevOps à automação é um equívoco.
Por exemplo, embora 14 entrevistados citam \ cat {automation} como um importante
 facilitador para adotar o DevOps, alguns entrevistados também ponderam que considerar
 a automação com maior importância do que outras partes pode realmente ser um risco:

\begin{mq}
``\emph{Eu penso que a expansão da colaboração entre as equipes envolveu outras
coisas, não foi apenas automação. Deve haver um alinhamento com o
necessidades de negócios. (...) Eu acho que o DevOps possibilitou um entendimento
+mais amplo da produção de software e estávamos percebendo exatamente que não se
trata de automatizar tudo. (...) Então, vejo com cautela uma suposta visão que
automatizar as coisas pode ser a maneira de implementar o DevOps.} "(P7, Analista
de Suporte, Brasil
)
\end{mq}

%\begin{mq}
%``\emph{Despite of we actually use automation in a reasonable number of scenarios,
%we have been able to develop our culture significantly without automation and I think that you can reach a
%good DevOps level with little or even no automation.}" (P8, DevOps Engineer, Brazil)
%\end{mq}

%% That is, although \cat{automation} is a very commonly used enabler, it is possible to
%% increase the level of \cc without focus on automating. And
%% this premise is valid to the other enablers.

\subsection{Saídas DevOps}

{\emph Saídas DevOps} é aquele grupo de categorias que não produz primariamente
o efeito esperado de um {\emph facilitador DevOps}

DevOps outcomes is that group of categories that does not produces primarily the
expected effect of a DevOps enabler, typically concepts that are expected as
consequences of an adoption of DevOps. We have identified four categories that
can work as DevOps outcomes: \cat{agility}, \cat{continuous measurement},
\cat{quality assurance}, and \cat{software resilience}. Note that,
as mentioned before, \cat{continuous measurement} and \cat{quality assurance}
are both enablers and outcomes.

That is, a well succeeded DevOps adoption typically increases the potential of
\cat{agility} of teams and enables \cat{continuous measurement}, \cat{quality assurance} and
\cat{resilience} of applications.
However, in some situations, this potential is not completely used due business
decisions. For example, one respondent cited that, at a first moment, the
company did not allowed the continuous deployment (more potential of agility)
of applications in production:

\begin{mq}
``\emph{We had conditions and security to continuously publish in production,
however, in the beginning the managers were afraid and decided that the
publication would happen weekly.}" (P9, IT Manager, Brazil)
\end{mq}
\fi
