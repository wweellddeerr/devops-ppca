\section{Uma Teoria Sobre a Adoção de \textit{DevOps}}\label{secao_teoria}

\iffalse
Os resultados de um estudo de grounded theory, como o próprio nome do método sugere, baseiam-se nos dados coletados, de modo que as hipóteses emergem dos dados.
Grounded Theory deve descrever as relações-chave entre as
categorias que a compõem, ou seja, um conjunto de hipóteses inter-relacionadas ~ \ cite {hoda2017becoming}.

Apresentamos as categorias de nossa grounded theory sobre a adoção do DevOps como uma rede das três categorias de facilitadores  (\cat{automation}, \cat{sharing and transparency}) que são comumente usados para desenvolver a categoria central\ cc, conforme discutido na seção anterior.
Baseado em nosso entendimento, implementar os facilitadores para desenvolver o \ cc normalmente leva aos conceitos relacionados a duas categorias de resultados esperados:
\ cat {agilidade} e \ cat {resiliência}.

Além disso, existem duas categorias que podem ser consideradas
tanto como facilitadoras quanto como resultados: \ cat {medição contínua} e \ cat {garantia de qualidade}.
Nesta seção, descrevemos as relações entre essas categorias, construindo uma teoria
de adoção do DevOps.
\subsection{A General Path for DevOps Adoption}

Na seção~\ref{sec:introduction} apresentamos a questão geral desta
pesquisa: existe algum caminho recomendado para adotar DevOps? Aqui, nós elaboramos uma resposta,
com base nas análises realizadas conforme detalhado na seção~\ref{sec:research_method}. O principal
ponto que deve ser formulado é a construção de um \cc entre as equipes de desenvolvimento e operações de software e
atividades relacionadas. De acordo com nossas descobertas, as outras categorias,
muitas das quais também estão presentes em outros estudos que investigaram DevOps,
só fazem sentido se as práticas e conceitos relacionados à elas contribuírem para o nível de um \ cc ou levarem às conseqüências esperadas
de um \ cc. Esse entendimento induz à várias hipóteses, como discutido a seguir.


\begin{mh}
\ textbf {Hypothesis 1:} \ textit {Existe um grupo de categorias relacionadas à adoção de DevOps que só fazem sentido se usadas para aumentar o nível} \ cc \ emph {. Nós
chamamos esse grupo de categorias de \ textbf {enablers}}.
\ end {mh}

Com base nesta primeira hipótese, a maturidade da adoção de DevOps não
avança em situações em que apenas uma equipe é responsável por entender, adaptar ou
evoluir a automação --- mesmo quando essa automação suportar diferentes atividades, como implantação, provisionamento de infraestrutura,
monitoramento. O mesmo vale para as outras categorias \emph{enable}. Ou seja, nas situações em que
\cat{transparência e compartilhamento} não contribuem para
o \cc, eles não contribuem para a adoção de DevOps como um todo. Alguns exemplos
que suportam nossa primeira hipótese incluem:

%\begin{mq}
%``\emph{Olhe, dentro do setor de operações houve algum grau de automação. O cara
% tinha armazenado em sua própria máquina bash scripts que o ajudaram na criação de um
% server ou ao criar uma nova instância de banco de dados. No entanto, não houve DevOps
% porque não havia relação intrínseca dessa automação com o
% processo de desenvolvimento} "(P11, DevOps Supervisor, Brazil)
% \ end {mq}


\begin{mq}
``\emph{DevOps envolve ferramentas, mas DevOps não é uma ferramenta. Isto é, as pessoas muitas vezes
focam no uso de ferramentas que são chamadas de `DevOps tools ', acreditando que DevOps é
isto. Eu sempre insisto que DevOps não é uma ferramenta, DevOps envolve o uso de
ferramentas para melhorar os procedimentos de desenvolvimento de software.} "(P2, DevOps
Consultor, Brasil)
\end{mq}


%% \begin{mq}
%% ``\emph{Keeping the culture alive remains a challenge to us, and it is very
%% important. Here in our company, for example, we have Tech Talks that are
%% monthly conversations that we have with the teams. The purpose of these Tech
%% Talks is to share knowledge about technologies and work processes increasing the
%% transparency of how everything works. We also have a Slack channel called
%% DevOps as Culture where we discuss things of DevOps culture. The idea is not to
%% let the culture die, we are always feeding it with something, because that is
%% the DevOps essence for us.}" (P12, Cloud Engineer, United States)
%% \end{mq}

\begin{mh}
\textbf{Hypothesis 2:} \textit{There is a group of categories related to DevOps adoption
that does not contribute to increase the} \cc \emph{level, but that instead are
pointed out as DevOps adoption related, because they emerge as an expected or
necessary consequence of the adoption. These categories represent the group of
\textbf{outcomes}}.
\end{mh}

In a first moment, the simple fact that a team is more
\cat{agile} in delivering software, or more \cat{resilient} in failure recovery, does not
contribute directly to bring operations teams closer to development teams.
Nevertheless, a signal of a mature DevOps adoption is an increasing of the capacity for continuously
delivering software (and thus being more \cat{agile})
and for building \cat{resilient} infrastructures.

\begin{mh}
\textbf{Hypothesis 3:} \textit{The categories \cat{Continuous Measurement} and \cat{Quality Assurance}
are both related to DevOps enabling capacity and to DevOps outcomes}.
\end{mh}

Measurement is cited as a typical responsibility of the operations team.
At the same time that sharing this responsibility reduces silos,
it is also cited that measurement is a necessary consequence of DevOps adoption. Particularly because
the continuous delivery of software requires more control,
which is supplied by concepts related to the \cat{continuous measurement} category.
The same premise is valid to the \cat{quality assurance} category. At first glance,
\cat{quality assurance} appears as one response to the context of agility in operations
as a result of DevOps adoption. But, the efforts in quality assurance of software products
increase the confidence between the development and operations teams, increasing the level
of \cc.

\subsection{DevOps Enablers}

Altogether, DevOps enablers are the means commonly used to increase the level of
the \cc in a DevOps adoption process.
We have identified five categories of DevOps enablers:
\cat{ Automation}, \cat{Continuous Measurement}, \cat{Quality Assurance},
\cat{Sharing}, and {\cat{Transparency}. Another finding of our
study leads to our fourth hypothesis.

\begin{mh}
\textbf{Hypothesis 4:} \textit{There is no precedence between enablers in a DevOps adoption process}.
\end{mh}

We have realized that the adoption process might not have
to priorize any enabler, and a company that aims to implement
DevOps should start with  the enablers that seem more appropriate (in terms
of its specificities). Accordingly, we did not find any evidence that an enbler
is more efficient than another for creating a \cc. \cat{Automation} is the category
that appears more frequently in our study, though several participants make
clear that associating DevOps with automation is a misconception.
%% For
%% instance, although 14 respondents cite \cat{automation} as an important
%% enabler to adopt DevOps, some respondents also ponder that considering
%% automation with greater importance than other parts can actually be a risk:

\begin{mq}
``\emph{I think that the expansion of collaboration between teams involved other
things, it was not just automation. There must be an alignment with the
business needs. (...) I think that DevOps made possible a broader understanding
of software production and we were realizing exactly that it is not about
automating everything. (...) So, I see with caution a supposed vision that automate things can
be the way to implement DevOps.}" (P7, Support Analyst, Brazil)
\end{mq}

%\begin{mq}
%``\emph{Despite of we actually use automation in a reasonable number of scenarios,
%we have been able to develop our culture significantly without automation and I think that you can reach a
%good DevOps level with little or even no automation.}" (P8, DevOps Engineer, Brazil)
%\end{mq}

%% That is, although \cat{automation} is a very commonly used enabler, it is possible to
%% increase the level of \cc without focus on automating. And
%% this premise is valid to the other enablers.

\subsection{DevOps Outcomes}

DevOps outcomes is that group of categories that does not produces primarily the
expected effect of a DevOps enabler, typically concepts that are expected as
consequences of an adoption of DevOps. We have identified four categories that
can work as DevOps outcomes: \cat{agility}, \cat{continuous measurement},
\cat{quality assurance}, and \cat{software resilience}. Note that,
as mentioned before, \cat{continuous measurement} and \cat{quality assurance}
are both enablers and outcomes.

That is, a well succeeded DevOps adoption typically increases the potential of
\cat{agility} of teams and enables \cat{continuous measurement}, \cat{quality assurance} and
\cat{resilience} of applications.
However, in some situations, this potential is not completely used due business
decisions. For example, one respondent cited that, at a first moment, the
company did not allowed the continuous deployment (more potential of agility)
of applications in production:

\begin{mq}
``\emph{We had conditions and security to continuously publish in production,
however, in the beginning the managers were afraid and decided that the
publication would happen weekly.}" (P9, IT Manager, Brazil)
\end{mq}
\fi
