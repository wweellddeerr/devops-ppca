\section{Uma Teoria Sobre a Adoção de \textit{DevOps}}\label{secao_teoria}

Os resultados de um estudo utilizando {\it grounded theory}, como o próprio
nome do método sugere, são fundamentados nos dados coletados, de modo que as
hipóteses emergem dos dados, ao invés de serem definidas no início da pesquisa
e validadas utilizando algum modelo estatítico. Hoda et al.~\cite{hoda2017becoming}
explicam que uma teoria produzida utilizando \emph{grounded theory} deve conter
um conjunto de hipóteses inter-relacionadas e que o termo \emph{hipótese} neste
contexto se refere às descrições das relações-chave entre as categorias que
compõem uma determinada teoria.

Nas próximas seções serão apresentadas as hipóteses que descrevem as principais
relações entre as categorias que fazem parte da adoção de {\it DevOps}. Estas
hipóteses constituem a teoria produzida e indicam que a adoção
de {\it DevOps} é explicada por meio de uma rede de categorias que contribuem
para o desenvolvimento da \cc, os \emph{facilitadores}. E que algumas outras
categorias relacionadas são consideradas resultados esperados do processo de
adoção de {\it DevOps}, as \emph{saídas DevOps}. É ainda explicado que algumas
categorias em alguns momentos podem ser consideradas facilitadores e em outros
saídas.

\subsection{Uma Abordagem Geral para a Adoção de DevOps}

Aqui é apresentado um possível caminho que pode ser utilizado por
novos praticantes que desejam adotar {\it DevOps}. Esta abordagem foi construída
com base nas análises realizadas conforme detalhado na subseção~\ref{subsecao:abordagem_gt}.
Conforme já mencionado anteriormente, a categoria principal na adoção de {\it DevOps}
é \cc. Isto implica que a principal preocupação na adoção de {\it DevOps} deve
ser a formação e o desenvolvimento de uma \cc entre as equipes de desenvolvimento
e operações de {\it software}. De acordo com as análises, as outras categorias,
muitas das quais também estão presentes em outros estudos que investigaram
{\it DevOps}, só podem ser consideradas como efetivas para a adoção de {\it DevOps}
se as práticas e conceitos relacionados a elas contribuírem para o nível de \cc
ou levarem às conseqüências esperadas de uma \cc. Esse entendimento induz algumas
hipóteses, como discutido a seguir.

\begin{mh}
{\bf Hipótese 1:} {\it Existe um conjunto de categorias relacionadas à adoção de
{\it DevOps} que apenas fazem sentido se usadas para aumentar o nível da \cc.
Esse conjunto de categorias é chamado de {\bf \emph{facilitadores DevOps}}}.
\end{mh}

De modo geral, os \emph{facilitadores DevOps} são os meios comumente usados para
aumentar o nível da \cc em um processo de adoção de {\it DevOps}. Conforme
detalhado nas seções \ref{secao_facilitadores} e \ref{secao_facilitadores_e_saidas},
foram identificadas quatro categorias que atuam como \emph{facilitadores DevOps}:
\cat{automação}, \cat{medição contínua}, \cat{garantia da qualidade} e
\cat{compartilhamento e transparência}.

Com base nesta primeira hipótese, a maturidade da adoção de {\it DevOps} não
avança em situações em que apenas uma equipe é responsável por entender, adaptar ou
evoluir algum aspecto. Por exemplo, um sofisticado grau de automação, mesmo
quando ela suportar diferentes atividades, como implantação, provisionamento de
infraestrutura ou monitoramento, não contribui para \emph{DevOps} se não
contribuir para o incremento da \cc. O mesmo vale para as outras categorias
de \emph{facilitadores}. Ou seja, nas situações em que \cat{compartilhamento e
transparência}, \cat{garantia da qualidade} e \cat{medição contínua} não
contribuem para a \cc, também não possibilitam um avanço na adoção de
{\it DevOps} como um todo. Alguns exemplos que suportam essa primeira hipótese
incluem:

\begin{mq}
``\emph{Olhe, dentro do setor de operações havia algum grau de automação. O cara
 tinha bash scripts na máquina dele que ajudavam na criação de um servidor ou
 de uma nova instância de banco de dados. Só que isso na minha visão,
 não havia DevOps porque não tinha relação intrínseca dessa automação com o
 processo de desenvolvimento}'' (P11, Supervisor de {\it DevOps}, Brasil)
\end{mq}

\begin{mq}
``\emph{Manter a cultura viva continua sendo um desafio para nós e isso é muito
importante. Aqui na empresa, por exemplo, temos tech talks que são conversas
mensais que temos com as equipes. O objetivo destes Tech Talks é compartilhar
conhecimentos sobre tecnologias e processos de trabalho aumentando a
transparência de como tudo funciona. Nós também temos um canal Slack chamado
DevOps como cultura, onde discutimos as coisas da cultura DevOps. A ideia é
não deixar a cultura morrer, estamos sempre alimentando-a com alguma coisa,
porque isso é a essência de DevOps para nós.}'' (P12, {\it Cloud Engineer}, Estados Unidos)
\end{mq}

\begin{mh}
\textbf{Hipótese 2:} {\it Há um grupo de categorias relacionadas à adoção de
DevOps que não são mencionadas por contribuem para aumentar o nível da \cc, mas
que são apontados como relacionadas à adoção de DevOps, porque elas emergem
como um consequência da adoção. Essas categorias representam o grupo de
{\bf \emph{saídas DevOps}}}.
\end{mh}

Em um primeiro momento, o simples fato de que uma equipe é mais ágil na entrega
de {\it software}, ou produz aplicações mais resilientes na recuperação de
falhas, não é apontado como responsável por contribuir diretamente para
aproximar as equipes de operações das equipes de desenvolvimento. No entanto,
um sinal de uma adoção madura de {\it DevOps} é um aumento da capacidade de
entregar {\it software} continuamente (de maneira mais ágil) e de construir
aplicações e infraestruturas resilientes.

Em síntese, as \emph{saídas DevOps} são o conjunto de categorias que não
produzem primariamente o efeito esperado de um \emph{facilitador}. Como já
detalhado nas seções \ref{secao_saidas} e \ref{secao_facilitadores_e_saidas},
foram identificadas quatro categorias que podem aparecer como saídas da adoção
de {\it DevOps}: \cat{agilidade}, \cat{resiliência}, \cat{garantia da qualidade}
e \cat{medição contínua}.

Convém destacar ainda que, em algumas situações, os potenciais gerados pela
adoção de {\it DevOps} podem não ser completamente explorados em um primeiro
momento devido a decisões de negócio. Por exemplo, um dos entrevistados citou
que a sua companhia não permitiu uma entrega contínua em ambiente de produção:

\begin{mq}
``\emph{Nós tínhamos condições e nos sentíamos seguros para publicar
continuamente também em produção, mas, no começo os gerentes ficaram meio
assustados e decidiram que a publicação deveria acontecer apenas semanalmente.}''
(P9, Gerente de TI, Brasil)
\end{mq}

\begin{mh}
\textbf{Hipótese 3:} {\it As categorias \cat{medição contínua} e \cat{garantia
da qualidade} são relacionadas à adoção de DevOps tanto como {\bf \emph{facilitadores}}
como quanto {\bf \emph{saídas}}}.
\end{mh}

Os conceitos relacionados à medição são citados como responsabilidades típicas
da equipe de operações. Ao mesmo tempo que compartilhar essa responsabilidade
entre os times reduz os silos, também é mencionado que a \cat{medição contínua}
é uma consequência necessária da adoção de {\it DevOps}. Particularmente porque a
entrega contínua de {\it software} requer maior controle, que é fornecido por
meio da utilização dos conceitos relacionados à esta categoria.
A mesma premissa é válida para a categoria \cat{garantia da qualidade}.
À primeira vista, \cat{garantia da qualidade} aparece como uma resposta ao
contexto de agilidade nas operações decorrente da adoção do {\it DevOps}. Mas,
os esforços na garantia de qualidade de produtos de {\it software} também são
responsáveis por aumentar a confiança entre as equipes de desenvolvimento e
operações, aumentando o nível da \cc.

\begin{mh}
\textbf{Hipótese 4:} {\it Não há precedência entre os facilitadores em um
processo de adoção de DevOps}.
\end{mh}

Foi identificado que o processo de adoção de {\it DevOps} pode não ter que
priorizar algum dos facilitadores. Há casos de adoção de {\it DevOps} com maior
ênfase em automação e outros com maior ênfase em garantia da qualidade. Assim
sendo, uma organização que visa adotar {\it DevOps} deve começar com os
facilitadores que parecem mais apropriados em termos das suas especificidades.
Assim, não foi encontrada nenhuma evidência de que um facilitador é mais
eficiente que outro para fomentar a \cc. \cat{Automação}, por exemplo, é a
categoria que aparece com maior frequência na análise, todavia, vários
participantes mencionaram que consideram um equívoco associar {\it DevOps} à
automação.

\begin{mq}
``\emph{Eu penso que a expansão da colaboração entre as equipes envolveu outras
coisas, não foi apenas automação. É preciso ter um alinhamento com as
necessidades de negócio. (...) Eu acho que DevOps possibilitou inclusive um
entendimento mais amplo da produção do software como um todo e a gente percebeu
exatamente que não se trata de sair automatizando tudo. (...) Então, vejo com cautela uma
suposta visão que automatizar as coisas pode ser a maneira de implementar DevOps.}''
(P7, Analista de Suporte, Brasil)
\end{mq}

\begin{mq}
``\emph{Embora atualmente a gente use automação em um número até razoável de
cenários, nós conseguimos desenvolver nossa cultura significativamente com
coisas que não envolvem automação e eu penso que você pode sim conseguir um bom
nível de DevOps com pouco ou talvez até nada de automação.}'' (P8, {\it DevOps
Engineer}, Brasil)
\end{mq}

\subsection{Um Modelo para Adoção de \emph{DevOps}}

Com base nas hipóteses H1-H4 apresentadas, foi construído
um modelo de três etapas para guiar a adoção de {\it DevOps}. A Figura \ref{modelo_adocao_devops}
representa graficamente o modelo, cujas etapas são descritas na enumeração a
seguir:

\begin{enumerate}
\item Na primeira etapa, a companhia interessada em adotar {\it DevOps} deve
compreender e disseminar que o objetivo principal é o estabelecimento de uma
\cc entre os times de desenvolvimento e operações.

\item Na segunda etapa, devem ser selecionados e desenvolvidos os facilitadores
mais adequados para o contexto da organização. Os facilitadores são meios
tipicamente utilizados para desenvolver a \cc e seus conceitos relacionados.

\item Por fim, na terceira etapa, a organização deve verificar as saídas que o
processo está produzindo, visando alinhá-las com a prática de mercado e
explorá-las de acordo com a sua necessidade.

\end{enumerate}

\figura{modelo_adocao_devops}{Modelo para Adoção de \textit{DevOps}}{modelo_adocao_devops}{width=.8\textwidth}
