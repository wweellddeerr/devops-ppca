\newcommand{\texCommand}[1]{\texttt{\textbackslash{#1}}}%

\newcommand{\exemplo}[1]{%
\vspace{\baselineskip}%
\noindent\fbox{\begin{minipage}{\textwidth}#1\end{minipage}}%
\\\vspace{\baselineskip}}%

\newcommand{\exemploVerbatim}[1]{%
\vspace{\baselineskip}%
\noindent\fbox{\begin{minipage}{\textwidth}%
#1\end{minipage}}%
\\\vspace{\baselineskip}}%

Este capítulo apresenta uma revisão dos principais trabalhos relacionados a
\textit{DevOps}, estando estruturado da seguinte forma: a seção \ref{secao_abordagem_nao_devops}
apresenta a maneira como o desenvolvimento de \textit{software} tipicamente é
organizado quando não se utiliza \textit{DevOps} como abordagem; a seção \ref{secao_historico_devops}
contém um breve histórico do surgimento do termo \textit{DevOps}; na seção
\ref{secao_definicao_devops} são apresentadas as principais definições de
\textit{DevOps} existentes na literatura; a seção \ref{secao_elementos_devops}
contém um detalhamento dos principais elementos que fazem parte das caracterizações
de \textit{DevOps} presentes na literatura.

\section{Abordagem ``Não \textit{DevOps}'' de Desenvolvimento de \textit{Software}}
\label{secao_abordagem_nao_devops}

De acordo com Davis e Daniels \cite{effective_devops}, no início o desenvolvedor
era o operador, o trabalho de construir um \textit{software} embutia todos
os aspectos relacionados ao \textit{hardware} no qual ele iria operar. Apenas
com o advento dos \textit{softwares} proprietários, no final do anos 1960,
\textit{software} e \textit{hardware} foram desacoplados, em termos de atividades
de desenvolvimento. Ainda segundo as autoras, o crescimento de popularidade da
internet gerou uma maior complexidade nos produtos de \textit{software}
requeridos para atender as necessidades das organizações. Essa maior
complexidade gerou uma proliferação de papéis e habilidades necessárias durante
a construção de \textit{software}. Grupos como qualidade, segurança, bancos de
dados tornaram-se áreas separadas durante o desenvolvimento de \textit{software}.

A questão de como o desenvolvimento de \textit{software} deve ser organizado
visando entregar soluções com maior velocidade, melhor qualidade e menor custo
é discutida há décadas nos círculos de engenharia de \textit{software}
\cite{empirical_studies_agile}. Entre as abordagens de desenvolvimento de
\textit{software} consideradas tradicionais, a mais antiga e mais amplamente
utilizada é o modelo cascata, que divide o ciclo de vida de desenvolvimento
de \textit{software} em estágios distintos e lineares
\cite{software_quality_agile}.

O modelo cascata possui algumas desvantagens, tais como inflexibilidade para
mudanças de requisitos, e um processo altamente cerimonioso, que tipicamente
despreza a natureza e o tamanho dos projetos e as características individuais
das pessoas envolvidas \cite{software_quality_agile}.

Diante da pressão por desenvolvimento acelerado de novos produtos e a
contrastante exigência de que esses produtos sejam cada vez mais confiáveis,
muitas soluções de melhorias tem sido sugeridas, desde padronização e controle
do processo de \textit{software} a um considerável número de ferramentas,
técnicas e práticas.

Os métodos ágeis de desenvolvimento de \textit{software} emergiram como
alternativa aos modelos tradicionais (p.ex.: cascata) com a proposta de
desburocratizar o processo, dispensando tudo que não seja essencial. Uma das
consequências do uso de metodologias ágeis é a redução dos silos presentes no
processo de desenvolvimento. As atividades que outrora eram executadas de
maneira linear passaram a ser unificadas em um processo iterativo e contínuo
de desenvolvimento \cite{a_decade_of_agile}.

Embora as metodologias ágeis tenham possibilitado uma aproximação das
atividades de desenvolvimento de \textit{software}, que passaram a ser
executadas de maneira iterativa e contínua, as atividades de operações têm
ficado de fora do contexto de práticas ágeis, gerando um gargalo na entrega
de valor do \textit{software} \cite{understanding_devops}.

Neste cenário, a equipe de operações tipicamente participa do processo
apenas em um último momento, que é o de disponibilizar o
\textit{software} em ambiente de produção. O time de desenvolvimento realiza
diversas iterações, produz diversas versões intermediárias dos produtos de
\textit{software} e o time de operações não tem participação ativa neste
momento. Segundo H\"uttermann
\cite{devops_for_developers}, essa abordagem gera uma barreira organizacional
e cultural que desencadeia três principais tipos de problemas: (1) cada time
(desenvolvimento e operações) defende seus próprios interesses, ao invés de
um interesse comum; (2) desenvolvimento de idiomas/vocabulários distintos pelos
dois times que trabalham em problemas isoladamente, sem a busca de um objetivo
comum; e (3) não compartilhamento de conhecimento entre os times, por falta
de confiança e receio de perda de poder ou de confrontamento.

Um outro problema típico dessa abordagem ``não \textit{DevOps}'' é a existência
de um \textit{blame game} onde, na ocorrência de um problema, um lado acusa
o outro de tê-lo causado e o foco deixa de ser a resolução do problema em si
e passa a ser apenas em evitar a responsabilização \cite{toward_unified_model}.

\section{A História de \textit{DevOps}}\label{secao_historico_devops}

O termo \textit{DevOps} foi cunhado em 2009 durante a organização da
conferência \textit{DevOpsDays} na Bélgica \cite{devops_for_developers}.

Todavia, as origens de \textit{DevOps} remetem para antes mesmo da existência
do termo. Primeiro, em 2008, foi publicado o artigo ``\textit{Agile
Infrastructure and Operations: How Infra-gile are You?}''
\cite{agile_infra_operations} no qual
Patrick Debois propôs analogias entre o sucesso com o qual os métodos ágeis
se adaptam às constantes mudanças das necessidades de negócio e a maneira como
a infraestrutura também poderia responder de maneira ágil à essas mudanças.
Desafiando um suposto oxímoro no qual agilidade e infraestrutura não
encaixam-se, o autor relata a experiência bem sucedida da aplicação de técnicas
ágeis na construção e gerenciamento da infraestrutura de três projetos reais.
Na mesma conferência onde Debois apresentou suas ideias de aplicar técnicas
ágeis a contextos de infraestrutura, Andrew Shafer propôs uma sessão sobre
infraestrutura ágil, momento em os dois começaram a interagir a respeito do
tema \cite{effective_devops}.

Pouco tempo depois, já em 2009, Hammond e Allspaw apresentaram ``\textit{10+ Deploys
per Day: Dev and Ops Cooperation at Flickr}'' \cite{flickr} em uma conferência
chamada \textit{Velocity}, ilustrando não somente o alto número de
\textit{deploys} realizados na \textit{Flickr}, mas o poder revolucionário que
a colaboração entre os times de desenvolvimento e operações possibilitou lá.

E foi neste momento que o termo \textit{DevOps} finalmente surgiu. Segundo
J. Davis e K. Daniels \cite{effective_devops}, em um \textit{tweet} de Andrew
Shaffer convidando para a próxima edição da conferência \textit{Velocity}, que
iria se concentrar em infraestrutura ágil, continha a \textit{hashtag}
``\textit{\#devops}''. Ao ler este \textit{tweet}, Patrick Debois lamentou que
não iria poder estar presente, pois estaria na Bélgica. Shaffer então teria
sugerido a Debois que organizasse sua própria edição da conferência
\textit{Velocity}. Foi quando Debois, utilizando o texto da \textit{hashtag}
organizou a primeira edição do evento \textit{DevOpsDays} na Bélgica,
iniciando a disseminação do termo \textit{DevOps}.

\section{O que é \textit{DevOps}?}

\textit{DevOps} emergiu na indústria de desenvolvimento de \textit{software}
sem uma clara definição ou delimitação teórica.

Em \textit{What is DevOps?} \cite{what_is_devops}, Loukides tenta ilustrar o
conceito de \textit{DevOps} por meio da diferenciação entre o passado e o
presente do trabalho de operações de \acrshort{TI}. O texto ilustra os novos
desafios existentes para os times de operações e destaca a importância da
existência de colaboração com os times de desenvolvimento para superá-los.
Todavia, não é apresentada uma definição final de \textit{DevOps}.

H\"uttermann afirma que ``\textit{DevOps} descreve práticas que simplificam o
processo de entrega de \textit{software}, enfatizando o aprendizado por meio da
transmissão contínua de \textit{feedback} da produção para o desenvolvimento''.
Quando comparada com outras, essa definição de \textit{DevOps} como um conjunto
de práticas relacionadas à entrega de \textit{software}, demonstra carecer
de uma maior ênfase no aspecto cultural. Por exemplo, de acordo com
Walls \cite{building_devops_culture}, o termo foi introduzido justamente para
definir uma cultura organizacional que as organizações podem buscar. De acordo
com ele, \textit{DevOps} é ``um movimento cultural combinado com várias
práticas relacionadas a \textit{software} que possibilitam um desenvolvimento
rápido dos produtos''. O autor descreve quatro características culturais chave em um
contexto de \textit{DevOps}: comunicação aberta, incentivo e alinhamento de
responsabilidades, respeito e confiança.

O aspecto cultural de \textit{DevOps} é também ressaltado por Davis e Daniels
\cite{effective_devops} que o definem como um movimento cultural que altera
como os indivíduos pensam sobre seu trabalho, valoriza a diversidade, apóia
processos que aceleram o ritmo da obtenção de valor e medem o efeito da mudança
social e técnica.

Como ressaltado por H\"uttermann \cite{devops_for_developers}, \textit{DevOps}
é um termo multifacetado e de difícil definição. A seguir, são apresentadas
três definições de \textit{DevOps} obtidas em estudos que dedicaram-se a
investigar essa questão em específico.

Por meio de uma revisão de literatura, o trabalho de J. Smeds et al.
\cite{devops_a_definition} define \textit{DevOps} como um conjunto de
capacidades do processo de engenharia de \textit{software} que é sustentado
por certos facilitadores culturais e tecnológicos. Os autores enfatizam
que as \emph{capacidades} definem processos que uma organização deve ser capaz
de executar, enquanto que os \emph{facilitadores} possibilitam uma maneira
fluente, flexível e eficiente de se trabalhar.

Dyck et al \cite{towards_definitions}, por sua vez, definem \textit{DevOps}
como uma abordagem organizacional que enfatiza a empatia e a colaboração
interfuncional dentro e entre as equipes - especialmente o desenvolvimento e
operações de TI - em organizações de desenvolvimento de \textit{software}, para
operar sistemas resilientes e acelerar a entrega de mudanças.

Por fim, França et al. \cite{characterizing_devops} definem \textit{DevOps}
como um neologismo que representa um movimento de profissionais de tecnologia
da informação e comunicação buscando uma diferente atitude no que se refere à
entrega de \textit{software}, através da colaboração entre o desenvolvimento de
sistemas de \textit{software} e as funções de operações, com base em um conjunto
de práticas, tais como cultura, automação, medição e compartilhamento.

\section{Elementos de \textit{DevOps}}

As caracterizações de \textit{DevOps} tipicamente utilizam enumerações e
detalhamentos de elementos relacionados. Nesta seção são apresentados os
principais elementos de \textit{DevOps} identificados na literatura relacionada.
O termo ``elemento'' é aqui usado para unificar o vocabulário da literatura
relacionada no qual constam termos distintos, como ``conceitos'', ``dimensões''
e ``princípios'' de \textit{DevOps}.

Em 2010, John Willis propôs uma caracterização de \textit{DevOps}
\cite{what_devops_means_2010} que se tornou influente. De acordo com esta
proposta, \textit{DevOps} é constituído de quatro elementos cujas iniciais
compõem o acrônimo \acrshort{CAMS}: cultura, automação, medição e compartilhamento
(\textit{sharing}, em inglês). Esse modelo é conhecido como \acrshort{CAMS}
\textit{Framework}.

Uma revisão de literatura de Erich et al. \cite{cooperation_dev_ops_esem_2014}
identificou oito conceitos principais relacionados a \textit{DevOps}, os quatro
presentes no \acrshort{CAMS} \textit{framework} acrescidos de (1) serviços,
(2) garantia da qualidade, (3) estruturas e (4) padrões.

Posteriormente, Lwakatare et al. \cite{dimensions_of_devops} apresentaram uma
caracterização de \textit{DevOps} como um \textit{framework} conceitual que
contém quatro dimensões: colaboração, automação, medição e
\textbf{monitoramento}, este último sendo o único elemento novo, em comparação
com os estudos anteriores. Para identificar as dimensões de \textit{DevOps} foi
utilizada uma revisão de literatura combinada com entrevistas a quatro
praticantes de três companhias de um mesmo grupo empresarial. Uma extensão
deste trabalho foi publicada posteriormente em \cite{extending_dimensions} e
acrescentou uma nova dimensão chamada de colaboração. Nessa extensão foi
conduzida uma revisão de literatura multivocal, utilizando dados da literatura
cinza.

Em \emph{Characterizing DevOps by Hearing Multiple Voices}, França et al.
\cite{characterizing_devops} organizam os elementos de \textit{DevOps} em seis
categorias de princípios: (1) automação, (2) garantia da qualidade,
(3) compartilhamento, (4) medição, (5) \textbf{aspectos sociais} e (6)
\textbf{\textit{Leanness}}, os dois últimos são elementos novos em relação
aos já apresentados aqui. Para chegar a estes resultados os autores conduziram
uma revisão de literatura multivocal, e os dados foram analisados usando
algumas técnicas de codificação de \textit{\acrfull{GT}}.

Por fim, por meio de uma nova revisão sistemática de literatura, Erich et al.
\cite{qualitative_devops_journalsw_17} identificaram sete agrupamentos de
conceitos presentes na literatura sobre \textit{DevOps}: (1) cultura de
colaboração, (2) automação (3) medição; (4) compartilhamento, (5) serviços,
(6) Garantia da qualidade e (7) \textbf{Governança}. Apenas \emph{governança}
não aparece em nenhum dos trabalhos apontados previamente.

É possível identificar elementos de \textit{DevOps} que aparecem em vários
trabalhos bem como outros que aparecem apenas em um deles, a tabela
\ref{tabela_elementos_devops} lista os elementos identificados e aponta em
quais trabalhos cada um deles aparece. Após comparação dos estudos, foi
identificado que os elementos \emph{cultura}, \emph{cultura de colaboração},
\emph{colaboração} e \emph{aspectos sociais} são similares e, para efeitos de
comparação, foi designiada uma única coluna na tabela, a coluna 2 (cultura de
colaboração).

\begin{table}[hb!]
\centering
\caption{Elementos de \textit{DevOps}}
\label{tabela_elementos_devops}
\begin{tabular}{|p{0.3cm}|p{2.7cm}|p{2.1cm}|p{2cm}|p{2cm}|p{2cm}|p{2cm}|}
\hline

        & Elemento               & \acrshort{CAMS} \textit{Framework} \cite{what_devops_means_2010} & Erich et al. \cite{cooperation_dev_ops_esem_2014} & Lwakatare et al. \cite{extending_dimensions} & França et al. \cite{characterizing_devops} & Erich et al. \cite{qualitative_devops_journalsw_17} \\
\end{tabular}

\begin{tabular}{|p{0.3cm}|p{2.7cm}|m{2.1cm}|m{2cm}|m{2cm}|m{2cm}|m{2cm}|}

\hline

1  & Cultura de Colaboração & X & X & X & X & X \\

\hline

2  & Automação              & X & X & X & X & X \\

\hline

3  & Medição                & X & X & X & X & X \\

\hline

4  & Compartilha-\newline mento & X & X &    & X & X \\

\hline

5  & Serviços               &        & X &        &        & X \\

\hline

6  & Garantia da Qualidade  &        & X &        & X & X \\

\hline

7  & Estruturas             &        & X &        &        &        \\

\hline

8  & Padrões                &        & X &        &        &        \\

\hline

9  & Monitoramento          &        &        & X &        &        \\

\hline

10 & \textit{Leanness}      &        &        &        & X &        \\

\hline
11 & Governança             &        &        &        &        & X \\

\hline

\end{tabular}
\end{table}

A tabela \ref{tabela_elementos_devops} reforça a relevância dos elementos do
\acrshort{CAMS} \textit{framework}. Apenas \emph{compartilhamento} deixa de
aparecer em um dos estudos citados, os demais aparecem em todos. Com relação
aos elementos que não fazem parte do \textit{framework}, \emph{garantia
da qualidade} e \emph{serviços} aparecem em três e dois estudos, respectivamente,
enquanto que os demais aparecem apenas em um único estudo.

\subsection{Cultura de Colaboração}

No \acrshort{CAMS} \textit{framework} \cite{what_devops_means_2010}, este
elemento é chamado apenas de \emph{cultura} e é apresentado de maneira sucinta
como um direcionamento de que pessoas e processos devem vir primeiro, sob pena
de os demais esforços relacionados a \textit{DevOps} serem infrutíferos.

Já no trabalho de Lwakatare et al. \cite{dimensions_of_devops,extending_dimensions},
este elemento é apresentado primeiramente apenas como \emph{colaboração}, e em
um segundo momento como duas dimensões distintas: \emph{cultura} e
\emph{colaboração}. Ao tratar do elemento como \emph{colaboração}, os autores
mencionam especificamente que \textit{DevOps} envolve uma \emph{cultura de colaboração}.
Neste trabalho não é apresentada uma caracterização do que seria essa \emph{cultura
de colaboração}, apenas é posto que ela é reforçada por meio de
compartilhamento de informações, ampliação de qualificações e transferência de
responsabilidades entre as duas equipes, além de incutir um senso de
responsabilidade compartilhada. Após o acréscimo de \emph{cultura} como elemento
\textit{DevOps}, não é também apresentada uma diferenciação entre essas duas
dimensões que são apresentadas separadamente. Sobre \emph{cultura}, apenas é
dito que muitas das práticas \textit{DevOps} envolvem uma mudança de cultura e
mentalidade para incentivar a empatia, o apoio mútuo e um bom ambiente de
trabalho para os envolvidos no desenvolvimento de \textit{software} e
nos processos de entrega.

No trabalho de França et al. \cite{characterizing_devops}, este elemento é
apresentado como um princípio denominado \emph{aspectos sociais}. De acordo com os
autores, apesar de existirem diversos princípios técnicos, muitas das
características de \textit{DevOps} estão associadas a aspectos sociais entre as
equipes de desenvolvimento de \textit{software} e operações. A \textbf{cultura
\textit{DevOps}} reconhece a confiança como uma característica relevante para
influenciar a mudança organizacional exigida por \textit{DevOps}.

Erich et al. \cite{qualitative_devops_journalsw_17} limitam-se a afirmar que
organizações que praticam \textit{DevOps} tentam remover a barreira cultural
entre o pessoal de desenvolvimento e de operações.

Em síntese, o elemento \emph{cultura de colaboração}, consiste na existência de
 uma maneira de se organizar as atividades de construção
de \textit{software} que reforça a colaboração entre os times de desenvolvimento
e operações, buscando-se evitar exatamente que se organizem em silos com
comunicação burocrática, sem confiança mútua e buscando objetivos distintos.

\subsubsection{Práticas Relacionadas}

\subsection{Automação}

\subsection{Medição}

\subsection{Compartilhamento}

\section{Impedimentos de \textit{DevOps}}

\section{Adoção de \textit{DevOps}}
