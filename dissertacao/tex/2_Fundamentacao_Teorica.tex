\newcommand{\texCommand}[1]{\texttt{\textbackslash{#1}}}%

\newcommand{\exemplo}[1]{%
\vspace{\baselineskip}%
\noindent\fbox{\begin{minipage}{\textwidth}#1\end{minipage}}%
\\\vspace{\baselineskip}}%

\newcommand{\exemploVerbatim}[1]{%
\vspace{\baselineskip}%
\noindent\fbox{\begin{minipage}{\textwidth}%
#1\end{minipage}}%
\\\vspace{\baselineskip}}%

Este capítulo apresenta uma revisão dos principais trabalhos relacionados a
\textit{DevOps} estando estruturado da seguinte forma:

\section{Abordagem ``Não \textit{DevOps}'' de Desenvolvimento de \textit{Software}}

A questão de como o desenvolvimento de \textit{software} deve ser organizado
visando entregar soluções com maior velocidade, melhor qualidade e menor custo
é discutida há décadas nos círculos de engenharia de \textit{software}
\cite{empirical_studies_agile}. Entre as abordagens de desenvolvimento de
\textit{software} consideradas tradicionais, a mais antiga e mais amplamente
utilizada é o modelo cascata, que divide o ciclo de vida de desenvolvimento
de \textit{software} em estágios distintos e lineares
\cite{software_quality_agile}.

O modelo cascata possui algumas desvantagens, tais como inflexibilidade para
mudanças de requisitos, e um processo altamente cerimonioso, que tipicamente
despreza a natureza e o tamanho dos projetos e as características individuais
das pessoas envolvidas \cite{software_quality_agile}.

Diante da pressão por desenvolvimento acelerado de novos produtos e a
contrastante exigência de que esses produtos sejam cada vez mais confiáveis,
muitas soluções de melhorias tem sido sugeridas, desde padronização e controle
do processo de \textit{software} a um considerável número de ferramentas,
técnicas e práticas.

Os métodos ágeis de desenvolvimento de \textit{software} emergiram como
alternativa aos modelos tradicionais (p.ex.: cascata) com a proposta de
desburocratizar o processo, dispensando tudo que não seja essencial. Uma das
consequências do uso de metodologias ágeis é a redução dos silos presentes no
processo de desenvolvimento. As atividades que outrora eram executadas de
maneira linear passaram a ser unificadas em um processo iterativo e contínuo
de desenvolvimento \cite{a_decade_of_agile}.

Embora as metodologias ágeis tenham possibilitado uma aproximação das
atividades de desenvolvimento de \textit{software}, que passaram a ser
executadas de maneira iterativa e contínua, as atividades de operações têm
ficado de fora do contexto de práticas ágeis, gerando um gargalo na entrega
de valor do \textit{software} \cite{understanding_devops}.

Neste cenário, a equipe de operações tipicamente participa do processo
apenas em um último momento, que é o de disponibilizar o
\textit{software} em ambiente de produção. O time de desenvolvimento realiza
diversas iterações, produz diversas versões intermediárias dos produtos de
\textit{software} e o time de operações não tem participação ativa neste
momento. Segundo H\"uttermann
\cite{devops_for_developers}, essa abordagem gera uma barreira organizacional
e cultural que desencadeia três principais tipos de problemas: (1) cada time
(desenvolvimento e operações) defende seus próprios interesses, ao invés de
um interesse comum; (2) desenvolvimento de idiomas/vocabulários distintos pelos
dois times que trabalham em problemas isoladamente, sem a busca de um objetivo
comum; e (3) não compartilhamento de conhecimento entre os times, por falta
de confiança e receio de perda de poder ou de confrontamento.

Um outro problema típico dessa abordagem ``não \textit{DevOps}'' é a existência
de um \textit{blame game} onde, na ocorrência de um problema, um lado acusa
o outro de tê-lo causado e o foco deixa de ser a resolução do problema em si
e passa a ser apenas em evitar a responsabilização \cite{toward_unified_model}.

\section{A História de \textit{DevOps}}

O termo \textit{DevOps} foi cunhado em 2009 durante a organização da
conferência \textit{DevOpsDays} na Bélgica \cite{devops_for_developers}.

Todavia, as origens de \textit{DevOps} remetem para antes mesmo da existência
do termo. Primeiro, em 2008, foi publicado o artigo ``\textit{Agile
Infrastructure and Operations: How Infra-gile are You?}''
\cite{agile_infra_operations} no qual
Patrick Debois propôs analogias entre o sucesso com o qual os métodos ágeis
se adaptam às constantes mudanças das necessidades de negócio e a maneira como
a infraestrutura também poderia responder de maneira ágil à essas mudanças.
Desafiando um suposto oxímoro no qual agilidade e infraestrutura não
encaixam-se, o autor relata a experiência bem sucedida da aplicação de técnicas
ágeis na construção e gerenciamento da infraestrutura de três projetos reais.
Na mesma conferência onde Debois apresentou suas ideias de aplicar técnicas
ágeis a contextos de infraestrutura, Andrew Shafer propôs uma sessão sobre
infraestrutura ágil, momento em os dois começaram a interagir a respeito do
tema \cite{effective_devops}.

Pouco tempo depois, já em 2009, Hammond e Allspaw apresentaram ``\textit{10+ Deploys
per Day: Dev and Ops Cooperation at Flickr}'' \cite{flickr} em uma conferência
chamada \textit{Velocity}, ilustrando não somente o alto número de
\textit{deploys} realizados na \textit{Flickr}, mas o poder revolucionário que
a colaboração entre os times de desenvolvimento e operações possibilitou lá.

E foi neste momento que o termo \textit{DevOps} finalmente surgiu. Segundo
J. Davis e K. Daniels \cite{effective_devops}, em um \textit{tweet} de Andrew
Shaffer convidando para a próxima edição da conferência \textit{Velocity}, que
iria se concentrar em infraestrutura ágil, continha a \textit{hashtag}
``\textit{\#devops}''. Ao ler este \textit{tweet}, Patrick Debois lamentou que
não iria poder estar presente, pois estaria na Bélgica. Shaffer então teria
sugerido a Debois que organizasse sua própria edição da conferência
\textit{Velocity}. Foi quando Debois, utilizando o texto da \textit{hashtag}
organizou a primeira edição do evento \textit{DevOpsDays} na Bélgica,
iniciando a disseminação do termo \textit{DevOps}.

\section{O que é \textit{DevOps}?}

\textit{DevOps} emergiu na indústria de desenvolvimento de \textit{software}
sem uma clara definição ou delimitação teórica.

Em \textit{What is DevOps?} \cite{what_is_devops}, Loukides tenta ilustrar o
conceito de \textit{DevOps} por meio da diferenciação entre o passado e o
presente do trabalho de operações de \acrshort{TI}. O texto ilustra os novos
desafios existentes para os times de operações e destaca a importância da
existência de colaboração com os times de desenvolvimento para superá-los.
Todavia, não é apresentada uma definição final de \textit{DevOps}.

H\"uttermann \cite{devops_for_developers} salienta a dificuldade que é definir
\textit{DevOps} e ressalta que o termo é multifacetado. O autor afirma que
``\textit{DevOps} descreve práticas que simplificam o processo de entrega
de \textit{software}, enfatizando o aprendizado por meio da transmissão contínua de
\textit{feedback} da produção para o desenvolvimento''. Quando comparada com
outras, essa definição de \textit{DevOps} como um conjunto de práticas
relacionadas à entrega de \textit{software}, demonstra carecer
de uma maior ênfase no aspecto cultural. Por exemplo, de acordo com
Walls \cite{building_devops_culture}, o termo foi introduzido justamente para
definir uma cultura organizacional que as organizações podem buscar. De acordo
com ele, \textit{DevOps} é ``um movimento cultural combinado com várias
práticas relacionadas a \textit{software} que possibilitam um desenvolvimento
rápido dos produtos''. O autor descreve quatro características culturais chave em um
contexto de \textit{DevOps}: comunicação aberta, incentivo e alinhamento de
responsabilidades, respeito e confiança.

Tão intrincado e difícil é definir \textit{DevOps} que três estudos recentes
investigaram especificamente essa questão.

Por meio de uma revisão de literatura, o trabalho de J. Smeds et al.
\cite{devops_a_definition} define \textit{DevOps} como um conjunto de
capacidades do processo de engenharia de \textit{software} que é sustentado
por certos facilitadores culturais e tecnológicos. Os autores enfatizam
que as \emph{capacidades} definem processos que uma organização deve ser capaz
de executar, enquanto que os \emph{facilitadores} possibilitam uma maneira
fluente, flexível e eficiente de se trabalhar.

Dyck et al \cite{towards_definitions}, por sua vez, definem \textit{DevOps}
como uma abordagem organizacional que enfatiza a empatia e a colaboração
interfuncional dentro e entre as equipes - especialmente o desenvolvimento e
operações de TI - em organizações de desenvolvimento de \textit{software}, para operar
sistemas resilientes e acelerar a entrega de mudanças.

Por fim, França et al. \cite{characterizing_devops} definem \textit{DevOps}
como um neologismo que representa um movimento de profissionais de tecnologia
da informação e comunicação buscando uma diferente atitude no que se refere à
entrega de \textit{software}, através da colaboração entre o desenvolvimento de
sistemas de \textit{software} e as funções de operações, com base em um conjunto
de práticas, tais como cultura, automação, medição e compartilhamento.

\section{Elementos de \textit{DevOps}}

As caracterizações de \textit{DevOps} tipicamente utilizam enumerações e
detalhamentos de elementos relacionados. Esses elementos podem ser práticas,
técnicas, princípios ou dimensões de \textit{DevOps}. Aqui são apresentados
os principais elementos de \textit{DevOps} identificados na literatura
relacionada.

Em 2010, John Willis propôs uma caracterização de \textit{DevOps} \cite{}
que se tornou influente. De acordo com esta proposta, \textit{DevOps} é
constituído de quatro elementos cujas iniciais compõem o acrônimo CAMS: cultura,
automação, medição e compartilhamento (sharing, em inglês). Esse modelo é
conhecido como CAMS \textit{Framework}.

Na revisão de literatura de fulano, foram acrescentados tais novos elementos.

Posteriormente Lwakatare et al.

Characterizing DevOps

A qualitative study of DevOps

\subsection{Cultura}
Willis é bastante sucinto com realação a cultura: pessoas e processos primeiro.

\subsection{Automação}

\subsection{Medição}

\subsection{Compartilhamento}

\section{Impedimentos de \textit{DevOps}}

\section{Adoção de \textit{DevOps}}

\section{\textit{Grounded Theory}}


%%%%%%%%%%%%%%%%%%%%%%%%%%%%%%%%%%%%%%%%%%%%%%%%%%%%%%%%%%%%%%%%%%%%%%%%%%%%%%%%
%%%%%%%%%%%%%%%%%%%%%%%%%%%%%%%%%%%%%%%%%%%%%%%%%%%%%%%%%%%%%%%%%%%%%%%%%%%%%%%%
%%%%%%%%%%%%%%%%%%%%%%%%%%%%%%%%%%%%%%%%%%%%%%%%%%%%%%%%%%%%%%%%%%%%%%%%%%%%%%%%


%%%%%%%%%%%%%%%%%%%%%%%%%%%%%%%%%%%%%%%%%%%%%%%%%%%%%%%%%%%%%%%%%%%%%%%%%%%%%%%%
%%%%%%%%%%%%%%%%%%%%%%%%%%%%%%%%%%%%%%%%%%%%%%%%%%%%%%%%%%%%%%%%%%%%%%%%%%%%%%%%
%%%%%%%%%%%%%%%%%%%%%%%%%%%%%%%%%%%%%%%%%%%%%%%%%%%%%%%%%%%%%%%%%%%%%%%%%%%%%%%%
