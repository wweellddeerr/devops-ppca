\newcommand{\texCommand}[1]{\texttt{\textbackslash{#1}}}%

\newcommand{\exemplo}[1]{%
\vspace{\baselineskip}%
\noindent\fbox{\begin{minipage}{\textwidth}#1\end{minipage}}%
\\\vspace{\baselineskip}}%

\newcommand{\exemploVerbatim}[1]{%
\vspace{\baselineskip}%
\noindent\fbox{\begin{minipage}{\textwidth}%
#1\end{minipage}}%
\\\vspace{\baselineskip}}%

Este capítulo apresenta uma revisão dos principais tópicos relacionados a
\textit{DevOps}, estando estruturado da seguinte forma: a seção \ref{secao_abordagem_nao_devops}
apresenta a maneira como o desenvolvimento de \textit{software} tipicamente é
organizado quando não se utiliza \textit{DevOps} como abordagem; a seção \ref{secao_historico_devops}
contém um breve histórico do surgimento do termo \textit{DevOps}; na seção
\ref{secao_definicao_devops} são apresentadas as principais definições de
\textit{DevOps} existentes na literatura. A seção \ref{secao_elementos_devops}
contém um detalhamento dos principais elementos que fazem parte das caracterizações
de \textit{DevOps}. E, por fim, na seção \ref{secao_desafios} são apresentados
desafios relacionados a \textit{DevOps}

\section{Abordagem ``Não \textit{DevOps}'' de Desenvolvimento de \textit{Software}}
\label{secao_abordagem_nao_devops}

%De acordo com Davis e Daniels \cite{effective_devops}, no início o desenvolvedor
%era o operador, o trabalho de construir um \textit{software} embutia todos
%os aspectos relacionados ao \textit{hardware} no qual ele iria operar. Apenas
%com o advento dos \textit{softwares} proprietários, no final do anos 1960,
%\textit{software} e \textit{hardware} foram desacoplados, em termos de atividades
%de desenvolvimento. Ainda segundo as autoras, o crescimento de popularidade da
%internet gerou uma maior complexidade nos produtos de \textit{software}
%requeridos para atender às necessidades das organizações. Essa maior
%complexidade gerou uma proliferação dos papéis existentes e das habilidades
%necessárias durante a construção de \textit{software}. Grupos como qualidade,
%requisitos, segurança e bancos de dados tornaram-se áreas dissociadas durante o
%desenvolvimento de \textit{software}.

A questão de como o desenvolvimento de \textit{software} deve ser
organizado visando entregar soluções com maior velocidade, melhor qualidade e
menor custo vem sendo discutida ao longo de décadas nos círculos de engenharia
de \textit{software} \cite{empirical_studies_agile}. Entre as abordagens de
desenvolvimento de \textit{software} consideradas tradicionais, a mais antiga e
que foi mais amplamente utilizada é o modelo cascata, que divide o ciclo de
vida de desenvolvimento de \textit{software} em estágios distintos e lineares
e que possui algumas desvantagens, tais como inflexibilidade para
mudanças de requisitos, e um processo altamente cerimonioso, que tipicamente
despreza a natureza e o tamanho dos projetos e as características individuais
das pessoas envolvidas \cite{software_quality_agile}.

Diante da pressão por desenvolvimento acelerado de novos produtos e a
contrastante exigência de que esses produtos sejam cada vez mais confiáveis,
as organizações têm buscado novas maneiras de melhorar seu processo de
desenvolvimento de \textit{software} para acompanhar as demandas de negócios e
de mercado.

Entre as formas de se trabalhar para desenvolver \textit{software} em ritmo
acelerado e de maneira confiável, os métodos ágeis de desenvolvimento de
\textit{software} emergiram como alternativa aos modelos tradicionais
(p.ex.: cascata) com a proposta de desburocratizar o processo, dispensando tudo
que não seja essencial. Uma das consequências do uso de metodologias ágeis é a
redução dos silos presentes no processo de desenvolvimento. As atividades que
outrora eram executadas de maneira linear passaram a ser unificadas em um
processo iterativo e contínuo de desenvolvimento \cite{a_decade_of_agile}.

Embora as metodologias ágeis tenham possibilitado uma aproximação das
atividades de desenvolvimento de \textit{software}, que passaram a ser
executadas de maneira iterativa e contínua, as atividades de operações têm
ficado de fora do contexto de práticas ágeis, gerando um gargalo na entrega
de valor do \textit{software} \cite{understanding_devops}.

Neste cenário, a equipe de operações tipicamente participa do processo
apenas em um último momento, que é o de disponibilizar o
\textit{software} em ambiente de produção. O time de desenvolvimento realiza
diversas iterações, produz diversas versões intermediárias dos produtos de
\textit{software}, e o time de operações não tem participação ativa neste
momento. Segundo H\"uttermann
\cite{devops_for_developers}, essa abordagem gera uma barreira organizacional
e cultural que desencadeia três principais tipos de problemas: (1) cada time
(desenvolvimento e operações) defende seus próprios interesses, ao invés de
um interesse comum; (2) desenvolvimento de idiomas/vocabulários distintos pelos
dois times que trabalham em problemas isoladamente, sem a busca de um objetivo
comum; e (3) não compartilhamento de conhecimento entre os times, por falta
de confiança e receio de perda de poder ou de confrontamento.

Um outro problema típico dessa abordagem ``não \textit{DevOps}'' é a existência
de um \textit{blame game} onde, na ocorrência de um problema, um lado acusa
o outro de tê-lo causado e o foco deixa de ser a resolução do problema em si
e passa a ser apenas em evitar a responsabilização \cite{toward_unified_model}.

\section{A História de \textit{DevOps}}\label{secao_historico_devops}

O termo \textit{DevOps} foi cunhado em 2009 durante a organização da
conferência \textit{DevOpsDays}\footnote{https://www.devopsdays.org/} na
Bélgica \cite{devops_for_developers}.

Todavia, as origens de \textit{DevOps} remetem para antes mesmo da existência
do termo. Primeiro, em 2008, foi publicado o artigo ``\textit{Agile
Infrastructure and Operations: How Infra-gile are You?}''
\cite{agile_infra_operations} no qual
Patrick Debois propôs analogias entre o sucesso com o qual os métodos ágeis
se adaptam às constantes mudanças das necessidades de negócio e a maneira como
a infraestrutura também poderia responder de maneira ágil à essas mudanças.
Desafiando um suposto oxímoro no qual agilidade e infraestrutura não
encaixam-se, o autor relata experiências bem sucedidas da aplicação de técnicas
ágeis na construção e gerenciamento da infraestrutura de três projetos reais.
Na mesma conferência onde Debois apresentou suas ideias de aplicar técnicas
ágeis a contextos de infraestrutura, Andrew Shafer propôs uma sessão sobre
infraestrutura ágil, momento em que os dois começaram a interagir a respeito do
tema \cite{effective_devops}.

Pouco tempo depois, já em 2009, Hammond e Allspaw apresentaram ``\textit{10+ Deploys
per Day: Dev and Ops Cooperation at Flickr}'' \cite{flickr} em uma conferência
chamada \textit{Velocity}, ilustrando não somente o alto número de
\textit{deploys} realizados na \textit{Flickr}, mas o poder revolucionário que
a colaboração entre os times de desenvolvimento e operações possibilitou lá.

E foi neste momento que o termo \textit{DevOps} finalmente surgiu. Segundo
J. Davis e K. Daniels \cite{effective_devops}, em um \textit{tweet} de Andrew
Shaffer convidando para a próxima edição da conferência \textit{Velocity}, que
iria se concentrar em infraestrutura ágil, continha a \textit{hashtag}
``\textit{\#devops}''. Ao ler este \textit{tweet}, Patrick Debois lamentou que
não iria poder estar presente, pois estaria na Bélgica. Shaffer então teria
sugerido a Debois que organizasse sua própria edição da conferência
\textit{Velocity}. Foi quando Debois, utilizando o texto da \textit{hashtag},
organizou a primeira edição do evento \textit{DevOpsDays} na Bélgica,
iniciando a disseminação do termo \textit{DevOps}.

\section{O que é \textit{DevOps}?}\label{secao_definicao_devops}

\textit{DevOps} emergiu na indústria de desenvolvimento de \textit{software}
sem uma clara definição ou delimitação teórica.

Em \textit{What is DevOps?} \cite{what_is_devops}, Loukides tenta ilustrar o
conceito de \textit{DevOps} por meio da diferenciação entre o passado e o
presente do trabalho de operações de \acrshort{TI}. O texto ilustra os novos
desafios existentes para os times de operações e destaca a importância da
existência de colaboração com os times de desenvolvimento para superá-los.
Todavia, não é apresentada uma definição final de \textit{DevOps}.

H\"uttermann afirma que ``\textit{DevOps} descreve práticas que simplificam o
processo de entrega de \textit{software}, enfatizando o aprendizado por meio da
transmissão contínua de \textit{feedback} da produção para o desenvolvimento''.
Quando comparada com outras, essa definição de \textit{DevOps} como um conjunto
de práticas relacionadas à entrega de \textit{software}, demonstra carecer
de uma maior ênfase no aspecto cultural. Por exemplo, de acordo com
Walls \cite{building_devops_culture}, o termo foi introduzido justamente para
definir uma cultura organizacional que as organizações podem buscar. De acordo
com ele, \textit{DevOps} é ``um movimento cultural combinado com várias
práticas relacionadas a \textit{software} que possibilitam um desenvolvimento
rápido dos produtos''. O autor descreve quatro características culturais chave em um
contexto de \textit{DevOps}: comunicação aberta, incentivo e alinhamento de
responsabilidades, respeito e confiança.

O aspecto cultural de \textit{DevOps} é também ressaltado por Davis e Daniels
\cite{effective_devops} que o definem como um movimento cultural que altera
como os indivíduos pensam sobre seu trabalho, valoriza a diversidade, apóia
processos que aceleram o ritmo da obtenção de valor e medem o efeito da mudança
social e técnica.

Como ressaltado por H\"uttermann \cite{devops_for_developers}, \textit{DevOps}
é um termo multifacetado e de difícil definição. A seguir, são apresentadas
três definições de \textit{DevOps} obtidas em estudos que dedicaram-se a
investigar essa questão em específico.

Por meio de uma revisão de literatura, o trabalho de J. Smeds et al.
\cite{devops_a_definition} define \textit{DevOps} como um conjunto de
capacidades do processo de engenharia de \textit{software} que é sustentado
por certos facilitadores culturais e tecnológicos. Os autores enfatizam
que as \emph{capacidades} definem processos que uma organização deve ser capaz
de executar, enquanto que os \emph{facilitadores} possibilitam uma maneira
fluente, flexível e eficiente de se trabalhar.

Dyck et al \cite{towards_definitions}, por sua vez, definem \textit{DevOps}
como uma abordagem organizacional que enfatiza a empatia e a colaboração
interfuncional dentro e entre as equipes - especialmente o desenvolvimento e
operações de TI - em organizações de desenvolvimento de \textit{software}, para
operar sistemas resilientes e acelerar a entrega de mudanças.

Por fim, França et al. \cite{characterizing_devops} definem \textit{DevOps}
como um neologismo que representa um movimento de profissionais de tecnologia
da informação e comunicação buscando uma diferente atitude no que se refere à
entrega de \textit{software}, através da colaboração entre o desenvolvimento de
sistemas de \textit{software} e as funções de operações, com base em um conjunto
de práticas, tais como cultura, automação, medição e compartilhamento.

\section{Elementos de \textit{DevOps}}\label{secao_elementos_devops}

As caracterizações de \textit{DevOps} tipicamente utilizam enumerações e
detalhamentos de elementos relacionados. Nesta seção são apresentados os
principais elementos de \textit{DevOps} identificados na literatura relacionada.
O termo ``elemento'' é aqui usado para unificar o vocabulário da literatura
relacionada no qual constam termos distintos, como ``conceitos'', ``dimensões''
e ``princípios'' de \textit{DevOps}.

Em 2010, John Willis propôs uma caracterização de \textit{DevOps}
\cite{what_devops_means_2010} que se tornou influente. De acordo com esta
proposta, \textit{DevOps} é constituído de quatro elementos cujas iniciais
compõem o acrônimo \acrshort{CAMS}: cultura, automação, medição e compartilhamento
(\textit{sharing}, em inglês). Esse modelo é conhecido como \acrshort{CAMS}
\textit{Framework}.

O trabalho de Hamunen \cite{challenges_in_adopting_devops}, após
investigar quais são os componentes-chave de {\it DevOps}, identificou que
diversas fontes adicionaram {\it Lean} como novo componente, culminando
na formação do que foi chamado de CALMS \textit{Framework}.

Uma revisão de literatura de Erich et al. \cite{cooperation_dev_ops_esem_2014}
identificou oito conceitos principais relacionados a \textit{DevOps}, os quatro
presentes no \acrshort{CAMS} \textit{framework} acrescidos de (1) serviços,
(2) garantia da qualidade, (3) estruturas e (4) padrões.

Posteriormente, Lwakatare et al. \cite{dimensions_of_devops} apresentaram uma
caracterização de \textit{DevOps} como um \textit{framework} conceitual que
contém quatro dimensões: colaboração, automação, medição e
\textbf{monitoramento}, este último sendo o único elemento novo, em comparação
com os estudos anteriores. Para identificar as dimensões de \textit{DevOps} foi
utilizada uma revisão de literatura combinada com entrevistas a quatro
praticantes de três companhias de um mesmo grupo empresarial. Uma extensão
deste trabalho foi publicada posteriormente em \cite{extending_dimensions} e
acrescentou uma nova dimensão chamada de \emph{colaboração}. Nessa extensão foi
conduzida uma revisão de literatura multivocal, utilizando dados da literatura
cinza.

Em \emph{Characterizing DevOps by Hearing Multiple Voices}, França et al.
\cite{characterizing_devops} organizam os elementos de \textit{DevOps} em seis
categorias de princípios: (1) automação, (2) garantia da qualidade,
(3) compartilhamento, (4) medição, (5) \textbf{aspectos sociais} e (6)
\textbf{\textit{leanness}}. Apenas a categoria \emph{aspectos sociais}
representa um elemento novo em relação aos já apresentados aqui. Para
chegar a estes resultados os autores conduziram uma revisão de literatura
multivocal, e os dados foram analisados usando algumas técnicas de
codificação de \textit{\acrfull{GT}}.

Por fim, por meio de uma nova revisão sistemática de literatura, Erich et al.
\cite{qualitative_devops_journalsw_17} identificaram sete agrupamentos de
conceitos presentes na literatura sobre \textit{DevOps}: (1) cultura de
colaboração, (2) automação, (3) medição, (4) compartilhamento, (5) serviços,
(6) garantia da qualidade e (7) \textbf{governança}. Apenas \emph{governança}
não aparece em nenhum dos trabalhos apontados previamente.

É possível identificar elementos de \textit{DevOps} que aparecem em vários
trabalhos bem como outros que aparecem apenas em um deles, a tabela
\ref{tabela_elementos_devops} lista os elementos identificados e aponta em
quais trabalhos cada um deles aparece. Após comparação dos estudos, foi
identificado que os elementos \emph{cultura}, \emph{cultura de colaboração},
\emph{colaboração} e \emph{aspectos sociais} são similares e, para efeitos de
comparação, foi designada uma única linha na tabela, a linha 1 (cultura de
colaboração).

\begin{table}[hb!]
\centering
\caption{Elementos de \textit{DevOps}}
\label{tabela_elementos_devops}
\begin{tabular}{|p{0.3cm}|p{2.7cm}|p{2cm}|p{1.55cm}|p{1.35cm}|p{1.7cm}|p{1.35cm}|p{1.45cm}|}
\hline

& Elemento & \acrshort{CAMS} \textit{Framework} \cite{what_devops_means_2010} & Hamunen \cite{challenges_in_adopting_devops} & Erich et al. \cite{cooperation_dev_ops_esem_2014} & Lwakatare et al. \cite{extending_dimensions} & França et al. \cite{characterizing_devops} & Erich et al. \cite{qualitative_devops_journalsw_17} \\
\end{tabular}

\begin{tabular}{|p{0.3cm}|p{2.7cm}|p{2cm}|p{1.55cm}|p{1.35cm}|p{1.7cm}|p{1.35cm}|p{1.45cm}|}

\hline

1  & Cultura de Colaboração & X & X & X & X & X & X \\

\hline

2  & Automação              & X & X & X & X & X & X \\

\hline

3  & Medição                & X & X & X & X & X & X \\

\hline

4  & Compartilha-\newline mento & X & X & X &    & X & X \\

\hline

5  & Serviços               & & & X &        &        & X \\

\hline

6  & Garantia da Qualidade  & & & X &        & X & X \\

\hline

7  & Estruturas             & & & X &        &        &        \\

\hline

8  & Padrões                & & & X &        &        &        \\

\hline

9  & Monitoramento          & & &        & X &        &        \\

\hline

10 & \textit{Leanness} & & X & & & X & \\

\hline
11 & Governança             & & &        &        &        & X \\

\hline

\end{tabular}
\end{table}

A tabela \ref{tabela_elementos_devops} reforça a relevância dos elementos do
\acrshort{CAMS} \textit{framework}. Apenas \emph{compartilhamento} deixa de
aparecer em um dos estudos citados, os demais aparecem em todos. Com relação
aos elementos que não fazem parte do \textit{framework}, \emph{garantia
da qualidade} e \emph{serviços} aparecem em três e dois estudos, respectivamente,
enquanto que os demais aparecem apenas em um único estudo.

Nas subseções a seguir, cada um dos elementos é detalhado. Alguns dos trabalhos
relacionados citam práticas relacionadas aos elementos. Quando aplicável, as
práticas também são listadas. Os elementos \emph{estruturas} e \emph{padrões}
foram citados apenas em \cite{cooperation_dev_ops_esem_2014} e não foi
apresentado um detalhamento sobre suas características. O mesmo grupo de
autores retirou esses elementos dos resultados apresentados a partir de uma
nova revisão de literatura realizada em \cite{qualitative_devops_journalsw_17}.
Assim sendo, esses elementos não fazem parte do detalhamento apresentado a
seguir.

\subsection{Cultura de Colaboração}

Uma parte essencial de {\it DevOps} é a mudança de cultura organizacional de
uma coleção de silos para uma forma de trabalho colaborativa. Isso implica
envolver o pessoal de operações no processo de projeto e transição de uma
aplicação. A cultura {\it DevOps} desfoca a linha entre os papéis dos
desenvolvedores e da equipe de operações. Em alguns casos, esse desfoque
elimina totalmente a distinção \cite{challenges_in_adopting_devops}.

No \acrshort{CAMS} \textit{framework} \cite{what_devops_means_2010}, este
elemento é chamado apenas de \emph{cultura} e é apresentado de maneira sucinta
como um direcionamento de que pessoas e processos devem vir primeiro, sob pena
de os demais esforços relacionados a \textit{DevOps} serem infrutíferos.

Já no trabalho de Lwakatare et al. \cite{dimensions_of_devops,extending_dimensions},
este elemento é apresentado primeiramente apenas como \emph{colaboração}, e em
um segundo momento como duas dimensões distintas: \emph{cultura} e
\emph{colaboração}. Ao tratar do elemento como \emph{colaboração}, os autores
mencionam especificamente que \textit{DevOps} envolve uma \emph{cultura de colaboração}.
Neste trabalho não é apresentada uma caracterização do que seria essa \emph{cultura
de colaboração}, apenas é posto que ela é reforçada por meio de
compartilhamento de informações, ampliação de qualificações e transferência de
responsabilidades entre as duas equipes, além de incutir um senso de
responsabilidade compartilhada. Após o acréscimo de \emph{cultura} como elemento
\textit{DevOps}, não é também apresentada uma diferenciação entre essas duas
dimensões que são apresentadas separadamente. Sobre \emph{cultura}, apenas é
dito que muitas das práticas \textit{DevOps} envolvem uma mudança de cultura e
mentalidade para incentivar a empatia, o apoio mútuo e um bom ambiente de
trabalho para os envolvidos no desenvolvimento de \textit{software} e
nos processos de entrega.

No trabalho de França et al. \cite{characterizing_devops}, este elemento é
apresentado como um princípio denominado \emph{aspectos sociais}. De acordo com os
autores, apesar de existirem diversos princípios técnicos, muitas das
características de \textit{DevOps} estão associadas a aspectos sociais entre as
equipes de desenvolvimento de \textit{software} e de operações. A \textbf{cultura
\textit{DevOps}} reconhece a confiança como uma característica relevante para
influenciar a mudança organizacional exigida por \textit{DevOps}.

Erich et al. \cite{qualitative_devops_journalsw_17} limitam-se a afirmar que
organizações que praticam \textit{DevOps} tentam remover a barreira cultural
entre o pessoal de desenvolvimento e de operações.

Em síntese, o elemento \emph{cultura de colaboração}, consiste na existência de
 uma maneira de se organizar as atividades de construção
de \textit{software} que reforça a colaboração entre os times de desenvolvimento
e operações, buscando-se evitar exatamente que se organizem em silos com
comunicação burocrática, sem confiança mútua e buscando objetivos distintos.

\subsubsection{Práticas Relacionadas}

Como práticas relacionadas ao elemento \emph{cultura de colaboração}, são
apresentadas as seis seguintes em \cite{extending_dimensions}:

\begin{enumerate}
\item Aumentar o escopo de responsabilidades;
\item Intensificar a cooperação e o envolvimento no trabalho diário de cada um
dos times;
\item Tanto desenvolvedores como operadores são considerados responsáveis para
lidar com incidentes;
\item Integrar o desenvolvimento em pós-mortem de produção;
\item Tornar a comunicação entre desenvolvimento e operações menos formal e
sem adversidades;
\item Respeito mútuo, apoio e vontade de trabalhar juntos e compartilhamento
de responsabilidades.
\end{enumerate}

Enquanto que em \cite{characterizing_devops}, são apresentadas mais oito
práticas: (1) avaliação coletiva de desempenho, (2) cultura de confiança, (3)
comunicação efetiva, (4) Aprendizagem mútua, (5) abertura para mudanças, (6)
responsabilidade pessoal, (7) relevância de aspectos culturais e (8) respeito
entre os membros dos times.

\subsection{Automação}

Organizações praticando \textit{DevOps} almejam ter um maior nível de automação
\cite{qualitative_devops_journalsw_17}. O objetivo por trás das práticas de
automação em {\it DevOps} está em alcançar {\it feedback} rápido e baixos
tempos de espera \cite{challenges_in_adopting_devops}.

No \acrshort{CAMS} \textit{framework} \cite{what_devops_means_2010}, é explicado
que as ferramentas \textit{DevOps} possibilitam uma ampla gama de possibilidades
para automação. Ferramentas para gerenciamento de versões, provisionamento,
gerenciamento de configuração, integração de sistemas, monitoramento e controle
e orquestração são tratatadas como peças importantes na construção de uma
estrutura \textit{DevOps}.

Para acompanhar o ritmo ágil das práticas de desenvolvimento de \textit{software}, os
processos de operações precisam ser flexíveis, repetíveis e rápidos, e a
automação é vista como um meio para viabilizar isso. Em ambientes complexos,
é difícil e demorado implantar funcionalidades e gerenciar configurações de
infraestrutura de \textit{software} manualmente. Além disso, existe automação
de testes visando garantir a qualidade das funcionalidades implantadas
\cite{dimensions_of_devops,extending_dimensions}.

França et al. \cite{characterizing_devops} acrescentam que automação é um dos
princípios fundamentais de \textit{DevOps} pois reduz os esforços repetitivos
e acelera a entrega de \textit{software}, melhorando não só a velocidade de
entrega, como também a consistência da infraestrutura, a produtividade das
equipes e a repetibilidade das tarefas.

\subsubsection{Práticas Relacionadas}

Como práticas relacionadas a automação, têm-se (1) infraestrutura como código
e (2) automação do processo de \textit{deployment}, apresentadas em
\cite{extending_dimensions}, enquanto que em \cite{characterizing_devops} são
apresentadas as seguintes: (1) aceleração por meio de automação, (2)
automatização de tarefas repetitivas, (3) automatização da coleta de métricas,
(4) automatização da infraestrutura e (5) consistência da infraestrutura.

\subsection{Medição}

Organizações que praticam \textit{DevOps} tentam usar métricas que envolvem
disciplinas de desenvolvimento e operações, ao invés de métricas separadas
para cada um dos times \cite{qualitative_devops_journalsw_17}. As métricas são
importantes para que os times tenham uma clara visão do que está acontecendo
com as aplicações em qualquer momento do tempo e para auxiliar na descoberta
de gargalos no processo \cite{challenges_in_adopting_devops}.

O processo de melhoria contínua requer a coleta de métricas, uma implementação
bem sucedida de \textit{DevOps} irá medir tudo o que for possível com a maior
frequência possível, há a coleta de métricas de desempenho, de processo e até
mesmo de pessoas \cite{what_devops_means_2010}.

A \emph{medição} em \textit{DevOps} é obtida quando os esforços vão além do
controle básico de qualidade, há medições usando dados de desempenho e uso em
tempo real das funcionalidades do \textit{software} em ambiente de produção.
Com uma \emph{medição} efetiva, os esforços de desenvolvimento de
\textit{software} são efetivamente medidos \cite{dimensions_of_devops}. \textit{DevOps}
envolve a coleta de métricas eficientes para apoiar a tomada de decisões no
ciclo de vida de desenvolvimento e operações de \textit{software} \cite{characterizing_devops}.

\subsubsection{Práticas Relacionadas}

As quatro práticas a seguir são apresentadas em \cite{extending_dimensions}:

\begin{enumerate}
\item As equipes de operações e de desenvolvimento são incentivadas e
recompensadas pelas mesmas métricas;
\item Tanto desenvolvimento quanto operações se concentram no valor de negócio
como a unidade essencial de medição;
\item O progresso no desenvolvimento é medido em termos de sistema
funcionando em ambiente de produção;
\item Os desenvolvedores usam \textit{feedback} de produção para orientar
decisões, melhorias e alterações nos sistemas.
\end{enumerate}

Enquanto que em \cite{characterizing_devops} são apresentadas as quatro
seguintes práticas: (1) medição para entender o código e o ambiente, (2)
monitoramento do negócio por meio de métricas, (3) monitoramento de operações
e (4) monitoramento de equipes por meio de métricas.

\subsection{Compartilhamento}

Segundo Hamunen \cite{challenges_in_adopting_devops}, uma cultura transparente
com mecanismos eficazes de disseminação de conhecimento entre as equipes é
essencial para derrubar as barreiras entre desenvolvimento e operações.
As organizações que praticam \textit{DevOps} tentam facilitar o desenvolvimento
e as operações por meio do uso de práticas para compartilhar o
conhecimento \cite{qualitative_devops_journalsw_17}.

O compartilhamento é considerado o \textit{loopback} no ciclo do \acrshort{CAMS}
\textit{framework}. Segundo Willis, criar uma cultura onde as pessoas
compartilham ideias e problemas é fundamental pois isso
gera um cenário de \textit{feedback} aberto que no final possibilita melhorias
na cultura \cite{what_devops_means_2010}.

\textit{DevOps} envolve cenários onde a informação e o conhecimento são
disseminados entre os indivíduos para promover o intercâmbio de aprendizado
pessoal e informações sobre projetos. Nesse sentido, os indivíduos divulgam
informações relevantes, por exemplo, sobre como implementar e executar
práticas recomendadas. Além disso, as informações sobre mudanças ou novas
características de produtos devem ser distribuídas entre os envolvidos
tanto da equipe de desenvolvimento quanto da de operações e isso promove a
visibilidade do fluxo de trabalho \cite{characterizing_devops}.

\subsubsection{Práticas Relacionadas}

Como práticas relacionadas a \emph{compartilhamento}, apenas são apresentadas
as práticas (1) compartilhamento do aprendizado pessoal e (2) compartilhamento
de informações de projeto, em \cite{characterizing_devops}.

\subsection{Serviços}

Este elemento está presente em \cite{cooperation_dev_ops_esem_2014} e em
\cite{qualitative_devops_journalsw_17}. Neste último estudo, é explicado que
considerar \emph{serviços} como elemento \textit{DevOps} consiste em considerar
uma característica das organizações que se estruturam em torno de serviços,
ao invés de em torno de disciplinas (como desenvolvimento e operações, por
exemplo) e que organizações que praticam \textit{DevOps}
tipicamente usam serviços em nuvem e a arquitetura de microsserviços.

\subsection{Garantia da Qualidade}

As organizações que praticam \textit{DevOps} tentam incorporar o controle de
qualidade ao processo de construção de \textit{software}, visando garantir
que os produtos e serviços possuam uma qualidade adequada
\cite{qualitative_devops_journalsw_17}.

A busca por \emph{garantia da qualidade} suporta a implementação de práticas
\textit{DevOps} uma vez que vincula diferentes partes interessadas
(desenvolvimento, operações, suporte e clientes) para realizar atividades de
forma eficiente e confiável. Ademais, representa o esforço para garantir que
os produtos e serviços atendam aos padrões de qualidade estabelecidos
\cite{characterizing_devops}.

\subsubsection{Práticas Relacionadas}

Em \cite{characterizing_devops} é apresentada uma única prática que possui
pouca expressividade e detalhamento: \emph{preocupações de garantia de
qualidade}.

\subsection{Monitoramento}

A proposta de \emph{monitoramento} como elemento \textit{DevOps} foi indicada
como uma das principais contribuições do primeiro trabalho de Lwakatare et al.
\cite{dimensions_of_devops} e mantida na extensão deste trabalho, publicada
em \cite{extending_dimensions}. Segundo os autores, a equipe de
operações é tipicamente responsável por monitorar os sistemas e a
infraestrutura subjacente para determinar a atribuição apropriada de recursos e
para detectar, relatar e corrigir os problemas que ocorrem nos sistemas.
Todavia, \textit{DevOps} propicia uma resposta aos desafios enfrentados na
realização de um monitoramento efetivo por meio da ênfase em colaboração entre
desenvolvedores e operadores. Adicionalmente, as análises são usadas para
integrar os dados de desempenho da infraestrutura e do sistema com o
comportamento de uso do cliente. As informações coletadas são fornecidas como
\textit{feedback} aos desenvolvedores e ao gerenciamento de produtos para serem
utilizadas em aprimoramentos e personalização dos produtos de \textit{software}.

Convém ressaltar, que as descrições contidas nos demais trabalhos para o
elemento \emph{medição}, tipicamente englobam aspectos de \emph{monitoramento}.
Por exemplo, entre as práticas apontadas em \cite{characterizing_devops} estão
\emph{monitoramento de operações} e \emph{monitoramento de equipes por meio de
métricas}. Essa separação de \emph{monitoramento} como um elemento autônomo em
relação a \emph{medição} apenas foi realizado por Lwakatare et al.

\subsection{\textit{Leanness}}

De acordo com França et al. \cite{characterizing_devops}, algumas práticas
\textit{DevOps} são baseadas nos princípios de \emph{pensamento enxuto}\footnote{Lean
Thinking \cite{lean_thinking}}. Considerando que \textit{DevOps} visa garantir
um fluxo contínuo para desenvolver e entregar {\it software} regularmente, em
pequenas e incrementadas mudanças, o processo precisa ser enxuto (\textit{lean}).

Já Hamunen \cite{challenges_in_adopting_devops}, cita a redução de desperdícios,
a limitação do trabalho em andamento e a redução de tamanho das entregas para
ilustrar os motivos de as metodologias ``{\it Lean}'' serem um importante
elemento de {\it DevOps}.

\subsubsection{Práticas Relacionadas}
França et al. \cite{characterizing_devops} apresentam seis práticas
relacionadas ao elemento \textit{Leanness}: (1) fluxo contínuo, (2) melhoria
contínua, (3) eliminação de desperdícios, (4) \textit{feedback} rápido, (5)
visão holística ou sistêmica e (6) simplicidade.

\section{Desafios na Adoção de \textit{DevOps}}\label{secao_desafios}

Implementar \textit{DevOps} é reconhecidamente uma atividade desafiadora. Nesta
seção são apresentados os principais desafios relatados na literatura
relacionada a respeito da adoção de \textit{DevOps}.

Riungu-Kalliosaari et al. \cite{devops_benefits_challenges} apontam quatro
desafios relacionados à adoção de \textit{Dev-Ops}. O primeiro é a existência
de \emph{comunicação insuficiente} entre os times de desenvolvimento e
operações. Em segundo lugar, como \textit{DevOps} requer
uma mudança cultural, existe o desafio da existência de \emph{culturas
profundamente arraigadas}. O terceiro desafio apontado é a possibilidade de
\emph{restrições legais} relacionadas, por exemplo, ao acesso, para execução de
testes, de dados protegidos. Por fim, o quarto desafio é que \textit{DevOps}
ainda não é completamente compreendido mas continua evoluindo rapidamente. Este
quarto desafio é endossado por Smeds et al. \cite{devops_a_definition}, que
afirmam que a definição e os objetivos de se adotar {\it DevOps} não são claros,
e também por Hamunen \cite{challenges_in_adopting_devops} que observou uma
falta de maturidade para o conceito de {\it DevOps}.

Mais dois desafios foram identificados tanto em \cite{devops_a_definition} quanto em \cite{challenges_in_adopting_devops}:
{\it DevOps} ser visto como uma \textit{buzzword} e a distribuição geográfica
dos times.

Adicionalmente, Hamunen \cite{challenges_in_adopting_devops} aponta que a falta
de suporte tanto a nível de gerência quanto a nível de time, bem como a falta
de confiança e as dificuldades para implementar tecnologias {\it DevOps} e para
adaptar os processos organizacionais são os outros principais agrupamentos de
desafios relacionados a {\it DevOps}.

Enquanto que Smeds et al. apontam ainda como desafios para {\it DevOps}:
dificuldades relacionadas a estruturas organizacionais não compatíveis, onde
clientes não vêem valor em {\it DevOps}; a falta de interesse de algum dos times;
e dificuldades técnicas como a existência e múltiplos ambientes de produção e
o fato de ambientes como desenvolvimento e teste não refletirem o ambiente de
produção.
