\newcommand{\texCommand}[1]{\texttt{\textbackslash{#1}}}%

\newcommand{\exemplo}[1]{%
\vspace{\baselineskip}%
\noindent\fbox{\begin{minipage}{\textwidth}#1\end{minipage}}%
\\\vspace{\baselineskip}}%

\newcommand{\exemploVerbatim}[1]{%
\vspace{\baselineskip}%
\noindent\fbox{\begin{minipage}{\textwidth}%
#1\end{minipage}}%
\\\vspace{\baselineskip}}%

Este capítulo apresenta uma revisão dos principais trabalhos relacionados a
\textit{DevOps} estando estruturado da seguinte forma:

\section{O que é \textit{DevOps}?}

\textit{DevOps} emergiu na indústria de desenvolvimento de software sem uma
clara definição. Todavia, estudos recentes propõem definições
dentre as quais desataca-se a de França et al. \cite{characterizing_devops}
que definem \textit{DevOps} como um neologismo que representa um movimento de
profissionais de tecnologia da informação e comunicação buscando uma
diferente atitude no que se refere à entrega de \textit{software},
através da colaboração entre o desenvolvimento de sistemas de
\textit{software} e as funções de operações, com base em um conjunto
de práticas, tais como cultura, automação, medição e compartilhamento.

\section{Elementos de \textit{DevOps}}

\section{Impedimentos de \textit{DevOps}}

\section{Adoção de \textit{DevOps}}

\section{\textit{Grounded Theory}}
Since the publication of the original version of GT [12], several
modifications and variations have been proposed to the method,
coming to exist at least seven different versions of Grounded Theory [10]. 

%%%%%%%%%%%%%%%%%%%%%%%%%%%%%%%%%%%%%%%%%%%%%%%%%%%%%%%%%%%%%%%%%%%%%%%%%%%%%%%%
%%%%%%%%%%%%%%%%%%%%%%%%%%%%%%%%%%%%%%%%%%%%%%%%%%%%%%%%%%%%%%%%%%%%%%%%%%%%%%%%
%%%%%%%%%%%%%%%%%%%%%%%%%%%%%%%%%%%%%%%%%%%%%%%%%%%%%%%%%%%%%%%%%%%%%%%%%%%%%%%%


%%%%%%%%%%%%%%%%%%%%%%%%%%%%%%%%%%%%%%%%%%%%%%%%%%%%%%%%%%%%%%%%%%%%%%%%%%%%%%%%
%%%%%%%%%%%%%%%%%%%%%%%%%%%%%%%%%%%%%%%%%%%%%%%%%%%%%%%%%%%%%%%%%%%%%%%%%%%%%%%%
%%%%%%%%%%%%%%%%%%%%%%%%%%%%%%%%%%%%%%%%%%%%%%%%%%%%%%%%%%%%%%%%%%%%%%%%%%%%%%%%
