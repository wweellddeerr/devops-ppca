\newcommand{\texCommand}[1]{\texttt{\textbackslash{#1}}}%

\newcommand{\exemplo}[1]{%
\vspace{\baselineskip}%
\noindent\fbox{\begin{minipage}{\textwidth}#1\end{minipage}}%
\\\vspace{\baselineskip}}%

\newcommand{\exemploVerbatim}[1]{%
\vspace{\baselineskip}%
\noindent\fbox{\begin{minipage}{\textwidth}%
#1\end{minipage}}%
\\\vspace{\baselineskip}}%

Este capítulo apresenta uma revisão dos principais trabalhos relacionados a
\textit{DevOps} estando estruturado da seguinte forma:

\section{O que é \textit{DevOps}?}

\textit{DevOps} emergiu na indústria de desenvolvimento de software sem uma
clara definição.

Em \textit{What is DevOps?}, Loukides tenta ilustrar o conceito de
\textit{DevOps} por meio da diferenciação entre o passado e o presente do
trabalho de operações de \acrshort{TI}. O texto ilustra os novos desafios
existentes para os times de operações e destaca a importância da existência
de colaboração com os times de desenvolvimento para superá-los. Todavia,
não é apresentada uma definição final de \textit{DevOps}.

H\"uttermann \cite{devops_for_developers} salienta a dificuldade que é definir
\textit{DevOps} e ressalta que o termo é multifacetado. O autor afirma que
``\textit{DevOps} descreve práticas que simplificam o processo de entrega
de software, enfatizando o aprendizado por meio da transmissão contínua de
\textit{feedback} da produção para o desenvolvimento''. Essa definição de
\textit{DevOps} como um conjunto de práticas relacionadas à entrega de
\textit{software} quando comparada com outras definições, demonstra carecer
de uma maior ênfase no aspecto cultural. Por exemplo, de acordo com
Walls \cite{building_devops_culture}, o termo foi introduzido justamente para
definir uma cultura organizacional que as organizações podem buscar. De acordo
com ele, \textit{DevOps} é ``um movimento cultural combinado com várias
práticas relacionadas a \textit{software} que possibilitam um desenvolvimento
rápido''. O autor descreve quatro características culturais chave em um
contexto de \textit{DevOps}: comunicação aberta, incentivo e alinhamento de
responsabilidades, respeito e confiança.






Todavia, estudos recentes propõem definições
dentre as quais desataca-se a de França et al. \cite{characterizing_devops}
que definem \textit{DevOps} como um neologismo que representa um movimento de
profissionais de tecnologia da informação e comunicação buscando uma
diferente atitude no que se refere à entrega de \textit{software},
através da colaboração entre o desenvolvimento de sistemas de
\textit{software} e as funções de operações, com base em um conjunto
de práticas, tais como cultura, automação, medição e compartilhamento.

\section{Elementos de \textit{DevOps}}

\section{Impedimentos de \textit{DevOps}}

\section{Adoção de \textit{DevOps}}

\section{\textit{Grounded Theory}}
Since the publication of the original version of GT [12], several
modifications and variations have been proposed to the method,
coming to exist at least seven different versions of Grounded Theory [10].

%%%%%%%%%%%%%%%%%%%%%%%%%%%%%%%%%%%%%%%%%%%%%%%%%%%%%%%%%%%%%%%%%%%%%%%%%%%%%%%%
%%%%%%%%%%%%%%%%%%%%%%%%%%%%%%%%%%%%%%%%%%%%%%%%%%%%%%%%%%%%%%%%%%%%%%%%%%%%%%%%
%%%%%%%%%%%%%%%%%%%%%%%%%%%%%%%%%%%%%%%%%%%%%%%%%%%%%%%%%%%%%%%%%%%%%%%%%%%%%%%%


%%%%%%%%%%%%%%%%%%%%%%%%%%%%%%%%%%%%%%%%%%%%%%%%%%%%%%%%%%%%%%%%%%%%%%%%%%%%%%%%
%%%%%%%%%%%%%%%%%%%%%%%%%%%%%%%%%%%%%%%%%%%%%%%%%%%%%%%%%%%%%%%%%%%%%%%%%%%%%%%%
%%%%%%%%%%%%%%%%%%%%%%%%%%%%%%%%%%%%%%%%%%%%%%%%%%%%%%%%%%%%%%%%%%%%%%%%%%%%%%%%
