Neste capítulo é detalhada a realização de um grupo focal \cite{focus_group_handbook,shull2007guide}
para identificar a percepção dos profissionais envolvidos tanto a respeito da
adoção de {\it DevOps} em geral, quanto da aplicabilidade e utilidade do modelo
proposto na adoção de {\it DevOps} no \acrshort{TCU}. Este capítulo contém ainda
detalhamento das ações citadas no grupo focal que foram desenvolvidas no \acrshort{TCU}
após o intercâmbio de experiências ocorrido durante a elaboração do modelo.

\section{Metodologia}

Para obtenção da avaliação empírica sobre a adoção de {\it DevOps} no \acrshort{TCU},
bem como da aplicabilidade e utilidade do modelo descrito, proposta como
objetivo deste trabalho de pesquisa, foi utilizado o método qualitativo de
pesquisa \emph{grupo focal} \cite{focus_group_handbook}.

A população envolvida no processo de adoção de \textit{DevOps} no TCU é
relativamente pequena (em torno de 12 desenvolvedores e 6 profissionais do time
de operações) e há um maior número de profissionais nos times de
desenvolvimento do que no time de operações. Para que as percepções obtidas
na avaliação não possuíssem o viés do time de desenvolvimento, foi adotada a
estratégia de ouvir um número similar de profissionais de cada um dos times.
Com essa limitação do número de profissionais do time de desenvolvimento que
poderiam ser ouvidos na pesquisa, o número total de participantes na amostra
(estimado entre 8 e 12 pessoas) mostrou-se pequeno para realização de um
questionário.

Ao mesmo tempo, a realização de entrevistas, além de alongar a obtenção do
\textit{feedback} desejado, poderia gerar um grande número de respostas
repetitivas, uma vez que são todos profissionais envolvidos no mesmo processo
de adoção de \textit{DevOps}. Optou-se, portanto, por se realizar um grupo
focal.

Grupo focal emergiu como método de pesquisa nas ciências sociais nos anos 1950
e atualmente é amplamente utilizado, por exemplo, em estudos sociológicos,
pesquisas de mercado, planejamento de produtos e estudos de usabilidade de
sistemas \cite{shull2007guide}. Morgan \cite{morgan1996focus} define grupo focal
como uma técnica de pesquisa que coleta dados através da interação de grupo em
um tópico específico determinado pelo pesquisador.

Segundo F. Shull et al. \cite{shull2007guide}, grupos focais tipicamente
possuem entre três e doze participantes, são projetados para obter percepções
pessoais de membros de um ou mais grupos envolvidos em uma área definida de
interesse de pesquisa e possuem como benefícios a produção de informações
cândidas, muitas vezes perspicazes, com um baixo custo e rápida execução.
Estas características tornam o grupo focal uma alternativa adequada aos
propósitos desta pesquisa. Ainda segundo os autores, a discussão é guiada e
facilitada por um pesquisador-moderador que segue uma estrutura predefinida
de questionamentos.

O grupo focal realizado no \acrshort{TCU} contou com a participação de quatro
profissionais, cujo perfil está brevemente descrito na Tabela \ref{tabela_participantes_grupo_focal}.

\begin{table}[hb!]
\centering
\label{tabela_participantes_grupo_focal}
\begin{tabular}{|p{0.7cm}|p{3.5cm}|p{2.8cm}|p{6.5cm}|} \hline
{\bf P\#} & {\bf Time de atuação} & {\bf Formação} & {\bf Experiência}\\ \hline
P1 & Desenvolvimento & Graduação & Atua há 3 anos no desenvolvimento de {\it software} do \acrshort{TCU} e possui 9 anos de experiência prévia \\ \hline
P2 & Desenvolvimento & Pós-graduação & Atua há 6 anos no desenvolvimento de {\it software} do \acrshort{TCU} e possui 7 anos de experiência prévia \\ \hline
P3 & Operações & Graduação & Atua há 3 anos no time de operações do \acrshort{TCU} e possui 8 anos de experiência prévia \\ \hline
P4 & Operações & Graduação & Atua há 3 anos no time de operações do \acrshort{TCU} e possui 10 anos de experiência prévia \\ \hline
\end{tabular}
\caption{Participantes do Grupo Focal}
\end{table}

Seguindo uma estrutura similar ao realizado por Lehtola et al. \cite{requirementes_priorization_in_practice},
o grupo focal foi conduzido da seguinte forma: o pesquisador-moderador atuou
como facilitador do grupo focal fornecendo aos participantes três tópicos de
discussão, listados na Tabela \ref{tabela_topicos}. No início da discussão de
cada tópico, as perguntas foram apresentadas aos participantes que escreveram
suas ideias e palavras-chave em notas {\it post-it}. Depois disso, as notas
foram postas em um quadro branco e serviram como ponto de partida para a
realização de discussões sobre o respectivo tópico com o propósito de se obter
conclusões a respeito da respectiva pergunta. Os resultados das discussões de
cada tópico são apresentados na próxima seção.

\begin{table}[hb!]
\centering
\label{tabela_topicos}
\begin{tabular}{|p{0.3cm}|p{6.4cm}|p{7cm}|} \hline
& \textbf{Tópico} & \textbf{Perguntas} \\ \hline

1 & Estágio atual da adoção de {\it DevOps} no \acrshort{TCU} &
1. Quais ações já desenvolvidas no \acrshort{TCU} você considera que fazem parte da adoção de {\it DevOps}?\newline\newline
2. Quais problemas existentes anteriormente foram resolvidos por essas ações? \\ \hline

2 & Aplicabilidade e utilidade do modelo proposto &
1. Você considera que este modelo que foi aprovado no \acrshort{CPA} tem contribuído para a adoção de {\it DevOps} no \acrshort{TCU}?\newline\newline
2. Caso positivo, quais as principais contribuições? \\ \hline

3 & Desafios enfrentados e próximos passos na adoção de {\it DevOps} &
1. Quais os principais desafios que a adoção de {\it DevOps} enfrenta atualmente no \acrshort{TCU}?\newline\newline
2. Quais os próximos passos para a adoção de {\it DevOps} no \acrshort{TCU}?\\ \hline

\end{tabular}
\caption{Tópicos do Grupo Focal}
\end{table}

A gravação do grupo focal foi proposta aos participantes e recusada.
Como os times de desenvolvimento e operações do \acrshort{TCU} atualmente são
partes de secretarias diferentes, foi levantado pelos participantes que a
gravação necessitava ser tratada diretamente com ambos os secretários. Ademais,
foi apontado que a gravação também limitaria a exposição de opiniões e ideias
pelo receio existente de gerar alguma interpretação indevida.

Foi então acordado que seria produzida uma ata de reunião cotendo o detalhamento
das discussões, e que essa ata deveria ser enviada a todos os participantes para
aprovação. Assim sendo, os resultados apontados na seção \ref{secao_resultados_grupo}
foram validadas pelos participantes por meio da ratificação da ata que a todos
foi enviada.

\section{Resultados}\label{secao_resultados_grupo}

O primeiro tópico de discussão no grupo focal foi o estágio atual da adoção de
{\it DevOps} no \acrshort{TCU}, onde foram discutidas as ações já desenvolvidas
e quais problemas essas ações resolveram. A seguir são enumerados os
resultados da discussão neste tópico.

\subsubsection{Disponibilização de \acrshort{VM}s para Ferramentas de Desenvolvimento}

A primeira ação indicada e discutida foi a disponibilização de ambientes (\acrshort{VM}s)
para instalação de ferramentas que são relacionadas ao trabalho de
desenvolvimento. Essa ação foi exemplificada com as recentes instalações bem
sucedidas das ferramentas Elasticsearch\footnote{https://www.elastic.co/} e Kafka\footnote{https://kafka.apache.org/}.
O problema existente anteriormente era que quando o desenvolvedor necessitava de
uma ferramenta desse tipo, inerente ao trabalho que realizava e necessária para
atender de maneira adequada problemas específicos, dependia de abrir uma
solicitação para que o time de operações a provesse, com prazos muitas vezes
tão dilatados que inviabilizavam o uso da solução mais adequada. Com o
provisionamento de \acrshort{VM}s e cooperação entre os dois times, essas
ferramentas ficaram disponíveis rapidamente para uso e são administradas de
maneira conjunta. Este é um claro exemplo de aplicação dos conceitos de
{\bf empoderamento do desenvolvimento de \emph{software}} e {\bf responsabilidade
compartilhada} da categoria principal \cat{cultura de colaboração}.

\subsubsection{Execução Conteinerizada de Aplicações}

Na sequência do grupo focal, a utilização de conteineres para execução de
aplicações foi debatida. O primeiro problema que essa ação solucionou, no
entendimento dos participantes, foi a anterior falta de paridade entre os
ambientes. Foram relembrados os problemas recorrentes de aplicações que
funcionavam em ambiente de desenvolvimento mas apresentavam problemas em
produção, o que deixou de acontecer a partir do uso dos conteineres.

Em seguida foi apontado que a disponibilização do {\it Dockerfile}\footnote{Arquivo
que descreve os passos a serem executados para construção de uma imagem
{\it Docker} que, ao ser executada, instancia um conteiner} no repositório de
código-fonte de cada projeto possibilita que tanto os desenvolvedores como o
pessoal de operações possam ter uma primeira ideia sobre o ambiente de execução
de cada aplicação.

Ainda sobre o uso de conteineres, foi debatida a exploração dos recursos providos
pelo {\it Kubernetes}\footnote{Ferramenta de orquestração de conteineres: https://kubernetes.io/}
nas aplicações. Com {\it Kubernetes} foi possível avançar ainda mais na
disponibilização de infraestrutura como código. Foi relembrado que o \acrfull{CPIC}
do \acrshort{TCU} determinou que as aplicações que executem em conteiner devem
possuir um diretório chamado de \emph{k8s} na raiz do repositório de código-fonte,
onde devem constar os arquivos de configuração do {\it Kubernetes}.
A aplicação do conceito de {\it deployment} da ferramenta tem permitido a
configuração de mecanismos para escalabilidade horizontal, alocação de recursos
para os containers e alta disponibilidade. Foi também citado que a
publicação de aplicações sem {\it down-time}, possibilitada pelos mecanismos do
{\it Kubernetes}, resolveu o problema de interromper o trabalho das secretarias
estaduais de controle externo que possuem fuso horário em relação a Brasília.

{bf Incluir aqui uma girua com a pasta do kubernetes e código do deployment}

Neste ponto da discussão, foi possível identificar a aplicação de sete
conceitos: (1) paridade entre ambientes, (2) automação do provimento da
infraestrutura, (3) serviços autônomos, (4) conteinerização, (5) auto {\it scaling},
(6) automação da recuperação e (7) zero {\it down-time}.

\subsubsection{Desburocratização da Comunicação}

A experiência obtida em outras organizações que adotaram {\it DevOps}
possibilitou o direcionamento de evitar o uso de meios burocráticos de
comunicação. Durante este tópico foi relembrado o processo de comunicação
extremamente cerimonioso contido no âmbito da publicação programada. Há uma
diretriz atual de se evitar o uso do {\it servicedesk} para resolução de
problemas simples. O uso do {\it Slack} tem sido institucionalizado e facilitado
o contato entre os dois times.

Aqui, é possível notar a aplicação do conceito de {\bf comunicação facilitada}
da categoria \cat{cultura de colaboração}.

\subsubsection{Pipelines Multidisciplinares}

As aplicações mais recentes do \acrshort{TCU} executam {\it pipelines}
multidisciplinares cujas etapas envolvem desde o {\it build}, passando por testes
automatizados e análise estática de código fonte, execução dos conteineres
utilizando {\it Kubernetes} e publicação de maneira isonômica nos diversos
ambientes (desenvolvimento, aceite e produção). Estes {\it pipelines} são
acessíveis por meio de um {\it Jenkinsfile} na raiz do repositório de código-fonte
de cada projeto e são automaticamente acionados a partir de cada {\it push}
executado. A Figura x.y contem um exemplo de {\it pipeline} em execução em uma
aplicação real do \acrshort{TCU}. Na visão dos participantes do grupo focal,
a implementação efetiva destes {\it pipelines} possibilita a entrega contínua
de {\it software} solucionando diversos problemas contidos na publicação
programada, principalmente as recorrentes demoras para publicação em produção
de funcionalidades já prontas.

A utilização de {\it Jenkinsfile} como maneira de se implementar um {\it pipeline}
compartilhado é outro exemplo de ação  que decorreu diretamente do intercâmbio
de experiências com praticantes de {\it DevOps} durante a condução deste
trabalho de pesquisa.

\subsubsection{Versionamento de Bancos de Dados Relacionais}

Foi relembrado que o uso de ferramentas de versionamento de bancos de dados
havia sido anteriormente declarado como incompatível com o processo de
desenvolvimento de {\it software} do \acrshort{TCU}. Apenas com a aproximação
dos times, foi possível gerar uma compreensão de quais procedimentos eram
executados quando se solicitava a alteração de uma estrutura de banco de dados.
Essencialmente, foi identificado que as alterações de bancos de dados geravam
atualizações nos mecanismos de auditoria executados pelo time de operações. Com
essa compreensão, foi possível automatizar essas atualizações de auditoria viabilizando o uso
da ferramenta para versionamento de bancos de dados escolhida, que é o {\it Flyway}\footnote{https://flywaydb.org/}.
Atualmente, quando se necessita, por exemplo, criar ou alterar uma tabela no
banco de dados, basta incluir o {\it script} equivalente no diretório adequado
de código fonte. Anteriormente era necessário abrir uma solicitação no {\it servicedesk}
e sincronizar os horários e procedimentos para que pudessem ser executados na
publicação programada.

O uso do {\it Flyway} é um exemplo da aplicação do conceito de {\bf automação do
gerenciamento da infraestrutura} da categoria \cat{automação}.

\subsubsection{Monitoramento Automatizado e Contínuo}

A respeito do monitoramento das aplicações, especificamente quando da ocorrência
de erros em tempo de execução, a solução construída de
monitoramento automatizado e contínuo foi destacada como parte das ações já
desenvolvidas e que resolveu o problema da falta de proatividade na correção de
erros existente anteriormente. A obtenção de acesso aos {\it logs} de aplicações
anteriormente precisava ser solicitada pelos desenvolvedores quando necessário,
tipicamente no momento da resolução de algum incidente.

Esta solução também foi implementada como fruto direto da experiência
coletada sobre a adoção de {\it DevOps} em uma das companhias entrevistadas
durante a condução desta pesquisa. A solução envolve a execução de um
procedimento utilizando a ferramenta {\it Fluentd}\footnote{https://www.fluentd.org/}
para coleta dos {\it logs} de aplicações e carga no banco de dados não
relacional {\it Elasticsearch}. Após essa carga, a ferramenta {\it ElastAlert}\footnote{http://elastalert.readthedocs.io/}
executa buscas semânticas pré-configuradas e, em caso de ocorrência de algum
padrão nos {\it logs} de aplicações (mensagem de erro, por exemplo), envia
automaticamente uma mensagem aos times de desenvolvimento e operações por meio
do {\it Slack}. A Figura \ref{monitoramento_logs} ilustra a solução
implementada no \acrshort{TCU}.

\figura{monitoramento_logs}{Solução do TCU para Monitoramento Contínuo}{monitoramento_logs}{width=.8\textwidth}

\section{Aplicabilidade e Utilidade do Modelo Proposto}

Todos os participantes do grupo focal concordaram que o modelo proposto possui
grande utilidade para a adoção de {\it DevOps} do \acrshort{TCU}. Relembraram
que boa parte das ações discutidas no tópico anterior foram fruto direto do
desenvolvimento deste modelo e que, portanto, consideram, que a sua aplicação
já está sendo efetiva e produzido resultados satisfatórios no sentido de colaborar
para a ampliação do uso de {\it DevOps} no desenvolvimento das aplicações
corporativas do {\it TCU}. A seguir, são apresentados os dois principais
benefícios da utilização do modelo, discutidos durante o grupo focal.

\subsubsection{Compreensão Institucional sobre \emph{DevOps}}

Como resposta à pergunta sobre as principais contribuições do modelo para o
\acrshort{TCU}, inicialmente foi apontado que, durante provas
de conceito realizadas anteriormente (que foi o primeiro encaminhamento dado
pelo \acrshort{CPA} para cumprimento do indicador do \acrshort{PDTI} sobre {\it DevOps}), ficou
nítido que o simples uso das ferramentas não estava aproximando os times, que
alguns desenvolvedores estavam agindo como se houvesse um salvo conduto para
``passar por cima'' dos procedimentos do time de operações e que o pessoal de
operações estava bastante preocupado em delimitar formalmente as fronteiras de
responsabilidade quanto à administração das ferramentas. O avanço da discussão
mostrou que todos os participantes concordaram com o entendimento de que
fomentar a cultura de colaboração não era um fator levado em conta pela maioria
dos profissionais envolvidos, e que
ver, através do modelo, que a adoção de {\it DevOps} no mercado passa
principalmente por este ponto tem possibilitado uma mudança de postura das
equipes, no sentido de colaborar mais ao invés de tentar defender interesses
próprios de cada uma das equipes.

Na sequência, foi discutido o conteúdo da nota {\it post-it} que continha uma
anotação a respeito da ``ampla gama de práticas e experiências'' presentes no
modelo. Foi mais uma vez relembrado que diversas práticas já foram implementadas usando como insumo experiências
de mercado coletadas durante a produção do modelo. O modelo também foi apontado
como uma ferramenta para avaliação de práticas que o \acrshort{TCU} ainda não
adota, fornecendo um {\it roadmap} robusto para guiar os próximos passos.

\subsubsection{Experiências de Mercado}

Por fim, foi destacado que o modelo foi construído levando em conta
experiências bem sucedidas no mercado e que isso representa grande valor para
o \acrshort{TCU}. No entendimento do grupo, embora o \acrshort{TCU} possua
muitas das peculiaridades presentes em ambientes governamentais, a busca pela
inovação tecnológica faz parte do mapa estratégico do órgão e não pode ser
alcançada olhando-se apenas para cenários similares ao atual. Foi ressaltado
que o mercado é um importante ator na definição de novas tecnologias que,
adaptadas em maior ou menor grau, podem ser plenamente aplicáveis aos órgãos
governamentais, como o TCU. Segundo o entendimento formado, o fato de um modelo
construído com base em experiências de mercado estar sendo efetivamente
aplicado, é mais uma constatação de que essa premissa é verdadeira.

\section{Desafios Enfrentados e Próximos Passos na Adoção de \emph{DevOps}}

As discussões do último tópico do grupo focal se concentraram em identificar
os desafios enfrentados na evolução do uso de {\it DevOps} no \acrshort{TCU},
bem como nos próximos passos para superar os desafios e institucionalizar
{\it DevOps} como abordagem de desenvolvimento.

\subsubsection{Maturidade do Entendimento Interno sobre \emph{DevOps}}
Inicialmente foi debatida a percepção apontada por um dos participantes de que
ainda há muita {\it hype} em torno do que seria a adoção de {\it DevOps}. Que
muitos desenvolvedores ainda pensam que isso possibilita a tomada de
iniciativas técnicas sem consultar outros profissionais e que algumas pessoas de
operações ainda não se sentem confortáveis com essa mudança de paradigma, pois
entendem que {\it DevOps} pode provocar uma desorganização em um ambiente que já
possuia estabilidade. Foi mencionado que o modelo apresentado
ajuda a lidar com esse desafio, mas que é necessária uma melhor conscientização
de todos o profissionais sobre fomentar a colaboração entre os times e não
apenas sair querendo resolver tudo de acordo com convicções pessoais.

Foi apontada também a dificuldade de disseminar o conhecimento relacionado
às novas ferramentas e processo que vieram junto com a adoção de {\it DevOps}.
Ações para mitigação desse desafio foram discutidas, incluindo a ampliação das
palestras internas do \acrshort{TCU} que são chamadas internamente de bate-bola
técnico, a participação nos eventos tais qual o \emph{DevOpsDays}, e os estímulos
que o \acrshort{TCU} já dá para os servidores, como licença capacitação,
reembolso de treinamentos e disponibilização da plataforma {\it safari books}.
Nesse sentido, formou-se o entendimento de que é um dos próximos passos a
ampliação da capacitação técnica das equipes nos temas relativos à modernização
das ferramentas e processos.

\subsubsection{Segurança da Informação}

Aqui, foi debatido que a adoção de {\it DevOps} aumentou consideravalmente a
superfície de vulnerabilidades que o \acrshort{TCU} possui. Foi apontado que
o time de operações tem grande preocupação com a segurança da informação, que
estão avaliando algumas das ferramentas implantadas e que irão propor
modificações. Foi então alinhado que este debate não pode ficar apenas no time
de operações, pois isso é exatamente uma manifestação da falta de colaboração.

Por iniciativa de um dos participantes, foi então debatido que essas preocupações
ampliam um pouco do escopo de {\it DevOps}, indo para um contexto de {\it DevSecOps},
quando as atividades de segurança também são integradas ao processo de
desenvolvimento. Tem-se, então, mais um próximo passo, que é a ampliação da
perspectiva de {\it DevSecOps}.

\subsubsection{Coleta de Métricas em Aplicações}

Nessa parte do debate, foi apontado que a solução de monitoramento contínuo
atual se restringe aos erros das aplicações, e que o modelo contém ideias a
respeito da coleta de métricas em aplicações para fomentar evoluções e decisões
de negócio. Foi pontuado por um dos participantes que a mesma solução pode ser
ampliada, desde que as aplicações sejam instrumentadas para gerar {\it logs} de
quaisquer outras métricas. Por conseguinte, a coleta contínua de outras métricas
de aplicações foi apontado como mais um dos próximos passos da adoção de
{\it DevOps} do \acrshort{TCU}.

\subsubsection{Portarias de Responsabilidade}

Aqui, foi apontado que, embora o modelo tenha possibilitado a compreensão de que
o mais importante deve ser fomentar a \cat{cultura de colaboração}, muitos
profissionais do órgão ainda pensam de maneira mais formal e a portaria
estabelece que as responsabilidades por questões relacionadas à infraestrutura
de aplicações é do \acrshort{SINAP}, o que dificulta a consolidação de um senso
de responsabilidade compartilhada.

Não houve um consenso a respeito de qual a melhor solução para resolver as
restrições contidas nas portarias de estruturação organizacional. Alguns (P1 e P4)
entendem que seria adequado que o setor de operações (\acrshort{SINAP}) poderia
ser transferido para a Secretaria de Soluções de \acrshort{TI} (atualmente é
parte da Secretaria de Infraestrutura de \acrshort{TI}); outros (P2 e P3)
demonstraram o entendimento de que basta uma alteração na portaria para definir
que há responsabilidade compartilhada por questões relacionadas a infraestrutura
de aplicações. Foi lembrado que há um grupo de trabalho constituído no intuito
de propor modificações nas portarias para ajustar essas atribuições ao
cenário de {\it DevOps}.

\subsubsection{Distanciamento Físico das Equipes}

A última dificuldade apontada foi que as equipes de desenvolvimento e operações
atualmente atuam em salas separadas. A distância física foi apontada como um fator
que dificulta a comunicação e atrapalha a formação da cultura de colaboração.
Os participantes concordaram que a aproximação física das equipes passa pela
questão da reestruturação das portarias discutidas acima. Caso o setor de
operações seja incorporado à Secretaria de Soluções de \acrshort{TI}, é provável
que a aproximação ocorra, caso contrário, é necessário buscar uma solução
viável.

\section{Considerações Gerais}

O grupo focal não se concentrou especificamente na avaliação do modelo proposto
nesta pesquisa porque a adoção de {\it DevOps} no \acrshort{TCU} não envolve
apenas esta pesquisa. O modelo é mais um mecanismo na busca pela maturação do
uso de {\it DevOps} no desenvolvimento das aplicações corporativas do \acrshort{TCU}.

Mesmo não tendo sido o único ponto de debate, considerações a respeito do modelo
permearam todos os tópicos debatidos, foram destacadas ações práticas que só
se materializaram devido ao intercâmbio de experiências ocorridas durnate a
pesquisa. Ademais, foi identificada uma conscientização a respeito da cultura
de colaboração que não existia anteriormente, as ações têm sido compreendidas
como parte dos esforços para se fomentar a cultura de colaboração.

Por fim, cabe ressaltar que a versão final do modelo foi introduzida há pouco
tempo (em torno de 1 mês) no \acrshort{TCU} e que uma percepção mais robusta
do seu impacto requer mais tempo para se formar. Duas ações específicas foram
realizadas para divulgação: (1) apresentação em um bate-bola técnico ({\it tech
talk}) para todos os profissionais; e (2) apresentação para o \acrfull{CPA} em
uma das suas reuniões quinzenais, oportunidade em que foi decidido que a
adoção de {\it DevOps} no \acrshort{TCU} deve ser pautada por este modelo.
