Neste capítulo é apresentado um modelo que pode ser utilizado para guiar a
implantação de \textit{DevOps} por novos praticantes. Este modelo foi construído
com base na teoria apresentada no capítulo anterior e está sendo
utilizado para guiar a adoção de DevOps no \acrshort{TCU}. Detalhes da
utilização no \acrshort{TCU} são apresentados em conjunto com as explicações
das etapas do modelo.

We instantiated our model in the Brazilian Federal Court of Accounts (hereafter TCU), a Brazilian Federal
Government institution. TCU was bogged down in implanting specific DevOps tools, repeating the same non-DevOps problems, with
conflicts between development and operations teams about how to divide the responsibilities related to different facets in the intersection between software development and software provisioning.
When instantiated, our model helped TCU to change its focus to
improve the collaboration between teams, and to use the tooling
to support (rather than being the goal of) the entire process.
