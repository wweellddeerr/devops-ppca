Neste capítulo é apresentado um modelo que pode ser utilizado para guiar a
implantação de \textit{DevOps} por novos praticantes. Este modelo foi construído
com base na teoria apresentada no capítulo anterior e está sendo
utilizado para fomentar a adoção de DevOps no \acrshort{TCU}. É detalhada a
realização de um grupo focal \cite{focus_group_handbook,shull2007guide} para
identificar a percepção dos profissionais envolvidos tanto a respeito da adoção
de {\it DevOps} em geral, quanto da aplicabilidade e utilidade do modelo proposto.
Este capítulo contém ainda detalhamento técnico de ações desenvolvidas no
\acrshort{TCU} no intuito de ampliar a utilização de {\it DevOps} no
desenvolvimento das aplicações corporativas.

\section{A Adoção de DevOps \acrshort{TCU}}
We instantiated our model in the Brazilian Federal Court of Accounts (hereafter TCU), a Brazilian Federal
Government institution. TCU was bogged down in implanting specific DevOps tools, repeating the same non-DevOps problems, with
conflicts between development and operations teams about how to divide the responsibilities related to different facets in the intersection between software development and software provisioning.
When instantiated, our model helped TCU to change its focus to
improve the collaboration between teams, and to use the tooling
to support (rather than being the goal of) the entire process.

\subsection{Grupo Focal}

Para obtenção da avaliação empírica sobre a adoção de {\it DevOps} no \acrshort{TCU},
bem como da aplicabilidade e utilidade do modelo descrito,
proposta como objetivo deste trabalho de pesquisa, foi utilizado o método qualitativo
de pesquisa \emph{grupo focal} \cite{focus_group_handbook}.

A população envolvida no processo de adoção de \textit{DevOps} no TCU é
relativamente pequena (em torno de 12 desenvolvedores e 6 profissionais do time
de operações) e há um maior número de profissionais nos times de
desenvolvimento do que no time de operações. Para que as percepções obtidas
na avaliação não possuíssem o viés do time de desenvolvimento, foi adotada a
estratégia de ouvir um número similar de profissionais de cada um dos times.
Com essa limitação do número de profissionais do time de desenvolvimento que
poderiam ser ouvidos na pesquisa, o número total de participantes na amostra
(estimado entre 8 e 12 pessoas) mostrou-se pequeno para realização de um
questionário.

Ao mesmo tempo, a realização de entrevistas, além de alongar a obtenção do
\textit{feedback} desejado, poderia gerar um grande número de respostas
repetitivas, uma vez que são todos profissionais envolvidos no mesmo processo
de adoção de \textit{DevOps}.

Grupo focal emergiu como método de pesquisa nas ciências sociais nos anos 1950
e atualmente é amplamente utilizado, por exemplo, em estudos sociológicos,
pesquisas de mercado, planejamento de produtos e estudos de usabilidade de
sistemas \cite{shull2007guide}. Morgan \cite{morgan1996focus} define grupo focal
como uma técnica de pesquisa que coleta dados através da interação de grupo em
um tópico específico determinado pelo pesquisador.

Segundo F. Shull et al. \cite{shull2007guide}, grupos focais tipicamente
possuem entre três e doze participantes e são projetados para obter percepções
pessoais de membros de um ou mais grupos envolvidos em uma área definida de
interesse de pesquisa e possuem como benefícios a produção de informações
cândidas, muitas vezes perspicazes, com um baixo custo e rápida execução.
Estas características tornam o grupo focal uma alternativa adequada aos
propósitos desta pesquisa. Ainda segundo os autores, a discussão é guiada e
facilitada por um pesquisador-moderador que segue uma estrutura predefinida
de questionamentos.

\begin{table}[hb!]
\centering
\caption{Participantes do Grupo Focal}
\label{tabela_participantes_grupo_focal}
\begin{tabular}{|p{1cm}|p{4cm}|p{3cm}|p{7cm}|} \hline
{\bf P\#} & {\bf Time de atuação} & {\bf Formação} & {\bf Experiência}\\ \hline
P1 & Desenvolvimento & Graduação & Atua há 3 anos no desenvolvimento de {\it software} do \acrshort{TCU} e possui 9 anos de experiência prévia \\ \hline
P2 & Desenvolvimento & Pós-graduação & Atua há 6 anos no desenvolvimento de {\it software} do \acrshort{TCU} e possui 7 anos de experiência prévia \\ \hline
P3 & Operações & Graduação & Atua há 3 anos no time de operações do \acrshort{TCU} e possui 8 anos de experiência prévia \\ \hline
P4 & Operações & Graduação & Atua há 3 anos no time de operações do \acrshort{TCU} e possui 10 anos de experiência prévia \\ \hline
\end{tabular}
\end{table}

Seguindo uma estrutura similar ao realizado por Lehtola et al. \cite{requirementes_priorization_in_practice},
o grupo focal foi conduzido da seguinte forma: (1) o pesquisador-moderador atuou
como facilitador do grupo focal fornecendo aos participantes três tópicos de
discussão, listados na tabela \ref{tabela_topicos}. No início da discussão de
cada tópico, as perguntas foram apresentadas aos participantes que escreveram
suas ideias e palavras-chave em notas {\it post-it}. Depois disso, as notas
foram postas em um quadro branco e serviram como ponto de partida para a
realização de discussões sobre o respectivo tópico com o propósito de se obter
conclusões a respeito da respectiva pergunta.

\begin{table}[hb!]
\centering
\caption{Tópicos do Grupo Focal}
\label{tabela_topicos}
\begin{tabular}{|p{0.3cm}|p{6.4cm}|p{7cm}|} \hline
& \textbf{Tópico} & \textbf{Perguntas} \\ \hline

1 & Estágio atual da adoção de {\it DevOps} no \acrshort{TCU} &
1. Quais ações já desenvolvidas no \acrshort{TCU} você considera que fazem parte da adoção de {\it DevOps}?\newline\newline
2. Quais problemas existentes anteriormente foram resolvidos por essas ações? \\ \hline

2 & Aplicabilidade e utilidade do modelo proposto &
1. Você considera que este modelo que foi aprovado no \acrshort{CPA} tem contribuído para a adoção de {\it DevOps} no \acrshort{TCU}?\newline\newline
2. Caso positivo, quais as principais contribuições? \\ \hline

3 & Desafios enfrentados e próximos passos na adoção de {\it DevOps} &
1. Quais os principais desafios que a adoção de {\it DevOps} enfrenta atualmente no \acrshort{TCU}?\newline\newline
2. Quais os próximos passos para a adoção de {\it DevOps} no \acrshort{TCU}?\\ \hline

\end{tabular}
\end{table}

\subsection{Estágio Atual da Adoção de DevOps no TCU}

\subsection{Aplicabilidade e Utilidade do Modelo proposto}

\subsection{Desafios enfrentados e Próximos Passos na Adoção de DevOps}
