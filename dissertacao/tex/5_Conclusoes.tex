Neste capítulo são apresentadas as considerações finais do trabalho de
dissertação contendo uma retrospectiva da pesquisa (seção \ref{secao_retro}) e
a indicação de limitações e possíveis trabalhos futuros (seção \ref{limitacoes}).

\section{Retrospectiva da Pesquisa}\label{secao_retro}

Os problemas existentes no \acrshort{TCU} decorrentes da baixa colaboração entre
seus times de desenvolvimento e operações durante o desenvolvimento de {\it software}
culminaram na inclusão de um indicador no \acrshort{PDTI} do órgão
tratando especificamente da ampliação do uso de {\it DevOps} no desenvolvimento
das suas aplicações corporativas. Decorrente da necessidade de cumprimento deste
indicador, o \acrshort{CPA}, comitê responsável por orientar e padronizar as
decisões tecnológicas e arquiteturais do \acrshort{TCU}, direcionou que esse
processo de adoção de {\it DevOps} deveria ser conduzido pautado no intercâmbio
de experiências com praticantes de mercado que foram bem sucedidos na adoção de
{\it DevOps}.

Alinhado com as necessidades do \acrshort{TCU}, neste trabalho de pesquisa foram
ouvidos 15 praticantes que contribuíram para a adoção de {\it DevOps} em suas
companhias e, por meio da utilização da metodologia {\it grounded theory}, foi
produzida uma teoria que explica como {\it DevOps} foi adotado nessas companhias
e um modelo para guiar novas adoções de {\it DevOps} com base nessas
experiências. Por meio de uma análise da literatura realizada após a condução
do estudo, foi identificado que a teoria e o modelo representam uma contribuição
acadêmica para a compreensão de {\it DevOps}, uma vez que: (1) não foi
identificado outro estudo que tenha investigado a adoção de {\it DevOps}
utilizando {\it grounded theory} como metodologia; (2) o número de companhias
estudadas é maior que o presente nos demais trabalhos, ampliando os cenários
dos quais foram extraídas compreensões sobre a adoção de {\it DevOps}; e (3) as
explicações a respeito de como os elementos {\it DevOps} se relacionam em um
processo de adoção, constitui um acréscimo à compreensão previamente existente.

Os resultados da pesquisa ilustram que a adoção de {\it DevOps} envolve 34
conceitos agrupados em sete categorias: \cat{agilidade}, \cat{automação},
\cat{compartilhamento e transparência} \cc, \cat{medição contínua},
\cat{garantia da qualidade} e \cat{resiliência}. A categoria principal da adoção de
{\it DevOps} é \cc. Algumas das categorias (\cat{automação} e \cat{compartilhamento e
transparência}) atuam como \emph{facilitadores} para formação desta \cc.
Outras categorias (\cat{agilidade} e \cat{resiliência}) são explicadas como
\emph{saídas} esperadas da adoção de {\it DevOps}. Por fim, outras duas
categorias (\cat{medição contínua} e \cat{garantia da qualidade}) atuam tanto
como \emph{facilitadores} quanto como \emph{saídas} na adoção de {\it DevOps}.
Essencialmente, este modelo simplifica o entendimento de como se organiza o
complexo conjunto de elementos que são parte da adoção de {\it DevOps},
possibilitando que o processo seja mais direto e com menores chances de focar
em coisas erradas.

As experiências coletadas durante a produção deste estudo já têm sido
efetivamente aplicadas no processo de adoção de {\it DevOps} do \acrshort{TCU}.
Já o modelo em si, foi introduzido por meio da apresentação em uma {\it tech talk}
para os profissionais e envolvidos no desenvolvimento de {\it software} do \acrshort{TCU}
e apresentado ao \acrshort{CPA} para análise. O \acrshort{CPA} entendeu que o
processo de adoção de {\it DevOps} do \acrshort{TCU} deve se pautar no modelo apresentado.
Embora a formação de percepções mais concretas sobre o papel efetivo do modelo
durante o processo de adoção de {\it DevOps} requeira tempo, as percepções de
profissionais do \acrshort{TCU} tanto a respeito do estado atual da adoção de
{\it DevOps} como um todo, quanto da aplicabilidade e utilidade do modelo foram
coletadas por meio da realização de um grupo focal. No grupo focal foram
identificadas (1) as principais ações já desenvolvidas no âmbito da adoção de
{\it DevOps} no \acrshort{TCU}; (2) quais as contribuições que já podem ser
notadas a respeito da utilização do modelo, e (3) os desafios enfrentados e
próximos passos na adoção de {\it DevOps}.

\section{Limitações e Sugestões para Trabalho Futuro}\label{limitacoes}

Embora tenha-se buscado a experiência de praticantes em companhias localizadas
em outros países, a busca por interessados em contribuir com uma pesquisa
acadêmica é uma tarefa ainda mais desafiadora quando se vai além das fronteiras
do Brasil. O resultado disso é que dois terços dos entrevistados são de
companhias brasileiras. Assim sendo, uma primeira sugestão de trabalho
futuro é a expansão da quantidade de entrevistas em outros países e contextos
para validar se a saturação teórica aqui atingida foi afetada por questões
geográficas.

Durante o trabalho buscou-se investigar organizações cujo processo de
adoção de {\it Dev-Ops} foi bem sucedido. Todavia, a percepção de o processo ter
sido bem sucedido é subjetiva e individual dos praticantes, não há uma maneira
quantificável de se verificar. Ademais, embora os entrevistados tenham fornecido algumas
ideias sobre caminhos incorretos que podem comprometer a adoção de {\it DevOps},
para se identificar orientações a respeito do que não deve ser feito, o ideal
seria investigar organizações onde a adoção de {\it DevOps} não foi bem
sucedida, o que representa mais uma oportunidade para trabalho futuro, onde
poderia inclusive se avaliar se o modelo aqui proposto teria sido útil nesses
cenários.

Existe uma considerável sobreposição de resultados com os de trabalhos
relacionados. Esse é um risco que os estudos utilizando {\it grounded theory}
enfrentam, por conta da limitação de exposição à literatura no início do
trabalho. Embora a sobreposição de fato exista, os resultados foram reintegrados
à literatura mostrando as semelhanças e ilustrando os pontos em que se
complementam.

Organizações possuem particularidades. Isso dificulta a geração de um
modelo mais prescritivo, com detalhamento de práticas que precisam
necessariamente ser aplicadas. Enquanto que em algumas organizações é possível
existir um alto grau de automação, com experimentação e entrega contínua, em
outras o cenário pode ser diferente. O resultado dessa limitação é que o modelo
proposto contém alguns aspectos que são decididos no contexto de cada
organização, como em quais facilitadores investir para fomentar a cultura de
colaboração ou de que maneira explorar as saídas do processo.

A única evidência de utilidade e aplicabilidade do modelo proposto foi
obtida durante a realização de um grupo focal no \acrshort{TCU}. A participação
do pesquisador como moderador do grupo focal certamente pode ter influenciado
a opinião dos demais profissionais no intuito de considerar que o modelo é útil.
Todavia, as próprias ações concretas apontadas, atenuam essa possibilidade. Isto
posto, trabalhos futuros podem conduzir investigações a respeito da
aplicabilidade e utilidade do modelo em cenários distintos.

Embora {\it grounded theory} ofereça procedimentos rigorosos para coleta e
análise de dados, as pesquisas qualitativas em geral estão sujeitas a conterem
algum grau de viés do pesquisador. Certamente outros pesquisadores podem formar
uma interpretação e uma teoria diferentes depois de analisar os mesmos dados,
porém, acredita-se que ao menos as principais percepções seriam preservadas.

Os estudos utilizando {\it grounded theory} em geral não reinvidicam ser
definitivos, a teoria resultante deve ser modificável em outros contextos \cite{hoda2012developing}.
A implicação disso é que não se reivindica que a teoria aqui apresentada seja
absoluta ou final. São bem-vindas extensões da teoria baseadas em aspectos não
percebidos, detalhes mais sutis das categorias e conceitos atuais, ou ainda
potencial descoberta de novas dimensões ou conceitos a partir de estudos
futuros.
