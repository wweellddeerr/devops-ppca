Neste capítulo são apresentadas as considerações finais do trabalho de
dissertação contendo um resumo da pesquisa (seção \ref{resumo_pesquisa})
e a indicação de limitações e possíveis trabalhos futuros (seção \ref{limitacoes}).

\section{Resumo da Pesquisa}\label{resumo_pesquisa}

\section{Limitações e Sugestões para Trabalho Futuro}\ref{limitacoes}

\begin{itemize}
\item Embora tenha-se buscado a experiência de praticantes em companhias
localizadas em outros países, a busca por interessados em contribuir para uma
pesquisa acadêmica em outros países é uma tarefa ainda mais desafiadora do que
conseguí-los no Brasil. O resultado disso é que dois terços dos entrevistados
são de companhias brasileiras. Assim sendo, uma primeira sugestão de trabalho
futuro é a expansão da quantidade de entrevistados em outros países e contextos
para validar se a saturação teórica aqui atingida foi afetada por questões
geográficas.

\item Durante o trabalho buscou-se investigar organizações cujo processo de
adoção de {\it DevOps} foi bem sucedido. Todavia, para obtenção da percepção
de o processo ter sido bem sucedido,
ser bem sucedido é subjetiva e individual dos praticantes, não há uma maneira
quantificável de se verificar. Ademais, embora os entrevistados tenham algumas
ideias sobre caminhos incorretos que podem comprometer a adoção de {\it DevOps},
para se identificar orientações a respeito do que não deve ser feito, o ideal
seria investigar organizações onde a adoção de {\it DevOps} não foi bem
sucedida, o que representa mais uma oportunidade para trabalho futuro, onde
poderia ainda ser validado se o modelo aqui proposto teria sido útil nesses
cenários.

\item Existe uma considerável sobreposição de resultados com os de trabalhos
relacionados. Esse é um risco que os estudos utilizando {\it grounded theory}
enfrentam, por conta da limitação de exposição à literatura no início do
trabalho. Embora a sobreposição de fato exista, os resultados foram reintegrados
à literatura mostrando as semelhanças e ilustrando os pontos em que se
complementam.

\item Organizações possuem particularidades. Isso dificulta a geração de um
modelo mais prescritivo, com detalhamento de práticas que precisam
necessariamente ser aplicadas. Enquanto que em algumas organizações é possível
existir um alto grau de automação, com experimentação e entrega contínua, em
outras o cenário pode ser diferente. O resultado dessa limitação é que o modelo
proposto contém alguns aspectos que são decididos no contexto de cada
organização, como em quais facilitadores investir para fomentar a cultura de
colaboração ou de que maneira explorar as saídas do processo.

\item A única evidência de utilidade e aplicabilidade do modelo proposto foi
obtida durante a realização de um grupo focal no \acrshort{TCU}. A participação
do pesquisador como moderador do grupo focal, certamente pode ter influenciado
a opinião dos demais profissionais no intuito de considerar que o modelo é útil.
Todavia, as próprias ações concretas apontadas, atenuam essa possibilidade. Isto
posto, trabalhos futuros podem conduzir investigações a respeito da
aplicabilidade e utilidade do modelo em cenários distintos. 

\item Embora {\it grounded theory} ofereça procedimentos rigorosos para coleta e
análise de dados, as pesquisas qualitativas em geral estão sujeitas a conterem
algum grau de viés do pesquisador. Certamente outros pesquisadores podem formar
uma interpretação e uma teoria diferentes depois de analisar os mesmos dados,
mas acredita-se que ao menos as principais percepções seriam preservadas.

\item Os estudos utilizando {\it grounded theory} em geral não reinvidicam ser
definitivos, a teoria resultante deve ser modificável em outros contextos \cite{hoda2012developing}.
A implicação disso é que não se reivindica que a teoria aqui apresentada seja
absoluta ou final. São bem-vindas extensões da teoria baseadas em aspectos não
percebidos ou detalhes mais sutis das categorias atuais ou potencial descoberta
de novas dimensões ou conceitos a partir de estudos futuros.
\end{itemize}
