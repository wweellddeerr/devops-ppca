\subsection{Trabalhos Relacionados}

Seguindo os preceitos da metodologia {\it Grounded Theory}, o contato com a
literatura existente sobre {\it DevOps} foi aprofundado apenas ao final da
produção da teoria apresentada nas seções anteriores. Aqui são apresentados
os principais trabalhos relacionados com indicação dass principais semelhanças
e diferenças entre o trabalho aqui apresentado e os demais.

O trabalho de Smeds et al.~\cite{devops_a_definition} propôs a existência de
{\it enablers} técnicos e culturais que compõem a definição de {\it DevOps} e
{\it capabilities} resultantes do uso das técnicas {\it DevOps}. Estes
{\it enablers} e {\it capabilities} se assemelham aos \emph{facilitadores} e
\emph{saídas DevOps} apresentados aqui. Ademais, alguns conceitos coincidem:
(1) automação de testes, {\it deployment}, monitoramento, infraestrutura e
recuperação; (2) integração, testes e {\it deployment} contínuos; (3) recuperação
de falhas em serviços sem {\it delay}; e (4) comunicação constante e {\it effortless}.
Todavia, algumas outras diferenças também podem ser identificadas: (1) o não
agrupamento de conceitos em categorias, a maior parte dos facilitadores são
relacionados a automação; (2) apresentam os aspectos culturais como responsáveis
por contribuir com a formação de capacidades {\it DevOps} e não como a
preocupação principal de {\it DevOps}; (3) o foco da parte empírica do estudo,
onde foram conduzidas entrevistas semi-estruturadas com 13 funcionários de uma única
companhia cujo processo de adoção de {\it DevOps} estava em um estágio inicial,
foi em construir uma lista de possíveis impedimentos na adoção de {\it DevOps} e
não em prover orientações para novos praticantes a respeito de \emph{como} se
adotar {\it DevOps}.

Já Lwakatare et al.~\cite{dimensions_of_devops,extending_dimensions} propuseram
um {\it framework} conceitual para explicar ``{\it DevOps} como um fenômeno''.
O {\it framework} foi construído por meio de uma revisão de literatura multivocal
utilizando dados da literatura cinza combinada com entrevistas a praticantes de
três companhias que estavam aplicando práticas {\it DevOps} e organizado em
torno de cinco dimensões (colaboração, automação, cultura, monitoramento e
medição) e estas dimensões foram apresentadas com práticas relacionadas. O foco
deste trabalho também não foi em investigar a adoção de {\it DevOps} na prática.
A principal semelhança identificada foi a presença de todas as dimensões aqui
também, embora \emph{cultura} e \emph{colaboração} sejam aqui apresentadas como
uma única abstração (cultura de colaboração) e os conceitos relacionados a
monitoramento estejam aqui presentes como parte da categoria de \cat{medição
contínua} e não em duas categorias separadas. Ademais, infraestrutura como
código foi lá apresentada como um conceito relacionado a automação enquanto que
aqui está relacionado a \cat{compartilhamento e transparência}.

Fran\c{c}a et al.~\cite{characterizing_devops}, por sua vez, realizaram uma
revisão multivocal de literatura, utilizando dados de diversas fontes, incluindo
a literatura cinza, com procedimentos de análise qualitativa de {\it grounded
theory} para prover uma definição para {\it DevOps} e identificar práticas
{\it DevOps}, habilidades requeridas, características, benefícios, e problemas
motivando a adoção de {\it DevOps}. O uso de {\it grounded theory} representa
uma semelhança entre os estudos, embora, neste caso tenham sido utilizados
apenas os procedimentos de análise e não a abordagem como um todo. Os resultados
também apresentam algumas semelhanças: (1) automação, compartilhamento, medição
e garantia da qualidade são apresentados em ambos os estudos como categorias;
(2) a categoria \emph{aspectos sociais} apresentada lá é similar à categoria
\emph{cultura de colaboração} apresentada aqui. Como diferenças, destacam-se:
(1) o trabalho não realizou entrevistas para coletas de dados; (2) os
resultados lá são apresentados como um conjunto de princípios {\it DevOps}
enquanto que aqui são conjuntos de facilitadores e saídas {\it DevOps}; e (3)
a categoria {\it leanness} é apresentada lá e não aqui, enquanto que
\emph{resiliência} é apresentada aqui e não lá.

A primeira parte do trabalho de Erich et al. \cite{qualitative_devops_journalsw_17}
se concentrou em identificar elementos que compõem {\it DevOps} por meio de uma
revisão de literatura. Cinco das sete categorias apresentadas como resultado
dessa parte do trabalho também foram apresentadas aqui: cultura de colaboração,
automação, medição, compartilhamento e garantia da qualidade. Uma segunda parte
deste trabalho se concentrou em investigar como {\it DevOps} é implementado na
prática por meio de entrevistas com praticantes de seis companhias. O propósito
inicial e a utilização de entrevistas para coleta de dados são similares aos
aqui apresentados, todavia, as percepções lá apresentadas foram apresentaram de
maneira individual para cada companhia, não direcionando os pontos comuns com o
intuito de formular um modelo geral de como {\it DevOps} pode ser adotado.

Já no trabalho de Hamunen \cite{challenges_in_adopting_devops}, a primeira parte
foi dedicada a identificar os componentes chave de {\it DevOps} por meio de
consultas à literatura. Como resultado foi apresentado o modelo ``CALMS'':
cultura, automação, {\it lean}, medição e compartilhamento. À exceção de {\it lean},
os demais componentes aparecem como categorias da adoção de {\it DevOps} aqui
também. A segunda parte buscou identificar os principais desafios que as
organizações enfrentam ao lidar com {\it DevOps}, onde foram realizadas nove
entrevistas com profissionais com experiência em iniciativas {\it DevOps} no
mercado finlandês. Embora o foco deste trabalho tenha sido em identificar os
desafios, há uma sobreposição quando, mesmo que secundariamente, são providos
meios de se superar estes desafios. Por exemplo, é mencionado que o principal
avanço tecnológico em {\it DevOps} é a criação de um {\it pipeline} automatizado
de entrega contínua, que contém aspectos dos facilitadores {\it DevOps} presentes
aqui. Destaca-se também que o estudo aponta como necessária para superar os
desafios de {\it DevOps} a adaptação dos processos organizacionais, em uma
linha similar ao proposto na categoria aqui denominada de cultura de colaboração.

O estudo de Feijter et al. \cite{feijter2017towards} resultou na proposição de
um modelo de maturidade para adoção de {\it DevOps}. Este estudo possui diversas
semelhanças com o aqui realizado: (1) o foco foi exatamente em explicar como evoluir
no nível de maturidade de {\it DevOps} em alguma organização; (2) foram realizadas
entrevistas com praticantes para obtenção desse modelo de maturidade; (3) a motivação
para a pesquisa envolve a necessidade de uma companhia de desenvolvimento de
{\it software}; (4) os procedimentos de análise de dados envolvem um modelo de
comparação constante que em muito se assemelha com os procedimentos de codificação
de {\it grounded theory}; e (5) foram realizados {\it workshops} com especialistas
da organização para avaliar níveis de maturidade de {\it DevOps} e validar
a aplicabilidade do modelo. Os modelos diferem apenas quanto ao método de
pesquisa utilizada

\subsubsection{Considerações Gerais a Respeito dos Trabalhos Relacionados}

Em relação aos trabalhos cujo propósito é de apresentar elementos {\it DevOps} e
práticas relacionadas, a teoria apresenta detalhes que ajudam a
compreender como podem ser respondidas algumas questões práticas a
respeito da adoção de {\it DevOps} que permaneciam em aberto: (1) Existe um caminho
recomendado para se adotar \textit{DevOps}? (2) Já que \textit{DevOps} é
constituído de múltiplos elementos, eles possuem a mesma relevância quando se
adota \textit{DevOps}? (3) Qual é o papel desempenhado por cada um desses
elementos - tais como medição, compartilhamento e automação - em uma adoção de
\textit{DevOps}?

De maneira geral, é possível destacar que: (1) nenhum dos estudos utilizou
{\it grounded theory} para investigar o processo de adoção de {\it DevOps};
(2) exceto o trabalho de Feijter et al. \cite{feijter2017towards}, os demais
focaram em caracterizar {\it DevOps} de maneira geral ou desafios relacionados
a {\it DevOps} e não em produzir orientações relacionadas ao processo de adoção
em si; (3) entre os trabalhos que realizaram entrevistas, o que maior número
de companhias distintas ouviu foi o de Hamunen \cite{challenges_in_adopting_devops},
com 9 companhias, enquanto que aqui foram coletadas experiência de 15 companhias.
