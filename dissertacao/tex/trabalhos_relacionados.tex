\section{Trabalhos Relacionados}

Após construída, a teoria descrita acima foi reintegrada com a literatura
existente sobre {\it DevOps}. A subseção \ref{secao_caracterizacoes_devops}
apresenta as semelhanças e diferenças de trabalhos que foram construídos
visando caracterizar {\it DevOps}, tipicamente por meio da identificação de
elementos relacionados. Já na subseção \ref{secao_adocao_6_companhias} é
feita a comparação com um estudo cujo foco foi similar, de caracterizar a
adoção de {\it DevOps} em companhias de mercado. A seção \ref{como_superar_desafios}
apresenta uma comparação de um trabalho cujo foco foi identificar desafios na
adoção de {\it DevOps} e que apresentou algumas orientações a respeito de como
superar estes desafios. Por fim, na subseção \ref{modelo_maturidade} a teoria é
comparada a um trabalho relacionado no qual é proposto um modelo de maturidade
para adoção de {\it DevOps} e também detalhes de como progredir nesse modelo,
aumentando o nível de maturidade de {\it DevOps} de uma companhia.

\subsection{Trabalhos com Foco em Caracterizar DevOps}\label{secao_caracterizacoes_devops}

Conforme mencionado acima, os trabalhos comparados nessa subseção tiveram como
foco principal caracterizar {\it DevOps}, o que difere do propósito da produção
da teoria que foi de caracterizar {emph como DevOps é adotado}. Embora
exista essa diferença de foco, existe uma sobreposição de resultados que se
justifica por ser natural que os elementos {\it DevOps} apareçam no processo de
adoção de {\it DevOps}. Além das diferenças de foco, existem algumas
diferenças metodológicas em relação ao estudo aqui proposto. Essas diferenças
são resumidas na tabela \ref{related_work_table}.

\begin{table}[hb!]
\centering
\caption{Comparação de Metodologia e Foco dos Trabalhos Relacionados}
\label{related_work_table}
\begin{tabular}{|p{3cm}|p{6cm}|p{6cm}|}
\hline

\textbf{Estudo}
& \textbf{Abordagem de Pesquisa}
& \textbf{Foco do Trabalho} \\

\hline

\textbf{Este estudo}

& Abordagem {\it grounded theory} entrevistando praticantes que contribuíram
para a adoção de {\it DevOps} em 15 companhias de 5 países.

& Explicar {\bf como} {\it DevOps} tem sido adotado de maneira bem sucedida na prática de mercado;\newline
\newline Prover orientações que possam ser usadas em novas adoções de {\it DevOps}. \\

\hline

\textbf{J. Smeds et al.~\cite{devops_a_definition}}

& Revisão sistemática de literatura; \newline \newline Entrevistas semi-estruturadas
com 13 funcionários de uma única companhia cujo processo de adoção de {\it DevOps}
estava em um estágio inicial.

& Identificar na literatura as principais características que definem {\it DevOps}
\newline \newline
Construir uma lista de possíveis impedimentos na adoção de {\it DevOps} por meio
de um estudo empírico. \\

\hline

\textbf{Lwakatare et al.~\cite{extending_dimensions}}

&
Revisão de literatura multivocal utilizando dados da literatura cinza; \newline \newline
Três entrevistas com praticantes de desenvolvimento de {\it software} em uma
companhia que estava aplicando práticas {\it DevOps} em um projeto.

& Identificar como praticantes descrevem {\it DevOps} como um fenômeno; \newline
\newline Identificar as práticas {\it DevOps} de acordo com os praticantes de
desenvolvimento de {\it software}. \\

\hline

\textbf{Fran\c{c}a et al.~\cite{characterizing_devops}}

& Revisão de literatura multivocal com procedimentos de análise qualitativa de
{\it grounded theory}.

& Prover uma definição para {\it DevOps}; \newline \newline
Identificar práticas {\it DevOps}, habilidades requeridas, características,
benefícios, e problemas motivando a adoção de {\it DevOps}. \\

\hline

\end{tabular}
\end{table}

Em relação aos trabalhos apresentados nessa subseção, a teoria apresentam detalhes que ajudam a
compreender como podem ser respondidas algumas questões práticas a
respeito da adoção de {\it DevOps} que permaneceram em aberto: (1) Existe um caminho
recomendado para se adotar \textit{DevOps}? (2) Já que \textit{DevOps} é
constituído de múltiplos elementos, eles possuem a mesma relevância quando se
adota \textit{DevOps}? (3) Qual é o papel desempenhado por cada um desses
elementos - tais como medição, compartilhamento e automação - em uma adoção de
\textit{DevOps}?

\subsection{A Caracterização da Adoção de DevOps em 6 Companhias Proposta por Erich et al. \cite{qualitative_devops_journalsw_17}}\label{secao_adocao_6_companhias}

Systematic literature review;
Interviews with practitioners from 6 companies across 3 countries.

Identify how literature defines DevOps;
Investigate how DevOps is being implemented in practice.

O trabalho de Erich et al. \cite{qualitative_devops_journalsw_17} investigou
o processo de adoção de \textit{DevOps} por meio de entrevistas com praticantes
de seis companhias. Todavia, as percepções obtidas não detalham a maneira como
as companhias aplicaram os elementos de \textit{DevOps} ao longo do processo,
e são apresentadas de maneira individualizada, não se colocando em perspectiva
o que há de comum no entendimento dessas companhias. Os dados de como cada
companhia relata ter implementado \textit{DevOps} são apresentados de maneira
sucinta e os principais pontos são descritos a seguir.

Na primeira companhia, é relatado que o pessoal de desenvolvimento e
operações passou a trabalhar de maneira conjunta diariamente, foi criado um
novo cargo na companhia chamado \textit{DevOps Engineer}, o pessoal de gestão
foi treinado para ser capaz de gerenciar tanto pessoas de desenvolvimento como
de operações, o pessoal de recursos humanos alterou os processos de avaliação e,
por fim, a organização está automatizando seus processos de infraestrutura, não
são apresentados detalhes de como isso está sendo feito.

Na segunda companhia, \textit{DevOps} é visto como uma extensão de
\textit{Scrum} e \textit{Lean}. A adoção de \textit{DevOps} lá foi guiada por
dois princípios: colaboração diária e ambiente de trabalho compartilhado.
Havia a premissa de que o \textit{software} deve ir para produção a cada
iteração. É relatado então que a organização usou \textit{frameworks}
comerciais para guiar a implantação de \emph{entrega contínua}, também sem maiores
detalhamentos de como foi feito.

Na terceira companhia, é relatado que três times que trabalhavam em
funcionalidades separadas foram agrupados em um \emph{time DevOps}. A organização
passou então a experimentar \textit{entrega contínua}, mas ainda estava em
um estágio inicial quando a entrevista foi realizada.

A quarta companhia criou uma única equipe de \textit{DevOps} para
experimentar ferramentas, princípios e práticas \textit{DevOps}. Os
membros dessa equipe possuem habilidades multidisciplinares. É relatado ainda
o uso de ferramentas para controle de versão e o desenvolvimento de ferramentas
próprias para criar facilmente ambientes de nuvem.

Na quinta companhia, é relatado que \textit{DevOps} é visto como
um papel responsável por gerenciamento de incidentes, gerenciamento de
capacidade, gerenciamento de riscos e suporte ao processo de criação.
O uso de computação em nuvem ocasionou a necessidade de se existir uma
abordagem mais sistemática para execução das atividades de operações, exigindo
que o pessoal responsável aprendesse técnicas de desenvolvimento de
\textit{software}. O time de desenvolvimento foi dividido em vários subtimes,
um dos quais focado em \textit{DevOps}, com responsabilidade de apoiar os demais
times. São também enumerados alguns princípios e práticas utilizados na companhia
durante a adoção de \textit{DevOps}.

Por fim, com relação a como a sexta companhia implementou \textit{DevOps},
é apresentado um detalhamento de como ocorre a \emph{integração contínua} na
organização. As evoluções de código são feitas por meio de \textit{pull
requests} que após aprovadas são integradas ao código principal. Após a
integração dos códigos das \textit{pull requests}, um \textit{pipeline} de entrega
contínua é executado para publicar o \textit{software}.

\subsection{As Considerações de Hamunen \cite{challenges_in_adopting_devops} Sobre Como Superar os Desafios na Adoção de DevOps}\label{como_superar_desafios}

\subsection{O Modelo de Maturidade Proposto por Feijter et al. \cite{feijter2017towards}}\label{modelo_maturidade}
