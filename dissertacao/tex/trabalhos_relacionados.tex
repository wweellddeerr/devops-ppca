\subsection{Trabalhos Relacionados}

O trabalho de Smeds et al.~\cite{devops_a_definition} propôs a existência de
{\it enablers} técnicos e culturais que compõem a definição de {\it DevOps} e
{\it capabilities} resultantes do uso das técnicas {\it DevOps}. Estes
{\it enablers} e {\it capabilities} se assemelham aos \emph{facilitadores} e
\emph{saídas DevOps} apresentados aqui. Ademais, alguns conceitos coincidem:
(1) automação de testes, {\it deployment}, monitoramento, infraestrutura e
recuperação; (2) integração, testes e {\it deployment} contínuos; (3) recuperação
de falhas em serviços sem {\it delay}; e (4) comunicação constante e {\it effortless}.
Todavia, algumas outras diferenças também podem ser identificadas: (1) o não
agrupamento de conceitos em categorias, a maior parte dos facilitadores são
relacionados a automação; (2) apresentam os aspectos culturais como responsáveis
por contribuir com a formação de capacidades {\it DevOps} e não como a
preocupação principal de {\it DevOps}; (3) o foco da parte empírica do estudo,
onde foram conduzidas entrevistas semi-estruturadas com 13 funcionários de uma única
companhia cujo processo de adoção de {\it DevOps} estava em um estágio inicial,
foi em construir uma lista de possíveis impedimentos na adoção de {\it DevOps} e
não em prover orientações para novos praticantes a respeito de \emph{como} se
adotar {\it DevOps}.

Já Lwakatare et al.~\cite{dimensions_of_devops,extending_dimensions} propuseram
um {\it framework} conceitual para explicar ``{\it DevOps} como um fenômeno''.
O {\it framework} foi construído por meio de uma revisão de literatura multivocal
utilizando dados da literatura cinza combinada com entrevistas a praticantes de
três companhias que estavam aplicando práticas {\it DevOps} e organizado em
torno de cinco dimensões (colaboração, automação, cultura, monitoramento e
medição) e estas dimensões foram apresentadas com práticas relacionadas. O foco
deste trabalho também não foi em investigar a adoção de {\it DevOps} na prática.
A principal semelhança identificada foi a presença de todas as dimensões aqui
também, embora \emph{cultura} e \emph{colaboração} sejam aqui apresentadas como
uma única abstração (cultura de colaboração) e os conceitos relacionados a
monitoramento estejam aqui presentes como parte da categoria de \cat{medição
contínua} e não em duas categorias separadas. Ademais, infraestrutura como
código foi lá apresentada como um conceito relacionado a automação enquanto que
aqui está relacionado a \cat{compartilhamento e transparência}.

Fran\c{c}a et al.~\cite{characterizing_devops}, por sua vez, realizaram uma
revisão multivocal de literatura, utilizando dados de diversas fontes () com
procedimentos de análise qualitativa de {\it grounded theory} para prover uma
definição para {\it DevOps} e identificar práticas {\it DevOps}, habilidades
requeridas, características, benefícios, e problemas motivando a adoção
de {\it DevOps}. O uso de {\it grounded theory} representa uma semelhança entre
os estudos, embora, neste caso tenham sido utilizados apenas os procedimentos
de análise e não a abordagem como um todo. Os resultados também apresentam
algumas semelhanças.

Uma parte do trabalho de Erich et al. \cite{qualitative_devops_journalsw_17} também
se concentrou em identificar elementos que compõem {\it DevOps} por meio de uma
revisão de literatura. Uma segunda parte do trabalho de Erich et al. \cite{qualitative_devops_journalsw_17}
se concentrou em investigar como {\it DevOps} é implementado na prática por
meio de entrevistas com praticantes de seis companhias. Todavia, as percepções
lá apresentadas foram apresentaram de maneira individual para cada companhia,
não direcionando os pontos comuns com o intuito de formular um modelo geral de
como {\it DevOps} pode ser adotado.

O trabalho de Hamunen \cite{challenges_in_adopting_devops}

O Modelo de Maturidade Proposto por Feijter et al. \cite{feijter2017towards}

\subsubsection{Considerações Gerais a Respeito dos Trabalhos Relacionados}

Em relação aos trabalhos cujo propósito é de apresentar elementos {\it DevOps} e
práticas relacionadas, a teoria apresenta detalhes que ajudam a
compreender como podem ser respondidas algumas questões práticas a
respeito da adoção de {\it DevOps} que permaneciam em aberto: (1) Existe um caminho
recomendado para se adotar \textit{DevOps}? (2) Já que \textit{DevOps} é
constituído de múltiplos elementos, eles possuem a mesma relevância quando se
adota \textit{DevOps}? (3) Qual é o papel desempenhado por cada um desses
elementos - tais como medição, compartilhamento e automação - em uma adoção de
\textit{DevOps}?

De maneira geral, é possível destacar que: (1) nenhum dos estudos utilizou
{\it grounded theory} para investigar o processo de adoção de {\it DevOps};
(2) exceto o trabalho de Feijter et al. \cite{feijter2017towards}, os demais
focaram em caracterizar {\it DevOps} de maneira geral ou desafios relacionados
a {\it DevOps} e não em produzir orientação para novos praticantes; (3) entre os
trabalhos que realizaram entrevistas, o que maior número de companhias distintas
ouviu foi o de Erich et al. \cite{qualitative_devops_journalsw_17}, com 6
companhias, enquanto que aqui foram coletadas experiência de 15 companhias.
