\section{Trabalhos Relacionados}

\subsection{Trabalhos Parcialmente Relacionados}

Conforme ilustrado no capítulo anterior, os trabalhos de pequisa existentes
apresentam algumas caracterizações de \textit{DevOps} por meio da enumeração
de conceitos, princípios e práticas relacionadas. Apesar de alguns desses
estudos conterem abordagens qualitativas para investigar empresas adotando
\textit{DevOps}, conduzindo entrevistas com praticantes, o foco desses estudos
tipicamente é em caracterizar \textit{DevOps}, ao invés de prover recomendações
na adoção de \textit{DevOps}. Além das diferenças de foco, existem algumas
diferenças metodológicas em relação ao estudo aqui proposto. Essas diferenças
são resumidas na tabela \ref{related_work_table}.

%% Table~\ref{related_work_table} compares the most relevant literature to
%% out work. Since the work of Lwakatare et al.~\cite{extending_dimensions} is an
%% extension of a previous contribution of the authors and her team \cite{dimensions_of_devops_xp_15},
%% we only compared the most recent study at Table~\ref{related_work_table}.

\begin{table}[hb!]
\centering
\caption{Comparação de Metodologia e Foco dos Trabalhos Relacionados}
\label{related_work_table}
\begin{tabular}{|p{3cm}|p{6cm}|p{6cm}|}
\hline

\textbf{Estudo}
& \textbf{Abordagem de Pesquisa}
& \textbf{Foco do Trabalho} \\

\hline

\textbf{Este estudo}

& Grounded Theory study interviewing practitioners which contributed to DevOps
adoption in 15 companies across 5 countries.

& Explain how DevOps has been successfully adopted in practice;\newline
\newline Provide guidelines to be used in new DevOps adoptions. \\

\hline

\textbf{J. Smeds et al.~\cite{devops_a_definition}}

& Systematic literature review; \newline \newline Semi structured interviews
with 13 employees of a single company whose DevOps adoption process was at
an initial stage.

& Identify in literature the main defining characteristics of DevOps; \newline
\newline Build a list of possible impediments to DevOps adoption through an
empirical study. \\

\hline

\textbf{Lwakatare et al.~\cite{extending_dimensions}}

&
Multivocal literature review using data from gray literature; \newline \newline
Three interviews with software practitioners in one company that was applying
DevOps practices in one project.

& Identify how do practitioners describe DevOps as a phenomenon; \newline
\newline Identify the DevOps practices according to software practitioners. \\

\hline

\textbf{Fran\c{c}a et al.~\cite{characterizing_devops}}

& Multivocal literature review with qualitative analysis procedures from
Grounded Theory.

& Provide a DevOps definition; \newline \newline
Identify DevOps practices, required skills, characteristics, benefits and
issues motivating its adoption.  \\

\hline

\textbf{Erich et al.~\cite{qualitative_devops_journalsw_17}}

& Systematic literature review; \newline \newline
Interviews with practitioners from 6 companies across 3 countries.

& Identify how literature defines DevOps; \newline \newline
Investigate how DevOps is being implemented in practice. \\

\hline

\end{tabular}
\end{table}

Como consequência dessas diferenças de foco, algumas questões práticas a
respeito da adoção de DevOps permancem em aberto: (1) Existe um caminho
recomendado para se adotar \textit{DevOps}? (2) Já que \textit{DevOps} é
constituído de múltiplos elementos, eles
possuem a mesma relevância quando se adota \textit{DevOps}? (3) Qual é o papel
desempenhado por cada um desses elementos - tais como medição, compartilhamento e
automação - em uma adoção de \textit{DevOps}? Para responder a essas questões,
é necessário um entendimento holístico dos caminhos seguidos por organizações
que adotaram \textit{DevOps} de uma maneira bem sucedida.

Os trabalhos listados anteriormente concentram-se em caracterizar
\textit{DevOps}, há pouco conteúdo dedicado a \emph{como} adotar \textit{DevOps},
com indicação de caminhos seguidos por companhias que passaram pelo processo
e que eventualmente poderiam ser usados por novos praticantes, com indicações
de lições aprendidas e possíveis abordagens que possam ser seguidas.

\subsection{A Caracterização da Adoção de {\it DevOps} em 6 Companhias Proposta por Erich et al. \cite{qualitative_devops_journalsw_17}}

O trabalho de Erich et al. \cite{qualitative_devops_journalsw_17} investigou
o processo de adoção de \textit{DevOps} por meio de entrevistas com praticantes
de seis companhias. Todavia, as percepções obtidas não detalham a maneira como
as companhias aplicaram os elementos de \textit{DevOps} ao longo do processo,
e são apresentadas de maneira individualizada, não se colocando em perspectiva
o que há de comum no entendimento dessas companhias. Os dados de como cada
companhia relata ter implementado \textit{DevOps} são apresentados de maneira
sucinta e os principais pontos são descritos a seguir.

Na primeira companhia, é relatado que o pessoal de desenvolvimento e
operações passou a trabalhar de maneira conjunta diariamente, foi criado um
novo cargo na companhia chamado \textit{DevOps Engineer}, o pessoal de gestão
foi treinado para ser capaz de gerenciar tanto pessoas de desenvolvimento como
de operações, o pessoal de recursos humanos alterou os processos de avaliação e,
por fim, a organização está automatizando seus processos de infraestrutura, não
são apresentados detalhes de como isso está sendo feito.

Na segunda companhia, \textit{DevOps} é visto como uma extensão de
\textit{Scrum} e \textit{Lean}. A adoção de \textit{DevOps} lá foi guiada por
dois princípios: colaboração diária e ambiente de trabalho compartilhado.
Havia a premissa de que o \textit{software} deve ir para produção a cada
iteração. É relatado então que a organização usou \textit{frameworks}
comerciais para guiar a implantação de \emph{entrega contínua}, também sem maiores
detalhamentos de como foi feito.

Na terceira companhia, é relatado que três times que trabalhavam em
funcionalidades separadas foram agrupados em um \emph{time DevOps}. A organização
passou então a experimentar \textit{entrega contínua}, mas ainda estava em
um estágio inicial quando a entrevista foi realizada.

A quarta companhia criou uma única equipe de \textit{DevOps} para
experimentar ferramentas, princípios e práticas \textit{DevOps}. Os
membros dessa equipe possuem habilidades multidisciplinares. É relatado ainda
o uso de ferramentas para controle de versão e o desenvolvimento de ferramentas
próprias para criar facilmente ambientes de nuvem.

Na quinta companhia, é relatado que \textit{DevOps} é visto como
um papel responsável por gerenciamento de incidentes, gerenciamento de
capacidade, gerenciamento de riscos e suporte ao processo de criação.
O uso de computação em nuvem ocasionou a necessidade de se existir uma
abordagem mais sistemática para execução das atividades de operações, exigindo
que o pessoal responsável aprendesse técnicas de desenvolvimento de
\textit{software}. O time de desenvolvimento foi dividido em vários subtimes,
um dos quais focado em \textit{DevOps}, com responsabilidade de apoiar os demais
times. São também enumerados alguns princípios e práticas utilizados na companhia
durante a adoção de \textit{DevOps}.

Por fim, com relação a como a sexta companhia implementou \textit{DevOps},
é apresentado um detalhamento de como ocorre a \emph{integração contínua} na
organização. As evoluções de código são feitas por meio de \textit{pull
requests} que após aprovadas são integradas ao código principal. Após a
integração dos códigos das \textit{pull requests}, um \textit{pipeline} de entrega
contínua é executado para publicar o \textit{software}.

\subsection{As Considerações de Hamunen \cite{challenges_in_adopting_devops} Sobre Como Superar os Desafios na Adoção de {\bf {\it DevOps}}}

\subsection{O Modelo de Maturidade Proposto por Feijter et al. \cite{feijter2017towards}}
