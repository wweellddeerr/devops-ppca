DevOps is a set of practices and cultural values that aims to reduce the
barriers between development and operations teams in software production. Due
to potential benefits of DevOps and to failed attempts to minimize the impacts
of a little collaboration between its development and operations teams, \acrshort{TCU} is
seeking to extend the use of the DevOps approach in the development of its
enterprise applications.

The Information Technology Master Plan (PDTI) of \acrshort{TCU} indicates that
one of the indicators for the 2017/2018 biennium addresses the extension of
use of the DevOps approach. TCU's Standing Committee on Architecture (CPA) then
defined that successful experiences of adopting DevOps in the market should be
sought and that the adoption of DevOps in TCU should be guided by this market
practice.

This document describes a study using the Grounded Theory approach in its
classical variation to characterize the adoption of DevOps in market practice.
Were interviewed practitioners who contributed to the successful adoption of
DevOps in 15 companies from five countries with different sizes and business
domains.

A model is presented to improve both understanding and guidance about adopting
DevOps. This model increments the existing view of DevOps by explaining the
role and motivation of each element (and their relationships) in the DevOps
adoption process. The model was organized in terms of DevOps enablers categories
and DevOps outcomes categories. Was identified that build a collaborative culture
is the core DevOps adoption concern, contrasting with a possible understanding
that implanting specific DevOps tools or automating specific operations tasks is
enough to achieve DevOps.

The model created in this study was presented to the CPA of TCU and considered
relevant, since then the evolution of the use of DevOps at TCU has been guided
by it. Perceptions of professionals involved regarding the current
stage of adopting DevOps at TCU, as well as the utility and applicability of
the proposed model, were collected through a focus group.

Altogether, the results contribute to (a) generating an adequate understanding
of DevOps adoption, from the perspective of practitioners; and (b) assisting
\acrshort{TCU}, and possibly other companies, in the migration path towards
DevOps adoption.
