DevOps is a set of practices and cultural values that aims to reduce the
barriers between development and operations teams. Due to potential benefits
of DevOps and to failed attempts to minimize the impacts of a litle
collaboration between its development and operations teams, \acrshort{TCU} is
seeking to extend the use of the DevOps approach in the development of its
enterprise applications.

Due to its increasing interest and imprecise definitions, existing research
works have tried to characterize DevOps — mainly using a set of concepts and
related practices.

Nevertheless, little is known about the practitioners understanding about
successful paths for DevOps adoption. The lack of such understanding might
hinder institutions, like \acrshort{TCU}, to adopt DevOps. Therefore, here is
presented a study about DevOps adoption, highlighting the main related concepts
that can contribute to its adoption for new practitioners.

This work builds upon the classical Grounded Theory approach. Practitioners that
contributed to the adoption of DevOps in 15 companies from different domains
and across five countries were interviewed.

A model is presented to improve both the understanding and guidance of
DevOps adoption. This model increments the existing view of DevOps by explaining
the role and motivation of each element (and their relationships) in the
DevOps adoption process. The model was organized in terms of DevOps enabler
categories and DevOps outcome categories. Was identified that build a
collaborative culture is the core DevOps adoption concern, contrasting with
an possible understanding that implanting specific DevOps tools or automating
specific operations tasks is enough to achieve DevOps.

The model created in this study was evaluated through a case study, whose goal
is to increase the maturity level of DevOps adoption at \acrshort{TCU}. The
perceptions of professionals involved in adopting DevOps at TCU were collected
through a focus group.

Altogether, the results contribute to (a) generating an adequate understanding
of DevOps adoption, from the perspective of practitioners; and (b) assisting
\acrshort{TCU} and other institutions in the migration path towards DevOps
adoption.
