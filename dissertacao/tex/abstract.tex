\emph{Background.} DevOps is a set of practices and cultural values
  that aims to reduce the
  barriers between development and operations
  teams during software development. Due to its increasing interest and imprecise
  definitions, existing research works have tried to
  characterize DevOps---mainly using a set of concepts and related practices.

  \emph{Aims.} Nevertheless, little is
  known about the \emph{practitioners understanding}
  about successful paths for DevOps adoption. The lack of such understanding
  might hinder institutions to adopt DevOps practices. Therefore, the goal
  here is to present a theory about DevOps adoption, highlighting how the
  main related concepts have contributed to its adoption in industry.

  \emph{Method.} This work builds upon the classical Grounded
  Theory approach. Were interviewed practitioners
  that contributed to the adoption of DevOps in 15 companies from different
  domains and across five countries. The model is being empirically evaluated
  through a case study, whose goal is to increase the maturity level of
  DevOps adoption at the Brazilian Federal Court of Accounts (TCU).

  \emph{Results.} This paper
  presents a model to improve both the understanding and guidance
  of DevOps adoption. The model increments the existing view of
  DevOps by explaining the role and motivation of each
  category (and their relationships) in the DevOps adoption process.
  This model was organized in terms of \emph{DevOps enabler categories} and
  \emph{DevOps outcome categories}. Was concluded that
  \emph{collaboration} is the core DevOps concern, contrasting with a possible
  understanding that implanting specific tools to \emph{automate building, deployment,
  and infrastructure provisioning and management} is enough to achieve DevOps.

  \emph{Conclusions.} Altogether, the results contribute to (a) generating
  an adequate understanding of DevOps adoption, from the perspective
  of practitioners; and (b) assisting institutions, like \acrshort{TCU}, in the
  migration path towards DevOps adoption.
