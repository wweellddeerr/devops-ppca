DevOps is a set of practices and cultural values that aims to reduce the
barriers between development and operations teams during software development.
Due to its increasing interest and imprecise definitions, existing research
works have tried to characterize DevOps---mainly using a set of concepts and
related practices. Nevertheless, little is known about the \emph{practitioners
understanding} about successful paths for DevOps adoption. The lack of such
understanding might hinder institutions to adopt DevOps practices. Therefore,
the goal here is to present a theory about DevOps adoption, highlighting how the
main related concepts have contributed to its adoption in industry.

This work uses a multi-method approach. Initially a Grounded Theory study was
conducted using its classical variation. In this step, practitioners
that contributed to the adoption of DevOps in 15 companies from different
domains and across five countries were interviewed. Based on results, a model
was produced to improve both the understanding and guidance of DevOps adoption.
In the second step of the study, the model was introduced in the adoption of
DevOps at the Brazilian Federal Court of Accounts (TCU), at which time a focus
group was held to evaluate the current stage of adoption and to validate the
applicability and utility of the model.

The model increments the existing view of DevOps by explaining the role and
motivation of each category (and their relationships) in the DevOps adoption
process. This model was organized in terms of \emph{DevOps enabler categories}
and \emph{DevOps outcome categories}. Was concluded that \emph{collaboration} is
the core DevOps concern, contrasting with a possible understanding that
implanting specific tools to \emph{automate building, deployment, and
infrastructure provisioning and management} is enough to achieve DevOps.

Altogether, the results contribute to (a) generating an adequate understanding
of DevOps adoption, from the perspective of practitioners; and (b) assisting
institutions, like \acrshort{TCU}, in the migration path towards DevOps
adoption. In addition, the experiences collected during the production of the
model have been applied in adoption of DevOps at TCU.
