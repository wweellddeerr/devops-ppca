Conforme ilustrado no capítulo anterior, os trabalhos de pequisa existentes
apresentam algumas caracterizações de \textit{DevOps} por meio da enumeração
de conceitos, princípios e práticas relacionadas. Apesar de alguns desses
estudos conterem abordagens qualitativas para investigar empresas adotando
\textit{DevOps}, conduzindo entrevistas com praticantes, o foco desses estudos
tipicamente é em caracterizar \textit{DevOps}, ao invés de prover recomendações
na adoção de \textit{DevOps}. Além das diferenças de foco, existem algumas
diferenças metodológicas em relação ao estudo aqui proposto. Essas diferenças
são resumidas na tabela \ref{related_work_table}.

%% Table~\ref{related_work_table} compares the most relevant literature to
%% out work. Since the work of Lwakatare et al.~\cite{extending_dimensions} is an
%% extension of a previous contribution of the authors and her team \cite{dimensions_of_devops_xp_15},
%% we only compared the most recent study at Table~\ref{related_work_table}.

\begin{table}[hb!]
\centering
\caption{Comparação de Metodologia e Foco dos Trabalhos Relacionados}
\label{related_work_table}
\begin{tabular}{|p{3cm}|p{6cm}|p{6cm}|}
\hline

\textbf{Estudo}
& \textbf{Abordagem de Pesquisa}
& \textbf{Foco do Trabalho} \\

\hline

\textbf{Este estudo}

& Grounded Theory study interviewing practitioners which contributed to DevOps
adoption in 15 companies across 5 countries.

& Explain how DevOps has been successfully adopted in practice;\newline
\newline Provide guidelines to be used in new DevOps adoptions. \\

\hline

\textbf{J. Smeds et al.~\cite{devops_a_definition}}

& Systematic literature review; \newline \newline Semi structured interviews
with 13 employees of a single company whose DevOps adoption process was at
an initial stage.

& Identify in literature the main defining characteristics of DevOps; \newline
\newline Build a list of possible impediments to DevOps adoption through an
empirical study. \\

\hline

\textbf{Lwakatare et al.~\cite{extending_dimensions}}

&
Multivocal literature review using data from gray literature; \newline \newline
Three interviews with software practitioners in one company that was applying
DevOps practices in one project.

& Identify how do practitioners describe DevOps as a phenomenon; \newline
\newline Identify the DevOps practices according to software practitioners. \\

\hline

\textbf{Fran\c{c}a et al.~\cite{characterizing_devops}}

& Multivocal literature review with qualitative analysis procedures from
Grounded Theory.

& Provide a DevOps definition; \newline \newline
Identify DevOps practices, required skills, characteristics, benefits and
issues motivating its adoption.  \\

\hline

\textbf{Erich et al.~\cite{qualitative_devops_journalsw_17}}

& Systematic literature review; \newline \newline
Interviews with practitioners from 6 companies across 3 countries.

& Identify how literature defines DevOps; \newline \newline
Investigate how DevOps is being implemented in practice. \\

\hline

\end{tabular}
\end{table}

Como consequência dessas diferenças de foco, algumas questões práticas a
respeito da adoção de DevOps permancem em aberto: (1) Existe um caminho
recomendado para se adotar \textit{DevOps}? (2) Já que \textit{DevOps} é
constituído de múltiplos elementos, eles
possuem a mesma relevância quando se adota \textit{DevOps}? (3) Qual é o papel
desempenhado por cada um desses elementos - tais como medição, compartilhamento e
automação - em uma adoção de \textit{DevOps}? Para responder a essas questões,
é necessário um entendimento holístico dos caminhos seguidos por organizações
que adotaram \textit{DevOps} de uma maneira bem sucedida.

Neste capítulo, apresenta-se um estudo utilizando a variação clássica da
metodologia \textit{\acrfull{GT}} para se caracterizar a adoção de DevOps
em organizações de mercado que foram bem sucedidas neste processo.

A teoria apresentada neste capítulo foi construída com base na percepção de
praticantes de quinze companhias de cinco países que foram bem sucedidos na
adoção de \textit{DevOps}. A teoria torna claro que praticantes interessados em
adotar \textit{DevOps} devem focar na construção de uma \cat{cultura de
colaboração}, o que previne pontos de falha comuns relacionados a focar em
ferramental ou em automação. O papel dos demais elementos que fazem parte da
adoção de \textit{DevOps} é explicado por meio da definição de dois grandes
agrupamentos de categorias de conceitos, denominados \emph{Facilitadores DevOps}
e \emph{Saídas DevOps}.

\section{\acrfull{GT}}

Introduzir GT

\acrshort{GT} foi utilizada como metodologia de pesquisa devido a três principais
razões. Primeiro, \acrshort{GT} é uma metodologia consolidada em outras áreas de
pesquisa, notadamente sociologia médica \cite{gt_medical_sociology}, nutrição
\cite{gt_nursing}, educação \cite{gt_education} e administração
\cite{gt_management}. \acrshort{GT} também tem sido cada vez mais aplicada
para estudar tópicos de engenharia de \textit{software}
\cite{hoda2017becoming,Waterman:2015:ICSE,stol2016grounded}. A segunda
razão é que \acrshort{GT} é considerada uma metodologia adequada para responder
questões de pesquisa que objetivam caracterizar cenários sob uma perspectiva
pessoal daqueles envolvidos em uma disciplina ou atividade \cite{stol2016grounded},
que é exatamente o cenário que se almeja caracterizar aqui: quais são os caminhos
seguidos por praticantes que adotaram \textit{DevOps} de uma maneira bem
sucedida? Finalmente, \acrshort{GT} possibilita a construção de um entendimento
independente e original, o que é adequado para coletar evidência empírica
diretamente da prática na indústria sem o viés das pesquisas anteriores.
A evidência coletada só é reintegrada com a literatura existente após a construção
da teoria \cite{}.

A theory, according to Strauss and Corbin [44], is ‘a set of well-developed
categories (e.g. themes, concepts) that are systematically interrelated through statements of
relationship to form a theoretical framework that explains some relevant social, psychological,
educational, nursing or other phenomenon’ (using GT to understand sw process improv)



\subsection{Variações de \acrshort{GT}}

Desde a publicação da versão original de \acrshort{GT}, diversas modificações
e variações foram propostas ao método, ocasionando a existência de pelo menos
sete diferentes versões de \acrfull{GT}~\cite{denzin2007grounded}. Segundo
Stol et al. \cite{stol2016grounded}, as principais versões são três: (1)
\acrshort{GT} de \textit{Glaser} (GT clássica ou \textit{Glaseriana}), (2)
\acrshort{GT} de \textit{Strauss} e \textit{Corbin} (GT \textit{Straussiana}) e
(3) \acrshort{GT} de \textit{Charmaz} (GT construtivista), e qualquer
estudo utilizando \acrshort{GT} deve especificar qual versão foi utilizada.

As principais diferenças entre as versões de \acrshort{GT} estão relacionadas
à maneira de utilizar questões de pesquisa, ao papel da literatura, às técnicas
de codificação, aos tipos de questionamentos que devem ser levantadas durante
a análise dos dados e aos critérios de avaliação da teoria construída. A
Tabela \ref{versoes_GT_tabela}, adaptada de \cite{stol2016grounded}
sintetiza as principais diferenças entre as três principais versões.

\begin{table}[hb!]
\centering
\caption{Principais diferenças entre as versões de GT}
\label{versoes_GT_tabela}
\resizebox{\textwidth}{!}{
\begin{tabular}{|p{3cm}|p{5.7cm}|p{5.7cm}|p{5.7cm}|}
\hline
\textbf{Elemento} & \textbf{Clássica} & \textbf{\textit{Straussiana}} & \textbf{Construtivista} \\
\hline
\textbf{Questão de pesquisa}
& Não é definida \textit{a priori}, emerge da pesquisa. O pesquisador inicia com uma área de interesse.
& Pode ser definida antecipadamente, derivada da literatura ou sugerida por um colega, é frequentemente ampla e aberta.
& A pesquisa inicia com questões de pesquisa iniciais, que evoluem ao longo do estudo. \\
\hline

\textbf{Papel da literatura}
& Uma revisão abrangente da literatura só deve ser conduzida após a literatura possuir estágio avançado de desenvolvimento, para evitar a influência de conceitos existentes na teoria emergente.
& A literatura pode ser consultada ao longo do processo para algumas necessidades específicas, como na formualação de perguntas para coleta de dados, para obter sugestões de áreas para exemplificação teórica, etc.
& Embora concorde com os motivos de \textit{Glaser} recomendar evitar contato com a literatura no início da pesquisa, defende uma consulta adaptada da literatura para adequar o propósito do estudo. \\
\hline

\textbf{Técnicas de codificação}
& \textit{Open coding}, \textit{selective coding} e \textit{theoretical coding}.
& \textit{Open coding}, \textit{axial coding} e \textit{selective coding}.
& \textit{Initial coding}, \textit{focused coding} e \textit{theoretical coding}. \\
\hline

\textbf{Questões levantadas durante a análise}
& Estes dados são um estudo de quê? Que categoria ou propriedade este incidente indica? O que está acontecendo atualmente nos dados?
& Questões do estilo quem, quando, onde, como, com quais consequências ou sob quais condições algum fenômeno ocorreu, ajudam a descobri ideias importantes para a teoria.
& Estes dados são um estudo de quê? O que este dado sugere? Sob o ponto de vista de quem? A qual categoria teórica esta unidade de dado indica pertencer? \\
\hline

\textbf{Critérios de avaliação}
& As categorias geradas devem estar alinhadas aos dados, a teoria deve funcionar, a teoria deve ter relevância para a ação da área, e a teoria deve ser modificável quando novos dados aparecerem.
& São definidos sete critérios para o processo de pesquisa, tais quais, informação na seleção de exemplos, principais categorias, hipóteses derivadas e discrepâncias. E oito critérios a respeito do crescimento empírico, tais quais ``conceitos foram gerados?'', ``existe variação construída na teoria?''
& Credibilidade, originalidade, utilidade e se a teoria faz sentido para os participantes. \\
\hline

\end{tabular}}
\end{table}


Neste trabalho foi utilizada a variação clássica, principalmente pela
inexistência de questões de pesquisa no início da pesquisa, exatamente como
sugerido nessa versão. O início da pesquisa estava ancorado em investigar uma
área de interesse: a adoção bem sucedida de \textit{DevOps} na indústria.
Adicionalmente, os trabalhos de pesquisa atualmente existentes na área
de engenharia de \textit{software} usam predominantemente a versão clássica
\cite{stol2016grounded} e, portanto, existe um maior referencial que pode ser
utilizado como base em um novo estudo.

\subsection{Abordagem \acrshort{GT}}

A figura \ref{abordagem_gt}, adaptada da proposta por Adolph et al.
\cite{using_gt_adolph}, ilustra uma abordagem de como conduzir um estudo
utilizando \acrshort{GT} como método. Esta abordagem foi utilizada no estudo
aqui proposto.

\figuraBib{GT}{O método GT}{using_gt_adolph}{abordagem_gt}{width=.45\textwidth}%

\subsection{Trabalhos Relacionados}

\iffalse
\section{Adoção de \textit{DevOps}}\label{secao_adocao_devops}

Os trabalhos listados anteriormente concentram-se em caracterizar
\textit{DevOps}, há pouco conteúdo dedicado a \emph{como} adotar \textit{DevOps},
com indicação de caminhos seguidos por companhias que passaram pelo processo
e que eventualmente poderiam ser usados por novos praticantes, com indicações
de lições aprendidas e possíveis abordagens que possam ser seguidas.

O trabalho de Erich et al. \cite{qualitative_devops_journalsw_17} investigou
o processo de adoção de \textit{DevOps} por meio de entrevistas com praticantes
de seis companhias. Todavia, as percepções obtidas não detalham a maneira como
as companhias aplicaram os elementos de \textit{DevOps} ao longo do processo,
e são apresentadas de maneira individualizada, não se colocando em perspectiva
o que há de comum no entendimento dessas companhias. Os dados de como cada
companhia relata ter implementado \textit{DevOps} são apresentados de maneira
sucinta e os principais pontos são descritos a seguir.

Na primeira companhia, é relatado que o pessoal de desenvolvimento e
operações passou a trabalhar de maneira conjunta diariamente, foi criado um
novo cargo na companhia chamado \textit{DevOps Engineer}, o pessoal de gestão
foi treinado para ser capaz de gerenciar tanto pessoas de desenvolvimento como
de operações, o pessoal de recursos humanos alterou os processos de avaliação e,
por fim, a organização está automatizando seus processos de infraestrutura, não
são apresentados detalhes de como isso está sendo feito.

Na segunda companhia, \textit{DevOps} é visto como uma extensão de
\textit{Scrum} e \textit{Lean}. A adoção de \textit{DevOps} lá foi guiada por
dois princípios: colaboração diária e ambiente de trabalho compartilhado.
Havia a premissa de que o \textit{software} deve ir para produção a cada
iteração. É relatado então que a organização usou \textit{frameworks}
comerciais para guiar a implantação de \emph{entrega contínua}, também sem maiores
detalhamentos de como foi feito.

Na terceira companhia, é relatado que três times que trabalhavam em
funcionalidades separadas foram agrupados em um \emph{time DevOps}. A organização
passou então a experimentar \textit{entrega contínua}, mas ainda estava em
um estágio inicial quando a entrevista foi realizada.

A quarta companhia criou uma única equipe de \textit{DevOps} para
experimentar ferramentas, princípios e práticas \textit{DevOps}. Os
membros dessa equipe possuem habilidades multidisciplinares. É relatado ainda
o uso de ferramentas para controle de versão e o desenvolvimento de ferramentas
próprias para criar facilmente ambientes de nuvem.

Na quinta companhia, é relatado que \textit{DevOps} é visto como
um papel responsável por gerenciamento de incidentes, gerenciamento de
capacidade, gerenciamento de riscos e suporte ao processo de criação.
O uso de computação em nuvem ocasionou a necessidade de se existir uma
abordagem mais sistemática para execução das atividades de operações, exigindo
que o pessoal responsável aprendesse técnicas de desenvolvimento de
\textit{software}. O time de desenvolvimento foi dividido em vários subtimes,
um dos quais focado em \textit{DevOps}, com responsabilidade de apoiar os demais
times. São também enumerados alguns princípios e práticas utilizados na companhia
durante a adoção de \textit{DevOps}.

Por fim, com relação a como a sexta companhia implementou \textit{DevOps},
é apresentado um detalhamento de como ocorre a \emph{integração contínua} na
organização. As evoluções de código são feitas por meio de \textit{pull
requests} que após aprovadas são integradas ao código principal. Após a
integração dos códigos das \textit{pull requests}, um \textit{pipeline} de entrega
contínua é executado para publicar o \textit{software}.
\fi
