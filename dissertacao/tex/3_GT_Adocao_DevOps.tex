Neste capítulo, apresenta-se um estudo utilizando a variação clássica da
metodologia \textit{\acrfull{GT}} \cite{glaser1967discovery} para se
caracterizar a adoção de \textit{DevOps} em organizações de mercado que foram bem
sucedidas neste processo.

A teoria aqui apresentada foi construída com base na percepção de
praticantes de quinze companhias de cinco países que foram bem sucedidas na
adoção de \textit{DevOps}. A teoria torna claro que praticantes interessados em
adotar \textit{DevOps} devem focar na construção de uma \cat{cultura de
colaboração}, o que previne pontos de falha comuns relacionados a focar em
ferramental ou em automação. O papel dos demais elementos que fazem parte da
adoção de \textit{DevOps} é explicado por meio da definição de dois grandes
agrupamentos de categorias de conceitos, denominados \emph{Facilitadores DevOps}
e \emph{Saídas DevOps}.

\section{\acrfull{GT}}

\acrfull{GT} é um método de pesquisa, desenvolvido originalmente pelos
sociólogos B. Glaser e A. Strauss, que possibilita a geração sistemática de
uma teoria a partir de dados analisados em um rigoroso processo \cite{glaser1967discovery}.
O objetivo de um estudo em \acrshort{GT} é entender como se dá a ação em uma
determinada área de conhecimento a partir do ponto de vista dos atores
envolvidos \cite{glaser_doing_1998}.

Ressalte-se que, de acordo com Strauss e Corbin \cite{corbin2014basics}, no
âmbito de \textit{Grounded Theory}, uma teoria é ``um conjunto bem desenvolvido
de categorias (incluindo conceitos e temas) que são sistematicamente
inter-relacionadas através de \textbf{sentenças de relacionamentos} para
formar um \textit{framework} teórico que explica algum fenômeno relevante''.

Hoda et al. \cite{hoda2017becoming} explicam que um estudo utilizando
\textit{Grounded Theory} é mais que apenas um conjunto de categorias. Ele deve
descrever os principais relacionamentos entre as categorias, e que essas
``sentenças de relacionamentos'' são aqui chamadas de hipóteses, sem prejuízo
para o uso deste mesmo termo com significado distinto em abordagens que usam
métodos estatísticos para confirmar ou refutar hipóteses pré-estabelecidas.

Segundo Locke \cite{locke2001grounded}, o diferencial de \textit{Grounded
Theory} é seu compromisso com a ``descoberta'' através do contato direto com o
mundo social de interesse, juntamente com uma limitação, a priori, do contato
com a teorização existente. Os estudos utilizando \acrshort{GT} não começam com
uma estrutura conceitual, ao invés disso pretendem culminar em uma \cite{miles1994qualitative}.
São estudos em que uma cobertura aprofundada da literatura é deliberadamente
adiada até que surjam as direções da análise de dados coletados diretamente
na fonte de interesse \cite{punch2013introduction}. Stol et al. \cite{stol2016grounded}
explicam que a principal razão para se limitar a exposição à literatura é previnir
que o pesquisador se concentre em pensar em termos de conceitos pré-estabelecidos,
testando teorias existentes, ao invés de desenvolver uma. De fato, \acrshort{GT}
é um método de desenvolvimento de teoria, e não de teste de teoria. Jantunen e
Gause \cite{jantunen_using_gt_approach}, utilizando a classificação de abordagens
de pesquisa proposta por Jarvinen \cite{jarvinen_mapping_2008}, apontam que
\acrshort{GT} é uma abordagem de desenvolvimento de teoria do grupo estudos
empíricos da realidade, conforme a área em cinza da Figura \ref{abordagens}.

\figuraBib{abordagens}{Classificação de GT nas Abordagens de Pesquisa}{jantunen_using_gt_approach}{abordagens}{width=.6\textwidth}

No entanto, conforme destaca Glaser \cite{glaser1992basics}, essa postura de
limitar a exposição à literatura é parte da abordagem apenas no início da
pesquisa. Quando a teoria proposta parece suficientemente amadurecida, então o
pesquisador pode começar a revisar a literatura no campo substantivo e
relacioná-la com seu próprio trabalho.

\subsection{Variações de \acrshort{GT}}

Desde a publicação da versão original de \acrshort{GT}, diversas modificações
e variações foram propostas ao método, ocasionando a existência de pelo menos
sete diferentes versões de \acrfull{GT}~\cite{denzin2007grounded}. Segundo
Stol et al. \cite{stol2016grounded}, as principais versões são três: (1)
\acrshort{GT} de \textit{Glaser} (GT clássica ou \textit{Glaseriana}), (2)
\acrshort{GT} de \textit{Strauss} e \textit{Corbin} (GT \textit{Straussiana}) e
(3) \acrshort{GT} de \textit{Charmaz} (GT construtivista), e qualquer
estudo utilizando \acrshort{GT} deve especificar qual versão foi utilizada.

As principais diferenças entre as versões de \acrshort{GT} estão relacionadas
à maneira de utilizar questões de pesquisa, ao papel da literatura, às técnicas
de codificação, aos tipos de questionamentos que devem ser levantadas durante
a análise dos dados e aos critérios de avaliação da teoria construída. A
Tabela \ref{versoes_GT_tabela}, adaptada de \cite{stol2016grounded}
sintetiza as principais diferenças entre as três principais versões.

\begin{table}[hb!]
\centering
\caption{Principais diferenças entre as versões de GT}
\label{versoes_GT_tabela}
\resizebox{\textwidth}{!}{
\begin{tabular}{|p{3cm}|p{5.7cm}|p{5.7cm}|p{5.7cm}|}
\hline
\textbf{Elemento} & \textbf{Clássica} & \textbf{\textit{Straussiana}} & \textbf{Construtivista} \\
\hline
\textbf{Questão de pesquisa}
& Não é definida \textit{a priori}, emerge da pesquisa. O pesquisador inicia com uma área de interesse.
& Pode ser definida antecipadamente, derivada da literatura ou sugerida por um colega, é frequentemente ampla e aberta.
& A pesquisa inicia com questões de pesquisa iniciais, que evoluem ao longo do estudo. \\
\hline

\textbf{Papel da literatura}
& Uma revisão abrangente da literatura só deve ser conduzida após a literatura possuir estágio avançado de desenvolvimento, para evitar a influência de conceitos existentes na teoria emergente.
& A literatura pode ser consultada ao longo do processo para algumas necessidades específicas, como na formualação de perguntas para coleta de dados, para obter sugestões de áreas para exemplificação teórica, etc.
& Embora concorde com os motivos de \textit{Glaser} recomendar evitar contato com a literatura no início da pesquisa, defende uma consulta adaptada da literatura para adequar o propósito do estudo. \\
\hline

\textbf{Técnicas de codificação}
& \textit{Open coding}, \textit{selective coding} e \textit{theoretical coding}.
& \textit{Open coding}, \textit{axial coding} e \textit{selective coding}.
& \textit{Initial coding}, \textit{focused coding} e \textit{theoretical coding}. \\
\hline

\textbf{Questões levantadas durante a análise}
& Estes dados são um estudo de quê? Que categoria ou propriedade este incidente indica? O que está acontecendo atualmente nos dados?
& Questões do estilo quem, quando, onde, como, com quais consequências ou sob quais condições algum fenômeno ocorreu, ajudam a descobri ideias importantes para a teoria.
& Estes dados são um estudo de quê? O que este dado sugere? Sob o ponto de vista de quem? A qual categoria teórica esta unidade de dado indica pertencer? \\
\hline

\textbf{Critérios de avaliação}
& As categorias geradas devem estar alinhadas aos dados, a teoria deve funcionar, a teoria deve ter relevância para a ação da área, e a teoria deve ser modificável quando novos dados aparecerem.
& São definidos sete critérios para o processo de pesquisa, tais quais, informação na seleção de exemplos, principais categorias, hipóteses derivadas e discrepâncias. E oito critérios a respeito do crescimento empírico, tais quais ``conceitos foram gerados?'', ``existe variação construída na teoria?''
& Credibilidade, originalidade, utilidade e se a teoria faz sentido para os participantes. \\
\hline

\end{tabular}}
\end{table}

Neste trabalho foi utilizada a variação clássica, principalmente pela
inexistência de questões de pesquisa no início, exatamente como
sugerido nessa versão. O início da pesquisa estava ancorado em investigar uma
área de interesse: a adoção de \textit{DevOps} em cenários reais
e tidos como bem sucedidos.
Adicionalmente, os trabalhos de pesquisa atualmente existentes na área
de engenharia de \textit{software} usam predominantemente a versão clássica
\cite{stol2016grounded} e, portanto, existe um maior referencial que pode ser
utilizado como base em um novo estudo.

\subsection{Abordagem \acrshort{GT}}\label{subsecao:abordagem_gt}

A Figura \ref{abordagem_gt}, adaptada da proposta por Adolph et al.
\cite{using_gt_adolph}, ilustra uma abordagem de como conduzir um estudo
utilizando \acrshort{GT} como método. Esta abordagem foi utilizada no estudo
aqui realizado.

\figuraBib{GT}{O método GT}{using_gt_adolph}{abordagem_gt}{width=.6\textwidth}%

\begin{enumerate}[label=(\Alph*)]

\item Inicialmente, foi observado o fenômeno da adoção de \textit{DevOps} em companhias que
o adotaram de maneira bem sucedida. Os dados foram coletados e
analisados simultaneamente, em um processo iterativo. O dado bruto foi
analisado por meio da busca de padrões de incidentes que indicam \emph{conceitos},
e estes conceitos foram agrupados em \emph{categorias}. Esta primeira etapa onde
todo o dado bruto é analisado, gerando novos conceitos e categorias, é chamada
de \emph{codificação aberta}.

\item As categorias vão sendo desenvolvidas por meio da \emph{comparação constante}
dos novos incidentes com os prévios. Todo estudo utilizando \textit{grounded theory}
deve identificar uma ``\emph{categoria principal}'' \cite{stol2016grounded}. A
categoria principal é responsável por possibilitar a integração das demais
categorias e estruturar os resultados em uma teoria densa e consolidada \cite{jantunen_using_gt_approach}.
A identificação da categoria principal representa o fim da \emph{codificação
aberta} e o início da \emph{codificação seletiva}. Na codificação seletiva,
são codificadas apenas variáveis específicas que são diretamente relacionadas
com a categoria principal, visando a produção de uma teoria harmônica \cite{using_gt_coleman,hoda_impact_inadequate}.
A \emph{codificação seletiva} termina quando é atingida a \emph{saturação
teórica}, o que ocorre quando os novos dados contém mais evidência e exemplos
do que já está desenvolvido, mas não contém novos conceitos ou categorias~\cite{glaser1967discovery}.

\item Após a \emph{saturação teórica} é iniciada uma nova etapa denominada
\emph{codificação teórica} cujo propósito é construir uma teoria que explica
como conceitos, categorias e relacionamentos se encaixam em uma unidade conceitual.
Adicionalmente, a teoria produzida pode ser reintegrada com a literatura no final
do processo por meio da comparação de resultados. Ou seja, quando se utiliza
{\it grounded theory}, uma revisão de literatura só deve ser conduzida nos
estágios finais da pesquisa, visando evitar influências externas na concepção
da teoria~\cite{reconciling_perspectives}.

\item Durante todo o processo, {\it memos} são escritos com o objetivo de
registrar as ideias e pensamentos do pesquisador durante a análise dos dados; os
{\it memos} são consultados para ajudar na descoberta de conceitos, no
agrupamento de conceitos em categorias e nas explicações a respeito das relações
entre as categorias~\cite{reconciling_perspectives}.

\end{enumerate}

\section{Coleta de Dados}

Na coleta de dados, foram conduzidas entrevistas semi-estruturadas com praticantes
de quinze companhias do Brasil, Espanha, Estados Unidos, Irlanda e Portugal.
Estes praticantes contribuíram para a adoção de \textit{DevOps} em suas
companhias.

Os participantes das entrevistas foram recrutados de três maneiras: (1) através
de chamadas gerais postadas em grupos de {\it DevOps} em redes sociais
(Facebook, Slack e Telegram); (2) através de mensagens diretas para
praticantes com perfil na rede social LinkedIn; e (3) através de contato direto
no evento \emph{DevOpsDays} que ocorreu em Brasília em novembro de 2017.

Aos praticantes que demonstravam interesse em participar da entrevista, foram
feitos breves questionamentos para assegurar que possuíam participação efetiva
em processos consolidados de adoção de {\it DevOps} em suas respectivas
companhias.

A Tabela~\ref{participant_table} apresenta as características dos participantes
que foram entrevistados. Para manter anonimidade, em conformidade com as
diretrizes éticas, daqui em diante os participantes serão referenciados como
P1--P15 (primeira coluna). Todos os entrevistados são do sexo masculino.

\begin{table}[t]
\centering
\caption{Perfil dos Participantes}
\label{participant_table}
\begin{tabular}{|p{0.6cm}|p{3.4cm}|p{0.5cm}|p{0.5cm}|p{0.5cm}|p{1.7cm}|p{0.5cm}|p{1cm}|p{2.7cm}|}
\hline \centering
{\bf P\#} & {\bf Cargo} & {\bf SX} & {\bf DX} & {\bf CN} & {\bf Domínio} & {\bf CS} & {\bf Idade} & {\bf Formação} \\ \hline \centering
P1 & {\it DevOps Developer} & 9 & 2 & IR & \acrshort{TI} & S & 29 & Graduação \\ \hline \centering

P2 & Consultor {\it DevOps} & 9 & 3 & BR & \acrshort{TI} & M & 35 & Pós-graduação \\ \hline \centering

P3 & {\it DevOps} Developer & 8 & 1 & IR & \acrshort{TI} & S & 28 & Graduação \\ \hline \centering

P4 & Técnico em Computação & 10 & 2 & BR & Saúde & S & 30 & Graduação \\ \hline \centering

P5 & {\it Systems Engineer} & 10 & 3 & SP & Telecom & XL & 30 & Graduação \\ \hline \centering

P6 & {\it Developer} & 3 & 1 & PO & \acrshort{TI} & S & 27 & Graduação \\ \hline \centering

P7 & Analista de Suporte & 15 & 2 & BR & Telecom & L & 34 & Pós-graduação \\ \hline \centering

P8 & {\it DevOps Engineer} & 20 & 9 & BR & Marketing & M & 39 & Graduação \\ \hline \centering

P9 & Gerente de \acrshort{TI} & 14 & 8 & BR & \acrshort{TI} & M & 32 & Pós-graduação \\ \hline \centering

P10 & Administrador de Redes & 15 & 3 & BR & IT & S & 33 & Graduação \\ \hline \centering

P11 & Supervisor de {\it DevOps} & 6 & 4 & BR & \acrshort{TI} & M & 31 & Graduação \\ \hline \centering

P12 & {\it Cloud Engineer} & 9 & 3 & US & IT & L & 27 & Técnico \\ \hline \centering

P13 & Gerente de Tecnologia & 18 & 6 & BR & Food & M & 35 & Pós-graduação \\ \hline \centering

P14 & Gerente de \acrshort{TI} & 7 & 2 & BR & \acrshort{TI} & S & 28 & Graduação \\ \hline \centering

P15 & Desenvolvedor & 3 & 2 & BR & \acrshort{TI} & S & 22 & Graduação \\ \hline
\end{tabular}
\begin{tablenotes}
  \footnotesize \centering
  \item SX = anos de experiência com desenvolvimento de {\it software};
  \item DX = anos de experiência com {\it DevOps};
  \item CN = nome do país (BR=Brasil, IR=Irlanda, SP=Espanha, PO=Portugal e US=Estados Unidos);
  \item CS = tamanho da companhia (S\textless100; M\textless1000; L\textless5000; XL\textgreater5000).
\end{tablenotes}
\end{table}


As entrevistas foram conduzidas entre agosto de 2017 e maio de 2018 por meio de
chamadas utilizando a ferramenta {\it skype}. As entrevistas duraram em média 30,93 minutos
com duração mínima de 17 minutos, máxima de 50 minutos e mediana de 31 minutos.
A realização das entrevistas e a análise dos respectivos dados ocorreram de
maneira iterativa, de modo que os dados coletados e analisados serviram de
insumo para guiar a realização das novas entrevistas por meio de adaptação do
roteiro. As questões evoluíram de acordo com o progresso da pesquisa. No início
haviam apenas quatro questões {\it open-ended}: (1) O que motivou a adoção de
{\it DevOps}? (2) O que adotar {\it DevOps} significa no contexto da sua
companhia? (3) Como {\it DevOps} foi adotado em sua companhia? E (4) Quais foram
as principais dificuldades e como foram superadas?

Conforme a análise evoluiu, novas perguntas relacionadas aos
conceitos, categorias e relacionamentos identificados nas entrevistas anteriores
foram adicionadas ao roteiro de entrevistas.
Exemplos de questões adicionadas incluem: (1) Qual é a relação entre automatizar o
{\it deployment} e a adoção de {\it DevOps}? (2) É possível adotar {\it DevOps}
sem automação? (3) Foi formada uma cultura de colaboração? Como? O que faz parte
dessa cultura?

\section{Análise de Dados}

Para analisar os dados, as entrevistas foram gravadas, transcritas e analisadas.
A primeira fase da análise, chamada de \emph{codificação aberta}, iniciou
imediatamente após a transcrição da primeira entrevista.

A codificação aberta durou até que não restassem dúvidas a respeito de qual era
a \emph{categoria principal} do estudo. Similarmente ao descrito por Adolph et
al.~\cite{reconciling_perspectives}, no início da análise, uma categoria
demonstrou potencial para ser a categoria principal, mas não se consolidou e foi
alterada depois. A primeira candidata a categoria principal foi \cat{automação}
que aparecia de maneira recorrente nos relatos de adoção de {\it DevOps}.
Todavia, foi notado que muitos dos padrões identificados nos dados não eram
facilmente explicados em torno de \cat{automação}. Por exemplo, o senso de
responsabilidade compartilhada entre os times para resolução de problemas, ou
a noção de formação de uma visão de produto durante o processo de desenvolvimento
de {\it software}, eventos comumente descritos nas entrevistas, não aparentavam
se relacionar com \cat{automação}. Passou-se então a observar que a formação de
uma \cc era outra categoria recorrente na análise dos dados e com maior
potencial para explicar os eventos. Assim sendo, perguntas tratando
especificamente a respeito do papel da \cat{automação} em uma adoção de {\it DevOps}
e sobre a formação de uma \cc foram adicionadas ao roteiro de entrevista.

Considerando as adaptações no roteiro e as análises de novos dados em um
processo de \emph{comparação constante}, levando em conta as análises prévias e
os respectivos {\it memos} escritos durante todo o processo, após a décima
entrevista concluiu-se que \cc era inequivocamente a \emph{categoria principal}
sobre como {\it DevOps} foi adotado de maneira bem sucedida nas companhias
investigadas.

Neste momento, a \emph{codificação aberta} foi encerrada e a \emph{codificação
seletiva} foi iniciada. A codificação passou a ser restrita àquelas variáveis
diretamente relacionadas à categoria principal e seus relacionamentos. Realizando
mais três entrevistas e respectiva análise, percebeu-se que os novos dados
forneciam cada vez menos conteúdo à teoria que estava emergindo. Ou seja, as
explicações a respeito de como a categoria \cc é desenvolvida em uma adoção de
{\it DevOps} mostravam sinais de saturação. Assim sendo, foram conduzidas mais
duas entrevistas para concluir que a \emph{saturação teórica} havia sido atingida.

Seguindo a abordagem descrita anteriormente, iniciou-se a última etapa da
análise que é a \emph{codificação teórica} cujo objetivo é encontrar uma forma
de se integrar todos os conceitos, categorias e {\it memos} em uma teoria coesa
e homogênea. Percebeu-se neste momento que as categorias identificadas
desempenhavam os papés de \emph{saídas} ou \emph{facilitadores} {\it DevOps}.
Esses papéis serão melhor detalhados na seção \ref{secao_teoria}.

Para ilustrar com maior precisão os detalhes de como foi realizado o processo
de codificação, que resultou na descoberta dos conceitos e categorias relacionados
à adoção de {\it DevOps}, a seguir
é apresentado um exemplo da evolução de um trecho de transcrição de entrevista
(dado bruto) até uma categoria.

Durante as entrevistas, diversos detalhes acessórios são citados
pelos entrevistados. Isso resulta em dados brutos (transcrições) cheios de
ruídos. Assim sendo, inicialmente foram removidos os ruídos do dado bruto,
resultando na identificação dos \emph{pontos-chave}. Pontos-chave são a síntese
obtida dos trechos de entrevistas~\cite{georgieva2008best}. Por exemplo:

\textbf{Dado bruto:} {\it ``Então, aqui nós adotamos este tipo de estratégia
que é a infraestrutura como código, consequentemente, nós temos um versionamento
de toda a nossa infraestrutura em uma linguagem comum, de tal maneira que
qualquer pessoa, um desenvolvedor, um arquiteto, o cara de operações, ou mesmo
o gerente, ele olha e consegue descrever que a configuração da aplicação x é
y. Então, isto agrega muito valor para nós exatamente com mais transparência''}.

\textbf{Ponto-chave:} \textit{``Infraestrutura como código contribui para
transparência por possibilitar o versionamento da infraestrutura em uma
linguagem comum para todos os profissionais"}.

Após a identificação dos pontos-chave, um ou mais códigos foram então atribuídos
a cada um deles. Um código é uma frase que sumariza o ponto-chave,
e um ponto-chave pode levar a diversos códigos \cite{hoda2017becoming}.

{\bf Código:} \textit{Infraestrutura como código contribui para transparência}.

{\bf Código:} \textit{Infraestrutura como código provê uma linguagem comum}.

Neste exemplo, o conceito que emergiu foi o de ``\emph{infraestrutura como
código}''. O trecho de dado bruto acima foi retirado da décima segunda entrevista,
momento em que este conceito já existia como resultado das entrevistas anteriores.
Abaixo é apresentado um exemplo de {\it memo} escrito neste momento da análise.

\textbf{Memo:} \textit{O conceito de infraestrutura como código já apareceu
diversas vezes nas análises anteriores, esse é mais um exemplo. É mais um sinal
de que os conceitos estão apenas se repetindo e pode-se estar próximo da
saturação teórica. Este exemplo também reforça o que já estava sendo cogitado
de que infraestrutura como código está mais relacionado a transparência do que
a automação}.

O {\it memo} descrito acima foi útil para posteriormente ajudar a concluir que
a saturação teórica havia sido atingida. Também foi utilizado como auxílio no
processo de agrupamento de conceitos na categoria que foi denominada
\cat{Compartilhamento e Transparência}. Em um estudo utilizando \acrshort{GT},
uma categoria é um agrupamento de  conceitos relacionados, um novo nível de
abstração. A Figura \ref{categoria} ilustra o agrupamento dos conceitos
que formaram a referida categoria.

\figura{categoria}{Codificação: Construindo Categorias}{categoria}{width=.8\textwidth}

A expressão correspondente ao conceito de \emph{infraestrutura como código}
veio direto do dado bruto, mas isso não é uma regra. É comum que o conceito
seja uma abstração, sem que tenha emergido diretamente de uma expressão contida na
transcrição de entrevista. Por exemplo, abaixo é apresentado como o conceito
de ``\emph{empoderamento do desenvolvimento de software}'' emergiu sem que esta
expressão específica esteja presente no dado bruto.

\textbf{Dado bruto:} {\it ``Nós temos diversas pessoas trabalhando no
desenvolvimento, a quantidade de desenvolvedores é realmente impressionante.
Não era viável a gente ter tantos desenvolvedores produzindo artefatos e
parando o trabalho deles para aguardar outro time completamente separado
publicar. Ou também precisar de um ambiente de teste e ter que esperar o time
de operações prover isto quando possível. Essas atividades tinham que estar
disponíveis para servir rapidamente ao time de desenvolvimento. Com DevOps
nós conseguimos suprir essa necessidade de liberdade e mais poder para executar
algumas tarefas que são intrisicamente relacionadas ao trabalho deles''}.

{\bf Ponto-chave:} \textit{``Faz parte da adoção de DevOps a incorporação
de maior liberdade e poder para o time de desenvolvimento executar tarefas que
são intrisicamente relacionadas ao seu trabalho, como publicação de artefatos
e provimento de ambientes de testes''}.

{\bf Código:} Facilitar a publicação de artefatos empodera o desenvolvimento de
{\it software}.

{\bf Código:} Facilitar o provimento de ambientes de testes empodera o
desenvolvimento de {\it software}.

{\bf Conceito:} Empoderamento do desenvolvimento de {\it software}.

A análise dos dados resultou na geração de 34 conceitos relacionados à adoção
de {\it Dev-Ops}, que foram agrupados em sete categorias. Os conceitos e
categorias são enumerados na Tabela \ref{tabela_conceitos} e detalhados nas
seções \ref{secao_core_category} a \ref{secao_facilitadores_e_saidas}.
Durante estas seções, são exibidos trechos de transcrições de entrevistas que
foram selecionados por ilustrar de maneira clara as ideias envolvidas na
caracterização da adoção de {\it DevOps}.

\begin{table}[hb!]
\centering
\label{tabela_conceitos}
\begin{tabular}{|K{0.2cm}|K{6.5cm}|K{0.2cm}|K{7.3cm}|}
\hline
 & \textbf{Categoria} & & \textbf{Conceito}  \\ \hline
\multirow{6}{*}{1} & \multirow{6}{*}{Cultura de Colaboração}
   & 1 & Operações no dia a dia de desenvolvimento \\ \cline{3-4}
 & & 2 & Empoderamento do desenvolvimento de {\it software} \\ \cline{3-4}
 & & 3 & Pensamento de produto \\ \cline{3-4}
 & & 4 & Comunicação direta \\ \cline{3-4}
 & & 5 & Responsabilidade compartilhada \\ \cline{3-4}
 & & 6 & {\it Blameless} \\ \hline

\multirow{8}{*}{2} & \multirow{8}{*}{Automação} &
     1 & Automação do {\it deployment} \\ \cline{3-4}
 & & 2 & Automação de testes \\ \cline{3-4}
 & & 3 & Automação do provimento da infraestrutura \\ \cline{3-4}
 & & 4 & Automação do gerenciamento da infraestrutura \\ \cline{3-4}
 & & 5 & Serviços autônomos \\ \cline{3-4}
 & & 6 & Conteinerização \\ \cline{3-4}
 & & 7 & Automação do monitoramento \\ \cline{3-4}
 & & 8 & Automação da recuperação \\ \hline

\multirow{6}{*}{3} & \multirow{6}{*}{Compartilhamento e Transparência} &
     1 & Compartilhamento de conhecimento \\ \cline{3-4}
 & & 2 & Compartilhamento de atividades \\ \cline{3-4}
 & & 3 & Compartilhamento de processos \\ \cline{3-4}
 & & 4 & Infraestrutura como código \\ \cline{3-4}
 & & 5 & Compartilhamento em bases regulares \\ \cline{3-4}
 & & 6 & {\it Pipelines} compartilhados \\ \hline

\multirow{3}{*}{4} & \multirow{3}{*}{Agilidade}
   & 1 & Integração contínua  \\ \cline{3-4}
 & & 2 & Provimento contínuo da infraestrutura \\ \cline{3-4}
 & & 3 & {\it Deployment} contínuo \\ \hline

\multirow{3}{*}{5} & \multirow{3}{*}{Resiliência} &
     1 & Auto {\it scaling} \\ \cline{3-4}
 & & 2 & Automação da recuperação \\ \cline{3-4}
 & & 3 & Zero {\it down-time} \\ \hline

\multirow{3}{*}{6} & \multirow{3}{*}{Medição Contínua}
   & 1 & Monitoramento de {\it logs} de aplicações \\ \cline{3-4}
 & & 2 & Monitoramento contínuo de aplicações \\ \cline{3-4}
 & & 3 & Monitoramento contínuo de infraestrutura \\ \hline

\multirow{5}{*}{7} & \multirow{5}{*}{Garantia da Qualidade}
   & 1 & Ramificação de código \\ \cline{3-4}
 & & 2 & Serviços coesos \\ \cline{3-4}
 & & 3 & Teste contínuo \\ \cline{3-4}
 & & 4 & Análise estática de código fonte \\ \cline{3-4}
 & & 5 & Paridade entre ambientes \\ \hline

\end{tabular}
\caption{Enumeração de Categorias e Conceitos Relacionados}
\end{table}

\section{A Categoria Principal: Cultura de Colaboração}\label{secao_core_category}

A categoria principal da adoção de {\it DevOps} é \cc. A \cc visa essencialmente
remover os silos entre os times de desenvolvimento e operações, o que até certo
ponto se confunde com os próprios objetivos de {\it DevOps}. Inicialmente, uma
\cc envolve que as tarefas tipicamente de operaçõe---como {\it deployment},
provisionamento e gerenciamento de infraestrutura e monitoramento---devem ser
consideradas atividades regulares, parte do dia a dia de desenvolvimento de
{\it software}. As organizações relatam que não aguardam mais momentos específicos
para performar essas atividades, elas são executadas continuamente. Isso leva
ao primeiro conceito relacionado à categoria principal: {\bf operações no dia
a dia de desenvolvimento}.

\begin{mq}
``\emph{Uma etapa muito importante foi trazer o {\it deployment} para dentro do
dia a dia de desenvolvimento, sem ter que ficare esperando um dia específico
da semana ou do mês. Nós queríamos fazer {\it deploy} o tempo inteiro, mesmo
que em um primeiro momento não fosse em produção, um ambiente de {\it staging}
era suficiente. [...] O que a gente queria era incorporar o {\it deployment} ao
desenvolvimento. É claro que para que a gente pudesse fazer o {\it deployment}
continuamente, a gente tinha que prover toda a infraestrutura necessária no
mesmo ritmo.}" (P14, Gerente de \acrshort{TI}, Brasil)
\end{mq}

Sem {\it DevOps}, um cenário comum é a existência de um processo acelerado de
desenvolvimento de {\it software} sem preocupações relacionadas a operações.
No final, quando o time de desenvolvimento tem um mínimo produto viável, ele
o envia para o time de operações para publicação. Conhecendo poucas coisas sobre
a natureza do {\it software} e como ele foi produzido, o time de operações tem
que criar e configurar o ambiente necessário e publicar o {\it software}. Neste
cenário, a entrega de {\it software} tipicamente atrasa e os conflitos entre
os times se manifestam. Quando uma \cc é fomentada, os times colaboram para
executar as tarefas desde o primeiro dia do desenvolvimento do {\it software}.
Com o constante exercício das práticas de provisionamento, gerenciamento,
configuração e {\it deployment}, a entrega de {\it software} se torna mais
natural, reduzindo os atrasos e, consequentemente, os conflitos entre os times.

\begin{mq}
``\emph{Nós trabalhamos usando uma abordagem ágil, com {\it sprints} de quinze
dias, onde a gente foca em produzir {\it software} e gera novas versões em
altíssima frequência. Mas, na hora da entrega do {\it software} é que as
complicações começavam a aparecer. O trabalho de construir todo o ambiente
e fazer o {\it deploy} não fazia parte das {\it sprints}, a gente focava apenas
em codificar a aplicação. As entregas frequentemente atrasavam,
e a gente tinha que entregar com atrasos de semanas, o que não era bom nem para
nós e nem para os clientes.}'' (P6, {\it Developer}, Portugal)
\end{mq}

Um dos resultados da construção de uma \cc é que o time de desenvolvimento não
precisa mais parar o seu trabalho aguardando pela disponibilização de um
servidor de aplicação, ou pela execução de um {\it script} na base de dados, ou
ainda pela publicação de uma nova versão do {\it software} em um ambiente de
testes. Todos os envolvidos precisam conhecer a maneira como todas essas coisas
são feitas e, com a colaboração do time de operações, isso pode ser executado
de maneira regular. Se uma tarefa pode ser executada pelo time de desenvolvimento
e há confiança entre os times, essa tarefa será incorporada ao processo de
desenvolvimento de uma maneira natural, manifestando o segundo conceito
relacionado à categoria \cc: {\bf empoderamento do desenvolvimento de {\emph software}}.

\begin{mq}
``\emph{Não era viável a gente ter tantos desenvolvedores produzindo artefatos e
parando o trabalho deles para aguardar outro time completamente separado
publicar. Ou também precisar de um ambiente de teste e ter que esperar o time
de operações prover isto quando possível. Essas atividades tinham que estar
disponíveis para servir rapidamente ao time de desenvolvimento. Com DevOps
nós conseguimos suprir essa necessidade de liberdade e mais poder para executar
algumas tarefas que são intrisicamente relacionadas ao trabalho deles.}''
(P5, {\it Systems Engineer}, Espanha)
\end{mq}

Uma \cc requer {\bf pensamento de produto}, em substituição a pensamento de
desenvolvimento ou de operações. O time de desenvolvimento precisa compreender
que o {\it software} é um produto que não se encerra após o ``{\it push}'' do código
para o repositório de código-fonte, e o time de operações precisa entender que
o processo também não se inicia quando um artefato é recebido para publicação.
{\bf Pensamento de produto} é o terceiro conceito relacionado à categoria
principal.

\begin{mq}
``\emph{Nós alteramos o perfil profissional buscado em nossas contratações. Nós
queríamos contratar pessoas que tivessem uma visão de produto. Pessoas que eram
capazes de olhar para um problema e pensar na melhor solução para ele. Mas não
apenas pensar em uma solução de software, pensar também no momento em que essa
aplicação vai ser publicada. Nós também reunimos os desenvolvedores para
reforçar que todos deviam atuar dessa maneira. Todos deviam pensar no produto e
não apenas em seu código ou em sua infraestrutura.}" (P12, Cloud Engineer,
Estados Unidos)
\end{mq}

Deve haver uma {\bf comunicação direta} entre os times. Sistemas de {\it
ticket} são citados como um meio típico e inadequado de comunicação entre os
times de desenvolvimento e operações. A comunicação face a face é a melhor
opção, mas considerando que nem sempre é viável, o uso contínuo de ferramentas
como \emph{Slack}\footnote{https://slack.com/} e \emph{Hipchat}\footnote{https://www.hipchat.com/}
é citado como opção apropriada. {\bf Comunicação direta} é o quarto conceito
relacionado à categoria principal.

\begin{mq}
``\emph{Nós também usamos essa ferramenta ({\it Hipchat})
como uma maneira de facilitar a comunicação entre os times de desenvolvimento e
operações. O ritmo de trabalho é bastante acelerado, e por isso não é viável ter
uma comunicação burocrática. Pra agilizar as coisas, a gente usa bastante o
{\it Hipchat}, todos estão sempre atentos às mensagens, as respostas são
rápidas e a gente tem um controle bem adequado por lá. Isso deu muita liberdade
nas tarefas de desenvolvimento, em caso de qualquer dúvida, a equipe de
operações está ao alcance de uma mensagem.}" (P5, Engenheiro de Sistemas, Espanha)
\end{mq}

Existe uma {\bf responsabilidade compartilhada} de identificar e corrigir os
problemas de um {\bf software} ao fazer a transição para produção. A estratégia
de evitar a responsabilidade deve ser evitada. A equipe de desenvolvimento não
deve afirmar que uma determinada questão é um problema na infraestrutura, então
é responsabilidade da equipe de operações. Ou o contrário, a equipe de operações
não deve afirmar que uma determinada falha foi motivada por um problema na
aplicação, então é responsabilidade da equipe de desenvolvimento. Um contexto de
{\bf blameless} deve ser fomentado. Os times precisam focar na resolução dos
problemas e não em encontrar um culpado e fugir da responsabilidade. O contexto
de {\bf responsabilidade compartilhada} envolve não apenas a resolução de
problemas, mas também qualquer outra responsabilidade inerente ao produto de
{\it software} deve ser compartilhada. {\bf Blameless} e {\bf responsabilidade
compartilhada} são os conceitos restantes da categoria principal.

\begin{mq}
``\emph{Como consequência dessa busca contínua por melhoria da qualidade, em um
momento já avançado do processo, quando nós já tínhamos um bom nível de
colaboração, automação e tudo mais, nós identificamos um ponto de melhoria na
nossa cultura. Nós percebemos que algumas pessoas tinham medo de cometer erros.
Nossa cultura não era forte o suficiente para fazer com que todos se sentissem
à vontade para inovar e experimentar sem medo de errar. Nós fizemos um grande
esforço para espalhar essa ideia de que não há culpados por qualquer problema
que possa ocorrer. Nós fazemos todo o possível para evitar falhas, mas elas
vão acontecer, e apenas sem essa busca por culpados nós vamos ser capazes de
resolver os problemas rapidamente.}" (P8, {\it DevOps Engineer}, Brasil)
\end{mq}

À primeira vista, considerar a criação e o fortalecimento da \cc como o passo
mais importante para a adoção de {\it DevOps} pode parecer um tanto óbvio, mas
os próprios entrevistados citaram alguns equívocos que consideram recorrentes
em não priorizar esse aspecto na adoção de {\it DevOps}:

\begin{mq}``\emph{Na adoção de {\it DevOps}, há uma questão cultural muito forte
que os times muitas vezes não estão adaptados. Relacionado a isso, uma coisa
que me incomoda muito e que eu vejo acontecer muito é que as pessoas tomam
{\it DevOps} exclusivamente por ferramentas ou automação}'' (P9, Gerente de
\acrshort{TI}, Brasil)
\end{mq}

Além da categoria principal (\cc), foram identificados três outros conjuntos de
categorias: os \emph{facilitadores} da adoção de {\it DevOps}, as \emph{saídas}
da adoção de {\it DevOps}, e as categorias que são tanto facilitadores como
saídas. Uma explicação mais abrangente sobre esses dois papéis desempenhados
pelas categorias de conceitos em uma adoção de {\it DevOps} será apresentada
na seção \ref{secao_teoria}. Nas próximas seções as categorias são descritas
por meio dos seus conceitos relacionados.

\section{Facilitadores \emph{DevOps}}\label{secao_facilitadores}

Aqui serão detalhadas as duas categorias que sustentam a adoção das práticas
{\it DevOps}: \cat{automação} e \cat{compartilhamento e transparência}.

\subsection{Automação} \label{ssec:automation}

Essa é a categoria que apresenta o maior número de conceitos relacionados. Isso
ocorre porque procedimentos manuais são considerados fortes candidatos para
propiciar a formação de um silo, dificultado a criação de uma \cc. Se uma tarefa
é manual, uma pessoa ou um time será responsável por executá-la. Apesar de
\cat{compartilhamento e transparência} poderem ser usados para garantir a
colaboração mesmo em tarefas manuais, com a \cat{automação}, os pontos onde os
silos podem surgir são minimizados.

\begin{mq}
``\emph{Quando um desenvolvedor precisava criar uma nova aplicação, o workflow
antigo exigia que ele criasse um ticket para a equipes de operações, que então
avaliava e resolveria manualmente o problema solicitado. Essa tarefa podia levar
muito tempo e não havia visibilidade entre os times sobre o que estava
acontecendo [\ldots]. Hoje, esses silos não existem mais dentro da empresa,
em particular porque não é mais necessário executar todas essas tarefas
manualmente, tudo foi automatizado.}" (P12, Cloud Engineer, Estados Unidos)
\end{mq}

Além de contribuir para transparência, a \cat{automação} de procedimentos
também é considerada importante para garantir \emph{reprodutibilidade} de
tarefas, reduzindo retrabalho e risco de falha humana. Consequentemente,
\cat{automação} aumenta a confiança entre os times o que é um aspecto importante
da \cc.

\begin{mq}
``\emph{Antes de nós adotarmos DevOps, havia muito trabalho manual. Por exemplo,
se você precisasse criar um esquema no banco de dados, era um processo manual;
se você precisava criar um servidor de banco de dados, era um processo manual;
se você precisasse criar instâncias EC2\footnote{Amazon Elastic Compute Cloud}
adicionais, mais uma vez um processo manual. Este trabalho manual era demorado
e muitas vezes causava erros e retrabalho.}'' (P1, {\it DevOps Developer}, Irlanda)
\end{mq}

\begin{mq}
``\emph{Nossa principal motivação para adotar DevOps foi basicamente reduzir o
retrabalho. Quase toda semana a gente tinha que basicamente construir novos
servidores e iniciá-los manualmente, o que era muito demorado.}'' (P4, Técnico
em Computação, Brasil)
\end{mq}

Os oito conceitos da categoria \cat{automação} são detalhados a seguir.
Todas as entrevistas continham explicações sobre (1) {\bf automação do \emph{deployment}},
como parte da adoção de {\it DevOps}. A entrega de {\it software} é a
manifestação mais clara da entrega de valor no desenvolvimento de {\it software}.
Em caso de problemas no {\it deployment}, a expectativa de entregar valor ao
negócio pode rapidamente gerar conflitos e manifestar a existência de silos.
Desta forma, a \cat{automação} normalmente aumenta a agilidade e a confiabilidade.
Alguns outros conceitos de automação giram exatamente em torno da automação do
{\it deployment}.

É importante observar que a ocorrência frequente de {\it deployments} bem
sucedidos não é suficiente para garantir a geração de valor para o negócio.
Certamente, a qualidade do software é mais relevante. Portanto, para que possam
fazer parte do {\it pipeline} do {\it deployment}, as verificações de qualidade
também precisam ser automatizadas, como é o caso da (2) {\bf automação de testes}.
Além disso, para automatizar o {\it deployment} de aplicações, o ambiente em
que elas serão executadas precisa estar disponível. Portanto, a (3) {\bf automação
do provimento da infraestrutura} também deve ser considerada no processo. Além
de estar disponível, o ambiente precisa ser configurado adequadamente,
incluindo a quantidade de memória e CPU disponibilizada, as versões corretas de
bibliotecas  e a estrutura do banco de dados. Se a configuração de algum
desses aspectos não tiver sido automatizada, o {\it deployment} automatizado
pode não funcionar. Portanto, a (4) {\bf automação do gerenciamento da
infraestrutura} é outro conceito da categoria \cat{automação}.

{\it Software} moderno é tipicamente construído em torno de serviços.
Microsserviços foram comumente citados como um aspecto da adoção de {\it DevOps}.
Para Fowler e Lewis \cite{martinfowler2014microservices}, no estilo arquitetural
de microsserviços, os serviços precisam ser independentemente implantáveis por
mecanismos de {\it deployment} totalmente automatizados. Essa parte das
características de microsserviços relacionada a \cat{automação} foi aqui
denominada de (5) {\bf serviços autônomos}. Os relatos sobre a adoção de {\it DevOps}
tipicamente citam a (6) {\bf conteinerização} como uma maneira de automatizar o
provisionamento do ambiente onde esses serviços autônomos são executados: os
contêineres. (7) {\bf Automação do monitoramento} e (8) {\bf automação da
recuperação} são os conceitos restantes. O primeiro refere-se à capacidade de
monitorar as aplicações e a infraestrutura subjacente sem intervenção humana.
Um exemplo clássico é o uso generalizado de ferramentas para enviar mensagens
relatando alarmes - por meio de SMS, {\it Slack} / {\it Hipchat}, ou até mesmo
chamadas de celular - em caso de incidentes relacionados às aplicações
detectados automaticamente. E o segundo está relacionado à capacidade de
substituir um componente que não está funcionando adequadamente ou reverter uma
falha no {\it deployment} sem intervenção humana.

\subsection{Transparency and Sharing} Represents the grouping of concepts
emerged from recurrent interviews that help to disseminate concepts and
activities. Training, tech talks, committees lectures, and round tables
are examples of these events. Creating
channels using communication tools is another recurrent topic
related to \cat{sharing} along the processes of DevOps adoption.
According with the content of what is shared, we
have identified three main concepts:

\begin{itemize}
\item Knowledge sharing: the professionals interviewed mention a wide range of
skills they need to acquire during the adoption of DevOps, citing
structured events of sharing to smooth the learning curve of both technical and
cultural knowledge.


\item Activities sharing: where the focus is on sharing how simple tasks can or
should be performed. Communication tools, committees, and round tables are the common
forum for sharing this type of content.

\item Process sharing: here, the focus is on sharing whole working processes. The
content is more comprehensive than in sharing activities. Tech talks and
lectures are the common forum for sharing processes.

\end{itemize}

Sharing concepts contribute with the \cc. For example,
all team members gain best insight about the entire software production
process, with a solid understanding of shared responsibilities. A shared vocabulary also
emerged from \cat{sharing} and this facilitates communication.

The use of \textbf{infrastructure as code} was
recurrently cited as a means for guaranteeing that everyone knows how the execution environment of
an application is provided and managed. Bellow, we present an interview
transcript which sums up this concept.

\begin{mq}
``\emph{So, here we adopted this type of strategy that is the infrastructure as code,
consequently we have the versioning of our entire infrastructure in a common
language, in such a way that any person, a developer, an architect, the
operations guy of even the manager, he looks and is able to describe that the
configuration of application x is y. So, it aggregates too much value for us
exactly with more transparency.}" (P12, Cloud Engineer, United States)
\end{mq}

Regarding cat{transparency and sharing}, we also found the concept of \textbf{sharing on a regular basis}, which suggests
that sharing should be embedded in the process of software
development, in order to contribute effectively to transparency.
As we will detail in the \emph{continuous integration} concept of
the \cat{agility} category, a common way to integrate all tasks is a pipeline. Here, there is the
concept of \textbf{shared pipelines}, which indicates that the code of pipelines
must be accessible to everyone, in order to foment transparency.

\begin{mq}
``\emph{The code of how the infrastructure is
made is open to developers and the sysadmins need to know some aspects of how
the application code is built. The code of our pipelines is accessible to
everyone in the company to know how activities are automated}" (P13, Technology
Manager, Brazil)
\end{mq}


\section{\emph{Saídas} da Adoção de {\emph DevOps}}\label{secao_saidas}

%In this section we detail the categories that correspond to
%the expected consequences with the adoption of
%DevOps practices, including \cat{agility} and \cat{resilience};
%as discussed as follows.

\subsection{Agility}

Agility is frequently discussed as a major outcome of DevOps adoption. With more
collaboration between teams, \textbf{continuous integration} with execution of
multidisciplinary pipelines is possible and it is an agile related concept
frequently explored. These pipelines might contain
infrastructure provisioning, automated regression testing, code analysis,
automated deployment and any other task considered important to
execute continuously.

These pipelilnes encourage two other agile concepts: \textbf{continuous
infrastructure provisioning} and \textbf{continuous deployment}. The latter is
one of the most recurrent concepts identified in the interview analysis. Before
DevOps, deployment had been seen as a major event with high risk of downtime and
failure involved. After DevOps, the sensation of risk in deployment decreases
and this activity became more natural and frequent. Some practitioners claim
to perform dozens of deployments daily.

\subsection*{Resilience}

Also related to an expected outcome of adopting DevOps, \cat{resilience} in this
context refers to the ability of applications to adapt quickly to adverse situations.
The first related concept is \textbf{auto scaling}---i.e.,
allocating more or less resources to applications that increase or
decrease on demand. Another concept related to
the \cat{resilience} category is \textbf{recovery automation}, that is
the capability of the applications and infrastructure to recovery itself in case of
failures. There are two typical cases of recovery automation: (1) in cases
of some instability in the execution environment of an application (a
container, for example) occurs an auto restart of that environment; and (2) in
cases of new version deployment, if the new version does not work properly, the
previous one must be restored. This auto restore of a previous version
decrease the chance of downtimes due to errors in specific versions, which
is the concept of \textbf{zero down-time}, the last one of the \cat{resilience} category.

\section{Categorias que são tanto \emph{Facilitadores} como \emph{Saídas} da Adoção de \emph{DevOps}}
\label{secao_facilitadores_e_saidas}

Finally, here we detail the categories that are both enablers
and outcomes, including \cat{continuous measurement}
and \cat{quality assurance}; as discussed as follows.

\subsection{Continuous Measurement}

As an enabler, performing regularly the
activities of measurement and sharing
contributes to avoid existing silos and reinforce the \cc, because it is
considered a typical responsibility of the operations team.

\begin{mq}
``\emph{Before, we had only sporadic looks to
zabbix\footnote{\url{https://www.zabbix.com/}} to check if everything was OK.
At most someone would stop to look memory and CPU consumption. To maintain a
the quality of services, we expanded this view of metrics collection so that it
became part of the software product. We then started to collect metrics continuously
and with shared responsibilities. For example, if an overflow occurred in the
number of database connections, everyone received an alert and had
the responsibility to find solutions to that problem. %This is an interesting example of
%metrics that everyone started to be more attentive, not only the operations
%team.
}" (P3, DevOps Developer, Ireland)
\end{mq}

As an outcome, the continuously collection of metrics from applications and
infrastructure is a required consequence of DevOps adoption. It occurs because
the resultant agility increases the risk of something going wrong. The team
should be able to react quickly in case of problems, and the continuous
measurement allows it to be proactive and resilient.

\begin{mq}
``\emph{With DevOps we can do deployment all the time and, consequently, there was
the need of greater control of what was happening. So, we used
grafana\footnote{\url{https://grafana.com/}} and
prometheus\footnote{\url{https://prometheus.io/}} to follow everything that is
happening in the infrastructure and in the applications. We have a complete
dashboard in real time, we extract reports and, when something goes wrong, we
are the first to know.}" (P10, Network Administrator, Brazil)
\end{mq}

Continuous monitoring involves \textbf{application log monitoring} (1), a
concept that corresponds to the use of the log produced by
applications and infrastructure as data source. The concept of
\textbf{continuous infrastructure monitoring} (2) indicates that the monitoring
is not performed by a specific person or team in a specific moment. The
responsibility to monitor the infrastructure is shared and it is executed in
daily. \textbf{Continuous application measurement} (3), in turn, refers to
the instrumentation to provide metrics that are used to evaluate aspects and
often direct evolution or business decisions. All these monitoring/measurement
can occur in an automated way, the \textbf{monitoring automation} already been
detailed in subsection \ref{ssec:automation}.

\subsection{Quality Assurance}

In the same way as continuous measurement, quality assurance is a category that
can work both as enabler and as outcome. As enabler because an increasing quality
leads to more confidence between the teams, which in the end generates a virtuous
cycle of collaboration. As outcome, the principle is that it is not
feasible to create a scenario of continuous delivery of software without control
about the quality of the products and its production processes.

Respondents pointed to the need for sophisticated control of which code should
be part of deliverables that are continuously delivered. Git Flow was
recurrently cited as suitable \textbf{code branching} (1) model, the first
concept of quality assurance.
In a previous section, we explored the automation face of
microservices and testing. These elements have also a quality assurance face.
Another characteristic of microservices is the need for small services focusing
in doing only one thing. These small services are easier to scale and
structure, which manifest a quality assurance concept: \textbf{cohesive
services} (2). Regarding testing, another face is \textbf{continuous
testing} (3). To ensure quality in software products, we found that
tests (as well as other quality checks) should occur continuously. Continuous testing
is considered challenging without automation, and this reinforces the need for automated
tests.

Another two concepts cited as part of quality assurance in DevOps adoption are
the use of \textbf{source code static analysis} (4) to compute quality metrics in
source code and the \textbf{parity between environments} to
reinforce transparency and collaboration during software development.


\section{Uma Teoria Sobre a Adoção de \textit{DevOps}}\label{secao_teoria}

\iffalse
Os resultados de um estudo de grounded theory, como o próprio nome do método sugere, baseiam-se nos dados coletados, de modo que as hipóteses emergem dos dados.
Grounded Theory deve descrever as relações-chave entre as
categorias que a compõem, ou seja, um conjunto de hipóteses inter-relacionadas ~ \ cite {hoda2017becoming}.

Apresentamos as categorias de nossa grounded theory sobre a adoção do DevOps como uma rede das três categorias de facilitadores  (\cat{automation}, \cat{sharing and transparency}) que são comumente usados para desenvolver a categoria central\ cc, conforme discutido na seção anterior.
Baseado em nosso entendimento, implementar os facilitadores para desenvolver o \ cc normalmente leva aos conceitos relacionados a duas categorias de resultados esperados:
\ cat {agilidade} e \ cat {resiliência}.

Além disso, existem duas categorias que podem ser consideradas
tanto como facilitadoras quanto como resultados: \ cat {medição contínua} e \ cat {garantia de qualidade}.
Nesta seção, descrevemos as relações entre essas categorias, construindo uma teoria
de adoção do DevOps.
\subsection{A General Path for DevOps Adoption}

Na seção~\ref{sec:introduction} apresentamos a questão geral desta
pesquisa: existe algum caminho recomendado para adotar DevOps? Aqui, nós elaboramos uma resposta,
com base nas análises realizadas conforme detalhado na seção~\ref{sec:research_method}. O principal
ponto que deve ser formulado é a construção de um \cc entre as equipes de desenvolvimento e operações de software e
atividades relacionadas. De acordo com nossas descobertas, as outras categorias,
muitas das quais também estão presentes em outros estudos que investigaram DevOps,
só fazem sentido se as práticas e conceitos relacionados à elas contribuírem para o nível de um \ cc ou levarem às conseqüências esperadas
de um \ cc. Esse entendimento induz à várias hipóteses, como discutido a seguir.


\begin{mh}
\ textbf {Hypothesis 1:} \ textit {Existe um grupo de categorias relacionadas à adoção de DevOps que só fazem sentido se usadas para aumentar o nível} \ cc \ emph {. Nós
chamamos esse grupo de categorias de \ textbf {enablers}}.
\ end {mh}

Com base nesta primeira hipótese, a maturidade da adoção de DevOps não
avança em situações em que apenas uma equipe é responsável por entender, adaptar ou
evoluir a automação --- mesmo quando essa automação suportar diferentes atividades, como implantação, provisionamento de infraestrutura,
monitoramento. O mesmo vale para as outras categorias \emph{enable}. Ou seja, nas situações em que
\cat{transparência e compartilhamento} não contribuem para
o \cc, eles não contribuem para a adoção de DevOps como um todo. Alguns exemplos
que suportam nossa primeira hipótese incluem:

%\begin{mq}
%``\emph{Olhe, dentro do setor de operações houve algum grau de automação. O cara
% tinha armazenado em sua própria máquina bash scripts que o ajudaram na criação de um
% server ou ao criar uma nova instância de banco de dados. No entanto, não houve DevOps
% porque não havia relação intrínseca dessa automação com o
% processo de desenvolvimento} "(P11, DevOps Supervisor, Brazil)
% \ end {mq}


\begin{mq}
``\emph{DevOps envolve ferramentas, mas DevOps não é uma ferramenta. Isto é, as pessoas muitas vezes
focam no uso de ferramentas que são chamadas de `DevOps tools ', acreditando que DevOps é
isto. Eu sempre insisto que DevOps não é uma ferramenta, DevOps envolve o uso de
ferramentas para melhorar os procedimentos de desenvolvimento de software.} "(P2, DevOps
Consultor, Brasil)
\end{mq}


%% \begin{mq}
%% ``\emph{Keeping the culture alive remains a challenge to us, and it is very
%% important. Here in our company, for example, we have Tech Talks that are
%% monthly conversations that we have with the teams. The purpose of these Tech
%% Talks is to share knowledge about technologies and work processes increasing the
%% transparency of how everything works. We also have a Slack channel called
%% DevOps as Culture where we discuss things of DevOps culture. The idea is not to
%% let the culture die, we are always feeding it with something, because that is
%% the DevOps essence for us.}" (P12, Cloud Engineer, United States)
%% \end{mq}

\begin{mh}
\textbf{Hypothesis 2:} \textit{There is a group of categories related to DevOps adoption
that does not contribute to increase the} \cc \emph{level, but that instead are
pointed out as DevOps adoption related, because they emerge as an expected or
necessary consequence of the adoption. These categories represent the group of
\textbf{outcomes}}.
\end{mh}

In a first moment, the simple fact that a team is more
\cat{agile} in delivering software, or more \cat{resilient} in failure recovery, does not
contribute directly to bring operations teams closer to development teams.
Nevertheless, a signal of a mature DevOps adoption is an increasing of the capacity for continuously
delivering software (and thus being more \cat{agile})
and for building \cat{resilient} infrastructures.

\begin{mh}
\textbf{Hypothesis 3:} \textit{The categories \cat{Continuous Measurement} and \cat{Quality Assurance}
are both related to DevOps enabling capacity and to DevOps outcomes}.
\end{mh}

Measurement is cited as a typical responsibility of the operations team.
At the same time that sharing this responsibility reduces silos,
it is also cited that measurement is a necessary consequence of DevOps adoption. Particularly because
the continuous delivery of software requires more control,
which is supplied by concepts related to the \cat{continuous measurement} category.
The same premise is valid to the \cat{quality assurance} category. At first glance,
\cat{quality assurance} appears as one response to the context of agility in operations
as a result of DevOps adoption. But, the efforts in quality assurance of software products
increase the confidence between the development and operations teams, increasing the level
of \cc.

\subsection{DevOps Enablers}

Altogether, DevOps enablers are the means commonly used to increase the level of
the \cc in a DevOps adoption process.
We have identified five categories of DevOps enablers:
\cat{ Automation}, \cat{Continuous Measurement}, \cat{Quality Assurance},
\cat{Sharing}, and {\cat{Transparency}. Another finding of our
study leads to our fourth hypothesis.

\begin{mh}
\textbf{Hypothesis 4:} \textit{There is no precedence between enablers in a DevOps adoption process}.
\end{mh}

We have realized that the adoption process might not have
to priorize any enabler, and a company that aims to implement
DevOps should start with  the enablers that seem more appropriate (in terms
of its specificities). Accordingly, we did not find any evidence that an enbler
is more efficient than another for creating a \cc. \cat{Automation} is the category
that appears more frequently in our study, though several participants make
clear that associating DevOps with automation is a misconception.
%% For
%% instance, although 14 respondents cite \cat{automation} as an important
%% enabler to adopt DevOps, some respondents also ponder that considering
%% automation with greater importance than other parts can actually be a risk:

\begin{mq}
``\emph{I think that the expansion of collaboration between teams involved other
things, it was not just automation. There must be an alignment with the
business needs. (...) I think that DevOps made possible a broader understanding
of software production and we were realizing exactly that it is not about
automating everything. (...) So, I see with caution a supposed vision that automate things can
be the way to implement DevOps.}" (P7, Support Analyst, Brazil)
\end{mq}

%\begin{mq}
%``\emph{Despite of we actually use automation in a reasonable number of scenarios,
%we have been able to develop our culture significantly without automation and I think that you can reach a
%good DevOps level with little or even no automation.}" (P8, DevOps Engineer, Brazil)
%\end{mq}

%% That is, although \cat{automation} is a very commonly used enabler, it is possible to
%% increase the level of \cc without focus on automating. And
%% this premise is valid to the other enablers.

\subsection{DevOps Outcomes}

DevOps outcomes is that group of categories that does not produces primarily the
expected effect of a DevOps enabler, typically concepts that are expected as
consequences of an adoption of DevOps. We have identified four categories that
can work as DevOps outcomes: \cat{agility}, \cat{continuous measurement},
\cat{quality assurance}, and \cat{software resilience}. Note that,
as mentioned before, \cat{continuous measurement} and \cat{quality assurance}
are both enablers and outcomes.

That is, a well succeeded DevOps adoption typically increases the potential of
\cat{agility} of teams and enables \cat{continuous measurement}, \cat{quality assurance} and
\cat{resilience} of applications.
However, in some situations, this potential is not completely used due business
decisions. For example, one respondent cited that, at a first moment, the
company did not allowed the continuous deployment (more potential of agility)
of applications in production:

\begin{mq}
``\emph{We had conditions and security to continuously publish in production,
however, in the beginning the managers were afraid and decided that the
publication would happen weekly.}" (P9, IT Manager, Brazil)
\end{mq}
\fi


\section{Trabalhos Relacionados}

Após construída, a teoria descrita acima foi reintegrada com a literatura
existente sobre {\it DevOps}. A subseção \ref{secao_caracterizacoes_devops}
apresenta as semelhanças e diferenças de trabalhos que foram construídos
visando caracterizar {\it DevOps}, tipicamente por meio da identificação de
elementos relacionados. Já na subseção \ref{secao_adocao_6_companhias} é
feita a comparação com um estudo cujo foco foi similar, de caracterizar a
adoção de {\it DevOps} em companhias de mercado. A seção \ref{como_superar_desafios}
apresenta uma comparação de um trabalho cujo foco foi identificar desafios na
adoção de {\it DevOps} e que apresentou algumas orientações a respeito de como
superar estes desafios. Por fim, na subseção \ref{modelo_maturidade} a teoria é
comparada a um trabalho relacionado no qual é proposto um modelo de maturidade
para adoção de {\it DevOps} e também detalhes de como progredir nesse modelo,
aumentando o nível de maturidade de {\it DevOps} de uma companhia.

\subsection{Trabalhos com Foco em Caracterizar DevOps}\label{secao_caracterizacoes_devops}

Conforme mencionado acima, os trabalhos comparados nessa subseção tiveram como
foco principal caracterizar {\it DevOps}, o que difere do propósito da produção
da teoria que foi de caracterizar {emph como DevOps é adotado}. Embora
exista essa diferença de foco, existe uma sobreposição de resultados que se
justifica por ser natural que os elementos {\it DevOps} apareçam no processo de
adoção de {\it DevOps}. Além das diferenças de foco, existem algumas
diferenças metodológicas em relação ao estudo aqui proposto. Essas diferenças
são resumidas na tabela \ref{related_work_table}.

\begin{table}[hb!]
\centering
\caption{Comparação de Metodologia e Foco dos Trabalhos Relacionados}
\label{related_work_table}
\begin{tabular}{|p{3cm}|p{6cm}|p{6cm}|}
\hline

\textbf{Estudo}
& \textbf{Abordagem de Pesquisa}
& \textbf{Foco do Trabalho} \\

\hline

\textbf{Este estudo}

& Abordagem {\it grounded theory} entrevistando praticantes que contribuíram
para a adoção de {\it DevOps} em 15 companhias de 5 países.

& Explicar {\bf como} {\it DevOps} tem sido adotado de maneira bem sucedida na prática de mercado;\newline
\newline Prover orientações que possam ser usadas em novas adoções de {\it DevOps}. \\

\hline

\textbf{J. Smeds et al.~\cite{devops_a_definition}}

& Revisão sistemática de literatura; \newline \newline Entrevistas semi-estruturadas
com 13 funcionários de uma única companhia cujo processo de adoção de {\it DevOps}
estava em um estágio inicial.

& Identificar na literatura as principais características que definem {\it DevOps}
\newline \newline
Construir uma lista de possíveis impedimentos na adoção de {\it DevOps} por meio
de um estudo empírico. \\

\hline

\textbf{Lwakatare et al.~\cite{extending_dimensions}}

&
Revisão de literatura multivocal utilizando dados da literatura cinza; \newline \newline
Três entrevistas com praticantes de desenvolvimento de {\it software} em uma
companhia que estava aplicando práticas {\it DevOps} em um projeto.

& Identificar como praticantes descrevem {\it DevOps} como um fenômeno; \newline
\newline Identificar as práticas {\it DevOps} de acordo com os praticantes de
desenvolvimento de {\it software}. \\

\hline

\textbf{Fran\c{c}a et al.~\cite{characterizing_devops}}

& Revisão de literatura multivocal com procedimentos de análise qualitativa de
{\it grounded theory}.

& Prover uma definição para {\it DevOps}; \newline \newline
Identificar práticas {\it DevOps}, habilidades requeridas, características,
benefícios, e problemas motivando a adoção de {\it DevOps}. \\

\hline

\end{tabular}
\end{table}

Em relação aos trabalhos apresentados nessa subseção, a teoria apresentam detalhes que ajudam a
compreender como podem ser respondidas algumas questões práticas a
respeito da adoção de {\it DevOps} que permaneceram em aberto: (1) Existe um caminho
recomendado para se adotar \textit{DevOps}? (2) Já que \textit{DevOps} é
constituído de múltiplos elementos, eles possuem a mesma relevância quando se
adota \textit{DevOps}? (3) Qual é o papel desempenhado por cada um desses
elementos - tais como medição, compartilhamento e automação - em uma adoção de
\textit{DevOps}?

\subsection{A Caracterização da Adoção de DevOps em 6 Companhias Proposta por Erich et al. \cite{qualitative_devops_journalsw_17}}\label{secao_adocao_6_companhias}

Systematic literature review;
Interviews with practitioners from 6 companies across 3 countries.

Identify how literature defines DevOps;
Investigate how DevOps is being implemented in practice.

O trabalho de Erich et al. \cite{qualitative_devops_journalsw_17} investigou
o processo de adoção de \textit{DevOps} por meio de entrevistas com praticantes
de seis companhias. Todavia, as percepções obtidas não detalham a maneira como
as companhias aplicaram os elementos de \textit{DevOps} ao longo do processo,
e são apresentadas de maneira individualizada, não se colocando em perspectiva
o que há de comum no entendimento dessas companhias. Os dados de como cada
companhia relata ter implementado \textit{DevOps} são apresentados de maneira
sucinta e os principais pontos são descritos a seguir.

Na primeira companhia, é relatado que o pessoal de desenvolvimento e
operações passou a trabalhar de maneira conjunta diariamente, foi criado um
novo cargo na companhia chamado \textit{DevOps Engineer}, o pessoal de gestão
foi treinado para ser capaz de gerenciar tanto pessoas de desenvolvimento como
de operações, o pessoal de recursos humanos alterou os processos de avaliação e,
por fim, a organização está automatizando seus processos de infraestrutura, não
são apresentados detalhes de como isso está sendo feito.

Na segunda companhia, \textit{DevOps} é visto como uma extensão de
\textit{Scrum} e \textit{Lean}. A adoção de \textit{DevOps} lá foi guiada por
dois princípios: colaboração diária e ambiente de trabalho compartilhado.
Havia a premissa de que o \textit{software} deve ir para produção a cada
iteração. É relatado então que a organização usou \textit{frameworks}
comerciais para guiar a implantação de \emph{entrega contínua}, também sem maiores
detalhamentos de como foi feito.

Na terceira companhia, é relatado que três times que trabalhavam em
funcionalidades separadas foram agrupados em um \emph{time DevOps}. A organização
passou então a experimentar \textit{entrega contínua}, mas ainda estava em
um estágio inicial quando a entrevista foi realizada.

A quarta companhia criou uma única equipe de \textit{DevOps} para
experimentar ferramentas, princípios e práticas \textit{DevOps}. Os
membros dessa equipe possuem habilidades multidisciplinares. É relatado ainda
o uso de ferramentas para controle de versão e o desenvolvimento de ferramentas
próprias para criar facilmente ambientes de nuvem.

Na quinta companhia, é relatado que \textit{DevOps} é visto como
um papel responsável por gerenciamento de incidentes, gerenciamento de
capacidade, gerenciamento de riscos e suporte ao processo de criação.
O uso de computação em nuvem ocasionou a necessidade de se existir uma
abordagem mais sistemática para execução das atividades de operações, exigindo
que o pessoal responsável aprendesse técnicas de desenvolvimento de
\textit{software}. O time de desenvolvimento foi dividido em vários subtimes,
um dos quais focado em \textit{DevOps}, com responsabilidade de apoiar os demais
times. São também enumerados alguns princípios e práticas utilizados na companhia
durante a adoção de \textit{DevOps}.

Por fim, com relação a como a sexta companhia implementou \textit{DevOps},
é apresentado um detalhamento de como ocorre a \emph{integração contínua} na
organização. As evoluções de código são feitas por meio de \textit{pull
requests} que após aprovadas são integradas ao código principal. Após a
integração dos códigos das \textit{pull requests}, um \textit{pipeline} de entrega
contínua é executado para publicar o \textit{software}.

\subsection{As Considerações de Hamunen \cite{challenges_in_adopting_devops} Sobre Como Superar os Desafios na Adoção de DevOps}\label{como_superar_desafios}

\subsection{O Modelo de Maturidade Proposto por Feijter et al. \cite{feijter2017towards}}\label{modelo_maturidade}

