Conforme ilustrado no capítulo anterior, os trabalhos de pequisa existentes
apresentam algumas caracterizações de \textit{DevOps} por meio da enumeração
de conceitos, princípios e práticas relacionadas. Apesar de alguns desses
estudos conterem abordagens qualitativas para investigar empresas adotando
\textit{DevOps}, conduzindo entrevistas com praticantes, o foco desses estudos
tipicamente é em caracterizar \textit{DevOps}, ao invés de prover recomendações
na adoção de \textit{DevOps}. Além das diferenças de foco, existem algumas
diferenças metodológicas em relação ao estudo aqui proposto. Essas diferenças
são resumidas na tabela \ref{related_work_table}.

%% Table~\ref{related_work_table} compares the most relevant literature to
%% out work. Since the work of Lwakatare et al.~\cite{extending_dimensions_icsea_16} is an
%% extension of a previous contribution of the authors and her team \cite{dimensions_of_devops_xp_15},
%% we only compared the most recent study at Table~\ref{related_work_table}.

\begin{table}[hb!]
\centering
\caption{Comparação de Metodologia e Foco dos Trabalhos Relacionados}
\label{related_work_table}
\begin{tabular}{|p{3cm}|p{6cm}|p{6cm}|}
\hline

\textbf{Estudo}
& \textbf{Abordagem de Pesquisa}
& \textbf{Foco do Trabalho} \\

\hline

\textbf{Este estudo}

& Grounded Theory study interviewing practitioners which contributed to DevOps
adoption in 15 companies across 5 countries.

& Explain how DevOps has been successfully adopted in practice;\newline
\newline Provide guidelines to be used in new DevOps adoptions. \\

\hline

\textbf{J. Smeds et al.~\cite{devops_a_definition}}

& Systematic literature review; \newline \newline Semi structured interviews
with 13 employees of a single company whose DevOps adoption process was at
an initial stage.

& Identify in literature the main defining characteristics of DevOps; \newline
\newline Build a list of possible impediments to DevOps adoption through an
empirical study. \\

\hline

\textbf{Lwakatare et al.~\cite{extending_dimensions_icsea_16}}

&
Multivocal literature review using data from gray literature; \newline \newline
Three interviews with software practitioners in one company that was applying
DevOps practices in one project.

& Identify how do practitioners describe DevOps as a phenomenon; \newline
\newline Identify the DevOps practices according to software practitioners. \\

\hline

\textbf{Fran\c{c}a et al.~\cite{characterizing_devops}}

& Multivocal literature review with qualitative analysis procedures from
Grounded Theory.

& Provide a DevOps definition; \newline \newline
Identify DevOps practices, required skills, characteristics, benefits and
issues motivating its adoption.  \\

\hline

\textbf{Erich et al.~\cite{qualitative_devops_journalsw_17}}

& Systematic literature review; \newline \newline
Interviews with practitioners from 6 companies across 3 countries.

& Identify how literature defines DevOps; \newline \newline
Investigate how DevOps is being implemented in practice. \\

\hline

\end{tabular}
\end{table}

Como consequência dessas diferenças de foco, algumas questões práticas a
respeito da adoção de DevOps permancem em aberto: (1) Existe um caminho
recomendado para se adotar \textit{DevOps}? (2) Já que \textit{DevOps} é
constituído de múltiplos elementos, eles
possuem a mesma relevância quando se adota \textit{DevOps}? (3) Qual é o papel
desempenhado por cada um desses elementos - tais como medição, compartilhamento e
automação - em uma adoção de \textit{DevOps}? Para responder a essas questões,
é necessário um entendimento holístico dos caminhos seguidos por organizações
que adotaram \textit{DevOps} de uma maneira bem sucedida.

Neste capítulo, apresenta-se um estudo utilizando a variação clássica da
metodologia \textit{\acrfull{GT}} para se caracterizar a adoção de DevOps
em organizações de mercado que foram bem sucedidas neste processo.

A teoria apresentada neste capítulo foi construída com base na percepção de
praticantes de quinze companhias de cinco países que foram bem sucedidos na
adoção de \textit{DevOps}. A teoria torna claro que praticantes interessados em
adotar \textit{DevOps} devem focar na construção de uma \cat{cultura de
colaboração}, o que previne pontos de falha comuns relacionados a focar em
ferramental ou em automação. O papel dos demais elementos que fazem parte da
adoção de \textit{DevOps} é explicado por meio da definição de dois grandes
agrupamentos de categorias de conceitos, denominados \emph{Facilitadores DevOps}
e \emph{Saídas DevOps}.

\section{\acrfull{GT}}

Introduzir GT

\acrshort{GT} foi utilizada como metodologia de pesquisa devido a três principais
razões. Primeiro, \acrshort{GT} é uma metodologia consolidada em outras áreas de
pesquisa, notadamente sociologia médica \cite{gt_medical_sociology}, nutrição
\cite{gt_nursing}, educação \cite{gt_education} e administração
\cite{gt_management}. \acrshort{GT} também tem sido cada vez mais aplicada
para estudar tópicos de engenharia de \textit{software}
\cite{hoda2017becoming,Waterman:2015:ICSE,stol2016grounded}. A segunda
razão é que \acrshort{GT} é considerada uma metodologia adequada para responder
questões de pesquisa que objetivam caracterizar cenários sob uma perspectiva
pessoal daqueles envolvidos em uma disciplina ou atividade \cite{stol2016grounded},
que é exatamente o cenário que se almeja caracterizar aqui: quais são os caminhos
seguidos por praticantes que adotaram \textit{DevOps} de uma maneira bem
sucedida? Finalmente, \acrshort{GT} possibilita a construção de um entendimento
independente e original, o que é adequado para coletar evidência empírica
diretamente da prática na indústria sem o viés das pesquisas anteriores.
A evidência coletada só é reintegrada com a literatura existente após a construção
da teoria \cite{}.

\subsection{Variações de \acrshort{GT}}

Conforme explicado no capítulo 2, existem diversas variações de \acrshort{GT} e,
segundo Stol et al. \cite{}, qualquer estudo utilizando \acrshort{GT} deve
especificar qual versão foi utilizada.

A variação clássica foi utilizada aqui, principalmente pela inexistência de
questões de pesquisa no início da pesquisa, exatamente como sugerido nessa
versão. O início da pesquisa estava ancorado em investigar uma área de
interesse: a adoção bem sucedida de \textit{DevOps} na indústria.
Adicionalmente, os trabalhos de pesquisa atualmente existentes na área
de engenharia de \textit{software} usam predominantemente a versão clássica
\cite{} e, portanto, existe um maior referencial que pode ser utilizado
como base em um novo estudo.

\subsection{Abordagem \acrshort{GT}}

A figura \ref{abordagem_gt}, proposta por Adolph et al. \cite{using_gt_adolph},
ilustra como o método foi aplicado aqui.

\figuraBib{GT}{O método GT}{using_gt_adolph}{abordagem_gt}{width=.45\textwidth}%
