Neste capítulo, apresenta-se um estudo utilizando a variação clássica da
metodologia \textit{\acrfull{GT}} \cite{glaser1967discovery} para se
caracterizar a adoção de \textit{DevOps} em organizações de mercado que foram bem
sucedidas neste processo.

A teoria aqui apresentada foi construída com base na percepção de
praticantes de quinze companhias de cinco países que foram bem sucedidas na
adoção de \textit{DevOps}. A teoria torna claro que praticantes interessados em
adotar \textit{DevOps} devem focar na construção de uma \cat{cultura de
colaboração}, o que previne pontos de falha comuns relacionados a focar em
ferramental ou em automação. O papel dos demais elementos que fazem parte da
adoção de \textit{DevOps} é explicado por meio da definição de dois grandes
agrupamentos de categorias de conceitos, denominados \emph{Facilitadores DevOps}
e \emph{Saídas DevOps}.

\section{\acrfull{GT}}

\acrfull{GT} é um método de pesquisa desenvolvido originalmente pelos
sociólogos B. Glaser e A. Strauss que possibilita a geração sistemática de
uma teoria a partir de dados analisados em um rigoroso processo \cite{glaser1967discovery}.
O objetivo de um estudo em \acrshort{GT} é entender como se dá a ação em uma
determinada área de conhecimento a partir do ponto de vista dos atores
envolvidos \cite{glaser_doing_1998}.

Ressalte-se que, de acordo com Strauss e Corbin \cite{corbin2014basics}, no
âmbito de \textit{Grounded Theory}, uma teoria é ``um conjunto bem desenvolvido
de categorias (incluindo conceitos e temas) que são sistematicamente
inter-relacionadas através de \textbf{sentenças de relacionamentos} para
formar um \textit{framework} teórico que explica algum fenômeno relevante''.

Hoda et al. \cite{hoda2017becoming} explicam que um estudo utilizando
\textit{Grounded Theory} é mais que apenas um conjunto de categorias. Ele deve
descrever os principais relacionamentos entre as categorias, e que essas
``sentenças de relacionamentos'' são aqui chamadas de hipóteses, sem prejuízo
para o uso deste mesmo termo com significado distinto em abordagens que usam
métodos estatísticos para confirmar ou refutar hipóteses pré-estabelecidas.

Segundo Locke \cite{locke2001grounded}, o diferencial de \textit{Grounded
Theory} é seu compromisso com a ``descoberta'' através do contato direto com o
mundo social de interesse, juntamente com uma limitação, a priori, do contato
com a teorização existente. Os estudos utilizando \acrshort{GT} não começam com
uma estrutura conceitual, ao invés disso pretendem culminar em uma \cite{miles1994qualitative}.
São estudos em que uma cobertura aprofundada da literatura é deliberadamente
adiada até que surjam as direções da análise de dados coletados diretamente
na fonte de interesse \cite{punch2013introduction}. Stol et al. \cite{stol2016grounded}
explicam que a principal razão para se limitar a exposição à literatura é previnir
que o pesquisador se concentre em pensar em termos de conceitos pré-estabelecidos,
testando teorias existentes, ao invés de desenvolver uma. De fato, \acrshort{GT}
é um método de desenvolvimento de teoria, e não de teste de teoria. Jantunen e
Gause \cite{jantunen_using_gt_approach}, utilizando a classificação de abordagens
de pesquisa proposta por Jarvinen \cite{jarvinen_mapping_2008}, apontam que
\acrshort{GT} é uma abordagem de desenvolvimento de teoria do grupo estudos
empíricos da realidade, conforme a área em cinza da Figura \ref{abordagens}.

\figuraBib{abordagens}{Classificação de GT nas Abordagens de Pesquisa}{jantunen_using_gt_approach}{abordagens}{width=.45\textwidth}

No entanto, conforme destaca Glaser \cite{glaser1992basics}, essa postura de
limitar a exposição à literatura é parte da abordagem apenas no início da
pesquisa. Quando a teoria proposta parece suficientemente amadurecida, então o
pesquisador pode começar a revisar a literatura no campo substantivo e
relacioná-la com seu próprio trabalho.

\subsection{Variações de \acrshort{GT}}

Desde a publicação da versão original de \acrshort{GT}, diversas modificações
e variações foram propostas ao método, ocasionando a existência de pelo menos
sete diferentes versões de \acrfull{GT}~\cite{denzin2007grounded}. Segundo
Stol et al. \cite{stol2016grounded}, as principais versões são três: (1)
\acrshort{GT} de \textit{Glaser} (GT clássica ou \textit{Glaseriana}), (2)
\acrshort{GT} de \textit{Strauss} e \textit{Corbin} (GT \textit{Straussiana}) e
(3) \acrshort{GT} de \textit{Charmaz} (GT construtivista), e qualquer
estudo utilizando \acrshort{GT} deve especificar qual versão foi utilizada.

As principais diferenças entre as versões de \acrshort{GT} estão relacionadas
à maneira de utilizar questões de pesquisa, ao papel da literatura, às técnicas
de codificação, aos tipos de questionamentos que devem ser levantadas durante
a análise dos dados e aos critérios de avaliação da teoria construída. A
Tabela \ref{versoes_GT_tabela}, adaptada de \cite{stol2016grounded}
sintetiza as principais diferenças entre as três principais versões.

\begin{table}[hb!]
\centering
\caption{Principais diferenças entre as versões de GT}
\label{versoes_GT_tabela}
\resizebox{\textwidth}{!}{
\begin{tabular}{|p{3cm}|p{5.7cm}|p{5.7cm}|p{5.7cm}|}
\hline
\textbf{Elemento} & \textbf{Clássica} & \textbf{\textit{Straussiana}} & \textbf{Construtivista} \\
\hline
\textbf{Questão de pesquisa}
& Não é definida \textit{a priori}, emerge da pesquisa. O pesquisador inicia com uma área de interesse.
& Pode ser definida antecipadamente, derivada da literatura ou sugerida por um colega, é frequentemente ampla e aberta.
& A pesquisa inicia com questões de pesquisa iniciais, que evoluem ao longo do estudo. \\
\hline

\textbf{Papel da literatura}
& Uma revisão abrangente da literatura só deve ser conduzida após a literatura possuir estágio avançado de desenvolvimento, para evitar a influência de conceitos existentes na teoria emergente.
& A literatura pode ser consultada ao longo do processo para algumas necessidades específicas, como na formualação de perguntas para coleta de dados, para obter sugestões de áreas para exemplificação teórica, etc.
& Embora concorde com os motivos de \textit{Glaser} recomendar evitar contato com a literatura no início da pesquisa, defende uma consulta adaptada da literatura para adequar o propósito do estudo. \\
\hline

\textbf{Técnicas de codificação}
& \textit{Open coding}, \textit{selective coding} e \textit{theoretical coding}.
& \textit{Open coding}, \textit{axial coding} e \textit{selective coding}.
& \textit{Initial coding}, \textit{focused coding} e \textit{theoretical coding}. \\
\hline

\textbf{Questões levantadas durante a análise}
& Estes dados são um estudo de quê? Que categoria ou propriedade este incidente indica? O que está acontecendo atualmente nos dados?
& Questões do estilo quem, quando, onde, como, com quais consequências ou sob quais condições algum fenômeno ocorreu, ajudam a descobri ideias importantes para a teoria.
& Estes dados são um estudo de quê? O que este dado sugere? Sob o ponto de vista de quem? A qual categoria teórica esta unidade de dado indica pertencer? \\
\hline

\textbf{Critérios de avaliação}
& As categorias geradas devem estar alinhadas aos dados, a teoria deve funcionar, a teoria deve ter relevância para a ação da área, e a teoria deve ser modificável quando novos dados aparecerem.
& São definidos sete critérios para o processo de pesquisa, tais quais, informação na seleção de exemplos, principais categorias, hipóteses derivadas e discrepâncias. E oito critérios a respeito do crescimento empírico, tais quais ``conceitos foram gerados?'', ``existe variação construída na teoria?''
& Credibilidade, originalidade, utilidade e se a teoria faz sentido para os participantes. \\
\hline

\end{tabular}}
\end{table}

Neste trabalho foi utilizada a variação clássica, principalmente pela
inexistência de questões de pesquisa no início da pesquisa, exatamente como
sugerido nessa versão. O início da pesquisa estava ancorado em investigar uma
área de interesse: a adoção bem sucedida de \textit{DevOps} no mercado.
Adicionalmente, os trabalhos de pesquisa atualmente existentes na área
de engenharia de \textit{software} usam predominantemente a versão clássica
\cite{stol2016grounded} e, portanto, existe um maior referencial que pode ser
utilizado como base em um novo estudo.

\subsection{Abordagem \acrshort{GT}}

A figura \ref{abordagem_gt}, adaptada da proposta por Adolph et al.
\cite{using_gt_adolph}, ilustra uma abordagem de como conduzir um estudo
utilizando \acrshort{GT} como método. Esta abordagem foi utilizada no estudo
aqui realizado.

\figuraBib{GT}{O método GT}{using_gt_adolph}{abordagem_gt}{width=.45\textwidth}%

\begin{enumerate}[label=(\Alph*)]

\item Initially, we begin collecting data about the adoption
process from companies that have successfully adopted
it. As the data were collected, they were also analyzed
simultaneously. The raw data is analyzed by searching for
patterns of incidents to indicate concepts, and concepts
grouped into categories. This first step, where all raw data
is analyzed, is called open coding \cite{stol2016grounded}.

\item The categories are developed by constant comparison
of new incidents with previous. Every grounded theory
study has to identify a ”core category” \cite{stol2016grounded}. The core
category is responsible for enabling the integration of the
other categories and structuring the results into a dense
and consolidated grounded theory \cite{jantunen_using_gt_approach}. The identification
of the core category represents the end of open coding
and the beginning of the selective coding. In selective
coding, only specific variables that are directly related
to the core category and their relationships are coded, in
order to enable the production of a harmonic theory \cite{using_gt_coleman,hoda_impact_inadequate}.
Selective coding ends when we achieve a theoretical saturation, which
occurs when the last few participants provided more evidence and examples but
no new concepts or categories~\cite{glaser1967discovery}.

After saturation, we built a theory that explains the categories and the
relationships between the categories. Additionally, we reintegrated our theory
with the existing literature, which allowed us to compare our proposal with
other theories about DevOps. That is, using a Grounded Theory approach, one
should only conduct a literature review in later stages of a research, in order
to avoid external influences to conceive a theory~\cite{reconciling_perspectives}.

\item Throughout the process, memos are wrote capturing
thoughts and analytic processes; the memos support the
emerging concepts, categories, and their relationships~\cite{reconciling_perspectives}.

\end{enumerate}

Regarding data collection, we conducted semi-structured interviews with 15 practitioners of companies from
Brazil, Ireland, Portugal, Spain, and United States that
contributed to DevOps adoption processes in their companies. Participants
were recruited using two approaches: (1) through direct contact in a \emph{DevOpsDays}
event in Brazil and (2) through  general
calls for participation posted on DevOps user groups, social networks,
and local communities. In order to achieve a heterogeneous perspective
and increase the potential of generalization of the results,
we consulted practitioners from a variety of companies.
Table~\ref{participant_table} presents the characteristics of the participants
that accepted our invitation.
To maintain anonymity, in conformance with the human ethics guidelines,
hereafter we will refer to the participant as P1--P15 (first column).

\begin{table}[t]
\centering
\caption{Participant Profile. SX means software development experience in years,
DX means DevOps experience in years, CN means country of work, and CS means
company size (S\textless100; M\textless1000; L\textless5000; XL\textgreater5000).}
\label{participant_table}
\begin{tabular}{p{0.4cm}p{2.6cm}p{0.4cm}p{0.45cm}p{0.5cm}p{1.3cm}p{0.3cm}} \toprule \centering
\textbf{P\#}          & \textbf{Role}
       & \textbf{SX} & \textbf{DX} & \textbf{CN}   & \textbf{Domain}    & \multicolumn{1}{l}{\textbf{CS}} \\ \midrule \centering
P1                   & DevOps Developer      & 9            & 2           & IR            & IT                 & S                               \\ \centering

P2                   & DevOps Consult.       & 9            & 3           & BR            & IT                 & M                               \\ \centering

P3                   & DevOps Developer      & 8            & 1           & IR            & IT                 & S                               \\ \centering

P4                   & Computer Tech.        & 10           & 2           & BR            & Health             & S                               \\ \centering

P5                   & Systems Engineer      & 10           & 3           & SP            & Telecom            & XL                              \\ \centering

P6                   & Developer             & 3            & 1           & PO            & IT                 & S                               \\ \centering

P7                   & Support Analyst       & 15           & 2           & BR            & Telecom            & L                               \\ \centering

P8                   & DevOps Engineer       & 20           & 9           & BR            & Marketing              & M                               \\ \centering

P9                   & IT Manager            & 14           & 8           & BR            & IT                 & M                               \\ \centering

P10                  & Network Admin.        & 15           & 3           & BR            & IT                 & S                               \\ \centering

P11                  & DevOps Superv.                & 6            & 4           & BR            & IT                  & M                               \\ \centering

P12                  & Cloud Engineer              & 9            & 3           & US            & IT                  & L                               \\ \centering

P13                  & Technology Mngr.                 & 18            & 6           & BR            & Food                  & M                               \\ \centering

P14                  & IT Manager            & 7            & 2           & BR            & IT                  & S                               \\ \centering

P15                  & Developer        & 3            & 2           & BR            & IT                  & S \\ \bottomrule
\end{tabular}
\end{table}



The interviews were conducted over one year using Skype calls.
Data collection and analysis were iterative so the collected data helped to guide
future interviews. Questions evolved according to
the progress of the research. We start with five open-ended questions: (1) What
motivated the adoption of DevOps? (2) What DevOps adoption means in the context of
your company? (3) How was DevOps adopted in your company? (4) What were the
results of adopting DevOps? And (5) What were the main difficulties?

As the analyzes were being carried out, new questions were added to the script.
These new questions were related to the concepts and categories identified in
previous interviews. Examples of new questions include: (1) What is the
relationship between deployment automation and DevOps adoption? (2) Is it
possible to adopt DevOps without automation? (3) How does your company fostered a collaborative culture?

With respect to \emph{data analysis}, the interviews were
recorded, transcribed, and analyzed. The first moment
of the analysis, called open coding in GT, starts immediately after the
transcription of the first interview.
%% , which
%% was used to evolve the interview script to be used in
%% the second interview, and so on.
Open coding lasted until there was no
doubt about the core category of the study. Similar to that described by
Adolph et al.~\cite{reconciling_perspectives}, we started
considering a core category candidate and changed later. The first core category
candidate was \cat{automation}, but we quickly realized that this category did not
explain most of the behaviors
or events in data. We then started to understand that
\cc also appeared recurrently in the analysis and with more
potential to explain the remaining events. Thus, we asked explicitly
about the role of \cat{automation} and how the \cc is formed
in a DevOps adoption process.

Considering the script adaptations and the analysis of new data in a constant
comparison process, taking into account the previous analyses and the
respective memos written during all the process, after the tenth
interview we concluded that \cc was unequivocally the core
category regarding how DevOps was successfully adopted.
At this moment, the open coded ended and the selective coding started.
We started by restricting the coding only
to specific variables that were directly related to the core category and their
relationships. Following three more interviews and respective analysis, we realized that
the new data added less and less content to the emerging theory. That is, the
explanation around how the \cc category is developed showed signs of saturation.
We then conducted two more interviews to conclude that we had reached a
theoretical saturation.

At this point, we started the theoretical coding to find a way to integrate
all the concepts, categories, and memos in the form of a cohesive and
homogeneous theory, where we have pointed out the role of the categories as
enablers and outcomes. We present more details about
the results of our theoretical coding phase in the next section.
To illustrate the coding procedures, we present an working example from an
interview transcription to a category.
It is important to note that \emph{raw interview transcripts} are full of noise.
We start the coding by removing this noise and identifying the key points.
Key points are summarized points from sections of the interview~\cite{georgieva2008best}. Example:

\textbf{Raw data:} \textit{``So, here we adopted this type of strategy that is
the infrastructure as code, consequently we have the versioning of our entire
infrastructure in a common language, in such a way that any person, a
developer, an architect, the operations guy, or even the manager, he looks and
is able to describe that the configuration of application x is y. So, it
aggregates too much value for us exactly with more transparency"}

\textbf{Key point:} \textit{``Infrastructure as code contributes to
transparency because it enables the infrastructure versioning in a common
language to all professionals"}

We then assigned codes to the key point. A code is a phrase that summarizes
the key point and one key point can lead to several codes \cite{hoda2017becoming}.

\textbf{Code:} \textit{Infrastructure as code contributes to transparency}

\textbf{Code:} \textit{Infrastructure as code provides a common language}

In this example, the concept that emerged was ``infrastructure as code". The
expression corresponding to this concept comes directly from raw data, but this
is not a rule. It is common for the concept to be an abstraction, without
emerging from an expression present in raw data.
At this moment, we already identified other concepts that
contribute to transparency. We wrote the following memo:

\textbf{Memo:} \textit{Similar to sharing on a regular basis and shared
pipelines, the concept of infrastructure as code is related as important to
transparency. These transparency related concepts have often been cited as
means to achieve greater collaboration between teams}.

The constant comparison method was repeated on the concepts to produce a third
level of abstraction called categories. Infrastructure as code was grouped
together with five other concepts into the \textbf{sharing and transparency} category.


\section{Categorias e Conceitos da Adoção de \textit{DevOps}}

Como resultado da pesquisa, foram identificados 33 conceitos que foram agrupados
em sete categorias. Nesta seção são detalhados todos os conceitos e categorias
que foram  identificados como parte da adoção de \textit{DevOps}.

\subsection{A Categoria Principal: Cultura de Colaboração}

The \cc is the core category
for DevOps adoption. A \cc essentially aims to remove
the silos between development and operations teams and activities.
As a result, operations tasks---like deployment, infrastructure provisioning
management, and monitoring--- should be considered as regular, day-to-day,
development activities. This leads to the first concept related to
this core category: {\bf operations tasks should be performed by
the development teams in a seamless way.}

\begin{mq}
``\emph{A very important step was to bring the deployment into day-to-day
development, no more waiting for a specific day of the week or month. We wanted
to do deployment all the time, even if in a first moment it were not in
production, a staging environment was enough. [...] Of course, to carry out the deployment
continuously, we had to provide all the necessary infrastructure at the same
pace.}" (P14, IT Manager, Brazil)
\end{mq}

Without DevOps, a common scenario is an accelerated software development
without concerns about operations. At the end, when the development team has a
minimum viable software product, it is sent to the operations team for
publication. Knowing few things about the nature of the software and how it
was produced, the operations team has to create and configure an environment
and to publish the software. In this scenario, software delivery is typically
delayed and conflicts between teams manifest themselves. When a \cc is fomented, teams collaborate to perform the tasks from the first day
of software development. With the constant exercise of provisioning, management,
configuration and deployment practices, software delivery becomes more natural,
reducing delays and, consequently, the conflicts between teams.

\begin{mq}
``\emph{We work using an agile approach, planning 15-day sprints where we focused on
producing software and producing new releases at a high frequency. However, at the time of
delivering the software, complications start to appear. (...) Deliveries was often delayed,
having to deliver with delays of weeks, which was not good either for us or for
the stakeholders.}" (P6, Developer, Portugal)
\end{mq}

As a result of constructing a \cc, the development
team no longer needs to stop their work waiting for the creation
of one application server, or for the execution of some database script, or for
the publication of a new version of the software in a staging environment.
Everyone needs to know the way this is done and, with the collaboration of the
operations team, this can be performed in a regular basis. If any task can be
performed by the development team and there is trust between the teams, this task is
incorporated into the development process in a natural way, manifesting the
second concept related to \cc category: \textbf{software
development empowerment}.

\begin{mq}
``\emph{
%We had several people working on development, the amount of developers is
%impressive.
It was not feasible to have so many developers generating artifacts and
stopping their work to wait another completely separate team to publish. Or
needing a test environment and having to wait for the operations team to
provide it only when possible. These activities have to be available to quickly
serve the development team. With DevOps we supply the need of freedom and
more power to execute some tasks which are intrinsically linked to their work.}"
(P5, Systems Engineer, Spain)
\end{mq}

A \cc requires  \textbf{product thinking}, in substitution to
\textbf{operations or development thinking}. The development team has to understand that
the software is a product that does not end after ``pushing'' the code to a
project's repository and the operations team has to understand that its processes does not
start when an artifact is received for publication. \textbf{Product thinking}
is the third concept related to our core category.

\begin{mq}
``\emph{We wanted to
contract people who could have a product vision. People who could see the
problem and think in the better solution to it. But not only think in a
software solution, also think about the moment when that application will be
published. We also bring together developers to reinforce that everyone
%should think that way. Everyone
has to think in the product and not only in
their code or in their infrastructure}" (P12, Cloud Engineer, United States)
\end{mq}

There should be a \textbf{straightforward communication} between teams. Ticketing
systems are cited as a typical and inappropriate means of communication
between development and operations teams. Face-to-face communication is the best
option, but considering that it is not always feasible, the continuous use of
tools like \emph{Slack} and \emph{Hip Chat} was cited as appropriate options.

\begin{mq}
``\emph{We also use this tool (Hip Chat) as a way to facilitate communication between
development and operations teams. The pace of work there is very accelerated, and thus
it is not feasible to have a bureaucratic communication. (...) This gave us a lot of
freedom to the development activities, in case of any doubt, the operations staff
is within reach of a message.}" (P5, Systems Engineer, Spain)
\end{mq}

There is a \emph{shared} responsibility to identify and fix the issues
of a software when transitioning to production. The strategy of avoiding liability should be kept away.
The development team must not say that a given issue is a problem in the infrastructure, then
it is responsibility of the operations team. Or the opposite, the operations team
must not say that a given failure was motivated by a problem in the application, then it is
responsibility of development team. A \textbf{blameless} context must exist.
The teams need to focus on solving problems, not on finding blame and
running away from responsibility. The context of \textbf{shared
responsibilities} involves not only solving problems, but also any other
responsibility inherent in the software product must be shared.
\textbf{Blameless} and \textbf{shared responsibilities} are the remaining
concepts of the core category.

\begin{mq}
``\emph{We realized that some people were afraid of making mistakes. Our
culture was not strong enough to make everyone feel comfortable to innovate and
experiment without fear of making mistakes. We made a great effort to spread
this idea that there are no blame for any problem that may occur. We take every possible
measure to avoid failures, but they
will happen, and only without blame we will be able to solve a problem quickly.}" (P8,
DevOps Engineer, Brazil)
\end{mq}

At first glance, considering the creation and strengthening of the \cc as the most important step towards DevOps adoption seems somewhat obvious, but
the respondents cited some mistakes that they consider recurrent in not
prioritizing this aspect in a DevOps adoption:

\begin{mq}``\emph{In a DevOps adoption, there is a very strong cultural issue that the teams
sometimes are not adapted. Related to this, one thing that bothers me a lot and
that I see happen a lot is people hitching DevOps exclusively by tooling or
automation.}" (P9, IT Manager, Brazil)
\end{mq}

Besides the core category (\cc), we identified
three other sets of categories: the enablers
of DevOps adoption, the consequences of adopting
DevOps, and the categories that are both enablers and consequences.

%\subsection{Enabler Categories}

%Here we detail the categories that support the adoption of
%DevOps practices, including \cat{automation}, \cat{sharing and transparency}.

\subsection{Automation} \label{ssec:automation}

This category presents the higher number of related concepts. This
occurs because manual proceedings are considered strong candidates to
propitiate the formation of a silo, hindering the construction
of a \cc. If a task is manual, a single person or
team will be responsible to execute it. Although \cat{transparency} and \cat{sharing} can
be used to ensure collaboration even in manual tasks, with automation the
points where silos may arise are minimized.

\begin{mq}
``\emph{When a developer needed to build a new application, the previous workflow demanded her
to create a ticket to the operations teams, which should then manually evaluate and solve
the requested issue. This task could take a lot of time and there was no
visibility between teams about what was going on (\ldots). Today, those silos do not exist
anymore within the company, in particular because it is not necessary to execute all these tasks manually,
everything has been automated.}" (P12, Cloud Engineer, United States)
\end{mq}

In addition to contribute to \cat{transparency}, \cat{automation} is also considered
important to ensure \emph{reproducibility} of tasks, reducing rework and risk of
human failure. Consequently, \cat{automation} increases the confidence
between teams, which is an important aspect of the \cc.

\begin{mq}
``\emph{Before we adopted DevOps, there was a lot of manual work. For example, if you
needed to create a database schema, it was a manual process; if you needed to create a
database server, it was a manual process; if you needed to create additional EC2 \footnote{Amazon Elastic
Compute Cloud} instances, such a process was also manual.
This manual work was time consuming and often caused errors and
rework.}" (P1, DevOps Developer, Ireland)
\end{mq}

%\begin{mq}
%``\emph{Our main motivation to adopt DevOps was basically reduce rework. Almost every
%week, we had to basically make new servers and start them manually, which was
%very time-consuming.}" (P4, Computer Technician, Brazil)
%\end{mq}

The eight concepts of the \cat{automation} category will be detailed next.
In all interviews we extracted explanations about \textbf{deployment
automation} (1), as part of DevOps adoption. Software delivery is the clearest
manifestation of value delivery in software development. In case of problems
in deployment, the expectation of delivering value to business can quickly
generate conflicts and manifest the existence of silos.
In this way, \cat{automation} typically increases agility and reliability. Some other
concepts of automation go exactly around deployment automation.

It is important to note that frequent and successfully
deployments are not sufficient to generate value to business. Surely, the quality of
the software is more relevant. Therefore, quality checks need to be automated as well, so they can be part of the
deployment pipeline, as is the case of \textbf{test automation} (2). In addition, to
automate application deployment, the environment where the
application will run needs to be available. So, \textbf{infrastructure
provisioning automation} (3) must be also considered in the process. Beyond being available,
the environment needs to be properly configured, including the amount of memory and CPU,
availability of the correct libraries versions, and database structure. If the configuration of some of these concerns
has not been automated, the deployment activity can go wrong. Therefore,
the automation of \textbf{infrastructure management} (4) is another
concept of the \cat{automation} category.

Modern software is built around services. Microservices  was commonly cited
as one aspect of DevOps adoption. To Fowler and Lewis
\cite{martinfowler2014microservices}, in the
microservice architectural style, services need to be independently deployable
by fully automated deployment machinery. We call this part of microservices
characteristics of \textbf{autonomous services} (5). \textbf{Containerization}
(6) is also mentioned as a way to automate the provisioning of containers---the
environment where these autonomous services will execute.
\textbf{Monitoring automation} (7) and \textbf{recovery automation} (8) are the
remaining concepts. The first refers to the ability to monitor the
applications and infrastructure without human intervention. One classic example
is the widespread use of tools for sending messages reporting
alarms---through SMS, Slack/Hip Chat, or even
cellphone calls-- in case of incidents. And the second is related to the ability
to either replace a component that is not working or
roll back a failed deployment without human intervention.

\subsection{Transparency and Sharing} Represents the grouping of concepts
emerged from recurrent interviews that help to disseminate concepts and
activities. Training, tech talks, committees lectures, and round tables
are examples of these events. Creating
channels using communication tools is another recurrent topic
related to \cat{sharing} along the processes of DevOps adoption.
According with the content of what is shared, we
have identified three main concepts:

\begin{itemize}
\item Knowledge sharing: the professionals interviewed mention a wide range of
skills they need to acquire during the adoption of DevOps, citing
structured events of sharing to smooth the learning curve of both technical and
cultural knowledge.


\item Activities sharing: where the focus is on sharing how simple tasks can or
should be performed. Communication tools, committees, and round tables are the common
forum for sharing this type of content.

\item Process sharing: here, the focus is on sharing whole working processes. The
content is more comprehensive than in sharing activities. Tech talks and
lectures are the common forum for sharing processes.

\end{itemize}

Sharing concepts contribute with the \cc. For example,
all team members gain best insight about the entire software production
process, with a solid understanding of shared responsibilities. A shared vocabulary also
emerged from \cat{sharing} and this facilitates communication.

The use of \textbf{infrastructure as code} was
recurrently cited as a means for guaranteeing that everyone knows how the execution environment of
an application is provided and managed. Bellow, we present an interview
transcript which sums up this concept.

\begin{mq}
``\emph{So, here we adopted this type of strategy that is the infrastructure as code,
consequently we have the versioning of our entire infrastructure in a common
language, in such a way that any person, a developer, an architect, the
operations guy of even the manager, he looks and is able to describe that the
configuration of application x is y. So, it aggregates too much value for us
exactly with more transparency.}" (P12, Cloud Engineer, United States)
\end{mq}

Regarding cat{transparency and sharing}, we also found the concept of \textbf{sharing on a regular basis}, which suggests
that sharing should be embedded in the process of software
development, in order to contribute effectively to transparency.
As we will detail in the \emph{continuous integration} concept of
the \cat{agility} category, a common way to integrate all tasks is a pipeline. Here, there is the
concept of \textbf{shared pipelines}, which indicates that the code of pipelines
must be accessible to everyone, in order to foment transparency.

\begin{mq}
``\emph{The code of how the infrastructure is
made is open to developers and the sysadmins need to know some aspects of how
the application code is built. The code of our pipelines is accessible to
everyone in the company to know how activities are automated}" (P13, Technology
Manager, Brazil)
\end{mq}


%\subsection{Categories related to the DevOps adoption outcomes}

%In this section we detail the categories that correspond to
%the expected consequences with the adoption of
%DevOps practices, including \cat{agility} and \cat{resilience};
%as discussed as follows.

\subsection{Agility}

Agility is frequently discussed as a major outcome of DevOps adoption. With more
collaboration between teams, \textbf{continuous integration} with execution of
multidisciplinary pipelines is possible and it is an agile related concept
frequently explored. These pipelines might contain
infrastructure provisioning, automated regression testing, code analysis,
automated deployment and any other task considered important to
execute continuously.

These pipelilnes encourage two other agile concepts: \textbf{continuous
infrastructure provisioning} and \textbf{continuous deployment}. The latter is
one of the most recurrent concepts identified in the interview analysis. Before
DevOps, deployment had been seen as a major event with high risk of downtime and
failure involved. After DevOps, the sensation of risk in deployment decreases
and this activity became more natural and frequent. Some practitioners claim
to perform dozens of deployments daily.

\subsection*{Resilience}

Also related to an expected outcome of adopting DevOps, \cat{resilience} in this
context refers to the ability of applications to adapt quickly to adverse situations.
The first related concept is \textbf{auto scaling}---i.e.,
allocating more or less resources to applications that increase or
decrease on demand. Another concept related to
the \cat{resilience} category is \textbf{recovery automation}, that is
the capability of the applications and infrastructure to recovery itself in case of
failures. There are two typical cases of recovery automation: (1) in cases
of some instability in the execution environment of an application (a
container, for example) occurs an auto restart of that environment; and (2) in
cases of new version deployment, if the new version does not work properly, the
previous one must be restored. This auto restore of a previous version
decrease the chance of downtimes due to errors in specific versions, which
is the concept of \textbf{zero down-time}, the last one of the \cat{resilience} category.

%\subsection{Categories that are both Enablers and Outcomes}

%Finally, here we detail the categories that are both enablers
%and outcomes, including \cat{continuous measurement}
%and \cat{quality assurance}; as discussed as follows.

\subsection{Continuous Measurement}

As an enabler, performing regularly the
activities of measurement and sharing
contributes to avoid existing silos and reinforce the \cc, because it is
considered a typical responsibility of the operations team.

\begin{mq}
``\emph{Before, we had only sporadic looks to
zabbix\footnote{\url{https://www.zabbix.com/}} to check if everything was OK.
At most someone would stop to look memory and CPU consumption. To maintain a
the quality of services, we expanded this view of metrics collection so that it
became part of the software product. We then started to collect metrics continuously
and with shared responsibilities. For example, if an overflow occurred in the
number of database connections, everyone received an alert and had
the responsibility to find solutions to that problem. %This is an interesting example of
%metrics that everyone started to be more attentive, not only the operations
%team.
}" (P3, DevOps Developer, Ireland)
\end{mq}

As an outcome, the continuously collection of metrics from applications and
infrastructure is a required consequence of DevOps adoption. It occurs because
the resultant agility increases the risk of something going wrong. The team
should be able to react quickly in case of problems, and the continuous
measurement allows it to be proactive and resilient.

\begin{mq}
``\emph{With DevOps we can do deployment all the time and, consequently, there was
the need of greater control of what was happening. So, we used
grafana\footnote{\url{https://grafana.com/}} and
prometheus\footnote{\url{https://prometheus.io/}} to follow everything that is
happening in the infrastructure and in the applications. We have a complete
dashboard in real time, we extract reports and, when something goes wrong, we
are the first to know.}" (P10, Network Administrator, Brazil)
\end{mq}

Continuous monitoring involves \textbf{application log monitoring} (1), a
concept that corresponds to the use of the log produced by
applications and infrastructure as data source. The concept of
\textbf{continuous infrastructure monitoring} (2) indicates that the monitoring
is not performed by a specific person or team in a specific moment. The
responsibility to monitor the infrastructure is shared and it is executed in
daily. \textbf{Continuous application measurement} (3), in turn, refers to
the instrumentation to provide metrics that are used to evaluate aspects and
often direct evolution or business decisions. All these monitoring/measurement
can occur in an automated way, the \textbf{monitoring automation} already been
detailed in subsection \ref{ssec:automation}.

\subsection{Quality Assurance}

In the same way as continuous measurement, quality assurance is a category that
can work both as enabler and as outcome. As enabler because an increasing quality
leads to more confidence between the teams, which in the end generates a virtuous
cycle of collaboration. As outcome, the principle is that it is not
feasible to create a scenario of continuous delivery of software without control
about the quality of the products and its production processes.

Respondents pointed to the need for sophisticated control of which code should
be part of deliverables that are continuously delivered. Git Flow was
recurrently cited as suitable \textbf{code branching} (1) model, the first
concept of quality assurance.
In a previous section, we explored the automation face of
microservices and testing. These elements have also a quality assurance face.
Another characteristic of microservices is the need for small services focusing
in doing only one thing. These small services are easier to scale and
structure, which manifest a quality assurance concept: \textbf{cohesive
services} (2). Regarding testing, another face is \textbf{continuous
testing} (3). To ensure quality in software products, we found that
tests (as well as other quality checks) should occur continuously. Continuous testing
is considered challenging without automation, and this reinforces the need for automated
tests.

Another two concepts cited as part of quality assurance in DevOps adoption are
the use of \textbf{source code static analysis} (4) to compute quality metrics in
source code and the \textbf{parity between environments} to
reinforce transparency and collaboration during software development.

\section{Uma Teoria Sobre a Adoção de \textit{DevOps}}

The results of a grounded theory study, as the name of the method itself
suggests, are grounded on the collected data, so the hypotheses emerge from
data. A grounded theory should describe the key relationships between the
categories that compose it, i.e., a set of inter-related hypotheses~\cite{hoda2017becoming}.
We present the categories of our grounded theory
about DevOps adoption as a network of the three categories of enablers (\cat{automation},
\cat{sharing and transparency}) that are commonly used to develop the core category
\cc, as discussed in the previous section. Based on our understanding,
implementing the enablers to develop the \cc typically leads
to concepts related to two categories of expected outcomes:
\cat{agility} and \cat{resilience}. Moreover, there are two categories that can be considered
both as enablers and as outcomes: \cat{continuous measurement} and \cat{quality assurance}.
In this section we describe the relationships between those categories, building a theory
of DevOps adoption.

\subsection{A General Path for DevOps Adoption}

In Section~\ref{sec:introduction} we presented the general question of this
research: is there any recommended path to adopt DevOps? Here, we elaborate a response,
based on the analyses conducted as detailed in Section~\ref{sec:research_method}. The main
point which should be formulated is the construction of a \cat{collaborative
culture} between the software development and operations teams and
related activities. According to our findings, the other categories,
many of which are also present in other studies that have investigated DevOps,
only make sense if the practices and
concepts related to them either contribute to the level of a \cc or lead to the expected consequences
of a \cc. This understanding induces several hypothesis, as discussed in
what follows.

\begin{mh}
\textbf{Hypothesis 1:} \textit{There is a group of categories related to DevOps adoption
that only make sense if used to increase the} \cc \emph{level. We
call this group of categories of \textbf{enablers}}.
\end{mh}

Based on this first hypothesis, the maturity of DevOps adoption does not
advance in situations where only one team is responsible to understand, adapt, or
evolve automation---even when such automation supports different activities like deployment, infrastructure provisioning,
monitoring. The same holds for the other \emph{enabling} categories. That is, in the situations which
\cat{transparency and sharing} do not contribute to
the \cc, they do not contribute to DevOps adoption as a whole. Some examples
that supports our first hypothesis include:

%\begin{mq}
%``\emph{Look, inside the operations sector there was some degree of automation. The guy
%had stored in his own machine bash scripts that helped him when setting up a
%server or when creating a new database instance. Nevertheless, there was no DevOps
%because there was no intrinsic relationship of this automation to the
%development process}" (P11, DevOps Supervisor, Brazil)
%\end{mq}


\begin{mq}
``\emph{DevOps involves tooling, but DevOps is not tooling. That is, people often
focus on using tools that are called `DevOps tools', believing that DevOps is
this. I always insist that DevOps is not tooling, DevOps involves the use of
tools properly, to improve software development procedures.}" (P2, DevOps
Consultant, Brazil)
\end{mq}


%% \begin{mq}
%% ``\emph{Keeping the culture alive remains a challenge to us, and it is very
%% important. Here in our company, for example, we have Tech Talks that are
%% monthly conversations that we have with the teams. The purpose of these Tech
%% Talks is to share knowledge about technologies and work processes increasing the
%% transparency of how everything works. We also have a Slack channel called
%% DevOps as Culture where we discuss things of DevOps culture. The idea is not to
%% let the culture die, we are always feeding it with something, because that is
%% the DevOps essence for us.}" (P12, Cloud Engineer, United States)
%% \end{mq}

\begin{mh}
\textbf{Hypothesis 2:} \textit{There is a group of categories related to DevOps adoption
that does not contribute to increase the} \cc \emph{level, but that instead are
pointed out as DevOps adoption related, because they emerge as an expected or
necessary consequence of the adoption. These categories represent the group of
\textbf{outcomes}}.
\end{mh}

In a first moment, the simple fact that a team is more
\cat{agile} in delivering software, or more \cat{resilient} in failure recovery, does not
contribute directly to bring operations teams closer to development teams.
Nevertheless, a signal of a mature DevOps adoption is an increasing of the capacity for continuously
delivering software (and thus being more \cat{agile})
and for building \cat{resilient} infrastructures.

\begin{mh}
\textbf{Hypothesis 3:} \textit{The categories \cat{Continuous Measurement} and \cat{Quality Assurance}
are both related to DevOps enabling capacity and to DevOps outcomes}.
\end{mh}

Measurement is cited as a typical responsibility of the operations team.
At the same time that sharing this responsibility reduces silos,
it is also cited that measurement is a necessary consequence of DevOps adoption. Particularly because
the continuous delivery of software requires more control,
which is supplied by concepts related to the \cat{continuous measurement} category.
The same premise is valid to the \cat{quality assurance} category. At first glance,
\cat{quality assurance} appears as one response to the context of agility in operations
as a result of DevOps adoption. But, the efforts in quality assurance of software products
increase the confidence between the development and operations teams, increasing the level
of \cc.

% \subsection{DevOps Enablers}

Altogether, DevOps enablers are the means commonly used to increase the level of
the \cc in a DevOps adoption process.
We have identified five categories of DevOps enablers:
\cat{ Automation}, \cat{Continuous Measurement}, \cat{Quality Assurance},
\cat{Sharing}, and {\cat{Transparency}. Another finding of our
study leads to our fourth hypothesis.

\begin{mh}
\textbf{Hypothesis 4:} \textit{There is no precedence between enablers in a DevOps adoption process}.
\end{mh}

We have realized that the adoption process might not have
to priorize any enabler, and a company that aims to implement
DevOps should start with  the enablers that seem more appropriate (in terms
of its specificities). Accordingly, we did not find any evidence that an enbler
is more efficient than another for creating a \cc. \cat{Automation} is the category
that appears more frequently in our study, though several participants make
clear that associating DevOps with automation is a misconception.
%% For
%% instance, although 14 respondents cite \cat{automation} as an important
%% enabler to adopt DevOps, some respondents also ponder that considering
%% automation with greater importance than other parts can actually be a risk:

\begin{mq}
``\emph{I think that the expansion of collaboration between teams involved other
things, it was not just automation. There must be an alignment with the
business needs. (...) I think that DevOps made possible a broader understanding
of software production and we were realizing exactly that it is not about
automating everything. (...) So, I see with caution a supposed vision that automate things can
be the way to implement DevOps.}" (P7, Support Analyst, Brazil)
\end{mq}

%\begin{mq}
%``\emph{Despite of we actually use automation in a reasonable number of scenarios,
%we have been able to develop our culture significantly without automation and I think that you can reach a
%good DevOps level with little or even no automation.}" (P8, DevOps Engineer, Brazil)
%\end{mq}

%% That is, although \cat{automation} is a very commonly used enabler, it is possible to
%% increase the level of \cc without focus on automating. And
%% this premise is valid to the other enablers.

% \subsection{DevOps Outcomes}

DevOps outcomes is that group of categories that does not produces primarily the
expected effect of a DevOps enabler, typically concepts that are expected as
consequences of an adoption of DevOps. We have identified four categories that
can work as DevOps outcomes: \cat{agility}, \cat{continuous measurement},
\cat{quality assurance}, and \cat{software resilience}. Note that,
as mentioned before, \cat{continuous measurement} and \cat{quality assurance}
are both enablers and outcomes.

That is, a well succeeded DevOps adoption typically increases the potential of
\cat{agility} of teams and enables \cat{continuous measurement}, \cat{quality assurance} and
\cat{resilience} of applications.
However, in some situations, this potential is not completely used due business
decisions. For example, one respondent cited that, at a first moment, the
company did not allowed the continuous deployment (more potential of agility)
of applications in production:

\begin{mq}
``\emph{We had conditions and security to continuously publish in production,
however, in the beginning the managers were afraid and decided that the
publication would happen weekly.}" (P9, IT Manager, Brazil)
\end{mq}

\section{Trabalhos Relacionados}

Conforme ilustrado no capítulo anterior, os trabalhos de pequisa existentes
apresentam algumas caracterizações de \textit{DevOps} por meio da enumeração
de conceitos, princípios e práticas relacionadas. Apesar de alguns desses
estudos conterem abordagens qualitativas para investigar empresas adotando
\textit{DevOps}, conduzindo entrevistas com praticantes, o foco desses estudos
tipicamente é em caracterizar \textit{DevOps}, ao invés de prover recomendações
na adoção de \textit{DevOps}. Além das diferenças de foco, existem algumas
diferenças metodológicas em relação ao estudo aqui proposto. Essas diferenças
são resumidas na tabela \ref{related_work_table}.

%% Table~\ref{related_work_table} compares the most relevant literature to
%% out work. Since the work of Lwakatare et al.~\cite{extending_dimensions} is an
%% extension of a previous contribution of the authors and her team \cite{dimensions_of_devops_xp_15},
%% we only compared the most recent study at Table~\ref{related_work_table}.

\begin{table}[hb!]
\centering
\caption{Comparação de Metodologia e Foco dos Trabalhos Relacionados}
\label{related_work_table}
\begin{tabular}{|p{3cm}|p{6cm}|p{6cm}|}
\hline

\textbf{Estudo}
& \textbf{Abordagem de Pesquisa}
& \textbf{Foco do Trabalho} \\

\hline

\textbf{Este estudo}

& Grounded Theory study interviewing practitioners which contributed to DevOps
adoption in 15 companies across 5 countries.

& Explain how DevOps has been successfully adopted in practice;\newline
\newline Provide guidelines to be used in new DevOps adoptions. \\

\hline

\textbf{J. Smeds et al.~\cite{devops_a_definition}}

& Systematic literature review; \newline \newline Semi structured interviews
with 13 employees of a single company whose DevOps adoption process was at
an initial stage.

& Identify in literature the main defining characteristics of DevOps; \newline
\newline Build a list of possible impediments to DevOps adoption through an
empirical study. \\

\hline

\textbf{Lwakatare et al.~\cite{extending_dimensions}}

&
Multivocal literature review using data from gray literature; \newline \newline
Three interviews with software practitioners in one company that was applying
DevOps practices in one project.

& Identify how do practitioners describe DevOps as a phenomenon; \newline
\newline Identify the DevOps practices according to software practitioners. \\

\hline

\textbf{Fran\c{c}a et al.~\cite{characterizing_devops}}

& Multivocal literature review with qualitative analysis procedures from
Grounded Theory.

& Provide a DevOps definition; \newline \newline
Identify DevOps practices, required skills, characteristics, benefits and
issues motivating its adoption.  \\

\hline

\textbf{Erich et al.~\cite{qualitative_devops_journalsw_17}}

& Systematic literature review; \newline \newline
Interviews with practitioners from 6 companies across 3 countries.

& Identify how literature defines DevOps; \newline \newline
Investigate how DevOps is being implemented in practice. \\

\hline

\end{tabular}
\end{table}

Como consequência dessas diferenças de foco, algumas questões práticas a
respeito da adoção de DevOps permancem em aberto: (1) Existe um caminho
recomendado para se adotar \textit{DevOps}? (2) Já que \textit{DevOps} é
constituído de múltiplos elementos, eles
possuem a mesma relevância quando se adota \textit{DevOps}? (3) Qual é o papel
desempenhado por cada um desses elementos - tais como medição, compartilhamento e
automação - em uma adoção de \textit{DevOps}? Para responder a essas questões,
é necessário um entendimento holístico dos caminhos seguidos por organizações
que adotaram \textit{DevOps} de uma maneira bem sucedida.

Os trabalhos listados anteriormente concentram-se em caracterizar
\textit{DevOps}, há pouco conteúdo dedicado a \emph{como} adotar \textit{DevOps},
com indicação de caminhos seguidos por companhias que passaram pelo processo
e que eventualmente poderiam ser usados por novos praticantes, com indicações
de lições aprendidas e possíveis abordagens que possam ser seguidas.

O trabalho de Erich et al. \cite{qualitative_devops_journalsw_17} investigou
o processo de adoção de \textit{DevOps} por meio de entrevistas com praticantes
de seis companhias. Todavia, as percepções obtidas não detalham a maneira como
as companhias aplicaram os elementos de \textit{DevOps} ao longo do processo,
e são apresentadas de maneira individualizada, não se colocando em perspectiva
o que há de comum no entendimento dessas companhias. Os dados de como cada
companhia relata ter implementado \textit{DevOps} são apresentados de maneira
sucinta e os principais pontos são descritos a seguir.

Na primeira companhia, é relatado que o pessoal de desenvolvimento e
operações passou a trabalhar de maneira conjunta diariamente, foi criado um
novo cargo na companhia chamado \textit{DevOps Engineer}, o pessoal de gestão
foi treinado para ser capaz de gerenciar tanto pessoas de desenvolvimento como
de operações, o pessoal de recursos humanos alterou os processos de avaliação e,
por fim, a organização está automatizando seus processos de infraestrutura, não
são apresentados detalhes de como isso está sendo feito.

Na segunda companhia, \textit{DevOps} é visto como uma extensão de
\textit{Scrum} e \textit{Lean}. A adoção de \textit{DevOps} lá foi guiada por
dois princípios: colaboração diária e ambiente de trabalho compartilhado.
Havia a premissa de que o \textit{software} deve ir para produção a cada
iteração. É relatado então que a organização usou \textit{frameworks}
comerciais para guiar a implantação de \emph{entrega contínua}, também sem maiores
detalhamentos de como foi feito.

Na terceira companhia, é relatado que três times que trabalhavam em
funcionalidades separadas foram agrupados em um \emph{time DevOps}. A organização
passou então a experimentar \textit{entrega contínua}, mas ainda estava em
um estágio inicial quando a entrevista foi realizada.

A quarta companhia criou uma única equipe de \textit{DevOps} para
experimentar ferramentas, princípios e práticas \textit{DevOps}. Os
membros dessa equipe possuem habilidades multidisciplinares. É relatado ainda
o uso de ferramentas para controle de versão e o desenvolvimento de ferramentas
próprias para criar facilmente ambientes de nuvem.

Na quinta companhia, é relatado que \textit{DevOps} é visto como
um papel responsável por gerenciamento de incidentes, gerenciamento de
capacidade, gerenciamento de riscos e suporte ao processo de criação.
O uso de computação em nuvem ocasionou a necessidade de se existir uma
abordagem mais sistemática para execução das atividades de operações, exigindo
que o pessoal responsável aprendesse técnicas de desenvolvimento de
\textit{software}. O time de desenvolvimento foi dividido em vários subtimes,
um dos quais focado em \textit{DevOps}, com responsabilidade de apoiar os demais
times. São também enumerados alguns princípios e práticas utilizados na companhia
durante a adoção de \textit{DevOps}.

Por fim, com relação a como a sexta companhia implementou \textit{DevOps},
é apresentado um detalhamento de como ocorre a \emph{integração contínua} na
organização. As evoluções de código são feitas por meio de \textit{pull
requests} que após aprovadas são integradas ao código principal. Após a
integração dos códigos das \textit{pull requests}, um \textit{pipeline} de entrega
contínua é executado para publicar o \textit{software}.
