Conforme ilustrado no capítulo anterior, os trabalhos de pequisa existentes
apresentam algumas caracterizações de \textit{DevOps} por meio da enumeração
de conceitos, princípios e práticas relacionadas. Apesar de alguns desses
estudos conterem abordagens qualitativas para investigar empresas adotando
\textit{DevOps}, conduzindo entrevistas com praticantes, o foco desses estudos
tipicamente é em caracterizar \textit{DevOps}, ao invés de prover recomendações
na adoção de \textit{DevOps}. Além das diferenças de foco, existem algumas
diferenças metodológicas em relação ao estudo aqui proposto. Essas diferenças
são resumidas na tabela \ref{tabela}.

Existing research works have proposed a number of DevOps
characterizations, for instance, as a set of concepts with related
practices [1, 9, 11, 22–24]. Although some of these studies leverage
qualitative approaches to gather practitioners perception (for
instance, conducting interviews with them), they focus on characterizing
DevOps, instead of providing recommendations to assist
on DevOps adoption. Consequently, the obtained DevOps characterizations
allow a comprehensive understanding of the elements
that constitute DevOps, but do not provide detailed guidance to
support newcomers interested in adopting DevOps practices. As a
consequence, many practical and timely questions still remain open,
for instance: (1) Is there any recommended path to adopt DevOps?
(2) Since DevOps is composed by multiple elements [22], do these
elements have the same relevance, when adopting DevOps? (3)
What is the role played by elements such as measurement, sharing,
and automation in a DevOps adoption? To provide answers to these
questions, we need a holistic understanding of the paths followed
in successful DevOps adoptions.


 os trabalhos relacionados apresentam
diversos elementos que contribuem para um melhor entendimento sobre
\textit{DevOps}, todavia, ainda existe uma lacuna de maior orientação que possa
guiar novos praticantes na adoção de \textit{DevOps}.

Neste capítulo são apresentados os resultados
A tabela x ilustra as diferenças
de foco e metodologia dos trabalhos relacionados.

Neste capítulo, apresenta-se um estudo utilizando a metodologia Grounded Theory
para caracterizar a adoção de DevOps em organizações que foram bem sucedidas
nesse processo de adoção.
