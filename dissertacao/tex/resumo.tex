\textit{DevOps} é um conjunto de práticas e valores culturas que objetiva
reduzir as barreiras entre os times de desenvolvimento e operações. Devido aos
potenciais benefícios de \textit{DevOps} e a tentativas falhas de minimizar os
impactos da baixa colaboração entre os seus times de desenvolvimento e operações,
o \acrshort{TCU} está buscando ampliar o uso da abordagem \textit{DevOps} no
desenvolvimento de suas aplicações corporativas.

Devido ao seu crescente interesse e definições imprecisas, os trabalhos de
pesquisa existentes têm tentado caracterizar \textit{DevOps} - principalmente
por meio do uso de conceitos e práticas relacionadas.

Todavia, pouco se sabe sobre o entendimento de praticantes a respeito dos caminhos
de sucesso para se adotar \textit{DevOps}.

Nevertheless, little is known about the practitioners understanding about
successful paths for DevOps adoption. A falta de tal entendimento pode impedir
que instituições, como o \acrshort{TCU}, adotem \textit{DevOps}. Portanto, aqui
é apresentado um estudo sobre a adoção de \textit{DevOps}, destacando os
principais conceitos relacionados que podem contribuir para sua adoção por
novos praticantes.

Este trabalho foi construído utilizando a abordagem \textit{Grounded Theory} em
sua variação clássica. Praticantes que contribuíram para a adoção de \textit{DevOps}
em 15 empresas de diferentes domínios e em cinco países foram entrevistados.

Um modelo é apresentado para melhorar o entendimento e a orientação sobre a
adoção de \textit{DevOps}. Este modelo incrementa a visão existente sobre
\textit{DevOps} explicando o papel e a motivação de cada elemento (e seus
relacionamentos) durante o processo de adoção de \textit{DevOps}. O modelo é
organizado em termos de categorias de \emph{facilitadores DevOps} e categorias
de \emph{saídas DevOps}. Foi identificado que construir uma cultura de
colaboração é a principal preocupação na adoção de \textit{DevOps}, contrastando
com um possível entendimento que implantar ferramentas \textit{DevOps} específicas
ou automatizar tarefas de operações específicas é suficiente para implantar
\textit{DevOps}.

O modelo criado no estudo foi avaliado através de um estudo de caso, cujo objetivo
foi aumentar o nível de maturidade na adoção de \textit{DevOps} no \acrshort{TCU}.
As percepções dos profissionais envolvidos na adoção de \textit{DevOps} no
\acrshort{TCU} foram coletadas através da realização de um grupo focal.

De modo geal, os resultados contribuem para (a) gerar um adequado entendimento
da adoção de \textit{DevOps}, a partir da perspectiva de praticantes; e (b)
auxiliar o \acrshort{TCU} e outras instituições no seu processo de migração para
adotar \textit{DevOps}.
