{\it DevOps} é um conjunto de práticas e valores culturais que visa reduzir as
barreiras entre os times de desenvolvimento e operações durante o
desenvolvimento de {\it software}. Devido ao seu crescente interesse e
definições imprecisas, trabalhos de pesquisa recentes têm tentado caracterizar
{\it DevOps}---tipicamente utilizando um conjunto de conceitos e práticas
relacionadas. Todavia, pouco se sabe a respeito do \emph{entendimento de
praticantes} sobre os caminhos bem sucedidos para se adotar {\it DevOps}. A falta
de tal entendimento pode impedir instituições de adotar práticas {\it DevOps}.
Portanto, o objetivo aqui é apresentar uma teoria a respeito da adoção de
{\it DevOps}, destacando a maneira como os principais conceitos relacionados
têm contribuído para a sua adoção na indústria.

Este trabalho utiliza uma abordagem multimétodo. Inicialmente, foi conduzido um
estudo utilizando a variação clássica de {\it Grounded Theory}. Nesta etapa,
profissionais que contribuíram para a adoção de {\it DevOps} em 15 companhias
de diferentes domínios de negócio e de cinco países diferentes foram
entrevistados. Com base nos resultados, um modelo foi produzido para melhorar
tanto o entendimento como a orientação a respeito da adoção de {\it DevOps}.
Na segunda etapa do estudo, o modelo foi introduzido na adoção de {\it DevOps}
do \acrfull{TCU}, momento em que um grupo focal foi conduzido para avaliar o
estágio atual da adoção de {\it DevOps} e para validar a aplicabilidade e
utilidade do modelo.

O modelo incrementa a visão existente de {\it DevOps} explicando o papel e a
motivação de cada categoria (e seus relacionamentos) no processo de adoção de {\it DevOps}.
Este modelo foi organizado em termos de \emph{categorias de facilitadores DevOps}
e \emph{categorias de saídas DevOps}. Concluiu-se que \emph{colaboração} é a
principal preocupação de {\it DevOps}, contrastando com um possível
entendimento de que implantar ferramentas específicas para automatizar o
{\it build}, o {\it deployment} e o provisionamento e gerenciamento da
infraestrutura é suficiente para se implantar {\it DevOps}.

Assim sendo, os resultados contribuem para (a) gerar um adequado entendimento a
respeito da adoção de {\it DevOps}, a partir das perspectivas dos praticantes;
e (b) auxiliar instituições, como o \acrshort{TCU}, no processo de migração
para adotar {\it DevOps}. Adicionalmente, as experiências coletadas durante a
produção do modelo têm sido aplicadas durante a adoção de {\it DevOps} no
\acrshort{TCU}.
