\textit{DevOps} é um conjunto de práticas e valores culturais que objetiva
reduzir as barreiras entre os times de desenvolvimento e operações na produção
de \textit{software}. Devido aos potenciais benefícios de \textit{DevOps} e a
tentativas falhas de minimizar os impactos da baixa colaboração entre os seus
times de desenvolvimento e operações, o \acrshort{TCU} está buscando ampliar o
uso da abordagem \textit{DevOps} no desenvolvimento de suas aplicações corporativas.

O \acrfull{PDTI} do \acrshort{TCU} indica que um dos indicadores para o
biênio 2017/2018 trata da ampliação do uso da abordagem \textit{DevOps}. O
\acrfull{CPA} do \acrshort{TCU} definiu então que deveriam ser buscadas
experiências bem sucedidas da adoção de \textit{DevOps} no mercado e que a
adoção de \textit{DevOps} no TCU deveria ser pautada nessa prática de mercado.

Neste trabalho é descrito um estudo utilizando a abordagem \textit{Grounded
Theory} em sua variação clássica para caracterizar a adoção de \textit{DevOps}
na prática de mercado. Foram entrevistados praticantes que contribuíram para a
adoção bem sucedida de \textit{DevOps} em 15 empresas de cinco países com
diferentes tamanhos e domínios de negócio.

Um modelo é apresentado para melhorar o entendimento e a orientação sobre a
adoção de \textit{DevOps}. Este modelo incrementa a visão existente sobre
\textit{DevOps} explicando o papel e a motivação de cada elemento (e seus
relacionamentos) durante o processo de adoção de \textit{DevOps}. O modelo foi
organizado em termos de categorias de
\emph{facilitadores DevOps} e categorias de \emph{saídas DevOps}. Foi
identificado que construir uma cultura de colaboração é a principal preocupação
na adoção de \textit{DevOps}, contrastando com um possível entendimento que
implantar ferramentas \textit{DevOps} específicas ou automatizar tarefas de
operações específicas é suficiente para se implantar \textit{DevOps}.

O modelo criado no estudo foi apresentado ao \acrshort{CPA} do \acrshort{TCU} e
considerado relevante, desde então a evolução do uso de \textit{DevOps} no
\acrshort{TCU} tem sido pautada por ele. As percepções dos profissionais
envolvidos quanto ao estágio atual da adoção de \textit{DevOps} no \acrshort{TCU},
bem como quanto à utilidade e aplicabilidade do modelo proposto, foram coletadas
através da realização de um grupo focal.

De modo geral, os resultados contribuem para (a) gerar um adequado entendimento
da adoção de \textit{DevOps}, a partir da perspectiva de praticantes de mercado;
e (b) auxiliar o \acrshort{TCU}, e eventualmente outras instituições, no seu
processo de evolução no uso de \textit{DevOps}.
